\chapter{概率统计试卷汇总}

\section{复习题 1}
\subsubsection{选择题(每题 $3$ 分, 共 $21$ 分)}
\begin{enumerate}
	\item 从 $0,1,2,\ldots,9$ 中任意选出 $3$ 个不同的数字, 三个数字中不含 $0$ 与 $5$ 的概率是 (\hspace{1pc})
	\fourch{$\frac{1}{15}$}{$\frac{2}{15}$}{$\frac{14}{15}$}{$\frac{7}{15}$}
	
	\item 某人射击中靶的概率为 $\frac{3}{4}$ . 若射击直到中靶为止, 则射击次数为 $3$ 的概率为 (\hspace{1pc})
	\fourch{$\left(\frac{3}{4}\right)^3$}{$\left(\frac{1}{4}\right)^2\times\frac{3}{4}$}{$\left(\frac{1}{4}\right)^3$}{$\left(\frac{3}{4}\right)^2\times\frac{1}{4}$}
	
	\item 设随机变量 $X$ 的概率密度 $f(x)$ 满足 $f(-x)=f(x)$ , $F(x)$ 是分布函数, 则 (\hspace{1pc})
	\twoch{$F(-a)=1-F(a)$}{$F(-a)=\frac{1}{2}F(a)$}{$F(-a)=F(a)$}{$F(-a)=\frac{1}{2}-F(a)$}
	
	\item 设二维随机变量 $(X,Y)$ 的分布律为 $P\left\{X=i,Y=j\right\}=c\cdot i\cdot j,i=1,2,3,j=1,2,3$ , 则 $c=$ (\hspace{1pc})
	\fourch{$\frac{1}{12}$}{$\frac{1}{3}$}{$\frac{1}{36}$}{$\frac{1}{2}$}
	
	\item 设随机变量 $X$ 服从均匀分布, 其概率密度为 $f(x)=
	\begin{cases}
	\frac{1}{2}, & 1<x<3\\
	0, & \text{其他}
	\end{cases}
	$ , 则 $D(X)=$ (\hspace{1pc})
	\fourch{$3$}{$\frac{1}{3}$}{$\frac{1}{2}$}{$2$}
	
	\item 设总体 $X\sim N\left(0,\sigma^2\right)$ , $X_1,X_2,\ldots,X_n$ 是总体 $X$ 的一个样本, $\overline{X},S^2$ 分别为样本均值和样本方差, 则下列样本函数中, 服从 $\chi^2(n)$ 分布的是 (\hspace{1pc})
	\fourch{$\sum_{i=1}^{n}X_i^2$}{$\frac{\overline{X}}{S/\sqrt{n-1}}$}{$\frac{(n-1)S^2}{\sigma^2}$}{$\frac{1}{\sigma^2}\sum_{i=1}^{n}X_i^2$}
	
	\item 设 $X_1,X_2,\ldots,X_n$ 是来自正态总体 $N\left(\mu,\sigma^2\right)$ 的一个样本, $\sigma^2$ 未知, $\overline{X}$是样本均值, $S^2=\frac{1}{n-1}\sum_{i=1}^{n}\left(X_i-\overline{X}\right)^2$ , 如果 $\overline{X}-k\frac{S}{\sqrt{n}}$ 是 $\mu$ 的置信度为 $1-\alpha$ 的单侧置信下限, 则 $k$ 应取 (\hspace{1pc})
	\fourch{$t_{1-\alpha}(n)$}{$t_{\alpha}(n)$}{$t_{\alpha}(n-1)$}{$t_{\alpha/2}(n-1)$}	
\end{enumerate}

\subsubsection{填空题(每题 $3$ 分, 共 $21$ 分)}
\begin{enumerate}
	\item 设 $A,B$ 为随机事件, $P(A)=0.8$ , $P(A-B)=0.3$ , 则 $P\left(\overline{AB}\right)=$\underline{\hspace{8pc}}
	
	\item 设随机变量 $X$ 的分布律为 $P\left\{x=k\right\}=c(0.5)^k,k=1,2,3,\ldots$ , 则常数 $c=$\underline{\hspace{8pc}}
	
	\item 设随机变量 $X$ 的概率密度为 $f(x)=
	\begin{cases}
	3x^2, & 0<x<1\\
	0, & \text{其他}
	\end{cases}
	$ , 则 $P\left\{\left|X\right|<0.2\right\}=$\underline{\hspace{8pc}}
	
	\item 设随机变量 $X$ 的概率密度为 $f(x)=
	\begin{cases}
	\frac{1}{c}, & 0<x<c\\
	0, & \text{其他}
	\end{cases}
	$ , 则 $\EE(X)=$\underline{\hspace{8pc}}
	
	\item 设二维随机变量 $(X,Y)$ 的概率密度为
	\begin{equation*}
		f(x,y)=
		\begin{cases}
		\sin x\cdot\cos y, & 0<x<\uppi/2,\ 0<y<\uppi/2\\
		0, & \text{其他}
		\end{cases}
		,
	\end{equation*}
	
	则 $P\left\{0<X<\uppi/4,\uppi/4<Y<\uppi/2\right\}=$\underline{\hspace{8pc}}
	
	\item 设随机变量 $X$ 的数学期望 $\EE(X)=\mu$ , 方差 $D(X)=\sigma^2$ , 则由切比雪夫不等式有 $P\{|X-\mu|\geq3\sigma\}\leq$\underline{\hspace{8pc}}
	
	\item 设 $X_1,X_2$ 是取自正态总体 $X\sim N\left(\mu,\sigma^2\right)$ 的一个容量为 $2$ 的样本, 则 $\mu$ 的无偏估计量 $\hat\mu_1=\frac{1}{2}X_1+\frac{1}{2}X_2$ , $\hat\mu_2=\frac{2}{3}X_1+\frac{1}{3}X_2$ , $\hat\mu_3=\frac{1}{4}X_1+\frac{3}{4}X_2$ 中最有效的是\underline{\hspace{8pc}}
\end{enumerate}

\subsubsection{解答题(共 $58$ 分)}
\begin{enumerate}
	\item ( $10$ 分)车间里有甲、乙、丙 $3$ 台机床生产同一种产品, 已知它们的次品率依次是 $0.05$ 、 $0.1$ 、 $0.2$ , 产品所占份额依次是 $20\%$ 、 $30\%$ 、 $50\%$ . 现从产品中任取 $1$ 件, 发现它是次品, 求次品来自机床乙的概率.
	
	\item ( $10$ 分)设随机变量 $X$ 的分布函数为 $F(x)=
	\begin{cases}
	k-k\ee^{-x^3}, & x>0\\
	0, & x\leq0
	\end{cases}
	$ , 试求:
	\begin{enumerate}
		\item[(1)] 常数 $k$ ;
		\item[(2)] $X$ 的概率密度 $f(x)$ .
	\end{enumerate}

	\item ( $10$ 分)设二维随机变量 $(X,Y)$ 的概率密度为:
	\begin{equation*}
		f(x,y)=
		\begin{cases}
		\frac{1}{4}, & 2\leq x\leq4,1\leq y\leq3\\
		0, & \text{其他}
		\end{cases},
	\end{equation*}
	试求 $(X,Y)$ 关于 $X$ 与 $Y$ 的边缘概率密度 $f_X(x)$ 与 $f_Y(y)$ , 并判断 $X$ 与 $Y$ 是否相互独立.
	
	\item ( $10$ 分)已知红黄两种番茄杂交的第二代结红果的植株与结黄果的植株的比率为 $3:1$ , 现种植杂交种 $400$ 株, 试用中心极限定理近似计算, 结红果的植株介于 $285$ 与 $315$ 之间的概率. $\left(\varPhi\left(\sqrt{3}\right)=0.9582,\varPhi\left(\sqrt{2}\right)=0.9207\right)$
	
	\item ( $8$ 分)设二维随机变量 $(X,Y)$ 的分布律为
	\begin{center}
		\begin{tabularx}{0.8\textwidth}{ZZZZ}
			\hline
			 & \multicolumn{3}{c}{$Y$}\\
			\cline{2-4}
			$X$ & $-1$ & $0$ & $1$\\
			\hline
			$-1$ & $\frac{1}{8}$ & $\frac{1}{8}$ & $\frac{1}{8}$\\
			$0$ & $\frac{1}{8}$ & $0$ & $\frac{1}{8}$\\
			$1$ & $\frac{1}{8}$ & $\frac{1}{8}$ & $\frac{1}{8}$\\
			\hline
		\end{tabularx}
	\end{center}
	求 $\mathrm{Cov}(X,Y)$ .
	
	\item ( $10$ 分)设 $X_1,X_2,\ldots,X_n$ 为总体 $X$ 的一个样本, 总体 $X$ 的概率密度为:
	\begin{equation*}
		f(x)=
		\begin{cases}
		(\alpha+1)x^\alpha, & 0<x<1\\
		0, & \text{其他}
		\end{cases},
	\end{equation*}
	求未知参数 $\alpha$ 的矩估计.
\end{enumerate}

\section{复习题 1 答案}
\subsubsection{选择题(每题 $3$ 分, 共 $21$ 分)}
\begin{enumerate}
	\item 从 $0,1,2,\ldots,9$ 中任意选出 $3$ 个不同的数字, 三个数字中不含 $0$ 与 $5$ 的概率是 (\hspace{0.25pc}D\hspace{0.25pc})
	\fourch{$\frac{1}{15}$}{$\frac{2}{15}$}{$\frac{14}{15}$}{$\frac{7}{15}$}
	
	\item 某人射击中靶的概率为 $\frac{3}{4}$ . 若射击直到中靶为止, 则射击次数为 $3$ 的概率为 (\hspace{0.25pc}B\hspace{0.25pc})
	\fourch{$\left(\frac{3}{4}\right)^3$}{$\left(\frac{1}{4}\right)^2\times\frac{3}{4}$}{$\left(\frac{1}{4}\right)^3$}{$\left(\frac{3}{4}\right)^2\times\frac{1}{4}$}
	
	\item 设随机变量 $X$ 的概率密度 $f(x)$ 满足 $f(-x)=f(x)$ , $F(x)$ 是分布函数, 则 (\hspace{0.25pc}A\hspace{0.25pc})
	\twoch{$F(-a)=1-F(a)$}{$F(-a)=\frac{1}{2}F(a)$}{$F(-a)=F(a)$}{$F(-a)=\frac{1}{2}-F(a)$}
	
	\item 设二维随机变量 $(X,Y)$ 的分布律为 $P\left\{X=i,Y=j\right\}=c\cdot i\cdot j,i=1,2,3,j=1,2,3$ , 则 $c=$ (\hspace{0.25pc}C\hspace{0.25pc})
	\fourch{$\frac{1}{12}$}{$\frac{1}{3}$}{$\frac{1}{36}$}{$\frac{1}{2}$}
	
	\item 设随机变量 $X$ 服从均匀分布, 其概率密度为 $f(x)=
	\begin{cases}
	\frac{1}{2}, & 1<x<3\\
	0, & \text{其他}
	\end{cases}
	$ , 则 $D(X)=$ (\hspace{0.25pc}B\hspace{0.25pc})
	\fourch{$3$}{$\frac{1}{3}$}{$\frac{1}{2}$}{$2$}
	
	\item 设总体 $X\sim N\left(0,\sigma^2\right)$ , $X_1,X_2,\ldots,X_n$ 是总体 $X$ 的一个样本, $\overline{X},S^2$ 分别为样本均值和样本方差, 则下列样本函数中, 服从 $\chi^2(n)$ 分布的是 (\hspace{0.25pc}D\hspace{0.25pc})
	\fourch{$\sum_{i=1}^{n}X_i^2$}{$\frac{\overline{X}}{S/\sqrt{n-1}}$}{$\frac{(n-1)S^2}{\sigma^2}$}{$\frac{1}{\sigma^2}\sum_{i=1}^{n}X_i^2$}
	
	\item 设 $X_1,X_2,\ldots,X_n$ 是来自正态总体 $N\left(\mu,\sigma^2\right)$ 的一个样本, $\sigma^2$ 未知, $\overline{X}$是样本均值, $S^2=\frac{1}{n-1}\sum_{i=1}^{n}\left(X_i-\overline{X}\right)^2$ , 如果 $\overline{X}-k\frac{S}{\sqrt{n}}$ 是 $\mu$ 的置信度为 $1-\alpha$ 的单侧置信下限, 则 $k$ 应取 (\hspace{0.25pc}C\hspace{0.25pc})
	\fourch{$t_{1-\alpha}(n)$}{$t_{\alpha}(n)$}{$t_{\alpha}(n-1)$}{$t_{\alpha/2}(n-1)$}	
\end{enumerate}

\subsubsection{填空题(每题 $3$ 分, 共 $21$ 分)}
\begin{enumerate}
	\item 设 $A,B$ 为随机事件, $P(A)=0.8$ , $P(A-B)=0.3$ , 则 $P\left(\overline{AB}\right)=$\underline{\hspace{1pc}$0.5$\hspace{1pc}}
	
	\item 设随机变量 $X$ 的分布律为 $P\left\{x=k\right\}=c(0.5)^k,k=1,2,3,\ldots$ , 则常数 $c=$\underline{\hspace{1pc}$1$\hspace{1pc}}
	
	\item 设随机变量 $X$ 的概率密度为 $f(x)=
	\begin{cases}
	3x^2, & 0<x<1\\
	0, & \text{其他}
	\end{cases}
	$ , 则 $P\left\{\left|X\right|<0.2\right\}=$\underline{\hspace{1pc}$\frac{1}{125}$\hspace{1pc}}
	
	\item 设随机变量 $X$ 的概率密度为 $f(x)=
	\begin{cases}
	\frac{1}{c}, & 0<x<c\\
	0, & \text{其他}
	\end{cases}
	$ , 则 $\EE(X)=$\underline{\hspace{1pc}$\frac{c}{2}$\hspace{1pc}}
	
	\item 设二维随机变量 $(X,Y)$ 的概率密度为
	\begin{equation*}
	f(x,y)=
	\begin{cases}
	\sin x\cdot\cos y, & 0<x<\uppi/2,\ 0<y<\uppi/2\\
	0, & \text{其他}
	\end{cases}
	,
	\end{equation*}
	
	则 $P\left\{0<X<\uppi/4,\uppi/4<Y<\uppi/2\right\}=$\underline{\hspace{1pc}$\left( \frac{2-\sqrt{2}}{2} \right)^2$\hspace{1pc}}
	
	\item 设随机变量 $X$ 的数学期望 $\EE(X)=\mu$ , 方差 $D(X)=\sigma^2$ , 则由切比雪夫不等式有 $P\{|X-\mu|\geq3\sigma\}\leq$\underline{\hspace{1pc}$\frac{1}{9}$\hspace{1pc}}
	
	\item 设 $X_1,X_2$ 是取自正态总体 $X\sim N\left(\mu,\sigma^2\right)$ 的一个容量为 $2$ 的样本, 则 $\mu$ 的无偏估计量 $\hat\mu_1=\frac{1}{2}X_1+\frac{1}{2}X_2$ , $\hat\mu_2=\frac{2}{3}X_1+\frac{1}{3}X_2$ , $\hat\mu_3=\frac{1}{4}X_1+\frac{3}{4}X_2$ 中最有效的是\underline{\hspace{1pc}$\hat\mu_1$\hspace{1pc}}
\end{enumerate}

\subsubsection{解答题(共 $58$ 分)}
\begin{enumerate}
	\item ( $10$ 分)车间里有甲、乙、丙 $3$ 台机床生产同一种产品, 已知它们的次品率依次是 $0.05$ 、 $0.1$ 、 $0.2$ , 产品所占份额依次是 $20\%$ 、 $30\%$ 、 $50\%$ . 现从产品中任取 $1$ 件, 发现它是次品, 求次品来自机床乙的概率.
	\begin{solution}
		设抽取的产品为次品的事件为 $A$ , 抽取的次品来自机床甲的事件为 $B_1$ , 抽取的次品来自机床乙的事件为 $B_2$ , 抽取的次品来自机床丙的事件为 $B_3$ .
		
		根据全概率公式
		\begin{equation*}
			\begin{aligned}
			P(A)&=P(A|B_1)P(B_1)+P(A|B_2)P(B_2)+P(A|B_3)P(B_3)\\
			&=0.05\times0.2+0.1\times0.3+0.2\times0.5=0.14
			\end{aligned}
		\end{equation*}
		根据贝叶斯公式
		\begin{equation*}
			P(B_2|A)=\frac{P(AB_2)}{P(A)}=\frac{P(A|B_2)P(B_2)}{P(A)}=\frac{0.1\times0.3}{0.14}=\frac{3}{14}
		\end{equation*}
	\end{solution}
	
	\item ( $10$ 分)设随机变量 $X$ 的分布函数为 $F(x)=
	\begin{cases}
	k-k\ee^{-x^3}, & x>0\\
	0, & x\leq0
	\end{cases}
	$ , 试求:
	\begin{enumerate}
		\item[(1)] 常数 $k$ ;
		\item[(2)] $X$ 的概率密度 $f(x)$ .
	\end{enumerate}
	\begin{solution}
	  \begin{enumerate}
		\item[(1)] 根据分布函数的性质 $\lim_{x\to+\infty}F(x)=k=1$
		\item[(2)] $F(x)=
		\begin{cases}
		1-\ee^{-x^3}, & x>0\\
		0, & x\leq0 
		\end{cases}
		$ , 则$f(x)=F'(x)=
		\begin{cases}
		3x^2\ee^{-x^3}, & x>0\\
		0, & x\leq0
		\end{cases}
		$
	  \end{enumerate}
	\end{solution}

	\item ( $10$ 分)设二维随机变量 $(X,Y)$ 的概率密度为:
	\begin{equation*}
	f(x,y)=
	\begin{cases}
	\frac{1}{4}, & 2\leq x\leq4,1\leq y\leq3\\
	0, & \text{其他}
	\end{cases},
	\end{equation*}
	试求 $(X,Y)$ 关于 $X$ 与 $Y$ 的边缘概率密度 $f_X(x)$ 与 $f_Y(y)$ , 并判断 $X$ 与 $Y$ 是否相互独立.
	\begin{solution}
		$f_X(x)=\int_{-\infty}^{+\infty}f(x,y)\dd y=
		\begin{cases}
		\int_{1}^{3}\frac{1}{4}\dd y, & 2\leq x\leq4\\
		0, & \text{其它}
		\end{cases}=\begin{cases}
		\frac{1}{2}, & 2\leq x\leq4\\
		0, & \text{其它}
		\end{cases}
		$
		
		同理 $f_Y(y)=
		\begin{cases}
		\frac{1}{2}, & 1\leq y\leq3\\
		0, & \text{其它}
		\end{cases}
		$ , $f_X(x)f_Y(y)=
		\begin{cases}
		\frac{1}{4}, & 2\leq x\leq4,1\leq y\leq 3\\
		0, & \text{其它}
		\end{cases}=f(x,y)
		$
		
		因此 $X$ 与 $Y$ 相互独立
	\end{solution}
	
	\item ( $10$ 分)已知红黄两种番茄杂交的第二代结红果的植株与结黄果的植株的比率为 $3:1$ , 现种植杂交种 $400$ 株, 试用中心极限定理近似计算, 结红果的植株介于 $285$ 与 $315$ 之间的概率. $\left(\varPhi\left(\sqrt{3}\right)=0.9582,\varPhi\left(\sqrt{2}\right)=0.9207\right)$
	\begin{solution}
		设结红果的植株的株数为 $X$ , $X\sim B(400,3/4)$ , 则 $\EE(X)=300$ , $D(X)=75$
		
		根据中心极限定理
			\begin{align*}
			 P(285\leq X\leq 315)&=P\left(\frac{-15}{\sqrt{75}}\leq\frac{X-300}{\sqrt{75}}\leq\frac{15}{\sqrt{75}}\right)=\varPhi\left(\sqrt{3}\right)-\varPhi\left(-\sqrt{3}\right)\\
			&=2\varPhi\left(\sqrt{3}\right)-1=0.9164
			\end{align*}
	\end{solution}
	
	\item ( $8$ 分)设二维随机变量 $(X,Y)$ 的分布律为
	\begin{center}
		\begin{tabularx}{0.8\textwidth}{ZZZZ}
			\hline
			\multirow{2}*{$X$} & \multicolumn{3}{c}{$Y$}\\
			\cline{2-4}
			 & $-1$ & $0$ & $1$\\
			\hline
			$-1$ & $\frac{1}{8}$ & $\frac{1}{8}$ & $\frac{1}{8}$\\
			$0$ & $\frac{1}{8}$ & $0$ & $\frac{1}{8}$\\
			$1$ & $\frac{1}{8}$ & $\frac{1}{8}$ & $\frac{1}{8}$\\
			\hline
		\end{tabularx}
	\end{center}
	求 $\mathrm{Cov}(X,Y)$ .
	\begin{solution}
		$\EE(X)=-1\times\frac{3}{8}+1\times\frac{3}{8}=0$ , 同理通过计算得 $\EE(Y)=0$ , $\EE(XY)=0$
		
		因此 $\text{Cov}(X,Y)=\EE(XY)-\EE(X)\EE(Y)=0$
	\end{solution}
	
	\item ( $10$ 分)设 $X_1,X_2,\ldots,X_n$ 为总体 $X$ 的一个样本, 总体 $X$ 的概率密度为:
	\begin{equation*}
	f(x)=
	\begin{cases}
	(\alpha+1)x^\alpha, & 0<x<1\\
	0, & \text{其他}
	\end{cases},
	\end{equation*}
	求未知参数 $\alpha$ 的矩估计.
	\begin{solution}
		$\EE(X)=\int_{0}^{1}(\alpha+1)x^{\alpha+1}\dd x=\frac{\alpha+1}{\alpha+2}$ , $\mu_1=\overline{X}=\sum_{i=1}^{n}\frac{X_i}{n}$ , 因此 $\alpha=\frac{2\overline{X}-1}{1-\overline{X}}$
	\end{solution}
\end{enumerate}