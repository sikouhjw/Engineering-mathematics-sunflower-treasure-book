\chapter{概率统计试卷汇总}

\section{复习题 1}
\subsubsection{选择题(每题 $3$ 分, 共 $21$ 分)}
\begin{enumerate}
	\item 从 $0,1,2,\ldots,9$ 中任意选出 $3$ 个不同的数字, 三个数字中不含 $0$ 与 $5$ 的概率是 (\hspace{1pc})
	\fourch{$\frac{1}{15}$}
	{$\frac{2}{15}$}
	{$\frac{14}{15}$}
	{$\frac{7}{15}$}
	
	\item 某人射击中靶的概率为 $\frac{3}{4}$ . 若射击直到中靶为止, 则射击次数为 $3$ 的概率为 (\hspace{1pc})
	\fourch{$\left(\frac{3}{4}\right)^3$}
	{$\left(\frac{1}{4}\right)^2\times\frac{3}{4}$}
	{$\left(\frac{1}{4}\right)^3$}
	{$\left(\frac{3}{4}\right)^2\times\frac{1}{4}$}
	
	\item 设随机变量 $X$ 的概率密度 $f(x)$ 满足 $f(-x)=f(x)$ , $F(x)$ 是分布函数, 则 (\hspace{1pc})
	\twoch{$F(-a)=1-F(a)$}
	{$F(-a)=\frac{1}{2}F(a)$}
	{$F(-a)=F(a)$}
	{$F(-a)=\frac{1}{2}-F(a)$}
	
	\item 设二维随机变量 $(X,Y)$ 的分布律为 $P\left\{X=i,Y=j\right\}=c\cdot i\cdot j,i=1,2,3,j=1,2,3$ , 则 $c=$ (\hspace{1pc})
	\fourch{$\frac{1}{12}$}
	{$\frac{1}{3}$}
	{$\frac{1}{36}$}
	{$\frac{1}{2}$}
	
	\item 设随机变量 $X$ 服从均匀分布, 其概率密度为 $f(x)=
	\begin{cases}
	\frac{1}{2}, & 1<x<3\\
	0, & \text{其他}
	\end{cases}
	$ , 则 $D(X)=$ (\hspace{1pc})
	\fourch{$3$}
	{$\frac{1}{3}$}
	{$\frac{1}{2}$}
	{$2$}
	
	\item 设总体 $X\sim N\left(0,\sigma^2\right)$ , $X_1,X_2,\ldots,X_n$ 是总体 $X$ 的一个样本, $\overline{X},S^2$ 分别为样本均值和样本方差, 则下列样本函数中, 服从 $\chi^2(n)$ 分布的是 (\hspace{1pc})
	\fourch{$\sum_{i=1}^{n}X_i^2$}
	{$\frac{\overline{X}}{S/\sqrt{n-1}}$}
	{$\frac{(n-1)S^2}{\sigma^2}$}
	{$\frac{1}{\sigma^2}\sum_{i=1}^{n}X_i^2$}
	
	\item 设 $X_1,X_2,\ldots,X_n$ 是来自正态总体 $N\left(\mu,\sigma^2\right)$ 的一个样本, $\sigma^2$ 未知, $\overline{X}$是样本均值, $S^2=\frac{1}{n-1}\sum_{i=1}^{n}\left(X_i-\overline{X}\right)^2$ , 如果 $\overline{X}-k\frac{S}{\sqrt{n}}$ 是 $\mu$ 的置信度为 $1-\alpha$ 的单侧置信下限, 则 $k$ 应取 (\hspace{1pc})
	\fourch{$t_{1-\alpha}(n)$}
	{$t_{\alpha}(n)$}
	{$t_{\alpha}(n-1)$}
	{$t_{\alpha/2}(n-1)$}	
\end{enumerate}

\subsubsection{填空题(每题 $3$ 分, 共 $21$ 分)}
\begin{enumerate}
	\item 设 $A,B$ 为随机事件, $P(A)=0.8$ , $P(A-B)=0.3$ , 则 $P\left(\overline{AB}\right)=$\underline{\hspace{8pc}}
	
	\item 设随机变量 $X$ 的分布律为 $P\left\{x=k\right\}=c(0.5)^k,k=1,2,3,\ldots$ , 则常数 $c=$\underline{\hspace{8pc}}
	
	\item 设随机变量 $X$ 的概率密度为 $f(x)=
	\begin{cases}
	3x^2, & 0<x<1\\
	0, & \text{其他}
	\end{cases}
	$ , 则 $P\left\{\left|X\right|<0.2\right\}=$\underline{\hspace{8pc}}
	
	\item 设随机变量 $X$ 的概率密度为 $f(x)=
	\begin{cases}
	\frac{1}{c}, & 0<x<c\\
	0, & \text{其他}
	\end{cases}
	$ , 则 $\EE(X)=$\underline{\hspace{8pc}}
	
	\item 设二维随机变量 $(X,Y)$ 的概率密度为
	\begin{equation*}
		f(x,y)=
		\begin{cases}
		\sin x\cdot\cos y, & 0<x<\uppi/2,\ 0<y<\uppi/2\\
		0, & \text{其他}
		\end{cases}
		,
	\end{equation*}
	
	则 $P\left\{0<X<\uppi/4,\uppi/4<Y<\uppi/2\right\}=$\underline{\hspace{8pc}}
	
	\item 设随机变量 $X$ 的数学期望 $\EE(X)=\mu$ , 方差 $D(X)=\sigma^2$ , 则由切比雪夫不等式有 $P\{|X-\mu|\geq3\sigma\}\leq$\underline{\hspace{8pc}}
	
	\item 设 $X_1,X_2$ 是取自正态总体 $X\sim N\left(\mu,\sigma^2\right)$ 的一个容量为 $2$ 的样本, 则 $\mu$ 的无偏估计量 $\hat\mu_1=\frac{1}{2}X_1+\frac{1}{2}X_2$ , $\hat\mu_2=\frac{2}{3}X_1+\frac{1}{3}X_2$ , $\hat\mu_3=\frac{1}{4}X_1+\frac{3}{4}X_2$ 中最有效的是\underline{\hspace{8pc}}
\end{enumerate}

\subsubsection{解答题(共 $58$ 分)}
\begin{enumerate}
	\item ( $10$ 分)车间里有甲、乙、丙 $3$ 台机床生产同一种产品, 已知它们的次品率依次是 $0.05$ 、 $0.1$ 、 $0.2$ , 产品所占份额依次是 $20\%$ 、 $30\%$ 、 $50\%$ . 现从产品中任取 $1$ 件, 发现它是次品, 求次品来自机床乙的概率.
	
	\item ( $10$ 分)设随机变量 $X$ 的分布函数为 $F(x)=
	\begin{cases}
	k-k\ee^{-x^3}, & x>0\\
	0, & x\leq0
	\end{cases}
	$ , 试求:
	\begin{enumerate}
		\item 常数 $k$ ;
		\item $X$ 的概率密度 $f(x)$ .
	\end{enumerate}

	\item ( $10$ 分)设二维随机变量 $(X,Y)$ 的概率密度为:
	\begin{equation*}
		f(x,y)=
		\begin{cases}
		\frac{1}{4}, & 2\leq x\leq4,1\leq y\leq3\\
		0, & \text{其他}
		\end{cases},
	\end{equation*}
	试求 $(X,Y)$ 关于 $X$ 与 $Y$ 的边缘概率密度 $f_X(x)$ 与 $f_Y(y)$ , 并判断 $X$ 与 $Y$ 是否相互独立.
	
	\item ( $10$ 分)已知红黄两种番茄杂交的第二代结红果的植株与结黄果的植株的比率为 $3:1$ , 现种植杂交种 $400$ 株, 试用中心极限定理近似计算, 结红果的植株介于 $285$ 与 $315$ 之间的概率. $\left(\varPhi\left(\sqrt{3}\right)=0.9582,\varPhi\left(\sqrt{2}\right)=0.9207\right)$
	
	\item ( $8$ 分)设二维随机变量 $(X,Y)$ 的分布律为
	\begin{center}
		\begin{tabularx}{0.8\textwidth}{ZZZZ}
			\hline
			 & \multicolumn{3}{c}{$Y$}\\
			\cline{2-4}
			$X$ & $-1$ & $0$ & $1$\\
			\hline
			$-1$ & $\frac{1}{8}$ & $\frac{1}{8}$ & $\frac{1}{8}$\\
			$0$ & $\frac{1}{8}$ & $0$ & $\frac{1}{8}$\\
			$1$ & $\frac{1}{8}$ & $\frac{1}{8}$ & $\frac{1}{8}$\\
			\hline
		\end{tabularx}
	\end{center}
	求 $\mathrm{Cov}(X,Y)$ .
	
	\item ( $10$ 分)设 $X_1,X_2,\ldots,X_n$ 为总体 $X$ 的一个样本, 总体 $X$ 的概率密度为:
	\begin{equation*}
		f(x)=
		\begin{cases}
		(\alpha+1)x^\alpha, & 0<x<1\\
		0, & \text{其他}
		\end{cases},
	\end{equation*}
	求未知参数 $\alpha$ 的矩估计.
\end{enumerate}

\section{复习题 1 答案}
\subsubsection{选择题(每题 $3$ 分, 共 $21$ 分)}
\begin{enumerate}
	\item 从 $0,1,2,\ldots,9$ 中任意选出 $3$ 个不同的数字, 三个数字中不含 $0$ 与 $5$ 的概率是 (\hspace{0.25pc}D\hspace{0.25pc})
	\fourch{$\frac{1}{15}$}{$\frac{2}{15}$}{$\frac{14}{15}$}{$\frac{7}{15}$}
	
	\item 某人射击中靶的概率为 $\frac{3}{4}$ . 若射击直到中靶为止, 则射击次数为 $3$ 的概率为 (\hspace{0.25pc}B\hspace{0.25pc})
	\fourch{$\left(\frac{3}{4}\right)^3$}{$\left(\frac{1}{4}\right)^2\times\frac{3}{4}$}{$\left(\frac{1}{4}\right)^3$}{$\left(\frac{3}{4}\right)^2\times\frac{1}{4}$}
	
	\item 设随机变量 $X$ 的概率密度 $f(x)$ 满足 $f(-x)=f(x)$ , $F(x)$ 是分布函数, 则 (\hspace{0.25pc}A\hspace{0.25pc})
	\twoch{$F(-a)=1-F(a)$}{$F(-a)=\frac{1}{2}F(a)$}{$F(-a)=F(a)$}{$F(-a)=\frac{1}{2}-F(a)$}
	
	\item 设二维随机变量 $(X,Y)$ 的分布律为 $P\left\{X=i,Y=j\right\}=c\cdot i\cdot j,i=1,2,3,j=1,2,3$ , 则 $c=$ (\hspace{0.25pc}C\hspace{0.25pc})
	\fourch{$\frac{1}{12}$}{$\frac{1}{3}$}{$\frac{1}{36}$}{$\frac{1}{2}$}
	
	\item 设随机变量 $X$ 服从均匀分布, 其概率密度为 $f(x)=
	\begin{cases}
	\frac{1}{2}, & 1<x<3\\
	0, & \text{其他}
	\end{cases}
	$ , 则 $D(X)=$ (\hspace{0.25pc}B\hspace{0.25pc})
	\fourch{$3$}{$\frac{1}{3}$}{$\frac{1}{2}$}{$2$}
	
	\item 设总体 $X\sim N\left(0,\sigma^2\right)$ , $X_1,X_2,\ldots,X_n$ 是总体 $X$ 的一个样本, $\overline{X},S^2$ 分别为样本均值和样本方差, 则下列样本函数中, 服从 $\chi^2(n)$ 分布的是 (\hspace{0.25pc}D\hspace{0.25pc})
	\fourch{$\sum_{i=1}^{n}X_i^2$}{$\frac{\overline{X}}{S/\sqrt{n-1}}$}{$\frac{(n-1)S^2}{\sigma^2}$}{$\frac{1}{\sigma^2}\sum_{i=1}^{n}X_i^2$}
	
	\item 设 $X_1,X_2,\ldots,X_n$ 是来自正态总体 $N\left(\mu,\sigma^2\right)$ 的一个样本, $\sigma^2$ 未知, $\overline{X}$是样本均值, $S^2=\frac{1}{n-1}\sum_{i=1}^{n}\left(X_i-\overline{X}\right)^2$ , 如果 $\overline{X}-k\frac{S}{\sqrt{n}}$ 是 $\mu$ 的置信度为 $1-\alpha$ 的单侧置信下限, 则 $k$ 应取 (\hspace{0.25pc}C\hspace{0.25pc})
	\fourch{$t_{1-\alpha}(n)$}{$t_{\alpha}(n)$}{$t_{\alpha}(n-1)$}{$t_{\alpha/2}(n-1)$}	
\end{enumerate}

\subsubsection{填空题(每题 $3$ 分, 共 $21$ 分)}
\begin{enumerate}
	\item 设 $A,B$ 为随机事件, $P(A)=0.8$ , $P(A-B)=0.3$ , 则 $P\left(\overline{AB}\right)=$\underline{\hspace{1pc}$0.5$\hspace{1pc}}
	
	\item 设随机变量 $X$ 的分布律为 $P\left\{x=k\right\}=c(0.5)^k,k=1,2,3,\ldots$ , 则常数 $c=$\underline{\hspace{1pc}$1$\hspace{1pc}}
	
	\item 设随机变量 $X$ 的概率密度为 $f(x)=
	\begin{cases}
	3x^2, & 0<x<1\\
	0, & \text{其他}
	\end{cases}
	$ , 则 $P\left\{\left|X\right|<0.2\right\}=$\underline{\hspace{1pc}$\frac{1}{125}$\hspace{1pc}}
	
	\item 设随机变量 $X$ 的概率密度为 $f(x)=
	\begin{cases}
	\frac{1}{c}, & 0<x<c\\
	0, & \text{其他}
	\end{cases}
	$ , 则 $\EE(X)=$\underline{\hspace{1pc}$\frac{c}{2}$\hspace{1pc}}
	
	\item 设二维随机变量 $(X,Y)$ 的概率密度为
	\begin{equation*}
	f(x,y)=
	\begin{cases}
	\sin x\cdot\cos y, & 0<x<\uppi/2,\ 0<y<\uppi/2\\
	0, & \text{其他}
	\end{cases}
	,
	\end{equation*}
	
	则 $P\left\{0<X<\uppi/4,\uppi/4<Y<\uppi/2\right\}=$\underline{\hspace{1pc}$\left( \frac{2-\sqrt{2}}{2} \right)^2$\hspace{1pc}}
	
	\item 设随机变量 $X$ 的数学期望 $\EE(X)=\mu$ , 方差 $D(X)=\sigma^2$ , 则由切比雪夫不等式有 $P\{|X-\mu|\geq3\sigma\}\leq$\underline{\hspace{1pc}$\frac{1}{9}$\hspace{1pc}}
	
	\item 设 $X_1,X_2$ 是取自正态总体 $X\sim N\left(\mu,\sigma^2\right)$ 的一个容量为 $2$ 的样本, 则 $\mu$ 的无偏估计量 $\hat\mu_1=\frac{1}{2}X_1+\frac{1}{2}X_2$ , $\hat\mu_2=\frac{2}{3}X_1+\frac{1}{3}X_2$ , $\hat\mu_3=\frac{1}{4}X_1+\frac{3}{4}X_2$ 中最有效的是\underline{\hspace{1pc}$\hat\mu_1$\hspace{1pc}}
\end{enumerate}

\subsubsection{解答题(共 $58$ 分)}
\begin{enumerate}
	\item ( $10$ 分)车间里有甲、乙、丙 $3$ 台机床生产同一种产品, 已知它们的次品率依次是 $0.05$ 、 $0.1$ 、 $0.2$ , 产品所占份额依次是 $20\%$ 、 $30\%$ 、 $50\%$ . 现从产品中任取 $1$ 件, 发现它是次品, 求次品来自机床乙的概率.
	\begin{solution}
		设抽取的产品为次品的事件为 $A$ , 抽取的次品来自机床甲的事件为 $B_1$ , 抽取的次品来自机床乙的事件为 $B_2$ , 抽取的次品来自机床丙的事件为 $B_3$ .
		
		根据全概率公式
		\begin{equation*}
			\begin{aligned}
			P(A)&=P(A|B_1)P(B_1)+P(A|B_2)P(B_2)+P(A|B_3)P(B_3)\\
			&=0.05\times0.2+0.1\times0.3+0.2\times0.5=0.14
			\end{aligned}
		\end{equation*}
		根据贝叶斯公式
		\begin{equation*}
			P(B_2|A)=\frac{P(AB_2)}{P(A)}=\frac{P(A|B_2)P(B_2)}{P(A)}=\frac{0.1\times0.3}{0.14}=\frac{3}{14}
		\end{equation*}
	\end{solution}
	
	\item ( $10$ 分)设随机变量 $X$ 的分布函数为 $F(x)=
	\begin{cases}
	k-k\ee^{-x^3}, & x>0\\
	0, & x\leq0
	\end{cases}
	$ , 试求:
	\begin{enumerate}
		\item 常数 $k$ ;
		\item $X$ 的概率密度 $f(x)$ .
	\end{enumerate}
	\begin{solution}
	  \begin{enumerate}
		\item 根据分布函数的性质 $\lim_{x\to+\infty}F(x)=k=1$
		\item $F(x)=
		\begin{cases}
		1-\ee^{-x^3}, & x>0\\
		0, & x\leq0 
		\end{cases}
		$ , 则 $f(x)=F'(x)=
		\begin{cases}
		3x^2\ee^{-x^3}, & x>0\\
		0, & x\leq0
		\end{cases}
		$
	  \end{enumerate}
	\end{solution}

	\item ( $10$ 分)设二维随机变量 $(X,Y)$ 的概率密度为:
	\begin{equation*}
	f(x,y)=
	\begin{cases}
	\frac{1}{4}, & 2\leq x\leq4,1\leq y\leq3\\
	0, & \text{其他}
	\end{cases},
	\end{equation*}
	试求 $(X,Y)$ 关于 $X$ 与 $Y$ 的边缘概率密度 $f_X(x)$ 与 $f_Y(y)$ , 并判断 $X$ 与 $Y$ 是否相互独立.
	\begin{solution}
		$f_X(x)=\int_{-\infty}^{+\infty}f(x,y)\dd y=
		\begin{cases}
		\int_{1}^{3}\frac{1}{4}\dd y, & 2\leq x\leq4\\
		0, & \text{其它}
		\end{cases}=\begin{cases}
		\frac{1}{2}, & 2\leq x\leq4\\
		0, & \text{其它}
		\end{cases}
		$
		
		同理 $f_Y(y)=
		\begin{cases}
		\frac{1}{2}, & 1\leq y\leq3\\
		0, & \text{其它}
		\end{cases}
		$ , $f_X(x)f_Y(y)=
		\begin{cases}
		\frac{1}{4}, & 2\leq x\leq4,1\leq y\leq 3\\
		0, & \text{其它}
		\end{cases}=f(x,y)
		$
		
		因此 $X$ 与 $Y$ 相互独立
	\end{solution}
	
	\item ( $10$ 分)已知红黄两种番茄杂交的第二代结红果的植株与结黄果的植株的比率为 $3:1$ , 现种植杂交种 $400$ 株, 试用中心极限定理近似计算, 结红果的植株介于 $285$ 与 $315$ 之间的概率. $\left(\varPhi\left(\sqrt{3}\right)=0.9582,\varPhi\left(\sqrt{2}\right)=0.9207\right)$
	\begin{solution}
		设结红果的植株的株数为 $X$ , $X\sim B(400,3/4)$ , 则 $\EE(X)=300$ , $D(X)=75$
		
		根据中心极限定理
			\begin{align*}
			 P(285\leq X\leq 315)&=P\left(\frac{-15}{\sqrt{75}}\leq\frac{X-300}{\sqrt{75}}\leq\frac{15}{\sqrt{75}}\right)=\varPhi\left(\sqrt{3}\right)-\varPhi\left(-\sqrt{3}\right)\\
			&=2\varPhi\left(\sqrt{3}\right)-1=0.9164
			\end{align*}
	\end{solution}
	
	\item ( $8$ 分)设二维随机变量 $(X,Y)$ 的分布律为
	\begin{center}
		\begin{tabularx}{0.8\textwidth}{ZZZZ}
			\hline
			\multirow{2}*{$X$} & \multicolumn{3}{c}{$Y$}\\
			\cline{2-4}
			 & $-1$ & $0$ & $1$\\
			\hline
			$-1$ & $\frac{1}{8}$ & $\frac{1}{8}$ & $\frac{1}{8}$\\
			$0$ & $\frac{1}{8}$ & $0$ & $\frac{1}{8}$\\
			$1$ & $\frac{1}{8}$ & $\frac{1}{8}$ & $\frac{1}{8}$\\
			\hline
		\end{tabularx}
	\end{center}
	求 $\mathrm{Cov}(X,Y)$ .
	\begin{solution}
		$\EE(X)=-1\times\frac{3}{8}+1\times\frac{3}{8}=0$ , 同理通过计算得 $\EE(Y)=0$ , $\EE(XY)=0$
		
		因此 $\text{Cov}(X,Y)=\EE(XY)-\EE(X)\EE(Y)=0$
	\end{solution}
	
	\item ( $10$ 分)设 $X_1,X_2,\ldots,X_n$ 为总体 $X$ 的一个样本, 总体 $X$ 的概率密度为:
	\begin{equation*}
	f(x)=
	\begin{cases}
	(\alpha+1)x^\alpha, & 0<x<1\\
	0, & \text{其他}
	\end{cases},
	\end{equation*}
	求未知参数 $\alpha$ 的矩估计.
	\begin{solution}
		$\EE(X)=\int_{0}^{1}(\alpha+1)x^{\alpha+1}\dd x=\frac{\alpha+1}{\alpha+2}$ , $\mu_1=\overline{X}=\sum_{i=1}^{n}\frac{X_i}{n}$ , 因此 $\alpha=\frac{2\overline{X}-1}{1-\overline{X}}$
	\end{solution}
\end{enumerate}

\section{复习题2}
此复习题非一份完整的考试卷, 而是多个不同的试卷拼凑.

\subsubsection{选择题(每题 $3$ 分)}
\begin{enumerate}
	\item 已知事件 $A$ , $B$ 满足 $P(AB)=P\left( \overline{A}\bigcap\overline{B} \right)$ , 且 $P(A)=0.4$ , 则 $P(B)=$ (\hspace{1pc})
	\fourch{0.4}
	{0.5}
	{0.6}
	{0.7}

	\item 有 $\gamma$ 个球, 随机地放在 $n$ 个盒子中 $(\gamma \leq n)$ , 则某指定的 $\gamma$ 个盒子中各有一球的概率为 (\hspace{1pc})
	\fourch{$\frac{\gamma !}{n^{\gamma}}$}
	{$\mathrm{C}_n^r\frac{\gamma !}{n^\gamma}$}
	{$\frac{n!}{\gamma^n}$}
	{$\mathrm{C}_\gamma^n\frac{n!}{\gamma^n}$}

	\item 设随机变量 $X$ 的概率密度为 $f(x)=c\ee^{-\left|x\right|}$ , 则 $c=$ (\hspace{1pc})
	\fourch{$-\frac{1}{2}$}{$0$}{$\frac{1}{2}$}{$1$}

	\item 掷一颗骰子$600$次, 求“一点”出现次数的均值为 (\hspace{1pc})
	\fourch{$50$}
	{$100$}
	{$120$}
	{$150$}

	\item 设每次试验成功的概率为 $p (0<p<1)$ , 重复进行试验直到第 $n$ 次才取得 $r (1\leq r\leq n)$ 次成功的概率为 (\hspace{1pc})
	\twoch{$\mathrm{C}_{n-1}^{r-1}p^r(1-p)^{n-r}$}
	{$\mathrm{C}_{n}^{r}p^r(1-p)^{n-r}$}
	{$\mathrm{C}_{n-1}^{r-1}p^{r-1}(1-p)^{n-r+1}$}
	{$p^r(1-p)^{n-r}$}

	\item 离散型随机变量 $X$ 的分布函数为 $F(x)$ , 则 $P(X=x_k)=$ (\hspace{1pc})
	\twoch{$P(x_{k-1}\leq X\leq x_k)$}
	{$F(x_{k+1})-F(x_{k-1})$}
	{$P(x_{k-1}<X<x_{k+1})$}
	{$F(x_{k})-F(x_{k-1})$}

	\item 设随机变量 $X, Y$ 是相互独立的两个随机变量, 其分布函数分别为 $F_X(x), F_Y(y)$ , 则随机变量 $Z=\max (X,Y)$ 的分布函数为 (\hspace{1pc})
	\twoch{$F_Z(z)=\max\left\{F_X(x),F_Y(y)\right\}$}
	{$F_Z(z)=\max\left\{\left|F_X(x)\right|,\left|F_Y(y)\right|\right\}$}
	{$F_Z(z)=F_X(x)F_Y(y)$}
	{$F_Z(z)=F_X(z)F_Y(z)$}

	\item 设随机变量 $(X, Y)$ 的方差 $D(X)=4, D(Y)=1$ , 相关系数 $\rho_{XY}=0.6$ , 则方差 $D(3X-2Y)=$ (\hspace{1pc})
	\fourch{$40$}
	{$34$}
	{$25.6$}
	{$17.6$}

	\item 设 $(X_1,X_2,\ldots,X_n)$ 为总体 $N\left(1,2^2\right)$ 的一个样本, $\overline X$ 为样本均值, 则下列结论中正确的是 (\hspace{1pc})
	\twoch{$\frac{\overline{X}-1}{2/\sqrt{n}}\sim t(n)$}
	{$\frac{1}{4}\sum_{i=1}^{n}(X_i-1)^2\sim F(n,1)$}
	{$\frac{\overline{X}-1}{\sqrt{2}/\sqrt{n}}\sim N(0,1)$}
	{$\frac{1}{4}\sum_{i=1}^{n}(X_i-1)^2\sim\chi^2(n)$}

	\item 设总体 $X$ 在 $(\mu-\rho,\mu+\rho)$ 上服从均匀分布, 则参数 $\mu$ 的矩估计量为 (\hspace{1pc})
	\fourch{$\frac{1}{\overline{X}}$}
	{$\frac{1}{n-1}\sum_{i=1}^{n}X_i$}
	{$\frac{1}{n-1}\sum_{i=1}^{n}X_i^2$}
	{$\overline{X}$}

	\item 设二维随机变量 $(X,Y)$ 的分布律为:
	\begin{center}
		\begin{tabularx}{0.8\textwidth}{ZZZ}
			\hline
			 & \multicolumn{2}{c}{$Y$}\\
			\cline{2-3}
			$X$ & 1 & 2\\
			\hline
			1 & $a$ & $\frac{2}{9}$\\
			\hline
			2 & $b$ & $\frac{4}{9}$\\
			\hline
		\end{tabularx}
	\end{center}
	若 $X$ 与 $Y$ 相互独立, 则 (\hspace{1pc})
	\fourch{$a=\frac{4}{9}, b=\frac{1}{9}$}
	{$a=\frac{1}{9}, b=\frac{4}{9}$}
	{$a=\frac{2}{9}, b=\frac{1}{9}$}
	{$a=\frac{1}{9}, b=\frac{2}{9}$}

	\item 设随机变量 $X$ 的概率密度为 $f(x)=
	\begin{cases}
	ax, & 0\leq x\leq 2\\
	0, & \text{其他}
	\end{cases}
	$, 则 $a=$ (\hspace{1pc})
	\fourch{$1$}
	{$\frac{1}{4}$}
	{$\frac{1}{2}$}
	{$\frac{1}{3}$}

	\item 设 $E(X)=2$ , $E(Y)=3$ , 则 $E(3X-4Y+5)=$ (\hspace{1pc})
	\fourch{1}
	{6}
	{$-6$}
	{$-1$}

	\item 设随机变量 $X\sim N(1,4)$ , 则下列随机变量中服从标准正态分布的是 (\hspace{1pc})
	\fourch{$Y=\frac{X-1}{4}$}
	{$Y=\frac{X-1}{2}$}
	{$Y=\frac{X+1}{4}$}
	{$Y=\frac{X+1}{2}$}

	\item 设盒中有 $4$ 支铅笔, $2$ 支钢笔, 从盒中任取 $2$ 支笔(不放回抽样), 则取得 $1$ 支铅笔和 $1$ 支钢笔的概率是 (\hspace{1pc})
	\fourch{$\frac{8}{15}$}
	{$\frac{4}{5}$}
	{$\frac{3}{5}$}
	{$\frac{7}{15}$}

	\item 设总体 $X$ 服从正态分布 $N(\mu,1)$ , $X_1$ , $X_2$ 是来自总体 $X$ 的一个样本, $\hat\mu_1=\frac{2}{3}X_1+\frac{1}{3}X_2$ , 
	$\hat\mu_2=\frac{1}{4}X_1+\frac{3}{4}X_2$ , $\hat\mu_3=\frac{1}{2}X_1+\frac{1}{2}X_2$ 都是 $\mu$ 的无偏估计量, 则其中方差最小的是 (\hspace{1pc})
	\fourch{$\hat\mu_3$}
	{$\hat\mu_2$}
	{$\hat\mu_1$}
	{一样大}

	\item 设随机变量 $X\sim\chi^2(m_1)$ , $Y\sim\chi^2(m_2)$ , 且 $X$ 与 $Y$ 相互独立, 则下列结果中不正确的是 (\hspace{1pc})
	\twoch{$X+Y\sim\chi^2(m_1+m_2)$}
	{$D(Y)=m_2$}
	{$D(X)=2m_1$}
	{$E(X)=m_1$}
 \end{enumerate}

 \subsubsection{填空题(每题 3 分)}
 \begin{enumerate}
	\item 已知 $P(B)=0.3$ , $P\left(\overline{A}\bigcup B\right)=0.7$ , 且 $A$ 与 $B$ 相互独立, 则 $P(A)=$\underline{\hspace{8pc}}
	
	\item 设随机变量 $X$ 服从参数为 $\lambda$ 的泊松分布, 且 $P\left\{X=0\right\}=\frac{1}{3}$ , 则 $\lambda=$\underline{\hspace{8pc}}
	
	\item 设 $X\sim N\left(2,\sigma^2\right)$ , 且 $P\left\{2<X<4\right\}=0.2$ , 则 $P\left\{X<0\right\}=$\underline{\hspace{8pc}}
	
	\item 已知 $D(X)=2$ , $D(Y)=1$ , 且 $X$ 和 $Y$ 相互独立, 则 $D(X-2Y)=$\underline{\hspace{8pc}}
	
	\item 设 $S^2$ 是从 $N(0,1)$ 中抽取容量为 $16$ 的样本方差, 则 $D\left(S^2\right)=$\underline{\hspace{8pc}}

	\item 一批电子元件共有 $100$ 个, 次品数为 $5$ . 连续两次不放回地从中任取一个, 则第二次才取得正品的概率为\underline{\hspace{8pc}}
	
	\item 设事件 $A$ 与 $B$ 相互独立, $P(A)=0.2$ , $P(B)=0.3$ , 则 $P\left(A\bigcup B\right)=$\underline{\hspace{8pc}}
	
	\item 设 $\overline{X}$ 为总体 $X\sim N(3,4)$ 中抽取的样本 $(X_1,X_2,X_3,X_4)$ 的均值, 则 $P\left(-1<\overline{X}<5\right)=$
	
	\underline{\hspace{8pc}}
	
	\item 甲、乙两人独立地进行射击, 甲击中的概率为 $0.9$ , 乙击中的概率为 $0.8$ , 则甲中乙不中的概率等于\underline{\hspace{8pc}}
	
	\item 设 $P(A)=P(B)=P(C)=\frac{1}{3}$ , $ P(AB)=P(AC)=0$ , $ P(BC)=\frac{1}{5}$. 则 $A, B, C$ 中至少有一个发生的概率为\underline{\hspace{8pc}}
	
	\item 设二维随机变量 $(X,Y)$ 的概率密度为 $f(x,y)=
	\begin{cases}
	 C\ee^{-3x}\sin y, & x>0, 0<y<\frac{\uppi}{2}\\
	0, & \text{其他}
	\end{cases}
	$ , 则 $C=$
	
	\underline{\hspace{8pc}}

	\item 设随机变量 $X$ 服从参数为 $3$ 的指数分布, 则 $P\left\{X\geqslant  1\right\}=$\underline{\hspace{8pc}}

	\item 总体 $Y\sim N\left(\mu,\sigma^2\right)$ , $\overline{Y}=\frac{1}{n}\sum_{i=1}^{n}Y_i$ 为样本均值, $S$ 为样本标准差, 当 $\sigma$ 为未知时, $\mu$ 的置信度为 $1-\alpha (0<\alpha<1)$ 的双侧置信区间为\underline{\hspace{8pc}}

	\item 设 $X$ , $Y$ 为两个随机变量, 已知 $E(X)=1$ , $E(Y)=2$ , $E(XY)=5$ , 则 $\Cov(X,Y-4)=$
	
	\underline{\hspace{8pc}}
	
	\item 若 $P(A)=0.5$ , $P\left(B\overline{A}\right)=0.2$ , 则 $P(A+B)=$\underline{\hspace{8pc}}
	
	\item 已知随机变量 $X\sim
	\begin{bmatrix}
	-1 & 0 & 2 & 5\\
	0.25 & 0.25 & 0.25 & 0.25
	\end{bmatrix}
	$ , 那么 $E(X)=$\underline{\hspace{8pc}}
	
	\item 设 $\hat\theta$ 是未知参数 $\theta$ 的一个无偏估计量, 则 $E(\hat\theta)=$\underline{\hspace{8pc}}

	\item 随机变量 $X$ 服从均匀分布 $U(1,3)$ , 则 $P(X>2)=$\underline{\hspace{8pc}}
	
	\item 设随机变量 $X\sim B(100,0.15)$ , 则 $E(X)=$\underline{\hspace{8pc}}
	
	\item 设随机变量 $X\sim N(3,4)$ , 已知 $\varPhi(1)=0.8413$ , 则 $P(X<1)=$\underline{\hspace{8pc}}
	
	\item 设随机变量 $X$ 的概率密度函数为 $f(x)=
	\begin{cases}
	3x^2, & 0\leq x\leq 1\\
	0, & \text{其它}
	\end{cases}
	$, 则$ P\left(X<\frac{1}{2}\right)=$
	
	\underline{\hspace{8pc}}

	\item 设随机变量 $X\sim
	\begin{bmatrix}
	0 & 1 & 3\\
	0.5 & 0.35 & 0.15
	\end{bmatrix}
	$ , 则 $P(X<2)=$\underline{\hspace{8pc}}
	
	\item 设随机变量 $X$ 的期望存在, 则 $E(X-E(X))=$\underline{\hspace{8pc}}
	
	\item 设 $X$ 为随机变量, 已知 $D(X)=2$ , 那么 $D(3X-5)=$\underline{\hspace{8pc}}
 \end{enumerate}

 \subsubsection{计算题}
 \begin{enumerate}
	\item (10分)设随机变量 $X$ 与 $Y$ 具有概率密度: $f(x,y)=
	\begin{cases}
	\frac{1}{8}(x+y) & 0\leq x\leq 2, 0\leq y\leq 2\\
	0 & \text{其它}
	\end{cases}
	$ . 试求: $D(X)$ , $D(Y)$ , 与 $D(2X-3Y)$ .

	\item (10分)某电子计算机主机有 $100$ 个终端, 每个终端有 $80\%$ 的时间被使用. 
	若各个终端是否被使用是相互独立的, 试求至少有 $15$ 个终端空闲的概率. $(\varPhi(1.25)=0.8944, \varPhi(0.31)=0.6217$
	
	\item (10分)试求正态总体 $N\left(\mu,0.5^2\right)$ 的容量分别为 $10$ , $15$ 的两独立样本均值差的绝对值大于 $0.4$ 的概率. $(\varPhi(1.96)=0.975)$
	
	\item (10分)设总体 $X$ 的密度函数为 $f(x)=
	\begin{cases}
	\frac{2}{\theta^2}(\theta-x), & 0<x<\theta\\
	0, & \text{其它}
	\end{cases}
	$ , $\theta>0$ , $X_1,X_2,\ldots,X_{10}$ 为来自总体 $X$ 的样本, 试求当样本观察值分别为 $0.5$ , $1.3$ , $0.6$ , $1.7$ , $2.2$ , $1.2$ , $0.8$ , $1.5$ , $2.0$ , $1.6$ 时未知参数 $\theta$ 的矩估计值.

	\item (10分)某商店拥有某产品共计 $12$ 件, 其中 $4$ 件次品, 已经售出 $2$ 件, 现从剩下的 $10$ 件产品中任取一件, 求这件是正品的概率.
	
	\item (10分)设某种电子元件的寿命服从正态分布 $N(40,100)$ , 随机地取 $5$ 个元件, 求恰有两个元件寿命小于 $50$ 的概率. $(\varPhi(1)=0.8413, \varPhi(2)=0.9772)$

	\item (12分)设总体 $X$ 的分布律为 $P\left\{X=k\right\}=(1-p)^{k-1}p, k=1,2,\ldots.$ ( $p$ 为未知参数), $X_1,X_2,\ldots,X_n$ 是总体 $X$ 的一个样本, 求 $p$ 的极大似然估计量.

	\item (10分)两台车床加工同样的零件, 第一台出现不合格品的概率是 $0.03$ , 第二台出现不合格品的概率是 $0.06$ , 加工出来的零件放在一起, 并且已知第一台加工的零件数比第二台加工的零件数多一倍.
	\begin{enumerate}
		\item 求任取一个零件是合格品的概率.
		\item 如果取出的零件是不合格品, 求它是由第二台车床加工的概率.
	\end{enumerate}

	\item (10分)某仪器装了 $3$ 个独立工作的同型号电子元件, 其寿命(单位: 小时)都服从同一指数分布, 密度函数为 $f(x)=
	\begin{cases}
	\frac{1}{600}\ee^{-\frac{x}{600}}, & x>0\\
	0, & \text{其它}
	\end{cases}
	$ , 试求此仪器在最初使用的 $200$ 小时内, 至少有一个此种电子元件损坏的概率.
\end{enumerate}