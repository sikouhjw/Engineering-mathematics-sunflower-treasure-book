\chapter{《概率论与数理统计》试卷汇总}

\section{秋}
\subsection{2018-2019 B14}
\subsubsection{选择题(每题 $3$ 分, 共 $21$ 分)}
\begin{ti}
	从 $0,1,2,\cdots,9$ 中任意选出 $3$ 个不同的数字, 三个数字中不含 $0$ 与 $5$ 的概率是 \kuo{}
	\fourch{$\frac{1}{15}$}
	{$\frac{2}{15}$}
	{$\frac{14}{15}$}
	{$\frac{7}{15}$}
\end{ti}

\begin{ti}
	某人射击中靶的概率为 $\frac{3}{4}$ . 若射击直到中靶为止, 则射击次数为 $3$ 的概率为 \kuo{}
	\fourch{$\left(\frac{3}{4}\right)^3$}
	{$\left(\frac{1}{4}\right)^2\times\frac{3}{4}$}
	{$\left(\frac{1}{4}\right)^3$}
	{$\left(\frac{3}{4}\right)^2\times\frac{1}{4}$}
\end{ti}

\begin{ti}
	设随机变量 $X$ 的概率密度 $f(x)$ 满足 $f(-x)=f(x)$ , $F(x)$ 是分布函数, 则 \kuo{}
	\twoch{$F(-a)=1-F(a)$}
	{$F(-a)=\frac{1}{2}F(a)$}
	{$F(-a)=F(a)$}
	{$F(-a)=\frac{1}{2}-F(a)$}
\end{ti}

\begin{ti}
	设二维随机变量 $(X,Y)$ 的分布律为 $P\left\{X=i,Y=j\right\}=c\cdot i\cdot j,i=1,2,3,j=1,2,3$, 则 $c=$ \kuo{}
	\fourch{$\frac{1}{12}$}
	{$\frac{1}{3}$}
	{$\frac{1}{36}$}
	{$\frac{1}{2}$}
\end{ti}

\begin{ti}
	设随机变量 $X$ 服从均匀分布, 其概率密度为 $f(x)=
	\begin{cases}
	\frac{1}{2}, & 1<x<3\\
	0, & \text{其他}
	\end{cases}
	$, 则 $D(X)=$ \kuo{}
	\fourch{$3$}
	{$\frac{1}{3}$}
	{$\frac{1}{2}$}
	{$2$}
\end{ti}

\begin{ti}
	设总体 $X\sim N\left(0,\sigma^2\right)$, $X_1,X_2,\cdots,X_n$ 是总体 $X$ 的一个样本, $\overline{X},S^2$ 分别为样本均值和样本方差, 则下列样本函数中, 服从 $\chi^2(n)$ 分布的是 \kuo{}
	\fourch{$\sum_{i=1}^{n}X_i^2$}
	{$\frac{\overline{X}}{S/\sqrt{n-1}}$}
	{$\frac{(n-1)S^2}{\sigma^2}$}
	{$\frac{1}{\sigma^2}\sum_{i=1}^{n}X_i^2$}
\end{ti}

\begin{ti}
	设 $X_1,X_2,\cdots,X_n$ 是来自正态总体 $N\left(\mu,\sigma^2\right)$ 的一个样本, $\sigma^2$ 未知, $\overline{X}$是样本均值, $S^2=\frac{1}{n-1}\sum_{i=1}^{n}\Bigl(X_i-\overline{X}\Bigr)^2$, 如果 $\overline{X}-k\frac{S}{\sqrt{n}}$ 是 $\mu$ 的置信度为 $1-\alpha$ 的单侧置信下限, 则 $k$ 应取 \kuo{}
	\fourch{$t_{1-\alpha}(n)$}
	{$t_{\alpha}(n)$}
	{$t_{\alpha}(n-1)$}
	{$t_{\alpha/2}(n-1)$}
\end{ti}

\subsubsection{填空题(每题 $3$ 分, 共 $21$ 分)}
\begin{ti}
	设 $A,B$ 为随机事件, $P(A)=0.8$ , $P(A-B)=0.3$ , 则 $P\left(\overline{AB}\right)=$ \hua{}
\end{ti}

\begin{ti}
	设随机变量 $X$ 的分布律为 $P\left\{x=k\right\}=c(0.5)^k,k=1,2,3,\cdots$, 则常数 $c=$ \hua{}
\end{ti}

\begin{ti}
	设随机变量 $X$ 的概率密度为 $f(x)=
	\begin{cases}
	3x^2, & 0<x<1\\
	0, & \text{其他}
	\end{cases}
	$, 则 $P\left\{\left|X\right|<0.2\right\}=$ \hua{}
\end{ti}

\begin{ti}
	设随机变量 $X$ 的概率密度为 $f(x)=
	\begin{cases}
	\frac{1}{c}, & 0<x<c\\
	0, & \text{其他}
	\end{cases}
	$, 则 $\EE(X)=$ \hua{}
\end{ti}

\begin{ti}
	设二维随机变量 $(X,Y)$ 的概率密度为
	\begin{equation*}
		f(x,y)=
		\begin{cases}
		\sin x\cdot\cos y, & 0<x<\uppi/2,\ 0<y<\uppi/2\\
		0, & \text{其他}
		\end{cases},
	\end{equation*}
	则 $P\left\{0<X<\uppi/4,\uppi/4<Y<\uppi/2\right\}=$ \hua{}
\end{ti}

\begin{ti}
	设随机变量 $X$ 的数学期望 $\EE(X)=\mu$, 方差 $D(X)=\sigma^2$, 则由切比雪夫不等式有 $P\{|X-\mu|\geq3\sigma\}\leq$ \hua{}
\end{ti}

\begin{ti}
	设 $X_1,X_2$ 是取自正态总体 $X\sim N\left(\mu,\sigma^2\right)$ 的一个容量为 $2$ 的样本, 则 $\mu$ 的无偏估计量 $\hat\mu_1=\frac{1}{2}X_1+\frac{1}{2}X_2$, $\hat\mu_2=\frac{2}{3}X_1+\frac{1}{3}X_2$, $\hat\mu_3=\frac{1}{4}X_1+\frac{3}{4}X_2$ 中最有效的是 \hua{}
\end{ti}

\subsubsection{解答题(共 $58$ 分)}
\begin{ti}[$10$ 分]
	车间里有甲、乙、丙 $3$ 台机床生产同一种产品, 已知它们的次品率依次是 $0.05$、 $0.1$、 $0.2$, 产品所占份额依次是 $20\%$、 $30\%$、 $50\%$. 现从产品中任取 $1$ 件, 发现它是次品, 求次品来自机床乙的概率
\end{ti}

\begin{ti}[$10$ 分]
	设随机变量 $X$ 的分布函数为 $F(x)=
	\begin{cases}
	k-k\ee^{-x^3}, & x>0\\
	0, & x\leq0
	\end{cases}
	$, 试求:
	\begin{enumerate}
		\item 常数 $k$;
		\item $X$ 的概率密度 $f(x)$
	\end{enumerate}
\end{ti}

\begin{ti}[$10$ 分]
	设二维随机变量 $(X,Y)$ 的概率密度为:
	\begin{equation*}
		f(x,y)=
		\begin{cases}
		\frac{1}{4}, & 2\leqslant x\leq4,1\leqslant y\leq3\\
		0, & \text{其他}
		\end{cases},
	\end{equation*}
	试求 $(X,Y)$ 关于 $X$ 与 $Y$ 的边缘概率密度 $f_X(x)$ 与 $f_Y(y)$, 并判断 $X$ 与 $Y$ 是否相互独立
\end{ti}

\begin{ti}[$10$ 分]
	已知红黄两种番茄杂交的第二代结红果的植株与结黄果的植株的比率为 $3:1$, 现种植杂交种 $400$ 株, 试用中心极限定理近似计算, 结红果的植株介于 $285$ 与 $315$ 之间的概率. $\Bigl(\varPhi\Bigl(\sqrt{3}\Bigr)=0.9582,\allowbreak\varPhi\Bigl(\sqrt{2}\Bigr)=0.9207\Bigr)$
\end{ti}

\begin{ti}[$8$ 分]
	设二维随机变量 $(X,Y)$ 的分布律为
	\begin{center}
		\begin{tabularx}{0.8\textwidth}{ZZZZ}
			\hline
			 & \multicolumn{3}{c}{$Y$}\\
			\cline{2-4}
			$X$ & $-1$ & $0$ & $1$\\
			\hline
			$-1$ & $\frac{1}{8}$ & $\frac{1}{8}$ & $\frac{1}{8}$\\
			$0$ & $\frac{1}{8}$ & $0$ & $\frac{1}{8}$\\
			$1$ & $\frac{1}{8}$ & $\frac{1}{8}$ & $\frac{1}{8}$\\
			\hline
		\end{tabularx}
	\end{center}
	求 $\Cov(X,Y)$
\end{ti}

\begin{ti}[$10$ 分]
	设 $X_1,X_2,\cdots,X_n$ 为总体 $X$ 的一个样本, 总体 $X$ 的概率密度为:
	\begin{equation*}
		f(x)=
		\begin{cases}
		(\alpha+1)x^\alpha, & 0<x<1\\
		0, & \text{其他}
		\end{cases},
	\end{equation*}
	求未知参数 $\alpha$ 的矩估计
\end{ti}

\subsection{2018-2019 B14 答案}
\subsubsection{选择题(每题 $3$ 分, 共 $21$ 分)}
\begin{ti}
	从 $0,1,2,\cdots,9$ 中任意选出 $3$ 个不同的数字, 三个数字中不含 $0$ 与 $5$ 的概率是 \kuoD{}
	\fourch{$\frac{1}{15}$}
	{$\frac{2}{15}$}
	{$\frac{14}{15}$}
	{$\frac{7}{15}$}
\end{ti}

\begin{ti}
	某人射击中靶的概率为 $\frac{3}{4}$ . 若射击直到中靶为止, 则射击次数为 $3$ 的概率为 \kuoB{}
	\fourch{$\left(\frac{3}{4}\right)^3$}
	{$\left(\frac{1}{4}\right)^2\times\frac{3}{4}$}
	{$\left(\frac{1}{4}\right)^3$}
	{$\left(\frac{3}{4}\right)^2\times\frac{1}{4}$}
\end{ti}

\begin{ti}
	设随机变量 $X$ 的概率密度 $f(x)$ 满足 $f(-x)=f(x)$ , $F(x)$ 是分布函数, 则 \kuoA{}
	\twoch{$F(-a)=1-F(a)$}
	{$F(-a)=\frac{1}{2}F(a)$}
	{$F(-a)=F(a)$}
	{$F(-a)=\frac{1}{2}-F(a)$}
\end{ti}

\begin{ti}
	设二维随机变量 $(X,Y)$ 的分布律为 $P\left\{X=i,Y=j\right\}=c\cdot i\cdot j,i=1,2,3,j=1,2,3$, 则 $c=$ \kuoC{}
	\fourch{$\frac{1}{12}$}
	{$\frac{1}{3}$}
	{$\frac{1}{36}$}
	{$\frac{1}{2}$}
\end{ti}

\begin{ti}
	设随机变量 $X$ 服从均匀分布, 其概率密度为 $f(x)=
	\begin{cases}
	\frac{1}{2}, & 1<x<3\\
	0, & \text{其他}
	\end{cases}
	$, 则 $D(X)=$ \kuoB{}
	\fourch{$3$}
	{$\frac{1}{3}$}
	{$\frac{1}{2}$}
	{$2$}
\end{ti}

\begin{ti}
	设总体 $X\sim N\left(0,\sigma^2\right)$, $X_1,X_2,\cdots,X_n$ 是总体 $X$ 的一个样本, $\overline{X},S^2$ 分别为样本均值和样本方差, 则下列样本函数中, 服从 $\chi^2(n)$ 分布的是 \kuoD{}
	\fourch{$\sum_{i=1}^{n}X_i^2$}
	{$\frac{\overline{X}}{S/\sqrt{n-1}}$}
	{$\frac{(n-1)S^2}{\sigma^2}$}
	{$\frac{1}{\sigma^2}\sum_{i=1}^{n}X_i^2$}
\end{ti}

\begin{ti}
	设 $X_1,X_2,\cdots,X_n$ 是来自正态总体 $N\left(\mu,\sigma^2\right)$ 的一个样本, $\sigma^2$ 未知, $\overline{X}$是样本均值, $S^2=\frac{1}{n-1}\sum_{i=1}^{n}\Bigl(X_i-\overline{X}\Bigr)^2$, 如果 $\overline{X}-k\frac{S}{\sqrt{n}}$ 是 $\mu$ 的置信度为 $1-\alpha$ 的单侧置信下限, 则 $k$ 应取 \kuoC{}
	\fourch{$t_{1-\alpha}(n)$}
	{$t_{\alpha}(n)$}
	{$t_{\alpha}(n-1)$}
	{$t_{\alpha/2}(n-1)$}
\end{ti}

\subsubsection{填空题(每题 $3$ 分, 共 $21$ 分)}
\begin{ti}
	设 $A,B$ 为随机事件, $P(A)=0.8$ , $P(A-B)=0.3$ , 则 $P\left(\overline{AB}\right)=$ \huaa{$0.5$}
\end{ti}

\begin{ti}
	设随机变量 $X$ 的分布律为 $P\left\{x=k\right\}=c(0.5)^k,k=1,2,3,\cdots$, 则常数 $c=$ \huaa{$1$}
\end{ti}

\begin{ti}
	设随机变量 $X$ 的概率密度为 $f(x)=
	\begin{cases}
	3x^2, & 0<x<1\\
	0, & \text{其他}
	\end{cases}
	$, 则 $P\left\{\left|X\right|<0.2\right\}=$ \huaa{$\frac{1}{125}$}
\end{ti}

\begin{ti}
	设随机变量 $X$ 的概率密度为 $f(x)=
	\begin{cases}
	\frac{1}{c}, & 0<x<c\\
	0, & \text{其他}
	\end{cases}
	$, 则 $\EE(X)=$ \huaa{$\frac{c}{2}$}
\end{ti}

\begin{ti}
	设二维随机变量 $(X,Y)$ 的概率密度为
	\begin{equation*}
		f(x,y)=
		\begin{cases}
		\sin x\cdot\cos y, & 0<x<\uppi/2,\ 0<y<\uppi/2\\
		0, & \text{其他}
		\end{cases},
	\end{equation*}
	则 $P\left\{0<X<\uppi/4,\uppi/4<Y<\uppi/2\right\}=$ \huaa{$\left( \frac{2-\sqrt{2}}{2} \right)^2$}
\end{ti}

\begin{ti}
	设随机变量 $X$ 的数学期望 $\EE(X)=\mu$, 方差 $D(X)=\sigma^2$, 则由切比雪夫不等式有 $P\{|X-\mu|\geq3\sigma\}\leq$ \huaa{$\frac{1}{9}$}
\end{ti}

\begin{ti}
	设 $X_1,X_2$ 是取自正态总体 $X\sim N\left(\mu,\sigma^2\right)$ 的一个容量为 $2$ 的样本, 则 $\mu$ 的无偏估计量 $\hat\mu_1=\frac{1}{2}X_1+\frac{1}{2}X_2$, $\hat\mu_2=\frac{2}{3}X_1+\frac{1}{3}X_2$, $\hat\mu_3=\frac{1}{4}X_1+\frac{3}{4}X_2$ 中最有效的是 \huaa{$\hat\mu_1$}
\end{ti}


\subsubsection{解答题(共 $58$ 分)}
\begin{ti}[$10$ 分]
	车间里有甲、乙、丙 $3$ 台机床生产同一种产品, 已知它们的次品率依次是 $0.05$、 $0.1$、 $0.2$, 产品所占份额依次是 $20\%$、 $30\%$、 $50\%$. 现从产品中任取 $1$ 件, 发现它是次品, 求次品来自机床乙的概率
	\begin{solution}
		设抽取的产品为次品的事件为 $A$, 抽取的次品来自机床甲的事件为 $B_1$, 抽取的次品来自机床乙的事件为 $B_2$, 抽取的次品来自机床丙的事件为 $B_3$
		
		根据全概率公式
		\begin{equation*}
			\begin{aligned}
			P(A)&=P(A|B_1)P(B_1)+P(A|B_2)P(B_2)+P(A|B_3)P(B_3)\\
			&=0.05\times0.2+0.1\times0.3+0.2\times0.5=0.14
			\end{aligned}
		\end{equation*}
		根据贝叶斯公式
		\begin{equation*}
			P(B_2|A)=\frac{P(AB_2)}{P(A)}=\frac{P(A|B_2)P(B_2)}{P(A)}=\frac{0.1\times0.3}{0.14}=\frac{3}{14}
		\end{equation*}
	\end{solution}
\end{ti}

\begin{ti}[$10$ 分]
	设随机变量 $X$ 的分布函数为 $F(x)=
	\begin{cases}
	k-k\ee^{-x^3}, & x>0\\
	0, & x\leq0
	\end{cases}
	$, 试求:
	\begin{enumerate}
		\item 常数 $k$;
		\item $X$ 的概率密度 $f(x)$
	\end{enumerate}
	\begin{solution}
		\begin{enumerate}
		  \item 根据分布函数的性质
		  \[
			  \lim_{x\to+\infty}F(x)=k=1
		  \]
		  \item 
		  \[
			  F(x)=
		  	\begin{cases}
		  	1-\ee^{-x^3}, & x>0\\
		  	0, & x\leq0 
		  	\end{cases},
		  \]
		  则
		  \[
			  f(x)=F'(x)=
		  	\begin{cases}
		  	3x^2\ee^{-x^3}, & x>0\\
		  	0, & x\leq0
		  \end{cases}
		  \]
		\end{enumerate}
	  \end{solution}
\end{ti}

\begin{ti}[$10$ 分]
	设二维随机变量 $(X,Y)$ 的概率密度为:
	\begin{equation*}
		f(x,y)=
		\begin{cases}
		\frac{1}{4}, & 2\leqslant x\leq4,1\leqslant y\leq3\\
		0, & \text{其他}
		\end{cases},
	\end{equation*}
	试求 $(X,Y)$ 关于 $X$ 与 $Y$ 的边缘概率密度 $f_X(x)$ 与 $f_Y(y)$, 并判断 $X$ 与 $Y$ 是否相互独立
	\begin{solution}
		\[
		f_X(x)=\int_{-\infty}^{+\infty}f(x,y)\dd y=
		\begin{cases}
		\int_{1}^{3}\frac{1}{4}\dd y, & 2\leqslant x\leq4\\
		0, & \text{其它}
		\end{cases}=\begin{cases}
		\frac{1}{2}, & 2\leqslant x\leq4\\
		0, & \text{其它}
		\end{cases}
		\]
		同理
		\[
		f_Y(y)=
		\begin{cases}
		\frac{1}{2}, & 1\leqslant y\leq3\\
		0, & \text{其它}
		\end{cases},
		f_X(x)f_Y(y)=
		\begin{cases}
		\frac{1}{4}, & 2\leqslant x\leq4,1\leqslant y\leqslant 3\\
		0, & \text{其它}
		\end{cases}=f(x,y)
		\]
		因此 $X$ 与 $Y$ 相互独立
	\end{solution}
\end{ti}

\begin{ti}[$10$ 分]
	已知红黄两种番茄杂交的第二代结红果的植株与结黄果的植株的比率为 $3:1$, 现种植杂交种 $400$ 株, 试用中心极限定理近似计算, 结红果的植株介于 $285$ 与 $315$ 之间的概率. $\Bigl(\varPhi\Bigl(\sqrt{3}\Bigr)=0.9582,\allowbreak\varPhi\Bigl(\sqrt{2}\Bigr)=0.9207\Bigr)$
	\begin{solution}
		设结红果的植株的株数为 $X$, $X\sim B(400,3/4)$, 则 $\EE(X)=300$, $D(X)=75$, 根据中心极限定理
			\begin{align*}
			 P(285\leqslant X\leqslant 315)&=P\left(\frac{-15}{\sqrt{75}}\leq\frac{X-300}{\sqrt{75}}\leq\frac{15}{\sqrt{75}}\right)=\varPhi\left(\sqrt{3}\right)-\varPhi\left(-\sqrt{3}\right)\\
			&=2\varPhi\left(\sqrt{3}\right)-1=0.9164
			\end{align*}
	\end{solution}
\end{ti}

\begin{ti}[$8$ 分]
	设二维随机变量 $(X,Y)$ 的分布律为
	\begin{center}
		\begin{tabularx}{0.8\textwidth}{ZZZZ}
			\hline
			 & \multicolumn{3}{c}{$Y$}\\
			\cline{2-4}
			$X$ & $-1$ & $0$ & $1$\\
			\hline
			$-1$ & $\frac{1}{8}$ & $\frac{1}{8}$ & $\frac{1}{8}$\\
			$0$ & $\frac{1}{8}$ & $0$ & $\frac{1}{8}$\\
			$1$ & $\frac{1}{8}$ & $\frac{1}{8}$ & $\frac{1}{8}$\\
			\hline
		\end{tabularx}
	\end{center}
	求 $\Cov(X,Y)$
	\begin{solution}
		$\EE(X)=-1\times\frac{3}{8}+1\times\frac{3}{8}=0$ , 同理通过计算得 $\EE(Y)=0$ , $\EE(XY)=0$, 因此
		\[
			\Cov(X,Y)=\EE(XY)-\EE(X)\EE(Y)=0
		\]
	\end{solution}
\end{ti}

\begin{ti}[$10$ 分]
	设 $X_1,X_2,\cdots,X_n$ 为总体 $X$ 的一个样本, 总体 $X$ 的概率密度为:
	\begin{equation*}
		f(x)=
		\begin{cases}
		(\alpha+1)x^\alpha, & 0<x<1\\
		0, & \text{其他}
		\end{cases},
	\end{equation*}
	求未知参数 $\alpha$ 的矩估计
	\begin{solution}
		\[
			\EE(X)=\int_{0}^{1}(\alpha+1)x^{\alpha+1}\dd x=\frac{\alpha+1}{\alpha+2}, \mu_1=\overline{X}=\sum_{i=1}^{n}\frac{X_i}{n},
		\]
		因此
		\[
			\alpha=\frac{2\overline{X}-1}{1-\overline{X}}
		\]
	\end{solution}
\end{ti}

\subsection{2019-2020 A20}
\subsubsection{选择题}
\begin{ti}
	已知 $P(A) = \frac{1}{2}$, $P(B|A) = \frac{1}{4}$, 则 $P(AB) = $ \kuo
	\fourch{$\frac{1}{2}$}{$\frac{1}{4}$}{$\frac{1}{6}$}{$\frac{1}{8}$}
\end{ti}

\begin{ti}
	某类灯泡使用时数在 $500$ 小时以上的概率为 $0.5$, 从中取 $3$ 个灯泡使用, 则在使用 $500$ 小时后无一损坏的概率为 \kuo
	\fourch{$\frac{1}{8}$}{$\frac{3}{8}$}{$\frac{5}{8}$}{$\frac{7}{8}$}
\end{ti}

\begin{ti}
	如果抛一枚硬币 $4$ 次, 设 $X$ 表示正面朝上的次数, 则 $P\{ X = 3 \} = $ \kuo
	\fourch{$\frac{1}{2}$}{$\frac{1}{8}$}{$\frac{1}{4}$}{$\frac{1}{3}$}
\end{ti}

\begin{ti}
	设随机变量 $(X,Y)$ 概率密度为 $f(x,y) = \begin{cases}
		x^{2} + \frac{1}{3} xy, & 0 \leqslant x \leqslant 1, 0 \leqslant y \leqslant 2\\
		0, & \text{其他}.
	\end{cases}$ 则 $P\{ 0 \leqslant X \leqslant 1, 0 \leqslant Y \leqslant 1 \} = $ \kuo
	\fourch{$\frac{1}{12}$}{$\frac{3}{12}$}{$\frac{5}{12}$}{$\frac{7}{12}$}
\end{ti}

\begin{ti}
	设随机变量 $X \sim N(2,4)$, 且 $aX - 1 \sim N(0,1)$, 则 \kuo
	\fourch{$a = -2$}{$a = -0.5$}{$a = 0.5$}{$a = 2$}
\end{ti}

\begin{ti}
	设随机变量 $X$ 的数学期望 $\EE(X) = \mu$, 方差 $D(X) = \sigma^{2}$, 则由切比雪夫不等式有 $P\{ |X - \mu| < 2 \sigma \} \geqslant $ \kuo
	\fourch{$\frac{1}{2}$}{$\frac{1}{3}$}{$\frac{2}{3}$}{$\frac{3}{4}$}
\end{ti}

\begin{ti}
	设 $\hat{\theta}$ 为未知参数 $\theta$ 的估计量, 若 $E \bigl( \hat{\theta} \bigr)$ 存在, 且有 $E \bigl( \hat{\theta} \bigr) = \theta$, 则称 $\hat{\theta}$ 为参数 $\theta$ 的 \kuo
	\fourch{有偏估计量}{无偏估计量}{一致估计量}{有效估计量}
\end{ti}

\subsubsection{填空题}
\begin{ti}
	袋中装有标号 $1,2,3,4,5$ 的 $5$ 个球, 从中任取一个, 则取到标号大于 $3$ 的球的概率为 \hua
\end{ti}

\begin{ti}
	设 $A,B$ 为两个事件, $P(A) = 1$, $P(B) = 0.7$, $P\bigl( \overline{B}\bigl|A \bigr) = 0.3$, 则 $P(AB) = $ \hua
\end{ti}

\begin{ti}
	设随机变量 $X$ 的概率密度为 $f(x) = \begin{cases}
		x^{\alpha}, & 0 < x < 1 \\
		0 & \text{其他}
	\end{cases}$, 则 $\alpha = $ \hua
\end{ti}

\begin{ti}
	设二维随机变量 $(X,Y)$ 的概率密度为 $f(x,y) = \begin{cases}
		C x^{2} \sin y, & 0 < x < 1, 0 < y < \frac{\uppi}{2}, \\
		0, & \text{其他}.
	\end{cases}$ 则 $C = $ \hua
\end{ti}

\begin{ti}
	设随机变量 $X$ 的概率密度为 $f(x) = \begin{cases}
		\frac{1}{b - a}, & a < x < b, \\
		0, & \text{其他}.
	\end{cases}$ 则 $D(X) = $ \hua
\end{ti}

\begin{ti}
	设总体 $X$ 服从 0-1 分布 $B(1,p)$, 即 $P\{ X = 1 \} = p$ (其中 $p$ 为未知参数)。$X_{1},X_{2},\cdots,X_{n}$ 是来自 $X$ 的一个样本, 则样本函数 $X_{1} + X_{2}$, $\max_{1 \leqslant i \leqslant n} X_{i}$, $X_{n} + 2p$, $(X_{n} - X_{1})^{2}$ 中不是统计量的是 \hua
\end{ti}

\begin{ti}
	设 $\theta$ 和 $X_{1},X_{2},\cdots,X_{n}$ 分别是总体 $X$ 的未知参数和一个样本, $\theta_{1},\theta_{2}$ 是由样本确定的两个统计量, 满足 $P\{ \theta_{1} < \theta < \theta_{2} \} = 1 - \alpha$, 则称随机区间 $(\theta_{1},\theta_{2})$ 为 $\theta$ 的置信度为 \hua{} 的置信区间
\end{ti}

\subsubsection{解答题}
\begin{ti}
	有两批种子, 第一批发芽率为 $0.9$, 第二批发芽率为 $0.96$, 现将数量相同两批种子混合在一起, 从中任取一粒, 求该粒种子发芽的概率
\end{ti}

\begin{ti}
	设随机变量 $X$ 服从 $[0,2]$ 的均匀分布, 求随机变量 $Y = X^{2}$ 的概率密度
\end{ti}

\begin{ti}
	设二维随机变量 $(X,Y)$ 分布律如表~\ref{tab:1}, 求 $(X,Y)$ 关于 $X$ 与 $Y$ 的边缘分布律, 并判断 $X$ 与 $Y$ 是否相互独立
	\begin{table}[htbp]
		\centering
		\caption{}\label{tab:1}
		\begin{tabularx}{0.8\textwidth}{ZZZZ}
			\hline
			 & \multicolumn{3}{c}{$Y$}\\
			\cline{2-4}
			$X$ & 0 & 1 & 2 \\
			\hline
			0 & $\frac{1}{4}$ & $\frac{1}{4}$ & $\frac{1}{16}$\\
			1 & $\frac{1}{4}$ & $\frac{1}{8}$ & 0\\
			2 & $\frac{1}{16}$ & 0 & 0\\
			\hline
		\end{tabularx}
	\end{table}
\end{ti}

\begin{ti}
	设 $X$ 与 $Y$ 两个随机变量 $\EE(X) = 2$, $\EE(X^{2}) = 20$, $\EE(Y) = 3$, $\EE(Y^{2}) = 34$, $\Cov(X,Y) = 10$, 求 $\EE(X - Y)$, $D(X + Y)$
\end{ti}

\begin{ti}
	某工厂有 $200$ 台同类型机器, 由于工艺等原因, 每台机器工作时间只有 \SI{80}{\percent}, 各台机器相互独立, 求任一时刻有 $156$ 至 $164$ 台机器正在工作的概率($\varPhi(0.71) = 0.7611, \varPhi(1.98) = 0.9762$)
\end{ti}

\begin{ti}
	设 $X_{1},X_{2},\cdots,X_{n}$ 为总体 $X$ 的一个样本, 总体 $X$ 概率密度为 $f(x) = \begin{cases}
		\frac{2}{\theta} \ee^{- \frac{2x}{\theta}}, & x > 0, \\
		0, & x \leqslant 0.
	\end{cases}$ 求参数 $\theta(\theta > 0)$ 的极大似然估计
\end{ti}

\section{春}
\subsection{2018-2019 B}
\subsubsection{选择题(每题 $3$ 分,共 $21$ 分)}
\begin{ti}
	设随机变量 $X$ 分布律为 $P\{ X=k \}=pq^{k-1}(p>0,q>0,k=1,2,\cdots)$ , 则 \kuo{}
	\fourch{$p+q=1$}
	{$p=\frac{1}{q}-1$}
	{$p+q=2$}
	{$pq=1$}
\end{ti}

\begin{ti}
	从 $0,1,2,\cdots,9$ 这 $10$ 个数字中随机抽取 $1$ 个数字, 则取到奇数的概率为 \kuo{}
	\fourch{$\frac{1}{2}$}
	{$\frac{3}{5}$}
	{$\frac{4}{9}$}
	{$\frac{5}{9}$}
\end{ti}

\begin{ti}
	一批产品, 优质品占 \SI{20}{\percent}, 进行重复抽样检查, 共取 $5$ 件, 则恰好 $3$ 件是优质品的概率为 \kuo{}
	\fourch{$10\times 0.2^3$}
	{$0.2^3$}
	{$10\times 0.2^3\times 0.8^2$}
	{$0.2^3\times 0.8^2$}
\end{ti}

\begin{ti}
	同时掷骰子, 以 $X$ 和 $Y$ 分别代表第 $1$ 和第 $2$ 个骰子的点数, 则 $P\{ X+Y=5 \}=$ \kuo{}
	\fourch{$\frac{1}{12}$}
	{$\frac{1}{9}$}
	{$\frac{1}{36}$}
	{$\frac{1}{18}$}
\end{ti}

\begin{ti}
	设 $X$ 为一个随机变量, 且 $\EE(X)=2$, $\EE\left( X^2 \right)=5$, 则 $D(2X)=$ \kuo{}
	\fourch{$6$}
	{$12$}
	{$2$}
	{$4$}
\end{ti}

\begin{ti}
	设总体 $X\sim N\left( \mu,\sigma^2 \right)$, $X_1,\cdots,X_n$ 为样本, $\overline{X}$, $S^2$ 分别为样本均值和样本方差, 则下列选项服从 $\chi^2(n-1)$ 分布的是 \kuo{}
	\fourch{$\frac{(n-1)S^2}{\sigma^2}$}
	{$\frac{1}{\sigma^2}\sum_{i=1}^n \left(X_i-\mu\right)^2$}
	{$\frac{\overline{X}-\mu}{\sigma/\sqrt{n}}$}
	{$\frac{\overline{X}-\mu}{S/\sqrt{n}}$}
\end{ti}

\begin{ti}
	设 $\hat{\theta}_1$ 和 $\hat{\theta}_2$ 都是由样本确定的两个统计量, 满足 \kuo{}, 则随机区间 $\left(\hat{\theta}_1,\hat{\theta}_2\right)$ 称为 $\theta$ 的置信水平为 \SI{95}{\percent} 的置信区间
	\twoch{$P\left\{ \hat{\theta}_1<\theta<\hat{\theta}_2 \right\}=0.05$}
	{$P\left\{ \hat{\theta}_1<\theta<\hat{\theta}_2 \right\}=0.95$}
	{$P\left\{ \hat{\theta}_1<\theta \right\}=0.95$}
	{$P\left\{ \theta<\hat{\theta}_2 \right\}=0.95$}
\end{ti}

\subsubsection{填空题(每题 $3$ 分,共 $21$ 分)}
\begin{ti}
	设随机变量 $X$ 的概率密度为 $f(x)=
	\begin{cases}
		\frac{1}{x^2}, & a<x<3a,\\
		0, & \text{其他}.
	\end{cases}$, 则 $a=$\hua{}
\end{ti}

\begin{ti}
	设随机变量 $X$ 的数学期望 $\EE(X)=\mu$, 方差 $D(X)=\sigma^2$, 则由切比雪夫不等式有 $P\{ |X-\mu|<2\sigma \} \geq$\hua{}
\end{ti}

\begin{ti}
	设随机事件 $A$, $B$ 互不相容, 且 $P(A)=0.3$, $P\left( \overline{B} \right)=0.5$, 则 $P\left( B\middle|\overline{A} \right)=$\hua{}
\end{ti}

\begin{ti}
	设 $\hat{\theta}=\hat{\theta}(X_1,X_2,\cdots,X_n)$ 是 $\theta$ 的一个估计量, 若 $\EE\left( \hat{\theta} \right)$ 存在, 且有\hua{}, 则 $\hat{\theta}$ 是 $\theta$ 的无偏估计量
\end{ti}

\begin{ti}
	事件 $A$ 在一次实验中发生的概率为 $\frac{1}{3}$, $X$ 表示在 $3$ 次重复独立试验中发生的次数, 则 $P\{ X\leqslant 2 \}=$\hua{}
\end{ti}

\begin{ti}
	设二维随机变量 $(X,Y)$ 的概率密度为: $f(x,y)=
	\begin{cases}
		4xy, & 0<x<1,0<y<1,\\
		0, & \text{其他}.
	\end{cases}
	$, 则 $P\bigl\{ 0<X<\frac{1}{2},\frac{1}{4}<Y<1 \bigr\}=$ \hua{}
\end{ti}

\begin{ti}
	设随机变量 $X$ 的概率密度为 $f(x)=
	\begin{cases}
		\frac{1}{\beta-\alpha}, & \alpha<x<\beta,\\
		0, & \text{其他}.
	\end{cases}
	$, 则 $\EE(X)=$ \hua{}
\end{ti}

\subsubsection{解答题(共 $58$ 分)}
\begin{ti}[$8$ 分]
	设随机变量 $X$ 的分布律为
	\begin{center}
		\begin{tabularx}{0.8\textwidth}{Z|ZZZZZ}
			$X$ & $-1$ & $0$ & $1$ & $4$ & $7$\\
			\hline
			$P$ & $0.1$ & $0.2$ & $0.4$ & $0.16$ & $0.14$
		\end{tabularx}
	\end{center}
	求:
	\begin{enumerate}
		\item $P\{ |X-2|\leqslant 1 \}$
		\item $Y=X^2$ 的分布律
	\end{enumerate}
\end{ti}

\begin{ti}[$10$ 分]
	设某区域内肥胖者占 \SI{10}{\percent}, 不胖不瘦者占 \SI{82}{\percent}, 瘦者占 \SI{8}{\percent}, 肥胖者患高血压的概率为 \SI{30}{\percent}, 不胖不瘦者患高血压的概率为 \SI{10}{\percent}, 瘦者患高血压的概率为 \SI{5}{\percent}, 求:
	\begin{enumerate}
		\item 该地区的居民患高血压的概率
		\item 若在该地区任选一人, 发现有高血压, 属于肥胖者的概率
	\end{enumerate}
\end{ti}

\begin{ti}[$10$ 分]
	设二维随机变量 $(X,Y)$ 的概率密度为 $f(x,y)=
	\begin{cases}
		\ee^{-x}, & x>y,y>0,\\
		0, & \text{其他}.
	\end{cases}
	$, 试求: $(X,Y)$ 关于 $X$ 与 $Y$ 的边缘概率密度 $f_X(x)$ 与 $f_Y(y)$
\end{ti}

\begin{ti}[$10$ 分]
	设 $X\sim N(0,2)$, $Y\sim N(0,1)$, 且相互独立, $U=X+Y+1$, $V=X-Y+1$, 求 $\Cov(U,V)$
\end{ti}

\begin{ti}[$10$ 分]
	某通信系统有 $60$ 台相互独立起作用的交换机, 每台交换机能清晰接受信号的概率为 $0.90$. 系统工作时, 要求能清晰接受信号的交换机至少 $54$ 台, 求该通信系统能正常工作的概率(结果取最接近的值, 其中 $\varPhi(2.58)=0.9951$, $\varPhi(1.36)=0.9131$)
\end{ti}

\begin{ti}[$10$ 分]
	设 $X_1,X_2,\cdots,X_n$ 为总体 $X$ 的一个样本, 总体 $X$ 服从参数为 $p(0<p<1)$ 的两点分布, 其分布律为 $P\{ X=x \}=p^x (1-p)^{1-x}(x=0,1)$, 求未知参数 $p$ 的矩估计
\end{ti}

\subsection{2019-2020 A}
\subsubsection{选择题}
\begin{ti}
	设袋中有 4 只白球, 2 只黑球, 从袋中任取 2 球(不放回), 取得 2 只白球的概率 \kuo{}
	\fourch{$\frac{3}{5}$}{$\frac{1}{5}$}{$\frac{2}{5}$}{$\frac{4}{5}$}
\end{ti}

\begin{ti}
	设随机变量 $X$ 服从正态分布 $N\bigl( 10,\sigma^2 \bigr)$, 且 $P\{ X < a \} = 0.5$, 则 $a =$ \kuo{}
	\fourch{$10$}{$10+\sigma$}{$\sigma$}{$10-\sigma$}
\end{ti}

\begin{ti}
	设二维随机变量 $(X,Y)$ 的分布律为
	\begin{center}
		\begin{tabularx}{0.8\textwidth}{Z*{2}{Z}}
			\hline
			 & \multicolumn{2}{c}{$Y$} \\
			\cline{2-3}
			$X$ & 1 & 2 \\
			\hline
			1 & $a$ & $\frac{1}{4}$ \\
			2 & $\frac{1}{4}$ & $b$ \\
			\hline
		\end{tabularx}
	\end{center}
	若 $X$ 与 $Y$ 相互独立, 则 \kuo{}
	\fourch{$a = \frac{1}{8}$, $b = \frac{3}{8}$}{$a = \frac{3}{8}$, $b = \frac{1}{8}$}{$a = \frac{1}{4}$, $b = \frac{1}{4}$}{$a = -\frac{1}{4}$, $b = \frac{3}{4}$}
\end{ti}

\begin{ti}
	设随机变量 $X$ 与 $Y$ 相互独立, 且 $D(X) = 3$, $D(Y) = 2$, 则 $D(2Y-X) =$ \kuo{}
	\fourch{1}{5}{7}{11}
\end{ti}

\begin{ti}
	设 $X$, $Y$ 为任意两个随机变量, 且 $\Cov(X,Y) = 0$, 则下述结论正确的是 \kuo{}
	\fourch{$D(2X-Y) = 2D(X) - D(Y)$}
	{$D(2X-Y) = 2D(X) + D(Y)$}
	{$D(2X-Y) = 4D(X) - D(Y)$}
	{$D(2X-Y) = 4D(X) + D(Y)$}
\end{ti}

\begin{ti}
	若从总体 $X$ 中抽出容量为 5 的一个样本, 样本观察值为 1, 3, 4, 4, 8, 则样本平均值为 \kuo{}
	\fourch{3}{4}{$4.2$}{$4.5$}
\end{ti}

\begin{ti}
	设总体 $X \sim B(1,p)$, 其中未知参数 $0<p<1$, $X_1$, $X_2$, $\cdots$, $X_n$ 是 $X$ 的一个样本, 则 $P$ 的矩估计为 \kuo{}
	\fourch{$\overline{X}$}{$\frac{1}{2} \overline{X}$}{$\frac{\overline{X}}{\overline{X}+1}$}{$1 - \overline{X}$}
\end{ti}

\subsubsection{填空题}
\begin{ti}
	已知 $A \subset B$, $P(A) = 0.2$, $P(B) = 0.3$, 则 $P\bigl( B \overline{A} \bigr) =$ \hua
\end{ti}

\begin{ti}
	甲、乙两人独立地进行射击, 甲击中的概率为 $0.9$, 乙击中的概率为 $0.8$, 则甲乙都击中的概率为 \hua
\end{ti}

\begin{ti}
	设随机变量 $X$ 的概率密度为 $f(x) = \begin{cases}
		\lambda \ee^{-3x}, & x > 0 \\
		0, & \text{其他}
	\end{cases}$, 则 $P\{ X \leqslant 0.1 \} =$ \hua
\end{ti}

\begin{ti}
	设二维随机变量 $(X,Y)$ 的概率密度 $f(x,y) = \begin{cases}
		Ax \ee^{-y}, & 0 < x < 2, y > 0 \\
		0, & \text{其他}
	\end{cases}$, 则 $A =$ \hua
\end{ti}

\begin{ti}
	设随机变量 $X$ 的概率密度为 $f(x) = \begin{cases}
		\ee^{-x}, & x > 0 \\
		0, & \text{其他}
	\end{cases}$, 则 $\EE(X) =$ \hua
\end{ti}

\begin{ti}
	设随机变量 $X$ 的数学期望 $\EE(X) = 4$, 方差 $D(X) = 5$, 则由切比雪夫不等式得 $P\{ |X - 4| < 3 \} \geqslant$ \hua
\end{ti}

\begin{ti}
	设总体 $X \sim N(\mu,1)$, $X_1$, $X_2$, $\cdots$, $X_n$ 为总体 $X$ 的一个样本, 则当 $C_1$, $\cdots$, $C_n$ 满足 \hua{} 时, $\sum_{i=1}^n C_i X_i$ 是 $\mu$ 的一个无偏估计量
\end{ti}

\subsubsection{计算题}
\begin{ti}
	两个箱子, 第一个箱子中有 4 个黑球, 一个白球, 第二个箱子中有 3 个黑球, 3 个白球. 先随机地取一个箱子, 再从这个箱子中取出一个球, 若已知取出的是白球, 问此球是第 2 个箱子的概率
\end{ti}

\begin{ti}
	设随机变量 $X$ 的分布律为
	\begin{center}
		\begin{tabular}{c|*{4}{c}}
			$X$ & 1 & 2 & $\cdots$ & 10 \\
			\hline
			$P$ & $2a$ & $4a$ & $\cdots$ & $20a$
		\end{tabular}
	\end{center}
	求:
	\begin{enumerate}
		\item 常数 $a$;
		\item $P\{ X>3 \}$
	\end{enumerate}
\end{ti}

\begin{ti}
	设二维随机变量 $(X,Y)$ 的概率密度为: $f(x,y) = \begin{cases}
		k (x+y)^2, & |x| \leqslant 1, |y| \leqslant 1 \\
		0, & \text{其他}
	\end{cases}$, 
	求:
	\begin{enumerate}
		\item $k$ 的值;
		\item $(X,Y)$ 关于 $X$ 与 $Y$ 的边缘概率密度 $f_X(x)$ 与 $f_Y(y)$
	\end{enumerate}
\end{ti}

\begin{ti}
	设二维随机变量 $(X,Y)$ 的概率密度为 $f(x,y) = \begin{cases}
		1, & 0 < x < 1, 0 < y < 1, \\
		0, & \text{其他}.
	\end{cases}$ 求 $D(X)$ 及 $D(Y)$
\end{ti}

\begin{ti}
	某保险公司第 $i$ 个月收到的保险费(万元)是随机变量 $X_i$, 且 $\EE(X_i) = 10$, $D(X_i) = 1$, 试用中心极限定理确定 100 个月收到的保险费超过 1010 万元的概率. (结果取最接近的值, 其中 $\varPhi(1) = 0.8413$, $\varPhi(1.5) = 0.9332$)
\end{ti}

\begin{ti}
	某种清漆的干燥时间 $X$(单位: 小时)服从正态分布 $N(\mu,\sigma^2)$, 随机抽取容量为 9 的一个样本, 测得其干燥时间的样本均值 $\overline{X} = 6$. 样本标准差 $S = 0.3$, 若 $\sigma^2$ 未知, 求 $\mu$ 的置信度为 \SI{95}{\percent} 的置信区间. (结果取最接近的值, 其中 $t_{0.025}(8) = 2.3060$, $t_{0.025}(9) = 2.2622$)
\end{ti}

\subsection{2020-2021 A}
只有重要的数据部分。
\subsubsection{选择题}
\begin{ti}
	$X \sim N(1,4)$, $D(2X-3) = $ \kuo
	\fourch{5}{8}{19}{16}
\end{ti}

\begin{ti}
	$f(x)$ 连续型 \kuo
	\fourch{
		$\int_{-\infty}^{+\infty} f(x) \dd{x} = 1$
	}{
		$\lim_{x \to +\infty} f(x) = 1$
	}{
		$0 \leqslant f(x) \leqslant 1$
	}{
		单调不减
	}
\end{ti}

\begin{ti}
	$X_1, \cdots, X_{16}$, $\cdots$ , $X \sim N(2,5^2)$, $\overline{X}$, 则 $\frac{\overline{X}^{-2}}{ \sigma/4 } \sim $ \kuo
	\fourch{$t(15)$}{$t(16)$}{$X^2(15)$}{$N(0,1)$}
\end{ti}

\begin{ti}
	$X \sim U(0, \theta)$, $\theta$ 的矩估计 \kuo
	\fourch{$\overline{X}$}{$2\overline{X}$}{$2\overline{X}-1$}{$2\overline{X}+1$}
\end{ti}

\begin{ti}
	6 只晶体管有一只次品, 取 2 只取得一次一正的概率 \kuo
	\fourch{$\frac{7}{15}$}{$\frac{1}{3}$}{$\frac{2}{3}$}{$\frac{3}{4}$}
\end{ti}

\begin{ti}
	$F(x,y)$, $X$ 的边缘分布函数 $F_X(x)$ \kuo
	\fourch{
		$F_X(x) = F(x, +\infty)$
	}{
		$F_X(x) = F(+\infty, y)$
	}{
		$F_X(x) = 1 - F(x, +\infty)$
	}{
		$F_X(x) = 1 - F(+\infty, y)$
	}
\end{ti}

\subsubsection{填空题}
\begin{ti}
	$P(A) + P(B) = 0.8$, $P(AB) = 0.3$, 则 $P \bigl( \overline{A \cup B} \bigr) = $ \hua
\end{ti}

\begin{ti}
	$f(x) = \begin{cases}
		\lambda \ee^{-3x}, 	& x \geqslant 0 \\
		0, 									& x < 0
	\end{cases}$, $\lambda = $ \hua
\end{ti}

\begin{ti}
	$f(x,y) = \begin{cases}
		1,	& 0<x<1,0<y<1	\\
		0,
	\end{cases}$, $P\bigl\{ 0 < x < \frac{1}{2}, -1 < y < 1 \bigr\} = $ \hua
\end{ti}

\begin{ti}
	$f(x) = \begin{cases}
		\frac{1}{b}, & -2<x<4, \\
		0,					 & \text{其它},
	\end{cases}$, $D(x) = $ \hua
\end{ti}

\begin{ti}
	$\EE(X) = 0.1$, $\EE(Y) = 10$, $X$, $Y$ 独立, $\EE(Y-4XY) = $ \hua
\end{ti}

\begin{ti}
	$\EE(X) = M$, $D(X) = \sigma^2$, $P\bigl\{ |x - \mu| < a \bigr\} \geqslant \frac{3}{4}$, $a = $ \hua
\end{ti}

\begin{ti}
	$\theta_1$, $\theta_2$ 无偏估计量, 若 \hua{} 则 $\theta_1$ 较 $\theta_2$ 有效
\end{ti}

\subsubsection{计算题}
\begin{ti}
	一仓库有10箱同种规格的产品, 其中由甲, 乙, 丙三厂生产的分别为5箱, 3箱, 2箱, 三厂产品的次品率依次为0.1, 0.2, 0.3, 从这10箱产品中任取一箱, 再从这箱中任取一件, 求取得正品的概率? 若确实取得正品, 求正品由甲厂生产的概率
\end{ti}

\begin{ti}
	$f(x) = \begin{cases}
		0,	& x < 0 \\
		a \sin x,	& 0 \leqslant x \leqslant \frac{\uppi}{2}	\\
		1,	&	x > \frac{\uppi}{2}
	\end{cases}$, 求
	\begin{enumerate}
		\item $a$
		\item $P \bigl\{ |x| < \frac{\uppi}{6} \bigr\}$
	\end{enumerate}
\end{ti}

\begin{ti}
	$f(x,y) = \begin{cases}
		k(6-x-y), & 0<x<2, 2<y<3,	\\
		0,	&	\text{其它},
	\end{cases}$
	\begin{enumerate}
		\item $k$
		\item 边缘概率 $f_X(x), f_Y(y)$
	\end{enumerate}
\end{ti}

\begin{ti}
	\mbox{}
	\begin{center}
		\begin{tabular}{cccccc}
			$x$	&	1	&	2	&	3	&	4	&	5	\\
			$P$	&	$0.05$	&	0.20	&	0.35	&	0.30	&	0.1
		\end{tabular}
	\end{center}
	求 $\EE(X)$, $D(X)$
\end{ti}

\begin{ti}
	木柱 100 根, 其中 \SI{80}{\percent} 长度不小于 \SI{3}{m}, 问其中至少 30 根小于 \SI{3}{m} 的概率 ($\Phi(2.5) = 0.9938$)
\end{ti}

\begin{ti}
	磁棒 $X$ 服从 $N(10,\sigma^2)$, $\overline{X} = 15$, $s^2 = 0.05$, $\sigma^2$ 未知. $\mu$ 置信 \SI{95}{\percent} 的置信区间. ($t_{0.025}(4) = 2.7764$)
\end{ti}

\section{拼凑卷}
此复习题非一份完整的考试卷, 而是多个不同的试卷拼凑(老师给的)。

\subsubsection{选择题(每题 $3$ 分)}
\begin{ti}
	已知事件 $A,B$ 满足 $P(AB)=P\left( \overline{A}\bigcap\overline{B} \right)$, 且 $P(A)=0.4$, 则 $P(B)=$ \kuo{}
	\fourch{0.4}
	{0.5}
	{0.6}
	{0.7}
\end{ti}

\begin{ti}
	有 $\gamma$ 个球, 随机地放在 $n$ 个盒子中 $(\gamma \leqslant n)$, 则某指定的 $\gamma$ 个盒子中各有一球的概率为 \kuo{}
	\fourch{$\frac{\gamma !}{n^{\gamma}}$}
	{$\mathrm{C}_n^r\frac{\gamma !}{n^\gamma}$}
	{$\frac{n!}{\gamma^n}$}
	{$\mathrm{C}_\gamma^n\frac{n!}{\gamma^n}$}
\end{ti}

\begin{ti}
	设随机变量 $X$ 的概率密度为 $f(x)=c\ee^{-\left|x\right|}$, 则 $c=$ \kuo{}
	\fourch{$-\frac{1}{2}$}{$0$}{$\frac{1}{2}$}{$1$}
\end{ti}

\begin{ti}
	掷一颗骰子 $600$ 次, 求“一点”出现次数的均值为 \kuo{}
	\fourch{$50$}
	{$100$}
	{$120$}
	{$150$}
\end{ti}

\begin{ti}
	设每次试验成功的概率为 $p (0<p<1)$, 重复进行试验直到第 $n$ 次才取得 $r (1\leqslant r\leqslant n)$ 次成功的概率为 \kuo{}
	\fourch{$\mathrm{C}_{n-1}^{r-1}p^r(1-p)^{n-r}$}
	{$\mathrm{C}_{n}^{r}p^r(1-p)^{n-r}$}
	{$\mathrm{C}_{n-1}^{r-1}p^{r-1}(1-p)^{n-r+1}$}
	{$p^r(1-p)^{n-r}$}
\end{ti}

\begin{ti}
	离散型随机变量 $X$ 的分布函数为 $F(x)$, 则 $P(X=x_k)=$ \kuo{}
	\fourch{$P(x_{k-1}\leqslant X\leqslant x_k)$}
	{$F(x_{k+1})-F(x_{k-1})$}
	{$P(x_{k-1}<X<x_{k+1})$}
	{$F(x_{k})-F(x_{k-1})$}
\end{ti}

\begin{ti}
	设随机变量 $X, Y$ 是相互独立的两个随机变量, 其分布函数分别为 $F_X(x), F_Y(y)$, 则随机变量 $Z=\max (X,\allowbreak Y)$ 的分布函数为 \kuo{}
	\twoch{$F_Z(z)=\max\left\{F_X(x),F_Y(y)\right\}$}
	{$F_Z(z)=\max\left\{\left|F_X(x)\right|,\left|F_Y(y)\right|\right\}$}
	{$F_Z(z)=F_X(x)F_Y(y)$}
	{$F_Z(z)=F_X(z)F_Y(z)$}
\end{ti}

\begin{ti}
	设随机变量 $(X, Y)$ 的方差 $D(X)=4, D(Y)=1$, 相关系数 $\rho_{XY}=0.6$, 则方差 $D(3X-2Y)=$ \kuo{}
	\fourch{$40$}
	{$34$}
	{$25.6$}
	{$17.6$}
\end{ti}

\begin{ti}
	设 $(X_1,X_2,\cdots,X_n)$ 为总体 $N\left(1,2^2\right)$ 的一个样本, $\overline X$ 为样本均值, 则下列结论中正确的是 \kuo{}
	\twoch{$\frac{\overline{X}-1}{2/\sqrt{n}}\sim t(n)$}
	{$\frac{1}{4}\sum_{i=1}^{n}(X_i-1)^2\sim F(n,1)$}
	{$\frac{\overline{X}-1}{\sqrt{2}/\sqrt{n}}\sim N(0,1)$}
	{$\frac{1}{4}\sum_{i=1}^{n}(X_i-1)^2\sim\chi^2(n)$}
\end{ti}

\begin{ti}
	设总体 $X$ 在 $(\mu-\rho,\mu+\rho)$ 上服从均匀分布, 则参数 $\mu$ 的矩估计量为 \kuo{}
	\fourch{$\frac{1}{\overline{X}}$}
	{$\frac{1}{n-1}\sum_{i=1}^{n}X_i$}
	{$\frac{1}{n-1}\sum_{i=1}^{n}X_i^2$}
	{$\overline{X}$}
\end{ti}

\begin{ti}
	设二维随机变量 $(X,Y)$ 的分布律为:
	\begin{center}
		\begin{tabularx}{0.8\textwidth}{ZZZ}
			\hline
			 & \multicolumn{2}{c}{$Y$}\\
			\cline{2-3}
			$X$ & 1 & 2\\
			\hline
			1 & $a$ & $\frac{2}{9}$\\
			\hline
			2 & $b$ & $\frac{4}{9}$\\
			\hline
		\end{tabularx}
	\end{center}
	若 $X$ 与 $Y$ 相互独立, 则 \kuo{}
	\fourch{$a=\frac{4}{9}, b=\frac{1}{9}$}
	{$a=\frac{1}{9}, b=\frac{4}{9}$}
	{$a=\frac{2}{9}, b=\frac{1}{9}$}
	{$a=\frac{1}{9}, b=\frac{2}{9}$}
\end{ti}

\begin{ti}
	设随机变量 $X$ 的概率密度为 $f(x)=
	\begin{cases}
	ax, & 0\leqslant x\leqslant 2\\
	0, & \text{其他}
	\end{cases}
	$, 则 $a=$ \kuo{}
	\fourch{$1$}
	{$\frac{1}{4}$}
	{$\frac{1}{2}$}
	{$\frac{1}{3}$}
\end{ti}

\begin{ti}
	设 $\EE(X)=2$, $\EE(Y)=3$, 则 $\EE(3X-4Y+5)=$ \kuo{}
	\fourch{1}
	{6}
	{$-6$}
	{$-1$}
\end{ti}

\begin{ti}
	设随机变量 $X\sim N(1,4)$, 则下列随机变量中服从标准正态分布的是 \kuo{}
	\fourch{$Y=\frac{X-1}{4}$}
	{$Y=\frac{X-1}{2}$}
	{$Y=\frac{X+1}{4}$}
	{$Y=\frac{X+1}{2}$}
\end{ti}

\begin{ti}
	设盒中有 $4$ 支铅笔, $2$ 支钢笔, 从盒中任取 $2$ 支笔(不放回抽样), 则取得 $1$ 支铅笔和 $1$ 支钢笔的概率是 \kuo{}
	\fourch{$\frac{8}{15}$}
	{$\frac{4}{5}$}
	{$\frac{3}{5}$}
	{$\frac{7}{15}$}
\end{ti}

\begin{ti}
	设总体 $X$ 服从正态分布 $N(\mu,1)$, $X_1$, $X_2$ 是来自总体 $X$ 的一个样本, $\hat\mu_1=\frac{2}{3}X_1+\frac{1}{3}X_2$, 
	$\hat\mu_2=\frac{1}{4}X_1+\frac{3}{4}X_2$, $\hat\mu_3=\frac{1}{2}X_1+\frac{1}{2}X_2$ 都是 $\mu$ 的无偏估计量, 则其中方差最小的是 \kuo{}
	\fourch{$\hat\mu_3$}
	{$\hat\mu_2$}
	{$\hat\mu_1$}
	{一样大}
\end{ti}

\begin{ti}
	设随机变量 $X\sim\chi^2(m_1)$, $Y\sim\chi^2(m_2)$, 且 $X$ 与 $Y$ 相互独立, 则下列结果中不正确的是 \kuo{}
	\twoch{$X+Y\sim\chi^2(m_1+m_2)$}
	{$D(Y)=m_2$}
	{$D(X)=2m_1$}
	{$\EE(X)=m_1$}
\end{ti}

\subsubsection{填空题(每题 3 分)}
\begin{ti}
	已知 $P(B)=0.3$, $P\left(\overline{A}\bigcup B\right)=0.7$, 且 $A$ 与 $B$ 相互独立, 则 $P(A)=$ \hua{}
\end{ti}

\begin{ti}
	设随机变量 $X$ 服从参数为 $\lambda$ 的泊松分布, 且 $P\left\{X=0\right\}=\frac{1}{3}$, 则 $\lambda=$ \hua{}
\end{ti}

\begin{ti}
	设 $X\sim N\left(2,\sigma^2\right)$, 且 $P\left\{2<X<4\right\}=0.2$, 则 $P\left\{X<0\right\}=$ \hua{}
\end{ti}

\begin{ti}
	已知 $D(X)=2$, $D(Y)=1$, 且 $X$ 和 $Y$ 相互独立, 则 $D(X-2Y)=$ \hua{}
\end{ti}

\begin{ti}
	设 $S^2$ 是从 $N(0,1)$ 中抽取容量为 $16$ 的样本方差, 则 $D\left(S^2\right)=$ \hua{}
\end{ti}

\begin{ti}
	一批电子元件共有 $100$ 个, 次品数为 $5$. 连续两次不放回地从中任取一个, 则第二次才取得正品的概率为 \hua{}
\end{ti}

\begin{ti}
	设事件 $A$ 与 $B$ 相互独立, $P(A)=0.2$, $P(B)=0.3$, 则 $P\left(A\bigcup B\right)=$ \hua{}
\end{ti}

\begin{ti}
	设 $\overline{X}$ 为总体 $X\sim N(3,4)$ 中抽取的样本 $(X_1,X_2,X_3,X_4)$ 的均值, 则 $P\left(-1<\overline{X}<5\right)=$ \hua{}
\end{ti}

\begin{ti}
	甲、乙两人独立地进行射击, 甲击中的概率为 $0.9$, 乙击中的概率为 $0.8$, 则甲中乙不中的概率等于 \hua{}
\end{ti}

\begin{ti}
	设 $P(A)=P(B)=P(C)=\frac{1}{3}$, $ P(AB)=P(AC)=0$, $ P(BC)=\frac{1}{5}$. 则 $A, B, C$ 中至少有一个发生的概率为 \hua{}
\end{ti}

\begin{ti}
	设二维随机变量 $(X,Y)$ 的概率密度为 $f(x,y)=
	\begin{cases}
	 C\ee^{-3x}\sin y, & x>0, 0<y<\frac{\uppi}{2}\\
	0, & \text{其他}
	\end{cases}
	$, 则 $C=$ \hua{}
\end{ti}

\begin{ti}
	设随机变量 $X$ 服从参数为 $3$ 的指数分布, 则 $P\left\{X\geqslant  1\right\}=$ \hua{}
\end{ti}

\begin{ti}
	总体 $Y\sim N\left(\mu,\sigma^2\right)$, $\overline{Y}=\frac{1}{n}\sum_{i=1}^{n}Y_i$ 为样本均值, $S$ 为样本标准差, 当 $\sigma$ 为未知时, $\mu$ 的置信度为 $1-\alpha (0<\alpha<1)$ 的双侧置信区间为 \hua{}
\end{ti}

\begin{ti}
	设 $X,Y$ 为两个随机变量, 已知 $\EE(X)=1$, $\EE(Y)=2$, $\EE(XY)=5$, 则 $\Cov(X,Y-4)=$ \hua{}
\end{ti}

\begin{ti}
	若 $P(A)=0.5$, $P\left(B\overline{A}\right)=0.2$, 则 $P(A+B)=$ \hua{}
\end{ti}

\begin{ti}
	已知随机变量 $X\sim
	\begin{bmatrix}
	-1 & 0 & 2 & 5\\
	0.25 & 0.25 & 0.25 & 0.25
	\end{bmatrix}
	$, 那么 $\EE(X)=$ \hua{}
\end{ti}

\begin{ti}
	设 $\hat\theta$ 是未知参数 $\theta$ 的一个无偏估计量, 则 $\EE(\hat\theta)=$ \hua{}
\end{ti}

\begin{ti}
	随机变量 $X$ 服从均匀分布 $U(1,3)$, 则 $P(X>2)=$ \hua{}
\end{ti}

\begin{ti}
	设随机变量 $X\sim B(100,0.15)$, 则 $\EE(X)=$ \hua{}
\end{ti}

\begin{ti}
	设随机变量 $X\sim N(3,4)$, 已知 $\varPhi(1)=0.8413$, 则 $P(X<1)=$ \hua{}
\end{ti}

\begin{ti}
	设随机变量 $X$ 的概率密度函数为 $f(x)=
	\begin{cases}
	3x^2, & 0\leqslant x\leqslant 1\\
	0, & \text{其它}
	\end{cases}
	$, 则$ P\left(X<\frac{1}{2}\right)=$ \hua{}
\end{ti}

\begin{ti}
	设随机变量 $X\sim
	\begin{bmatrix}
	0 & 1 & 3\\
	0.5 & 0.35 & 0.15
	\end{bmatrix}
	$, 则 $P(X<2)=$ \hua{}
\end{ti}

\begin{ti}
	设随机变量 $X$ 的期望存在, 则 $\EE(X-\EE(X))=$ \hua{}
\end{ti}

\begin{ti}
	设 $X$ 为随机变量, 已知 $D(X)=2$, 那么 $D(3X-5)=$ \hua{}
\end{ti}

\subsubsection{计算题}
\begin{ti}[10 分]
	设随机变量 $X$ 与 $Y$ 具有概率密度: $f(x,y)=
	\begin{cases}
	\frac{1}{8}(x+y) & 0\leqslant x\leqslant 2, 0\leqslant y\leqslant 2\\
	0 & \text{其它}
	\end{cases}
	$. 试求: $D(X)$, $D(Y)$ 与 $D(2X-3Y)$
\end{ti}

\begin{ti}[10 分]
	某电子计算机主机有 $100$ 个终端, 每个终端有 \SI{80}{\percent} 的时间被使用. 若各个终端是否被使用是相互独立的, 试求至少有 $15$ 个终端空闲的概率. $(\varPhi(1.25)=0.8944, \varPhi(0.31)=0.6217$
\end{ti}

\begin{ti}[10 分]
	试求正态总体 $N\left(\mu,0.5^2\right)$ 的容量分别为 $10,15$ 的两独立样本均值差的绝对值大于 $0.4$ 的概率. $(\varPhi(1.96)=0.975)$
\end{ti}

\begin{ti}[10 分]
	设总体 $X$ 的密度函数为 $f(x)=
	\begin{cases}
	\frac{2}{\theta^2}(\theta-x), & 0<x<\theta\\
	0, & \text{其它}
	\end{cases}
	$, $\theta>0$, $X_1,X_2,\cdots,X_{10}$ 为来自总体 $X$ 的样本, 试求当样本观察值分别为 $0.5$, $1.3$, $0.6$, $1.7$, $2.2$, $1.2$, $0.8$, $1.5$, $2.0$, $1.6$ 时未知参数 $\theta$ 的矩估计值
\end{ti}

\begin{ti}[10 分]
	某商店拥有某产品共计 $12$ 件, 其中 $4$ 件次品, 已经售出 $2$ 件, 现从剩下的 $10$ 件产品中任取一件, 求这件是正品的概率
\end{ti}

\begin{ti}[10 分]
	设某种电子元件的寿命服从正态分布 $N(40,100)$, 随机地取 $5$ 个元件, 求恰有两个元件寿命小于 $50$ 的概率. $(\varPhi(1)=0.8413, \varPhi(2)=0.9772)$
\end{ti}

\begin{ti}[12 分]
	设总体 $X$ 的分布律为 $P\left\{X=k\right\}=(1-p)^{k-1}p, k=1,2,\cdots.$ ($p$ 为未知参数), $X_1,X_2,\cdots,X_n$ 是总体 $X$ 的一个样本, 求 $p$ 的极大似然估计量
\end{ti}

\begin{ti}[10 分]
	两台车床加工同样的零件, 第一台出现不合格品的概率是 $0.03$, 第二台出现不合格品的概率是 $0.06$, 加工出来的零件放在一起, 并且已知第一台加工的零件数比第二台加工的零件数多一倍.
	\begin{enumerate}
		\item 求任取一个零件是合格品的概率.
		\item 如果取出的零件是不合格品, 求它是由第二台车床加工的概率
	\end{enumerate}
\end{ti}

\begin{ti}[10 分]
	某仪器装了 $3$ 个独立工作的同型号电子元件, 其寿命(单位: 小时)都服从同一指数分布, 密度函数为 $f(x)=
	\begin{cases}
	\frac{1}{600}\ee^{-\frac{x}{600}}, & x>0\\
	0, & \text{其它}
	\end{cases}
	$, 试求此仪器在最初使用的 $200$ 小时内, 至少有一个此种电子元件损坏的概率
\end{ti}