\chapter{复变函数试卷汇总}

\section{复习题 1}
\subsubsection{选择题(每小题 $3$ 分, 共 $15$ 分)}
\begin{enumerate}
	\item $\frac{(\sqrt{3}-\ii)^{4}}{(1-\ii)^{8}}=$ (\hspace{1pc})
	\twoch{$-\frac{1}{2}+\frac{\sqrt{3}}{2}\ii$}{$-\frac{1}{8}\left(1+\sqrt{3}\ii\right)$}{$\frac{1}{8}\left(-1+\sqrt{3} \ii\right)$}{$-\frac{1}{2}-\frac{\sqrt{3}}{2} \ii$}
	
	\item 设 $f(z)=2 x^{3}+3 y^{3} \ii$ , 则 $f(z)$ (\hspace{1pc})
	\twoch{处处不可导}{仅在 $6x^2=9y^2$ 上可导, 处处不解析}{处处解析}{仅在 $(0,0)$ 点可导}
	
	\item 下列等式正确的是 (\hspace{1pc})
	\twoch{$\Ln \mathrm{i}=\left(2 k \uppi-\frac{\uppi}{2}\right) \ii, \ln \ii=\frac{\uppi}{2} \ii$}{$\Ln \ii=\left( 2k\uppi+\frac{\uppi}{2}\right)\ii,\ln\ii=-\frac{\uppi}{2}\ii $}{$\Ln \ii=\left(2 k \uppi+\frac{\uppi}{2}\right) \ii, \ln \ii=\frac{\uppi}{2} \ii$}{$\Ln \ii=\left(2 k \uppi-\frac{\uppi}{2}\right) \ii, \ln \ii=-\frac{\uppi}{2} \ii$}
	
	\item $z=0$ 是函数 $\frac{1-\cos z}{z-\sin z}$ 的 (\hspace{1pc})
	\fourch{本性奇点}{可去奇点}{二级极点}{一级极点}
	
	\item 设 $\mathrm{C}$ 为 $z=(1-\ii)t$ , $t$ 从 $1$ 到 $0$ 的一段, 则 $\int_{\mathrm{C}} \overline{z} \dd z=$ (\hspace{1pc})
	\fourch{$-1$}{$1$}{$-\ii$}{$\ii$}
\end{enumerate}

\subsubsection{填空题(每小题 $3$ 分, 共 $15$ 分)}
\begin{enumerate}
	\item 若 $z+|z|=2+\ii$ , 则 $z=$\underline{\hspace{8pc}}
	
	\item 若 $\mathrm{C}$ 为正向圆周 $|z|=\frac{1}{2}$ , 则 $\oint_{\mathrm{C}} \frac{1}{z-2} \dd z=$\underline{\hspace{8pc}}
	
	\item 若 $z=2-\uppi\ii$ , 则 $\ee^{z}=$\underline{\hspace{8pc}}
	
	\item 若 $f(z)=\cos z^2$ , 则 $f(z)$ 在 $z=0$ 处泰勒展开式中 $z^4$ 项的系数 $a_4=$\underline{\hspace{8pc}}
	
	\item 函数 $f(t)=\sin t$ 的拉普拉斯变换 $F(s)=$\underline{\hspace{8pc}}
\end{enumerate}

\subsubsection{计算题(70分)}
\begin{enumerate}
	\item 设 $u(x,y)=x-2xy$ 且 $f(0)=0$ , 求解析函数 $f(z)=u+\ii v$ . ( $10$ 分)
	
	\item 计算积分 $\oint_{\mathrm{C}}\frac{2\ee^x}{z^5}\dd z$ 的值, 其中 $\mathrm{C}$ 为正向圆周 $|z|=1$ . ( $7$ 分)
	
	\item 计算积分 $\oint_{\mathrm{C}}\frac{3z+5}{z^2-z}\dd z$ 的值, 其中 $\mathrm{C}$ 为正向圆周 $|z|=\frac{1}{2}$ . ( $7$ 分)
	
	\item 求函数 $\frac{1-\cos z}{z^3}$ 在有限奇点处的留数. ( $7$ 分)
	
	\item 求函数 $\frac{2z^2+1}{z^2+2z}$ 在有限奇点处的留数. ( $7$ 分)
	
	\item 将 $f(z)=\frac{z}{(z-2)(z-6)}$ 在 $2<|z|<6$ 内展开为洛朗级数. ( $10$ 分)
	
	\item 若函数 $f(z)=a y^{3}+b x^{2} y+\ii\left(x^{3}+c x y^{2}\right)$ 是复平面上的解析函数, 求 $a,b,c$ 的值. ( $12$ 分)
	
	\item 利用拉普拉斯变换解常微分方程初值问题: $\begin{cases}
	x''(t)+6x'(t)+9x(t)=\ee^{-3t}\\
	x(0)=0, x'(0)=0
	\end{cases}$ . ( $10$ 分)
\end{enumerate}



\section{复习题 1 答案}
\subsubsection{选择题(每小题 $3$ 分, 共 $15$ 分)}
\begin{enumerate}
	\item $\frac{(\sqrt{3}-\ii)^{4}}{(1-\ii)^{8}}=$ (\hspace{0.25pc}D\hspace{0.25pc})
	\twoch{$-\frac{1}{2}+\frac{\sqrt{3}}{2}\ii$}{$-\frac{1}{8}\left(1+\sqrt{3}\ii\right)$}{$\frac{1}{8}\left(-1+\sqrt{3} \ii\right)$}{$-\frac{1}{2}-\frac{\sqrt{3}}{2} \ii$}
	
	\item 设 $f(z)=2 x^{3}+3 y^{3} \ii$ , 则 $f(z)$ (\hspace{0.25pc}B\hspace{0.25pc})
	\twoch{处处不可导}{仅在 $6x^2=9y^2$ 上可导, 处处不解析}{处处解析}{仅在 $(0,0)$ 点可导}
	
	\item 下列等式正确的是 (\hspace{0.25pc}C\hspace{0.25pc})
	\twoch{$\Ln \mathrm{i}=\left(2 k \uppi-\frac{\uppi}{2}\right) \ii, \ln \ii=\frac{\uppi}{2} \ii$}{$\Ln \ii=\left( 2k\uppi+\frac{\uppi}{2}\right)\ii,\ln\ii=-\frac{\uppi}{2}\ii $}{$\Ln \ii=\left(2 k \uppi+\frac{\uppi}{2}\right) \ii, \ln \ii=\frac{\uppi}{2} \ii$}{$\Ln \ii=\left(2 k \uppi-\frac{\uppi}{2}\right) \ii, \ln \ii=-\frac{\uppi}{2} \ii$}
	
	\item $z=0$ 是函数 $\frac{1-\cos z}{z-\sin z}$ 的 (\hspace{0.25pc}D\hspace{0.25pc})
	\fourch{本性奇点}{可去奇点}{二级极点}{一级极点}
	
	\item 设 $\mathrm{C}$ 为 $z=(1-\ii)t$ , $t$ 从 $1$ 到 $0$ 的一段, 则 $\int_{\mathrm{C}} \overline{z} \dd z=$ (\hspace{0.25pc}A\hspace{0.25pc})
	\fourch{$-1$}{$1$}{$-\ii$}{$\ii$}
\end{enumerate}

\subsubsection{填空题(每小题 $3$ 分, 共 $15$ 分)}
\begin{enumerate}
	\item 若 $z+|z|=2+\ii$ , 则 $z=$\underline{\hspace{1pc}$\frac{3}{4}+\ii$\hspace{1pc}}
	
	\item 若 $\mathrm{C}$ 为正向圆周 $|z|=\frac{1}{2}$ , 则 $\oint_{\mathrm{C}} \frac{1}{z-2} \dd z=$\underline{\hspace{1pc}$0$\hspace{1pc}}
	
	\item 若 $z=2-\uppi\ii$ , 则 $\ee^{z}=$\underline{\hspace{1pc}$-\ee^2$\hspace{1pc}}
	
	\item 若 $f(z)=\cos z^2$ , 则 $f(z)$ 在 $z=0$ 处泰勒展开式中 $z^4$ 项的系数 $a_4=$\underline{\hspace{1pc}$-\frac{1}{2}$\hspace{1pc}}
	
	\item 函数 $f(t)=\sin t$ 的拉普拉斯变换 $F(s)=$\underline{\hspace{1pc}$\frac{1}{s^2+1}$\hspace{1pc}}
\end{enumerate}

\subsubsection{计算题(70分)}
\begin{enumerate}
	\item 设 $u(x,y)=x-2xy$ 且 $f(0)=0$ , 求解析函数 $f(z)=u+\ii v$ . ( $10$ 分)
	\begin{solution}
		解析函数的 $u,v$ 必定满足 $\mathrm{C}.-\mathrm{R}.$ 方程, 即
		\begin{equation*}
			\begin{cases}
			\frac{\partial u}{\partial x}=\frac{\partial v}{\partial y}\\
			\frac{\partial u}{\partial y}=-\frac{\partial v}{\partial x}
			\end{cases}
		\end{equation*}
		$\frac{\partial v}{\partial y}=\frac{\partial u}{\partial x}=1-2 y$ , $\frac{\partial v}{\partial y}$ 对 $y$ 积分得 $v=y-y^{2}+\varphi(x)$
		
		$\frac{\partial u}{\partial y}=-2 x=-\frac{\partial v}{\partial x}=-\varphi^{\prime}(x)$ , 可以得出 $\varphi(x)=x^{2}+C$
		
		由于 $f(0)=0$ , 因此 $C=0$ ,即 $f(z)=x-2 x y+\ii\left(y-y^{2}+x^{2}\right)$
	\end{solution}
	
	\item 计算积分 $\oint_{\mathrm{C}}\frac{2\ee^x}{z^5}\dd z$ 的值, 其中 $\mathrm{C}$ 为正向圆周 $|z|=1$ . ( $7$ 分)
	\begin{solution}
		根据高阶导数公式 $f^{(n)}(z_0)=\frac{n!}{2\uppi\ii}\oint_{\mathrm{C}}\frac{f(z)}{(z-z_0)^{n+1}}\dd z$ , 那么
		\begin{equation*}
			\oint_{\mathrm{C}} \frac{2 \ee^{z}}{(z-0)^{5}} \dd z=\frac{2 \uppi \ii}{4 !}\left.\left(2 \ee^{z}\right)^{(4)}\right|_{z=0}=\frac{\uppi \ii}{6}
		\end{equation*}
	\end{solution}
	
	\item 计算积分 $\oint_{\mathrm{C}}\frac{3z+5}{z^2-z}\dd z$ 的值, 其中 $\mathrm{C}$ 为正向圆周 $|z|=\frac{1}{2}$ . ( $7$ 分)
	\begin{solution}
		\begin{equation*}
			\oint_{\mathrm{C}}\frac{3z+5}{z^2-z}\dd z=2\uppi\ii\underset{z=0}{\Res}\frac{3z+5}{z(z-1)}=2\uppi\ii\left.\frac{3z+5}{z-1}\right|_{z=0}=-10\uppi\ii
		\end{equation*}
	\end{solution}
	
	\item 求函数 $\frac{1-\cos z}{z^3}$ 在有限奇点处的留数. ( $7$ 分)
	\begin{solution}
		对 $\cos z$ 进行洛朗展开, $\cos z=1+\sum_{n=1}^{\infty}(-1)^n\frac{z^{2n}}{(2n)!}$ , 那么 $1-\cos z=\sum_{n=1}^{\infty}(-1)^{n+1}\frac{z^{2n}}{(2n)!}$
		
		那么 $\frac{1-\cos z}{z^3}=\sum_{n=1}^{\infty}(-1)^{n+1}\frac{z^{2n-3}}{(2n)!}$ , 根据洛朗系数公式, $\underset{z=0}{\Res}\frac{1-\cos z}{z^3}=c_{-1}=\frac{1}{2}$
	\end{solution}
	
	\item 求函数 $\frac{2z^2+1}{z^2+2z}$ 在有限奇点处的留数. ( $7$ 分)
	\begin{solution}
		\begin{equation*}
			\underset{z=0}{\Res}\frac{2z^2+1}{z^2+2z}=\left.\frac{2z^2+1}{z+2} \right|_{z=0}=\frac{1}{2} , 
			\underset{z=-2}{\Res}\frac{2z^2+1}{z^2+2z}=\left.\frac{2z^2+1}{z}\right|_{z=-2}=-\frac{9}{2}
		\end{equation*}
		
	\end{solution}
	
	\item 将 $f(z)=\frac{z}{(z-2)(z-6)}$ 在 $2<|z|<6$ 内展开为洛朗级数. ( $10$ 分)
	\begin{solution}
		\begin{align*}
			f(z)&=\frac{z}{4}\left( \frac{1}{z-6}-\frac{1}{z-2}\right) =\frac{z}{4}\left( -\frac{1}{6}\frac{1}{1-z/6}-\frac{1}{z}\frac{1}{1-2/z}\right) \\
			&=\frac{z}{4}\left( -\frac{1}{6}\sum_{n=0}^{\infty}(z/6)^n-\frac{1}{z}\sum_{n=0}^{\infty}(2/z)^n\right)\\
			&=-\frac{1}{4}\left( \sum_{n=0}^{\infty}(z/6)^{n+1}+\sum_{n=0}^{\infty}(2/z)^n\right)  
		\end{align*}
	\end{solution}
	
	\item 若函数 $f(z)=a y^{3}+b x^{2} y+\ii\left(x^{3}+c x y^{2}\right)$ 是复平面上的解析函数, 求 $a,b,c$ 的值. ( $12$ 分)
	\begin{solution}
		若 $f(z)$ 为解析函数, 则其实部、虚部满足 $\mathrm{C}.-\mathrm{R}.$ 方程, 设 $u=ay^3+bx^2y$ , $v=x^3+cxy^2$ , 则有
		\begin{equation*}
			\begin{cases}
			\frac{\partial u}{\partial x}=2 b x y=2 c x y=\frac{\partial v}{\partial y}\\
			\frac{\partial u}{\partial y}=3 a y^{2}+b x^{2}=-3 x^{2}-c y^{2}=-\frac{\partial v}{\partial x}
			\end{cases}
		\end{equation*}
		解得\begin{equation*}
			\begin{cases}
			a=1\\
			b=c=-3
			\end{cases}
		\end{equation*}
		
	\end{solution}
	
	\item 利用拉普拉斯变换解常微分方程初值问题: $\begin{cases}
	x''(t)+6x'(t)+9x(t)=\ee^{-3t}\\
	x(0)=0, x'(0)=0
	\end{cases}$ . ( $10$ 分)
	\begin{solution}
		设 $\LL[x]=X(s)$ , 对等式两边作拉普拉斯变换
		\begin{align*}
			\LL[x''+6x'+9x]&=s^2X(s)-sx(0)-x'(0)+6sX(s)-6x(0)+9X(s)\\
			&=s^2X(s)+6sX(s)+9X(s)=\frac{1}{s+3}
		\end{align*}
		那么有 $X(s)=\frac{1}{(s+3)^3}$ , 根据拉普拉斯变换的微分性质 $F''(s)=\LL[t^2f(t)]$
		\begin{equation*}
			\frac{1}{(s+3)^3}=\frac{1}{2}\left(\frac{1}{s+3} \right)''=\frac{\LL[t^2\ee^{-3t}]}{2}
		\end{equation*}
		那么 $x(t)=\frac{t^2\ee^{-3t}}{2}$
	\end{solution}
\end{enumerate}