\chapter{《复变函数》试卷汇总}
\section{秋}
\subsection{2018-2019 A}
\subsubsection{选择题(每小题 $3$ 分,共 $15$ 分)}
\begin{ti}
	$\frac{(\sqrt{3}-\ii)^{4}}{(1-\ii)^{8}}=$ \kuo{}
	\twoch{$-\frac{1}{2}+\frac{\sqrt{3}}{2}\ii$}{$-\frac{1}{8}\left(1+\sqrt{3}\ii\right)$}{$\frac{1}{8}\left(-1+\sqrt{3} \ii\right)$}{$-\frac{1}{2}-\frac{\sqrt{3}}{2} \ii$}
\end{ti}

\begin{ti}
	设 $f(z)=2 x^{3}+3 y^{3} \ii$, 则 $f(z)$ \kuo{}
	\twoch{处处不可导}{仅在 $6x^2=9y^2$ 上可导, 处处不解析}{处处解析}{仅在 $(0,0)$ 点可导}
\end{ti}

\begin{ti}
	下列等式正确的是 \kuo{}
	\twoch{$\Ln \ii =\left(2 k \uppi-\frac{\uppi}{2}\right) \ii, \ln \ii=\frac{\uppi}{2} \ii$}{$\Ln \ii=\left( 2k\uppi+\frac{\uppi}{2}\right)\ii,\ln\ii=-\frac{\uppi}{2}\ii $}{$\Ln \ii=\left(2 k \uppi+\frac{\uppi}{2}\right) \ii, \ln \ii=\frac{\uppi}{2} \ii$}{$\Ln \ii=\left(2 k \uppi-\frac{\uppi}{2}\right) \ii, \ln \ii=-\frac{\uppi}{2} \ii$}
\end{ti}

\begin{ti}
	$z=0$ 是函数 $\frac{1-\cos z}{z-\sin z}$ 的 \kuo{}
	\fourch{本性奇点}{可去奇点}{二级极点}{一级极点}
\end{ti}

\begin{ti}
	设 $\mathrm{C}$ 为 $z=(1-\ii)t$, $t$ 从 $1$ 到 $0$ 的一段, 则 $\int_{\mathrm{C}} \overline{z} \dd z=$ \kuo{}
	\fourch{$-1$}{$1$}{$-\ii$}{$\ii$}
\end{ti}

\subsubsection{填空题(每小题 $3$ 分,共 $15$ 分)}
\begin{ti}
	若 $z+|z|=2+\ii$, 则 $z=$ \hua{}
\end{ti}

\begin{ti}
	若 $\mathrm{C}$ 为正向圆周 $|z|=\frac{1}{2}$, 则 $\oint_{\mathrm{C}} \frac{1}{z-2} \dd z=$ \hua{}
\end{ti}

\begin{ti}
	若 $z=2-\uppi\ii$, 则 $\ee^{z}=$ \hua{}
\end{ti}

\begin{ti}
	若 $f(z)=\cos z^2$, 则 $f(z)$ 在 $z=0$ 处泰勒展开式中 $z^4$ 项的系数 $a_4=$ \hua{}
\end{ti}

\begin{ti}
	函数 $f(t)=\sin t$ 的拉普拉斯变换 $F(s)=$ \hua{}
\end{ti}

\subsubsection{计算题(70分)}
\begin{ti}[$10$ 分]
	设 $u(x,y)=x-2xy$ 且 $f(0)=0$, 求解析函数 $f(z)=u+\ii v$
\end{ti}

\begin{ti}[$7$ 分]
	计算积分 $\oint_{\mathrm{C}}\frac{2\ee^x}{z^5}\dd z$ 的值, 其中 $\mathrm{C}$ 为正向圆周 $|z|=1$
\end{ti}

\begin{ti}[$7$ 分]
	计算积分 $\oint_{\mathrm{C}}\frac{3z+5}{z^2-z}\dd z$ 的值, 其中 $\mathrm{C}$ 为正向圆周 $|z|=\frac{1}{2}$
\end{ti}

\begin{ti}[$7$ 分]
	求函数 $\frac{1-\cos z}{z^3}$ 在有限奇点处的留数
\end{ti}

\begin{ti}[$7$ 分]
	求函数 $\frac{2z^2+1}{z^2+2z}$ 在有限奇点处的留数
\end{ti}

\begin{ti}[$10$ 分]
	将 $f(z)=\frac{z}{(z-2)(z-6)}$ 在 $2<|z|<6$ 内展开为洛朗级数
\end{ti}

\begin{ti}[$12$ 分]
	若函数 $f(z)=a y^{3}+b x^{2} y+\ii\left(x^{3}+c x y^{2}\right)$ 是复平面上的解析函数, 求 $a,b,c$ 的值
\end{ti}

\begin{ti}[$10$ 分]
	利用拉普拉斯变换解常微分方程初值问题: $\begin{cases}
		x''(t)+6x'(t)+9x(t)=\ee^{-3t}\\
		x(0)=0, x'(0)=0
	\end{cases}$
\end{ti}

\subsection{2018-2019 A 答案}
\subsubsection{选择题(每小题 $3$ 分,共 $15$ 分)}
\begin{ti}
	$\frac{(\sqrt{3}-\ii)^{4}}{(1-\ii)^{8}}=$ \kuoD{}
	\twoch{$-\frac{1}{2}+\frac{\sqrt{3}}{2}\ii$}{$-\frac{1}{8}\left(1+\sqrt{3}\ii\right)$}{$\frac{1}{8}\left(-1+\sqrt{3} \ii\right)$}{$-\frac{1}{2}-\frac{\sqrt{3}}{2} \ii$}
\end{ti}

\begin{ti}
	设 $f(z)=2 x^{3}+3 y^{3} \ii$, 则 $f(z)$ \kuoB{}
	\twoch{处处不可导}{仅在 $6x^2=9y^2$ 上可导, 处处不解析}{处处解析}{仅在 $(0,0)$ 点可导}
\end{ti}

\begin{ti}
	下列等式正确的是 \kuoC{}
	\twoch{$\Ln \ii =\left(2 k \uppi-\frac{\uppi}{2}\right) \ii, \ln \ii=\frac{\uppi}{2} \ii$}{$\Ln \ii=\left( 2k\uppi+\frac{\uppi}{2}\right)\ii,\ln\ii=-\frac{\uppi}{2}\ii $}{$\Ln \ii=\left(2 k \uppi+\frac{\uppi}{2}\right) \ii, \ln \ii=\frac{\uppi}{2} \ii$}{$\Ln \ii=\left(2 k \uppi-\frac{\uppi}{2}\right) \ii, \ln \ii=-\frac{\uppi}{2} \ii$}
\end{ti}

\begin{ti}
	$z=0$ 是函数 $\frac{1-\cos z}{z-\sin z}$ 的 \kuoD{}
	\fourch{本性奇点}{可去奇点}{二级极点}{一级极点}
\end{ti}

\begin{ti}
	设 $\mathrm{C}$ 为 $z=(1-\ii)t$, $t$ 从 $1$ 到 $0$ 的一段, 则 $\int_{\mathrm{C}} \overline{z} \dd z=$ \kuoA{}
	\fourch{$-1$}{$1$}{$-\ii$}{$\ii$}
\end{ti}

\subsubsection{填空题(每小题 $3$ 分,共 $15$ 分)}
\begin{ti}
	若 $z+|z|=2+\ii$, 则 $z=$ \huaa{$\frac{3}{4}+\ii$}
\end{ti}

\begin{ti}
	若 $\mathrm{C}$ 为正向圆周 $|z|=\frac{1}{2}$, 则 $\oint_{\mathrm{C}} \frac{1}{z-2} \dd z=$ \huaa{$0$}
\end{ti}

\begin{ti}
	若 $z=2-\uppi\ii$, 则 $\ee^{z}=$ \huaa{$-\ee^2$}
\end{ti}

\begin{ti}
	若 $f(z)=\cos z^2$, 则 $f(z)$ 在 $z=0$ 处泰勒展开式中 $z^4$ 项的系数 $a_4=$ \huaa{$-\frac{1}{2}$}
\end{ti}

\begin{ti}
	函数 $f(t)=\sin t$ 的拉普拉斯变换 $F(s)=$ \huaa{$\frac{1}{s^2+1}$}
\end{ti}

\subsubsection{计算题(70分)}
\begin{ti}[$10$ 分]
	设 $u(x,y)=x-2xy$ 且 $f(0)=0$, 求解析函数 $f(z)=u+\ii v$
	\begin{solution}
		解析函数的 $u,v$ 必定满足 $\mathrm{C}.-\mathrm{R}.$ 方程, 即
		\begin{equation*}
			\begin{cases}
			\frac{\partial u}{\partial x}=\frac{\partial v}{\partial y}\\
			\frac{\partial u}{\partial y}=-\frac{\partial v}{\partial x}
			\end{cases}
		\end{equation*}
		\[
			\frac{\partial v}{\partial y}=\frac{\partial u}{\partial x}=1-2 y
		\]
		$\frac{\partial v}{\partial y}$ 对 $y$ 积分得
		\[
			v=y-y^{2}+\varphi(x)
		\]
		\[
			\frac{\partial u}{\partial y}=-2 x=-\frac{\partial v}{\partial x}=-\varphi^{\prime}(x)
		\]
		可以得出
		\[
			\varphi(x)=x^{2}+C
		\]
		由于 $f(0)=0$, 因此 $C=0$ ,即
		\[
			f(z)=x-2 x y+\ii\left(y-y^{2}+x^{2}\right)
		\]
	\end{solution}
\end{ti}

\begin{ti}[$7$ 分]
	计算积分 $\oint_{\mathrm{C}}\frac{2\ee^x}{z^5}\dd z$ 的值, 其中 $\mathrm{C}$ 为正向圆周 $|z|=1$
	\begin{solution}
		根据高阶导数公式
		\[
			f^{(n)}(z_0)=\frac{n!}{2\uppi\ii}\oint_{\mathrm{C}}\frac{f(z)}{(z-z_0)^{n+1}}\dd z
		\]
		那么
		\begin{equation*}
			\oint_{\mathrm{C}} \frac{2 \ee^{z}}{(z-0)^{5}} \dd z=\frac{2 \uppi \ii}{4 !}\left.\left(2 \ee^{z}\right)^{(4)}\right|_{z=0}=\frac{\uppi \ii}{6}
		\end{equation*}
	\end{solution}
\end{ti}

\begin{ti}[$7$ 分]
	计算积分 $\oint_{\mathrm{C}}\frac{3z+5}{z^2-z}\dd z$ 的值, 其中 $\mathrm{C}$ 为正向圆周 $|z|=\frac{1}{2}$
	\begin{solution}
		\begin{equation*}
			\oint_{\mathrm{C}}\frac{3z+5}{z^2-z}\dd z=2\uppi\ii\underset{z=0}{\Res}\frac{3z+5}{z(z-1)}=2\uppi\ii\left.\frac{3z+5}{z-1}\right|_{z=0}=-10\uppi\ii
		\end{equation*}
	\end{solution}
\end{ti}

\begin{ti}[$7$ 分]
	求函数 $\frac{1-\cos z}{z^3}$ 在有限奇点处的留数
	\begin{solution}
		对 $\cos z$ 进行洛朗展开
		\[
			\cos z=1+\sum_{n=1}^{\infty}(-1)^n\frac{z^{2n}}{(2n)!}
		\]
		那么
		\begin{align*}
			1-\cos z&=\sum_{n=1}^{\infty}(-1)^{n+1}\frac{z^{2n}}{(2n)!} \\
			\frac{1-\cos z}{z^3}&=\sum_{n=1}^{\infty}(-1)^{n+1}\frac{z^{2n-3}}{(2n)!}
		\end{align*}
		根据洛朗系数公式
		\[
			\underset{z=0}{\Res}\frac{1-\cos z}{z^3}=c_{-1}=\frac{1}{2}
		\]
	\end{solution}
\end{ti}

\begin{ti}[$7$ 分]
	求函数 $\frac{2z^2+1}{z^2+2z}$ 在有限奇点处的留数
	\begin{solution}
		\begin{equation*}
			\underset{z=0}{\Res}\frac{2z^2+1}{z^2+2z}=\left.\frac{2z^2+1}{z+2} \right|_{z=0}=\frac{1}{2},
			\underset{z=-2}{\Res}\frac{2z^2+1}{z^2+2z}=\left.\frac{2z^2+1}{z}\right|_{z=-2}=-\frac{9}{2}
		\end{equation*}
	\end{solution}
\end{ti}

\begin{ti}[$10$ 分]
	将 $f(z)=\frac{z}{(z-2)(z-6)}$ 在 $2<|z|<6$ 内展开为洛朗级数
	\begin{solution}
		\begin{align*}
			f(z)&=\frac{z}{4}\left( \frac{1}{z-6}-\frac{1}{z-2}\right) =\frac{z}{4}\left( -\frac{1}{6}\frac{1}{1-z/6}-\frac{1}{z}\frac{1}{1-2/z}\right) \\
			&=\frac{z}{4}\left( -\frac{1}{6}\sum_{n=0}^{\infty}(z/6)^n-\frac{1}{z}\sum_{n=0}^{\infty}(2/z)^n\right)\\
			&=-\frac{1}{4}\left( \sum_{n=0}^{\infty}(z/6)^{n+1}+\sum_{n=0}^{\infty}(2/z)^n\right)  
		\end{align*}
	\end{solution}
\end{ti}

\begin{ti}[$12$ 分]
	若函数 $f(z)=a y^{3}+b x^{2} y+\ii\left(x^{3}+c x y^{2}\right)$ 是复平面上的解析函数, 求 $a,b,c$ 的值
	\begin{solution}
		若 $f(z)$ 为解析函数, 则其实部、虚部满足 $\mathrm{C}.-\mathrm{R}.$ 方程, 设 $u=ay^3+bx^2y$, $v=x^3+cxy^2$, 则有
		\begin{equation*}
			\begin{cases}
			\frac{\partial u}{\partial x}=2 b x y=2 c x y=\frac{\partial v}{\partial y}\\
			\frac{\partial u}{\partial y}=3 a y^{2}+b x^{2}=-3 x^{2}-c y^{2}=-\frac{\partial v}{\partial x}
			\end{cases}
		\end{equation*}
		解得
		\begin{equation*}
			\begin{cases}
			a=1\\
			b=c=-3
			\end{cases}
		\end{equation*}
	\end{solution}
\end{ti}

\begin{ti}[$10$ 分]
	利用拉普拉斯变换解常微分方程初值问题: $\begin{cases}
		x''(t)+6x'(t)+9x(t)=\ee^{-3t}\\
		x(0)=0, x'(0)=0
	\end{cases}$
	\begin{solution}
		设 $\LL[x]=X(s)$, 对等式两边作拉普拉斯变换
		\begin{align*}
			\LL[x''+6x'+9x]&=s^2X(s)-sx(0)-x'(0)+6sX(s)-6x(0)+9X(s)\\
			&=s^2X(s)+6sX(s)+9X(s)=\frac{1}{s+3}
		\end{align*}
		那么有 $X(s)=\frac{1}{(s+3)^3}$, 根据拉普拉斯变换的微分性质 $F''(s)=\LL[t^2f(t)]$
		\begin{equation*}
			\frac{1}{(s+3)^3}=\frac{1}{2}\left(\frac{1}{s+3} \right)''=\frac{\LL\bigl[t^2\ee^{-3t}\bigr]}{2}
		\end{equation*}
		那么
		\[
			x(t)=\frac{t^2\ee^{-3t}}{2}
		\]
	\end{solution}
\end{ti}

\subsection{2019-2020 A13}
\subsubsection{选择题(每题 3 分,15 分)}
\begin{ti}
	在复平面上方程 $|z + 3| - |z - 1| = 0$ 表示 \kuo
	\fourch{直线}{圆周}{椭圆}{抛物线}
\end{ti}

\begin{ti}
	设 $f(z) = x^{2} - y^{2} + 2xy \ii$, 则 $f(1 + \ii) = $ \kuo
	\fourch{$2$}{$2\ii$}{$1 + \ii$}{$2 + 2\ii$}
\end{ti}

\begin{ti}
	幂级数 $\sum_{n=0}^{\infty} \frac{1}{2^{n}} z^{n}$ 的收敛半径 $R = $ \kuo
	\fourch{$\ee$}{$\frac{1}{2}$}{$2$}{$\frac{1}{\ee}$}
\end{ti}

\begin{ti}
	函数 $f(z) = z \Im z$ 在复平面上有定义且 \kuo
	\fourch{在 $z = 0$ 解析}{处处解析}{处处不解析}{以上都不对}
\end{ti}

\begin{ti}
	$z = 0$ 是函数 $\frac{\tan z}{z}$ 的 \kuo
	\fourch{本性奇点}{可去奇点}{一级极点}{二级极点}
\end{ti}

\subsubsection{填空题(每题 3 分,15 分)}
\begin{ti}
	设 $z = 2 - 2 \ii$. 则其三角表示式为 \hua
\end{ti}

\begin{ti}
	若幂级数 $\sum_{n = 0}^{\infty} c_{n} z^{n}$ 在 $z = 2 + \ii$ 处收敛, 那么该级数在 $z = 2$ 处的敛散性为 \hua
\end{ti}

\begin{ti}
	$\int_{0}^{\ii} 9 z^{2} \dd{z} = $ \hua
\end{ti}

\begin{ti}
	$\ln(\sqrt{3} + \ii) = $ \hua
\end{ti}

\begin{ti}
	$f(t) = \ee^{t} - \sin t$ 的拉普拉斯变换 \hua
\end{ti}

\subsubsection{计算题(70 分)}
\begin{ti}[10 分]
	已知调和函数 $u = x^{2} - y^{2} + xy, f(\ii) = - 1 + \ii$, 求解析函数 $f(z) = u + \ii v$, 并求 $f'(z)$
\end{ti}

\begin{ti}[7 分]
	计算积分 $\oint_{C} \frac{z \ee^{z}}{z - 1} \dd{z}$ 的值, 其中 $C$ 为正向圆周 $|z| = 2$
\end{ti}

\begin{ti}[7 分]
	计算积分 $\oint_{C} \frac{z \sin z}{\left( z - \frac{\uppi}{2} \right)^{2}} \dd{z}$ 的值, 其中 $C$ 为正向圆周 $|z| = 2$
\end{ti}

\begin{ti}[7 分]
	求函数 $\frac{\ee^{z}}{z (z + 1)}$ 在有限奇点处的留数
\end{ti}

\begin{ti}[7 分]
	求函数 $\frac{z^{2} - \cos z}{z^{3}}$ 在有限奇点处的留数
\end{ti}

\begin{ti}[10 分]
	求函数 $f(z) = \frac{1}{1 + z^{2}}$ 在 $2 < |z - \ii| < \infty$ 内展开成洛朗级数
\end{ti}

\begin{ti}[12 分]
	证明题:$f(z) = u + \ii v$ 与 $\overline{ f(z) }$ 在某区域 $D$ 内都解析, 试证 $f(z) = u + \ii v$ 是一常数
\end{ti}

\begin{ti}[10 分]
	利用拉普拉斯变换解常微分方程初值问题:
	\[
		\begin{cases}
			x''(t) + 4x'(t) + 3x(t) = \ee^{t},\\
			x(0) = 0, x'(0) = 0.
		\end{cases}
	\]
\end{ti}

\newpage
\guanggao