\chapter{高等数学试卷汇总}

\section{高数(一)期中}





\section{高数(一)期终}
\subsection{复习题 1}
\subsection{复习题 1 答案}




\section{高数(二)期中}
\subsection{复习题 1}%2018-2019B10
\subsubsection{选择题}
\begin{enumerate}
	\item 微分方程 $(y')^3+3\sqrt{y''}+x^4y'''=\sin x$ 的阶数是(\hspace{1pc})
	\fourch{1}{4}{2}{3}
	\item 设 $f(x,y)=x-y-\sqrt{x^2+y^2}$ , 则 $f_{x}(3,4)=$(\hspace{1pc})
	\fourch{$\frac{3}{5}$}{$\frac{2}{5}$}{$-\frac{2}{5}$}{$\frac{1}{5}$}
	\item 微分方程 $y'=\frac{y}{x}$ 的一个特解是(\hspace{1pc})
	\fourch{$y=2x$}{$\ee^y=x$}{$y=x^2$}{$y=\ln x$}
	\item 若 $z=\ln\sqrt{1+x^2+y^2}$ , 则 $\left.\dd z\right|_{(1,1)}=$(\hspace{1pc})
	\fourch{$\frac{\dd x+\dd y}{3}$}{$\frac{\dd x+\dd y}{2}$}{$\frac{\dd x+\dd y}{1}$}{$3(\dd x+\dd y)$}
	\item 设直线 $L:\begin{cases}
	x+3y+2z+1=0\\
	2x-y-10z+3=0
	\end{cases}$ , 平面 $\eta:\ 4x-2y+z-2=0$ , 则(\hspace{1pc})
	\fourch{$L$ 在 $\eta$ 上}{$L$ 平行于 $\eta$}{$L$ 垂直于 $\eta$}{$L$ 与 $\eta$ 斜交}
	\item 方程 $y'+3xy=6x^2y$ 是(\hspace{1pc})
	\twoch{二阶微分方程}{非线性微分方程}{一阶线性非齐次微分方程}{可分离变量的微分方程}
	\item 曲面 $\frac{x^2}{9}-\frac{y^2}{4}+\frac{z^2}{4}=1$ 与平面 $x=y$ 的交线是(\hspace{1pc})
	\fourch{两条直线}{双曲线}{椭圆}{抛物线}
	\item 设 $z=\ee^{x^2y}$ , 则 $\frac{\partial^2z}{\partial x\partial y}=$(\hspace{1pc})
	\twoch{$2y\left(1+x^3\right)\ee^{x^2y}$}{$\ee^{x^2y}$}{$2x\left(1+x^2y\right)\ee^{x^2y}$}{$2x\ee^{x^2y}$}
	\item 下列结论正确的是(\hspace{1pc})
	\twoch{$\vec{a}\times\left(\vec{b}-\vec{c}\right)=\vec{a}\times\vec{b}-\vec{a}\times\vec{c}$}{若 $\vec{a}\times\vec{b}=\vec{a}\times\vec{c}$ 且 $\vec{a}\ne\vec{0}$ , 则 $\vec{b}=\vec{c}$}{$\vec{a}\times\vec{b}=\vec{b}\times\vec{a}$}{若 $\left|\vec{a}\right|=1,\left|\vec{b}\right|=1$ , 则 $\left|\vec{a}\times\vec{b}\right|=1$}
\end{enumerate}

\subsubsection{填空题}
\begin{enumerate}
	\item 平面过点 $(2,0,0),(0,1,0),(0,0,0.5)$ , 则该平面的方程是\underline{\hspace{8pc}}
	\item 设 $y_1$ 是 $y''+p(x)y'+q(x)y=f(x)$ 的解, $y_2$ 是 $y''+p(x)y'+q(x)y=f(x)$ 的解, 则 $y_1+y_2$ 是\underline{\hspace{8pc}}方程的解
	\item 设 $z=y\arctan x$ , 则 $\left.\mathrm{grad}\,z\right|_{(1,2)}=$\underline{\hspace{8pc}}
	\item 过点 $P(0,2,4)$ 且与两平面 $x+2z=1$ 和 $y-2z=2$ 平行的直线方程是\underline{\hspace{8pc}}
	\item 设 $f(x,y)=\arcsin\frac{y}{x}$ , 则 $f_y(1,0)=$\underline{\hspace{8pc}}
	\item $y=\ee^x$ 是微分方程 $y''+py'+6y=0$ 的一个特解, 则 $p=$\underline{\hspace{8pc}}
	\item 已知平面 $\eta_1:\ A_1x+B_1y+C_1z+D_1=0$ 与平面 $\eta_2:\ A_2x+B_2y+C_2z+D_2=0$, 则 $\eta_1\perp\eta_2$ 的充要条件是\underline{\hspace{8pc}}
	\item 微分方程 $y''+2y'+5y=0$ 的通解为 $y=$\underline{\hspace{8pc}}
	\item 设 $z=\ee^{xy}+\cos\left(x^2+y\right)$, 则 $\frac{\partial z}{\partial y}=$\underline{\hspace{8pc}}
\end{enumerate}
\subsubsection{大题}
\begin{enumerate}
	\item 求方程 $\frac{\dd z}{\dd x}=-z+4x$ 的通解
	\item 求曲线 $2z+1=\ln(xy)+\ee^z$ 在点 $M_{0}(1,1,0)$ 处的切平面和法线方程
	\item 设由方程组 $\begin{cases}
	x+y+z=0\\
	x^2+y^2+z^2=1
	\end{cases}$
	确定了隐函数 $x=x(z),y=y(z)$ , 求 $\frac{\dd x}{\dd z},\frac{\dd y}{\dd z}$
	\item 求方程 $y''+6y'+13y=\ee^t$ 的通解
	\item 设 $z=x^2y+\sin x+\varphi(xy+1)$ , 且 $\varphi(u)$ 具有一阶连续导数, 求 $\frac{\partial z}{\partial x},\frac{\partial z}{\partial y}$
\end{enumerate}

\subsection{复习题 1 答案}
\subsubsection{选择题}
\begin{enumerate}
	\item 微分方程 $(y')^3+3\sqrt{y''}+x^4y'''=\sin x$ 的阶数是(\hspace{0.25pc}D\hspace{0.25pc})
	\fourch{1}{4}{2}{3}
	\item 设 $f(x,y)=x-y-\sqrt{x^2+y^2}$ , 则 $f_{x}(3,4)=$(\hspace{0.25pc}B\hspace{0.25pc})
	\fourch{$\frac{3}{5}$}{$\frac{2}{5}$}{$-\frac{2}{5}$}{$\frac{1}{5}$}
	\item 微分方程 $y'=\frac{y}{x}$ 的一个特解是(\hspace{0.25pc}A\hspace{0.25pc})
	\fourch{$y=2x$}{$\ee^y=x$}{$y=x^2$}{$y=\ln x$}
	\item 若 $z=\ln\sqrt{1+x^2+y^2}$ , 则 $\left.\dd z\right|_{(1,1)}=$(\hspace{0.25pc}A\hspace{0.25pc})
	\fourch{$\frac{\dd x+\dd y}{3}$}{$\frac{\dd x+\dd y}{2}$}{$\frac{\dd x+\dd y}{1}$}{$3(\dd x+\dd y)$}
	\item 设直线 $L:\begin{cases}
	x+3y+2z+1=0\\
	2x-y-10z+3=0
	\end{cases}$ , 平面 $\eta:\ 4x-2y+z-2=0$ , 则(\hspace{0.25pc}C\hspace{0.25pc})
	\fourch{$L$ 在 $\eta$ 上}{$L$ 平行于 $\eta$}{$L$ 垂直于 $\eta$}{$L$ 与 $\eta$ 斜交}
	\item 方程 $y'+3xy=6x^2y$ 是(\hspace{0.25pc}D\hspace{0.25pc})
	\twoch{二阶微分方程}{非线性微分方程}{一阶线性非齐次微分方程}{可分离变量的微分方程}
	\item 曲面 $\frac{x^2}{9}-\frac{y^2}{4}+\frac{z^2}{4}=1$ 与平面 $x=y$ 的交线是(\hspace{0.25pc}B\hspace{0.25pc})
	\fourch{两条直线}{双曲线}{椭圆}{抛物线}
	\item 设 $z=\ee^{x^2y}$ , 则 $\frac{\partial^2z}{\partial x\partial y}=$(\hspace{0.25pc}C\hspace{0.25pc})
	\twoch{$2y\left(1+x^3\right)\ee^{x^2y}$}{$\ee^{x^2y}$}{$2x\left(1+x^2y\right)\ee^{x^2y}$}{$2x\ee^{x^2y}$}
	\item 下列结论正确的是(\hspace{0.25pc}A\hspace{0.25pc})
	\twoch{$\vec{a}\times\left(\vec{b}-\vec{c}\right)=\vec{a}\times\vec{b}-\vec{a}\times\vec{c}$}{若 $\vec{a}\times\vec{b}=\vec{a}\times\vec{c}$ 且 $\vec{a}\ne\vec{0}$ , 则 $\vec{b}=\vec{c}$}{$\vec{a}\times\vec{b}=\vec{b}\times\vec{a}$}{若 $\left|\vec{a}\right|=1,\left|\vec{b}\right|=1$ , 则 $\left|\vec{a}\times\vec{b}\right|=1$}
\end{enumerate}

\subsubsection{填空题}
\begin{enumerate}
	\item 平面过点 $(2,0,0),(0,1,0),(0,0,0.5)$ , 则该平面的方程是\underline{\hspace{1pc}$\frac{x}{2}+y+2z=1$\hspace{1pc}}
	\item 设 $y_1$ 是 $y''+p(x)y'+q(x)y=f(x)$ 的解, $y_2$ 是 $y''+p(x)y'+q(x)y=f(x)$ 的解, 则 $y_1+y_2$ 是\underline{\hspace{1pc}$y''+p(x)y'+q(x)y=2f(x)$\hspace{1pc}}方程的解
	\item 设 $z=y\arctan x$ , 则 $\left.\mathrm{grad}\,z\right|_{(1,2)}=$\underline{\hspace{1pc}$\dd x+\frac{\uppi}{4}\dd y$\hspace{1pc}}
	\item 过点 $P(0,2,4)$ 且与两平面 $x+2z=1$ 和 $y-2z=2$ 平行的直线方程是\underline{\hspace{1pc}$\frac{x}{-2}=\frac{y-2}{2}=\frac{z-4}{1}$\hspace{1pc}}
	\item 设 $f(x,y)=\arcsin\frac{y}{x}$ , 则 $f_y(1,0)=$\underline{\hspace{1pc}$1$\hspace{1pc}}
	\item $y=\ee^x$ 是微分方程 $y''+py'+6y=0$ 的一个特解, 则 $p=$\underline{\hspace{1pc}$-7$\hspace{1pc}}
	\item 已知平面 $\eta_1:\ A_1x+B_1y+C_1z+D_1=0$ 与平面 $\eta_2:\ A_2x+B_2y+C_2z+D_2=0$, 则 $\eta_1\perp\eta_2$ 的充要条件是\underline{\hspace{1pc}$A_1A_2+B_1B_2+C_1C_2=0$\hspace{1pc}}
	\item 微分方程 $y''+2y'+5y=0$ 的通解为 $y=$\underline{\hspace{1pc}$C_1\ee^{-x}\sin(2x)+C_2\ee^{-x}\cos(2x)$\hspace{1pc}}
	\item 设 $z=\ee^{xy}+\cos\left(x^2+y\right)$, 则 $\frac{\partial z}{\partial y}=$\underline{\hspace{1pc}$x\ee^{xy}-\sin\left(x^2+y\right)$\hspace{1pc}}
\end{enumerate}
\subsubsection{大题}
\begin{enumerate}
	\item 求方程 $\frac{\dd z}{\dd x}=-z+4x$ 的通解
	\begin{solution}
		运用一阶线性非齐次微分方程公式, 得
		\begin{align*}
			z&=\ee^{-\int\dd x}\left( \int 4x\ee^{\int \dd x}\dd x+C\right) =\ee^{-x}\left( \int 4x\ee^{x}\dd x+C\right) \\
			&=\ee^{-x}\left( 4(x-1)\ee^{x}+C\right) =4(x-1)+C\ee^{-x}
		\end{align*}
	\end{solution}
	\item 求曲线 $2z+1=\ln(xy)+\ee^z$ 在点 $M_{0}(1,1,0)$ 处的切平面和法线方程
	\item 设由方程组 $\begin{cases}
	x+y+z=0\\
	x^2+y^2+z^2=1
	\end{cases}$
	确定了隐函数 $x=x(z),y=y(z)$ , 求 $\frac{\dd x}{\dd z},\frac{\dd y}{\dd z}$
	\begin{solution}
		对方程组 $\begin{cases}
		x+y+z=0\\
		x^2+y^2+z^2=1
		\end{cases}$ 两式求微分, 得
		\begin{equation*}
			\begin{cases}
			\dd x+\dd y+\dd z=0\\
			2x\dd x+2y\dd y+2z\dd z=0
			\end{cases}
		\end{equation*}
		解得
		\begin{equation*}
			\begin{cases}
			\frac{\dd x}{\dd z}=-\frac{x+2z}{2x+z}\\
			\frac{\dd y}{\dd z}=-\frac{y+2x}{2y+z}
			\end{cases}
		\end{equation*}
	\end{solution}
	\item 求方程 $y''+6y'+13y=\ee^t$ 的通解
	\begin{solution}
		方程 $y''+6y'+13y=\ee^t$ 对应的齐次方程 $y''+6y'+13y=0$ 的特征方程为 $r^2+6r+13=0$ , 解得 $r=-3\pm2\ii$ , 那么齐次方程的通解为 $C_1\ee^{-3t}\sin(2t)+C_2\ee^{-3t}\cos(2t)$
		
		设特解为 $a\ee^{t}$ , 代入方程 $y''+6y'+13y=\ee^t$ 后解得 $a=\frac{1}{20}$
		
		综上, 方程 $y''+6y'+13y=\ee^t$ 的通解为 $C_1\ee^{-3t}\sin(2t)+C_2\ee^{-3t}\cos(2t)+\frac{\ee^x}{20}$
	\end{solution}
	\item 设 $z=x^2y+\sin x+\varphi(xy+1)$ , 且 $\varphi(u)$ 具有一阶连续导数, 求 $\frac{\partial z}{\partial x},\frac{\partial z}{\partial y}$
	\begin{solution}
		$\frac{\partial z}{\partial x}=2xy+\cos x+y\varphi'(xy+1)$ , $\frac{\partial z}{\partial y}=x^2+x\varphi'(xy+1)$
	\end{solution}
\end{enumerate}
\subsection{复习题 2}
\subsection{复习题 2 答案}




\section{高数(二)期终}
\subsection{复习题 1}
\subsubsection{选择题(每小题 $3$ 分, 共 $24$ 分)}
\begin{enumerate}
	\item 方程 $y''-3 y'+2 y=\ee^{x}$ 的待定特解 $y^*$ 的一个形式是 $y^*=$ (\hspace{1pc})
	\fourch{$\ee^x$}{$ax^2\ee^x$}{$a\ee^x$}{$ax\ee^x$}
	
	\item 过点 $(3,1,-2)$ 且通过直线 $\frac{x-4}{5}=\frac{y+3}{2}=\frac{z}{1}$ 的平面方程 (\hspace{1pc})
	\twoch{$5x+2y+z-15=0$}{$\frac{x-3}{8}=\frac{y-1}{-9}=\frac{z+2}{-22}$}{$8x-9y-22z-59=0$}{$\frac{x-3}{5}=\frac{y-1}{2}=\frac{z+2}{1}$}
	
	\item 设 $f(x,y)=\ln\left(x+\frac{y}{2x} \right)$ , 则 $f_{y}(1,0)=$ (\hspace{1pc})
	\fourch{$1$}{$\frac{1}{2}$}{$\frac{1}{3}$}{$0$}
	
	\item $D=\{ (x,y)|0\leq x\leq 1,0\leq y\leq 2 \}$ , 利用二重积分的性质, $\iint_{D}\frac{1}{\sqrt{x^2+y^2+2xy+16}}\dd x\dd y$ 的最佳估值区间为 (\hspace{1pc})
	\fourch{$\left[ \frac{2}{5},\frac{1}{2} \right]$}{$\left[ \frac{1}{5},\frac{1}{2} \right]$}{$\left[ \frac{2}{5},1 \right]$}{$\left[ \frac{1}{2},1 \right]$}
	
	\item $\Omega$ 由柱面 $x^2+y^2=1$ 、平面 $z=1$ 及三个坐标面围成的在第一卦限内的闭区域, 则 $\iiint_{\Omega}xy\dd V=$ (\hspace{1pc})
	\twoch{$\int_{0}^{\uppi}\dd \theta\int_{0}^{1}\dd \rho\int_{0}^{1}\rho^3\sin\theta\cos\theta\dd z$}{$\int_{0}^{2\uppi}\int_{0}^{1}\dd\rho\int_{0}^{1}\rho^2\sin\theta\cos\theta\dd z$}{$\int_{0}^{\frac{\uppi}{2}}\dd \theta\int_{0}^{1}\dd\rho\int_{0}^{1}\rho^2\sin\theta\cos\theta\dd z$}{$\int_{0}^{\frac{\uppi}{2}} \dd \theta \int_{0}^{1} \dd \rho \int_{0}^{1} \rho^{3} \sin \theta \cos \theta \dd z$}
	
	\item 设 $L$ 是 $xoy$ 平面上的有向曲线, 下列曲线积分中, (\hspace{1pc}) 是与路径无关的
	\twoch{$\int_{L} 3 y x^{2} \dd x+x^{3} \dd y$}{$\int_{L} y \dd x-x \dd y$}{$\int_{L} 2 x y \dd x-x^{2} \dd y$}{$\int_{L} 3 y x^{2} \dd x+y^{3} \dd y$}
	
	\item 设 $L$ 为圆周 $\begin{cases}
	x=a\cos t\\
	y=a\sin t
	\end{cases}(0\leq t\leq 2\uppi)$ , 则 $\oint_{L}\left(x^{2}+y^{2}\right) \dd s=$ (\hspace{1pc})
	\fourch{$a^3$}{$\uppi a^3$}{$2\uppi a^3$}{$3\uppi a^3$}
	
	\item 下列级数中收敛的是 (\hspace{1pc})
	\fourch{$\sum_{n=1}^{\infty} \frac{n}{n+1}$}{$\sum_{n=1}^{\infty} \frac{1}{n \sqrt{n+1}}$}{$\sum_{n=1}^{\infty} \frac{1}{2(n+1)}$}{$\sum_{n=1}^{\infty} \frac{1}{\sqrt{n+1}}$}
\end{enumerate}

\subsubsection{填空题(每空 $3$ 分, 共 $24$ 分)}
\begin{enumerate}
	\item 微分方程 $\frac{\dd y}{\dd x}=-3 y+\ee^{2 x}$ 的通解是 $y=$\underline{\hspace{8pc}}
	
	\item 平行于 $y$ 轴且通过曲线 $\begin{cases}
	x^{2}+y^{2}+4 z^{2}=1\\
	x^{2}=y^{2}+z^{2}
	\end{cases}$ 的柱面方程是\underline{\hspace{8pc}}
	
	\item 设 $z=x^{2} y+x y^{2}$ , 则 $\dd z=$\underline{\hspace{8pc}}
	
	\item $\iint_{D} y^{2} \sin ^{3} x \dd x \dd y=$\underline{\hspace{8pc}}(区域 $D$ 为: $-4 \leq x \leq 4,-1 \leq y \leq 1$ )
	
	\item 设 $D$ 为平面闭区域: $x^{2}+y^{2} \leq 1$ , 则 $\iint_{D} \sqrt{x^{2}+y^{2}} \dd x \dd y$ 化为极坐标系下二次积分的表达式为\underline{\hspace{8pc}}
	
	\item 设 $L$ 是任意一条分段光滑的有向闭曲线, 则 $\oint_{L} 2 x y \dd x+x^{2} \dd y=$\underline{\hspace{8pc}}
	
	\item $I=\iint_{\Sigma}(x+z \sin y) \dd y \dd z+(y+x \sin z) \dd z \dd x+z \dd x \dd y=$\underline{\hspace{8pc}}, 其中 $\Sigma$ 为球面 $x^{2}+y^{2}+z^{2}=4(z \geq 0)$ 与平面 $z=0$ 围成区域的表面, 取外侧.
	
	\item 级数 $\sum_{n=1}^{\infty}(-1)^{n} \frac{1}{n} x^{n}$ 的收敛半径为\underline{\hspace{8pc}}
\end{enumerate}

\subsubsection{综合题(请写出求解过程, $8$ 小题, 共 $52$ 分)}
\begin{enumerate}
	\item 求过点 $(2,1,1)$ , 且与直线 $\begin{cases}
	x-y+3 z-7=0\\
	3 x+5 y-2 z+1=0
	\end{cases}$ 垂直的平面方程. ( $6$ 分)
	
	\item 设 $z=f\left(\ee^{x+y}, \sin (x y)\right)$ , 且 $f$ 具有一阶连续偏导数, 求 $\frac{\partial z}{\partial x}, \frac{\partial z}{\partial y}$ . ( $6$ 分)
	
	\item 计算 $\iint_{D}\left(x^{2}+y\right) \dd x \dd y$ , $D$ 是曲线 $y=x^{2}, x=y^{2}$ 围成的闭区域. ( $8$ 分)
	
	\item 计算 $\iiint_{\Omega}\left(x^{2}+y^{2}\right) \dd x \dd y \dd z$ , 其中 $\Omega$ 是由圆锥面 $z^{2}=x^{2}+y^{2}$ 及平面 $z=2$ 围成的闭区域. ( $6$ 分)
	
	\item 计算 $\int_{\Gamma} x^{3} \dd x+3 z y^{2} \dd y-x^{2} y \dd z$ , 其中 $\Gamma$ 是从点 $A(2,2,1)$ 到原点 $O$ 的直线段 $AO$ . ( $6$ 分)
	
	\item 空间区域 $\Omega$ 由开口向下的旋转抛物面 $z=1-x^{2}-y^{2}$ 与平面 $z=0$ 所围, $\Omega$ 的表面取外侧为 $\Sigma$ , 利用高斯公式计算 $\oiint_{\Sigma} x^{2} y z^{2} \dd y \dd z-x y^{2} z^{2} \dd z \dd x+z(1+x y z) \dd x \dd y$ . ( $8$ 分)
	
	\item 判断级数 $\sum_{n=1}^{\infty} \frac{n^{\ee}}{\ee^{n}}$ 的敛散性. ( $6$ 分)
	
	\item 求幂级数 $\sum_{n=0}^{\infty}(2 n+1) x^{2 n}(x \in(-1,1))$ 的和函数. ( $6$ 分)
\end{enumerate}


\subsection{复习题 1 答案}

\subsection{复习题 2}
\subsubsection{选择题(每小题 $3$ 分, 共 $24$ 分)}
\begin{enumerate}
	\item 微分方程 $y''-6 y'+9 y=\left(6 x^{2}+2\right) \ee^{x}$ 的待定特解的一个形式可为 (\hspace{1pc})
	\twoch{$y^{*}=\left(a x^{2}+b x+c\right) \ee^{x}$}{$y^{*}=x\left(a x^{2}+b x+c\right) \ee^{x}$}{$y^{*}=x^{2}\left(a x^{2}+b x+c\right) \ee^{x}$}{$y^{*}=x^{2}\left(x^{2}+1\right) \ee^{x}$}
	
	\item 设向量 $\vec{a}$ 的三个方向角为 $\alpha$ 、 $\beta$ 、 $\gamma$ , 且已知 $\alpha=60^{\circ}$ 、 $\beta=120^{\circ}$ , 则 $\gamma=$ (\hspace{1pc})
	\fourch{$120^{\circ}$}{$60^{\circ}$}{$45^{\circ}$}{$30^{\circ}$}
	
	\item 设 $z=\arctan \ee^{x y}$ , 则 $\frac{\partial z}{\partial y}=$ (\hspace{1pc})
	\fourch{$-\frac{x \ee^{x y}}{\sqrt{1-\ee^{2 x y}}}$}{$\frac{x \ee^{x y}}{\sqrt{1-\ee^{2 x y}}}$}{$-\frac{x \ee^{x y}}{1+\ee^{2 x y}}$}{$\frac{x \ee^{x y}}{1+\ee^{2 x y}}$}
	
	\item $D$ 为平面区域 $x^{2}+y^{2} \leq 4$ , 利用二重积分的性质, $\iint_{D}\left(x^{2}+4 y^{2}+9\right) \dd x \dd y$ 的最佳估值区间为 (\hspace{1pc})
	\fourch{$[36 \uppi, 52 \uppi]$}{$[36 \uppi, 100 \uppi]$}{$[52 \uppi, 100 \uppi]$}{$[9 \uppi, 25 \uppi]$}
	
	\item 设 $\Omega=\left\{(x, y, z) | x^{2}+y^{2}+z^{2} \leq 2, x \geq 0\right\}$ , 则以下等式错误的是 (\hspace{1pc})
	\fourch{$\iiint_{\Omega} x^{2} y \dd v=0$}{$\iiint_{\Omega}(x+y) \dd v=0$}{$\iiint_{\Omega} z \dd v=0$}{$\iiint_{\Omega} x y \dd v=0$}
	
	\item 设 $L$ 为直线 $y=y_0$ 上从点 $A(0,y_0)$ 到点 $B(3,y_0)$ 的有向直线段, 则 $\int_{L} 2 \dd y=$ (\hspace{1pc})
	\fourch{$6$}{$6y_0$}{$3y_0$}{$0$}
	
	\item $\Sigma$ 为平面 $x+y+z=1$ 与三坐标面所围区域表面的外侧, 则 $\iint_{\Sigma}(2 y+3 z) \dd y \dd z+(x+2 z) \dd z \dd x+(y+1) \dd x \dd y=$ (\hspace{1pc})
	\fourch{$0$}{$\frac{1}{6}$}{$\frac{2}{3}$}{$\frac{5}{3}$}
	
	\item 交错级数 $\sum_{n=1}^{\infty}(-1)^{n-1} \frac{1}{3^{n-1}}$ (\hspace{1pc})
	\fourch{发散}{条件收敛}{绝对收敛}{无法确定}
\end{enumerate}

\subsubsection{填空题(每空 $3$ 分,共 $24$ 分)}
\begin{enumerate}
	\item 以 $y_{1}=\ee^{x}, y_{2}=x \ee^{x}$ 为特解的阶数最低的常系数齐次线性微分方程是\underline{\hspace{8pc}}
	
	\item 直线 $L:\begin{cases}
	x=3t-2\\
	y=t+2\\
	z=2t-1
	\end{cases}$ 和平面 $\pi:\ 2 x+3 y+3 z-8=0$ 的交点是\underline{\hspace{8pc}}
	
	\item 设 $z=xy^3$ , 则 $\dd z=$\underline{\hspace{8pc}}
	
	\item 交换二次积分的积分次序后, $\int_{0}^{2} \dd y \int_{y^{2}}^{2 y} f(x, y) \dd x=$\underline{\hspace{8pc}}
	
	\item 设 $\Omega=\{-1 \leq x \leq 1,-1 \leq y \leq 3,0 \leq z \leq 2\}$ , 则 $\iiint_{\Omega} \dd x \dd y \dd z=$\underline{\hspace{8pc}}
	
	\item 设 $L$ 为由三点 $(0,0),(3,0),(3,2)$ 围成的平面区域 $D$ 的正向边界曲线, 由格林公式知 $\int_{L}(3 x-y+4) \dd x+(5 y+3 x-6) \dd y=$\underline{\hspace{8pc}}
	
	\item 设 $\Sigma$ 是上半圆锥面 $z=\sqrt{x^{2}+y^{2}}(0 \leq z \leq 1)$ , 则曲面积分 $\iint_{\Sigma}\left(x^2+y^2\right)\dd S=$\underline{\hspace{8pc}}
	
	\item 级数 $\sum_{n=1}^{\infty}\left(\frac{1}{n(n+1)}-\frac{1}{2^{n}}\right)$ 的和为\underline{\hspace{8pc}}
\end{enumerate}

\subsubsection{综合题( $8$ 小题, 共 $52$ 分)}
\begin{enumerate}
	\item 求方程 $\frac{\dd y}{\dd x}=\frac{x y}{1+x^{2}}$ 的通解. ( $6$ 分)
	
	\item 设 $z=\ln \left(x^{2}-y\right)$ , 而 $y=\tan x$ , 求 $\frac{\dd z}{\dd x}$ . ( $6$ 分)
	
	\item 计算 $\iint_{D}\left(x^{2}+y^{2}\right) \dd x \dd y$ , $D$ 为曲线 $x^{2}-2 x+y^{2}=0, y=0$ 围成的在第一象限的闭区域. ( $6$ 分)
	
	\item 计算三重积分 $\iiint_{\Omega} z \dd x \dd y \dd z$ , 其中 $\Omega$ 是由圆锥面 $z=\sqrt{x^{2}+y^{2}}$ 与球面 $z=\sqrt{2-x^{2}-y^{2}}$ 围成的区域. ( $6$ 分)
	
	\item 用高斯公式计算 $\oiint_{\Sigma}\left(a^{2} x+x^{3}\right) \dd y \dd z+y^{3} \dd z \dd x+z^{3} \dd x \dd y$ , 其中 $\Sigma$ 为球面 $x^{2}+y^{2}+z^{2}=a^{2}$ , 取外侧. ( $8$ 分)
	
	\item 用格林公式计算 $\oint_{C} x^{2} y \dd x-x y^{2} \dd y$ , 其中 $C$ 为圆周 $x^2+y^2=4$ , 取正向. ( $8$ 分)
	
	\item 判断级数 $\sum_{n=1}^{\infty} \frac{1}{2^{n-1}(2 n-1)}$ 的敛散性. ( $6$ 分)
	
	\item 在区间 $(-1,1)$ 内求幂级数 $\sum_{n=1}^{\infty} \frac{x^{n}}{n}$ 的和函数 $s(x)$ . ( $6$ 分)
\end{enumerate}


\subsection{复习题 2 答案}

\subsection{复习题 3}
\subsubsection{选择题}
miu抄下来
\subsubsection{填空题}
\begin{enumerate}
	\item 微分方程 $y'=3y$ 的通解是 $y=$\underline{\hspace{8pc}}
	
	\item 已知向量 $\vec{a}$ 与 $\vec{b}$ 方向相反, 且 $\left|\vec{b}\right|=3|\vec{a}|$ , 则 $\vec{b}$ 由 $\vec{a}$ 表示为 $\vec{b}=$\underline{\hspace{8pc}}
	
	\item 设 $z=x^3 y+\sin(x+y)$ , 则 $\frac{\partial z}{\partial x}=$\underline{\hspace{8pc}}
	
	\item 交换二次积分的积分次序后, $\int_0^1 \dd x\int_{x^3}^{x} f(x,y)\dd y=$\underline{\hspace{8pc}}
	
	\item 设 $\Omega=\left\{ 0\leq x\leq 3, 0\leq y\leq \frac{\uppi}{2}, 0\leq z\leq 2 \right\}$ , 则 $\iiint_{\Omega}xz\sin y\dd x\dd y\dd z=$\underline{\hspace{8pc}}
	
	\item $L$ 为曲线 $y=x^2$ 上从点 $(0,0)$ 到点 $(1,1)$ 的一段弧, 则 $\int_{L}\sqrt{y}\dd s=$\underline{\hspace{8pc}}
	
	\item $L$ 为圆周 $x^2+y^2=a^2$ 的正向边界曲线, 由格林公式知 $\oint_{L} \left( x^2y\cos x+2xy\sin x-y^2 \ee^x \right)\dd x + \left( x^2\sin x-2y\ee^x+x \right)\dd y=$\underline{\hspace{8pc}}
	
	\item 若 $p$ 满足条件\underline{\hspace{8pc}}, 则级数 $\sum_{n=1}^{+\infty}\frac{1}{n^{p-3}}$ 一定发散.
\end{enumerate}

\subsubsection{综合题}
\begin{enumerate}
	\item 求过点 $(3,1,3)$ , 且与直线 $
	\begin{cases}
		x-2y+2z-7=0\\
		x+5y-z+1=0
	\end{cases}
	$ 垂直的平面方程.

	\item 已知 $z=\ee^{2x-y}$ , 而 $x=\sin t$ , $y=t^2$ , 求 $\frac{\dd z}{\dd t}$ .
	
	\item 计算 $\iint_{D} x\ee^{-y^2}\dd x\dd y$ , $D$ 是曲线 $x=0$ , $y=x^2$ , $y=\sqrt{2}$ 围成的在第一象限的闭区域.
	
	\item 计算以 $xoy$ 面上的圆周 $x^2+y^2=4$ 围成的闭区域为底, 以旋转抛物面 $z=x^2+y^2$ 为顶的曲顶柱体的体积.
	
	\item 计算曲面积分 $\iint_{\Sigma}\left( x^2+y^2 \right)\dd S$ , 其中 $\Sigma$ 为锥面 $z=\sqrt{x^2+y^2}$ 被平面 $z=2$ 截下的带锥顶部分.
	
	\item 利用高斯公式计算 $I=\iint_{\Sigma}\left( x^3 z+2x \right)\dd y\dd z-x^2 yz\dd z\dd x-x^2 z^2\dd x\dd y$ , 其中 $\Sigma$ 是由开口向下的旋转抛物面 $z=2-x^2-y^2$ 与平面 $z=1$ 围成立体 $\Omega$ 的表面, 取外侧.
	
	\item 判断级数 $\sum_{n=1}^{+\infty}\frac{2\cdot 4\cdot 6\cdot\cdots\cdot (2n)}{1\cdot 4\cdot 7\cdot\cdots\cdot (3n-2)}$ 的敛散性.
	
	\item 求级数 $\sum_{n=1}^{+\infty}nx^{n+1}(x\in(-1,1))$ 的和函数.
\end{enumerate}





\subsection{难度与考试近似的题}
\subsubsection{选择题}
\begin{enumerate}
	\item 微分方程 $y'=p(x) y$ 的通解是 (\hspace{1pc})
	\fourch{$y=\ee^{\int p(x) \dd x}$}{$y=C \ee^{\int-p(x) \dd x}$}{$y=C \ee^{\int p(x) \dd x}$}{$y=C p(x)$}
	
	\item 已知曲线 $\begin{cases}
	x^{2}+y^{2}+z^{2}=2\\
	x+y+z=a
	\end{cases}$ 在 $yoz$ 坐标面上的投影曲线为 $\begin{cases}
		y^{2}+y z+z^{2}=1\\
		x=0
	\end{cases}$ , 则 $a=$ (\hspace{1pc})
	\fourch{$-1$}{$0$}{$1$}{$2$}
	
	\item 设 $z=\ee^{y} \tan x$ , 则 $\dd z=$ (\hspace{1pc})
	\twoch{$\ee^{y} \tan x \dd x+\ee^{y} \sec ^{2} x \dd y$}{$\frac{\ee^{y}}{1+x^{2}} \dd x+\ee^{y} \tan x \dd y$}{$\ee^{x} \tan y \dd x+\ee^{x} \sec ^{2} y \dd y$}{$\ee^{y} \sec ^{2} x \dd x+\ee^{y} \tan x \dd y$}
	
	\item 设积分区域 $D : x^{2}+y^{2} \leq 4$ , 则二重积分 $\iint_{D} \sqrt{x^{2}+y^{2}} \dd x \dd y=$ (\hspace{1pc})
	\fourch{$\int_{0}^{2 \uppi} \dd \theta \int_{0}^{2} \rho^{2} \dd \rho$}{$\int_{0}^{2 \uppi} \dd \theta \int_{\rho}^{4} \dd \rho$}{$\int_{0}^{2 \uppi} \dd \theta \int_{0}^{1} \rho^{2} \dd \rho$}{$\int_{0}^{2 \uppi} \dd \theta \int_{1}^{2} \rho \dd \rho$}
	
	\item 设 $\Omega$ 由圆锥面 $z=1-\sqrt{x^{2}+y^{2}}$ 与平面 $z=0$ 围成的闭区域, 则 $\iiint_{\Omega} z \dd v=$ (\hspace{1pc})
	\twoch{$\int_{0}^{\uppi} \dd \theta \int_{0}^{1} \rho \dd \rho \int_{0}^{1-\rho} z \dd z$}{$\int_{0}^{2 \uppi} \dd \theta \int_{0}^{1} \dd \rho \int_{0}^{1-\rho} z \dd z$}{$\int_{0}^{\uppi} \dd \theta \int_{0}^{1} \dd \rho \int_{0}^{1-\rho} z \dd z$}{$\int_{0}^{2 \uppi} \dd \theta \int_{0}^{1} \rho \dd \rho \int_{0}^{1-\rho} z \dd z$}
	
	\item 设 $L$ 为圆周 $\begin{cases}
	x=a \cos t\\
	y=a \sin t
	\end{cases}(0\leq t\leq 2\uppi)$ , 则 $\oint_{L}\left(x^{2}+y^{2}\right) \dd s=$ (\hspace{1pc})
	\fourch{$a^3$}{$\uppi a^3$}{$2\uppi a^3$}{$3\uppi a^3$}
	
	\item $L$ 为平面闭区域: $-1 \leq x \leq 1,0 \leq y \leq 1$ 的正向边界, 则 $\int_{L}\left(\frac{1}{2} y+3 x \ee^{x}\right) \dd x-\left(\frac{1}{2} x-y \sin y\right) \dd y=$ (\hspace{1pc})
	\fourch{$-2$}{$2$}{$-1$}{$1$}
	
	\item 设幂级数 $\sum_{n=1}^{\infty} a_{n} x^{n}$ 的收敛半径为 $R(0<R<+\infty)$ , 则幂级数 $\sum_{n=1}^{\infty} a_{n}\left(\frac{x}{2}\right)^{n}$ 的收敛半径为 (\hspace{1pc})
	\fourch{$\frac{R}{2}$}{$2R$}{$R$}{$\frac{2}{R}$}
	
	\item 微分方程 $\frac{\dd^{2} y}{\dd x^{2}}-3 \frac{\dd y}{\dd x}+2 y=x \ee^{3 x}$ 的待定特解 $y^{*}$ 的一个形式是 (\hspace{1pc})
	\twoch{$y^{*}=(a x+b)+c \ee^{3 x}$}{$y^{*}=(a x+b)+c x \ee^{3 x}$}{$y^{*}=(a x+b) \ee^{3 x}$}{$y^{*}=(a x+b) x \ee^{3 x}$}
	
	\item 过点 $(3,2,-7)$  且在三坐标轴上的截距相等, 则此平面方程是 (\hspace{1pc})
	\fourch{$x+y+z+2=0$}{$z+y+z-2=0$}{$x-y+z-2=0$}{$x-y-z-2=0$}
	
	\item 设 $L$ 是平面有向曲线, 下列曲线积分中, (\hspace{1pc}) 是与路径无关的
	\twoch{$\int_{L}\left(y \ee^{x}+x^{2}-y\right) \dd x+\left(x+\ee^{x}-2 y^{2}\right) \dd y$}{$\int_{L}(\cos x+y) \dd x+(x+\cos y) \dd y$}{$\int_{L}(\cos x-y) \dd x+(x+\cos y) \dd y$}{$\int_{L}\left(\frac{1}{2} y+3 x \ee^{x}\right) \dd x-\left(\frac{1}{2} x-y \sin y\right) \dd y$}
	
	\item 设 $\Sigma$ 是平面 $x=1, y=1, z=1$ 与三个坐标面围成区域的表面, 取外侧, 则曲面积分 $\iint_{\Sigma} 2 x \dd y \dd z+2 z \dd z \dd x+3 y \dd x \dd y=$ (\hspace{1pc})
	\fourch{$0$}{$2$}{$4$}{$7$}
	
	\item 级数 $1+\left(\frac{1}{2}\right)^{2}+\left(\frac{1}{3}\right)^{2}+\cdots+\left(\frac{1}{n}\right)^{2}+\cdots$ 是 (\hspace{1pc})
	\fourch{幂级数}{调和级数}{$p$ 级数}{等比级数}
	
	\item 方程 $\left(3 x^{2}+y \cos x\right) \dd x+\left(\sin x-4 y^{3}\right) \dd y=0$ 是 (\hspace{1pc})
	\twoch{可分离变量微分方程}{一阶线性方程}{全微分方程}{$(\mathrm{A})$ 、 $(\mathrm{B})$ 、 $(\mathrm{C})$ 均不对}
	
	\item $z=f(x, y)$ 在 $\left(x_{0}, y_{0}\right)$ 可微, 则 $\frac{\partial z}{\partial x}, \frac{\partial z}{\partial y}$ 在 $\left(x_{0}, y_{0}\right)$ (\hspace{1pc})
	\fourch{连续}{不连续}{不一定存在}{一定存在}
	
	\item 级数 $\sum_{n=2}^{\infty}\left(\frac{1}{\sqrt{n}-1}-\frac{1}{\sqrt{n}+1}\right)$ 是 (\hspace{1pc})
	\fourch{发散}{收敛}{条件收敛}{绝对收敛}
	
	\item 曲面 $z=\sqrt{x^{2}+y^{2}}$ 与平面 $z=1$ 所围立体的体积为 (\hspace{1pc})
	\twoch{$\iiint_{\Omega}\left(x^{2}+y^{2}\right) \dd v$}{$\int_{0}^{2 \pi} \dd \theta \int_{0}^{1} r \dd r \int_{r}^{1} \dd z$}{$\int_{-1}^{1} \dd x \int_{-\sqrt{1-x^{2}}}^{\sqrt{1-x^{2}}} \dd y \int_{0}^{x^{2}+y^{2}} \dd z$}{$\int_{0}^{2 \pi} \dd \theta \int_{0}^{1} r \dd r \int_{0}^{1} \dd z$}
	
	\item 方程 $y''-3 y'+2 y=3 x-\ee^{x}$ 的特解形式为 (\hspace{1pc})
	\fourch{$(a x+b) \ee^{x}$}{$a x+b+c x \ee^{x}$}{$a x+b+c \ee^{x}$}{$(a x+b) x \ee^{x}$}
	
	\item 设 $\overrightarrow{AB}$ 与 $u$ 轴的夹角为 $\frac{\uppi}{3}$ , 则 $\overrightarrow{AB}$ 在 $u$ 轴上的投影是 (\hspace{1pc})
	\fourch{$\overrightarrow{AB}\cos\frac{\uppi}{3}$}{$\overrightarrow{AB}\sin\frac{\uppi}{3}$}{$\left|\overrightarrow{AB}\right|\cos\frac{\uppi}{3}$}{$\left|\overrightarrow{AB}\right|\sin\frac{\uppi}{3}$}
	
	\item 过点 $M_{1}(3,-2,1), M_{2}(-1,0,2)$ 的直线方程是 (\hspace{1pc})
	\twoch{$-4(x-3)+2(y+2)+(z-1)=0$}{$\frac{x-3}{4}=\frac{y+2}{2}=\frac{z-1}{1}$}{$\frac{x+1}{4}=\frac{y}{2}=\frac{z-2}{1}$}{$\frac{x-3}{4}=\frac{y+2}{-2}=\frac{z-1}{-1}$}
	
	\item 直线 $\begin{cases}
	x+y+3 z=0\\
	x-y-z=0
	\end{cases}$ 与平面 $x-y-z+1=0$ 的夹角是 (\hspace{1pc})
	\fourch{$60^\circ$}{$0^\circ$}{$30^\circ$}{$90^\circ$}
	
	\item 设 $f(x,y)=\begin{cases}
	\frac{1}{xy}\sin\left(x^2y\right), & \text{当}xy\ne0,\\
	0, & \text{当}xy=0,
	\end{cases}$ 则当 $y\ne0$ 时, $f_{x}(0,y)=$ (\hspace{1pc})
	\fourch{$0$}{$1$}{$2$}{不存在}
	
	\item 曲线 $\begin{cases}
	z=\frac{x^2}{2}+\frac{y^2}{4},\\
	y=2.
	\end{cases}$ 在点 $\left( 1,2,\frac{3}{2} \right)$ 处的切线与 $x$ 轴的正向所成的倾角是 (\hspace{1pc})
	\fourch{$\arctan 1$}{$30^\circ$}{$60^\circ$}{$90^\circ$}
	
	\item $D$ 是矩形闭区域 $0 \leq x \leq 1, \quad 0 \leq y \leq 2 \quad, I=\iint_{D}(x+y+1) \dd x \dd y$ , 利用二重积分的性质, $I$ 的最佳估计区间为 (\hspace{1pc})
	\fourch{$[0,1]$}{$[0,2]$}{$[1,3]$}{$2,8$}
	
	\item $L$ 为 $y=x^2$ 上从 $A(1,1)$ 到 $B(0,0)$ 的一段弧, 则 $\int_{L}x\dd y=$ (\hspace{1pc})
	\fourch{$\int_{0}^{1} 2 x^{2} \dd x$}{$\int_{1}^{0} x \dd y$}{$\int_{1}^{0} 2 x^{2} \dd x$}{$\int_{0}^{1} \sqrt{y} \dd y$}

	\item 当 $\sum_{n=1}^{\infty}\left(a_{n}+b_{n}\right)$ 收敛时, $\sum_{n=1}^{\infty} a_{n}$ 与 $\sum_{n=1}^{\infty} b_{n}$ (\hspace{1pc})
	\fourch{可能不同时收敛}{不可能同时收敛}{必同时收敛}{必同时发散}
	
	\item 非齐次线性微分方程 $x''-2 x'+5 x=t \ee^{t} \sin 2 t$ 的特解形式 $x^*=$ (\hspace{1pc})
	\twoch{$(A t+B) \ee^{t} \sin 2 t$}{$\ee^{t}[(A t+B) \cos 2 t+(C t+D) \sin 2 t]$}{$t(A t+B) \ee^{t} \sin 2 t$}{$t \ee^{t}[(A t+B) \cos 2 t+(C t+D) \sin 2 t]$}
	
	\item 设向量 $\vec{a}=(1,2,3)$ 、 $\vec{b}=(2,0,1)$ , 则向量 $\vec{a}\times\vec{b}$ 在 $y$ 轴上的分向量为 (\hspace{1pc})
	\fourch{$5$}{$5\vec{j}$}{$-5$}{$-5\vec{j}$}
	
	\item 两向量 $\vec{a}$ 、 $\vec{b}$ 平行的充要条件是 (\hspace{1pc})
	\fourch{$\vec{a}\cdot\vec{b}=0$}{$\vec{a}\times\vec{b}=0$}{$\vec{a}\cdot\vec{b}=\vec{0}$}{$\vec{a}\times\vec{b}=\vec{0}$}
	
	\item $f(x,y)$ 在点 $(x_0,y_0)$ 处两个偏导数存在是 $f(x,y)$ 在 $(x_0,y_0)$ 处可微的 (\hspace{1pc})
	\fourch{必要条件}{充分条件}{充分必要条件}{以上都不是}
	
	\item 设上半球 $V=\left\{(x, y, z) | x^{2}+y^{2}+z^{2} \leq 1, z \geq 0\right\}$ , 则以下等式错误的是 (\hspace{1pc})
	\fourch{$\iiint_{V} x \dd V=0$}{$\iiint_{V} y \dd V=0$}{$\iiint_{V} z \dd V=0$}{$\iiint_{V} xy \dd V=0$}
	
	\item 设 $f(x)=\begin{cases}
	x, & x\in[-\uppi,0)\\
	1, & x\in[0,\uppi)
	\end{cases}$ 的傅里叶级数的和函数为 $S(x)$ , 则 $S(0)=$ (\hspace{1pc})
	\fourch{$0$}{$1$}{$\frac{1}{2}$}{$-\frac{1}{2}$}
	
	\item 级数 $\sum_{n=1}^{\infty}(-1)^{n} \frac{1}{n}$ (\hspace{1pc})
	\fourch{发散}{条件收敛}{绝对收敛}{以上都不对}
\end{enumerate}

\subsubsection{填空题}
\begin{enumerate}
	\item 以 $\ee^x,x\ee^x$ 为解的阶数最低的常系数线性齐次微分方程是\underline{\hspace{8pc}}
	
	\item 过点 $A(1,-2,1)$ 且以 $\vec{n}=(1,2,3)$ 为法向量的平面方程是\underline{\hspace{8pc}}
	
	\item 设 $z=\sin \left(x^{2}+y\right)$ , 则 $\frac{\partial^{2} z}{\partial x \partial y}=$\underline{\hspace{8pc}}
	
	\item 设 $D$ 是圆环形闭区域 $1 \leq x^{2}+y^{2} \leq 4$ , 那么 $\iint_{D} \sqrt{x^{2}+y^{2}} \dd \sigma=$\underline{\hspace{8pc}}
	
	\item 设 $\Omega$ 为球体: $x^{2}+y^{2}+z^{2} \leq 4$ , 则 $\iiint_{\Omega} x^{2} \sin (y z) \dd x \dd y \dd z=$\underline{\hspace{8pc}}
	
	\item $L$ 为抛物线 $x=y^2$ 上从点 $(1,-1)$ 到 $(1,1)$ 的一段弧, 则 $\int_{L} x y \dd y=$\underline{\hspace{8pc}}
	
	\item $\oiint_{\Sigma}(x y+z) \dd x \dd y+(x z+y) \dd x \dd z+(x+y z) \dd y \dd z=$\underline{\hspace{8pc}}, 其中 $\Sigma$ 是由六张平面 $x=1,x=2,y=1,y=2,z=1,z=3$ 围成的六面体的表面, 取内侧
	
	\item 级数 $\frac{1}{3}+\frac{1}{\sqrt{3}}+\frac{1}{\sqrt[3]{3}}+\cdots+\frac{1}{\sqrt[n]{3}}+\cdots$ 是\underline{\hspace{8pc}}(填收敛或发散)
	
	\item 微分方程 $y'=p(x) y$ 的通解是 $y=$\underline{\hspace{8pc}}
	
	\item 设 $\vec{a}$ 与轴 $\vec{l}$ 的夹角为 $\frac{\uppi}{6}$ , 且 $|\vec{a}|=4$ , 则 $\mathrm{Prj}_{\vec{l}} \vec{a}=$\underline{\hspace{8pc}}
	
	\item 设 $f(x, y)=\tan \left(x y^{2}\right)$ , 则 $f_{x}(0,2)=$\underline{\hspace{8pc}}
	
	\item  交换二次积分次序的积分次序后, $\int_{1}^{2} \dd x \int_{2-x}^{\sqrt{2 x-x^{2}}} f(x, y) \dd y=$\underline{\hspace{8pc}}
	
	\item  已知 $\Omega$ 是由旋转抛物面 $z=x^{2}+y^{2}$ 与上半球面 $z=\sqrt{2-x^{2}-y^{2}}$ 围成的区域, 则 $\iiint_{\Omega} x y z \dd x \dd y \dd z=$\underline{\hspace{8pc}}
	
	\item  设 $\Sigma$ 是球面 $x^{2}+y^{2}+z^{2}=1$ , 则 $\iint_{\Sigma}\left(x^{2}+y^{2}+z^{2}\right) \dd S=$\underline{\hspace{8pc}}
	
	\item 积分 $\oint_{L}\left(x^{2}-y\right) \dd x+\left(y^{2}+x\right) \dd y=$\underline{\hspace{8pc}}, 其中 $L$ 为圆周 $(x-1)^{2}+y^{2}=a^{2}$ 的正向
	
	\item 级数 $\sum_{n=1}^{\infty}(-1)^{n-1} \frac{1}{\sqrt{n}}$ 是\underline{\hspace{8pc}}收敛(填条件收敛或绝对收敛)
	
	\item 设 $z=x^{y}$ , 则 $\frac{\partial z}{\partial y}=$\underline{\hspace{8pc}}
	
	\item 积分 $\iint_{D} x y \dd x \dd y=$\underline{\hspace{8pc}}, 其中 $D$ 为 $0 \leq x \leq 2, 0 \leq y \leq 4$ .
	
	\item $L$ 为 $y=x^2$ 点 $(0,0)$ 到 $(1,1)$ 的一段弧, 则 $\int_{L} \sqrt{y} \dd s=$\underline{\hspace{8pc}}
	
	\item 级数 $\sum_{n=1}^{\infty} \frac{(-1)^{n}}{n^{p}}$ 当 $p$ 满足\underline{\hspace{8pc}}时条件收敛.
	
	\item 方程 $y \ee^{x} \dd x-\left(1+\ee^{x}\right) \dd y=0$ 的通解为\underline{\hspace{8pc}}
	
	\item 设 $z=\ln \sqrt{1+x^{2}+y^{2}}$ , 则 $\dd\left.z\right|_{(1,1)}=$\underline{\hspace{8pc}}
	
	\item 函数 $z=x^{2}+y^{2}$ 在点 $P(1,2)$ 沿从点 $(1,2)$ 到点 $(2,2+\sqrt{3})$ 的方向上的方向导数为\underline{\hspace{8pc}}
	
	\item 改换二次积分的积分次序: $\int_{0}^{1} \dd y \int_{0}^{y} f(x, y) \dd x=$\underline{\hspace{8pc}}
	
	\item 平面 $x+y+z=1$ 含在圆柱面 $x^{2}+y^{2}=2 x$ 内部的那部分平面面积为\underline{\hspace{8pc}}
	
	\item $L$ 为圆周 $x^{2}+y^{2}=1$ , 则 $\int_{L}\left(x^{2}+y^{2}\right) \dd s=$\underline{\hspace{8pc}}
	
	\item $\Sigma$ 是 $xoy$ 平面上的圆域: $x^{2}+y^{2} \leq 1$ , 取下侧, 则 $\iint_{\Sigma} \dd x \dd y=$\underline{\hspace{8pc}}
	
	\item 级数 $\sum_{n=1}^{\infty} \frac{3^{n}+4^{n}}{7^{n}}$ 的和为\underline{\hspace{8pc}}
	
	\item $\ee^{x^{2}}$ 的 $x$ 的幂级数展开式为\underline{\hspace{8pc}}
	
	\item 微分方程 $x'''-2x''-x'+2x=0$ 的通解是\underline{\hspace{8pc}}

	\item 过点 $(1,0,1)$ 及以 $(1,2,3)$ 为方向向量的直线的对称式方程为\underline{\hspace{8pc}}
	
	\item 函数 $z=x^y$ 的全微分 $\dd z=$\underline{\hspace{8pc}}
	
	\item 二元函数 $u=x^{2}-x y+y^{2}$ 在点 $(-1,1)$ 处沿方向\underline{\hspace{8pc}}的方向导数最大
	
	\item 交换二次积分的次序 $\int_0^1 \dd y\int_{-1}^{-y} f(x,y)\dd x=$\underline{\hspace{8pc}}
	
	\item 若 $L$ 为抛物线 $y^2=2x$ 上介于 $(2,-2)$ 与 $(2,2)$ 两点间的曲线段, 则 $\int_{L} y\dd s=$\underline{\hspace{8pc}}
	
	\item 若 $\Sigma$ 是曲面 $x^2+y^2+z^2=1$ , 则 $\iint_{\Sigma} \dd S=$\underline{\hspace{8pc}}
	
	\item 函数 $f(x)=3^x$ 的幂级数展开式为\underline{\hspace{8pc}}
\end{enumerate}

\subsubsection{综合题}
\begin{enumerate}
	\item 求过点 $(2,0,-3)$ , 且过直线 $\begin{cases}
	x-2 y+4 z-7=0\\
	3 x+5 y-2 z+1=0
	\end{cases}$ 垂直的平面方程. ( $6$ 分)
	
	\item 设 $z=x^{y}(x>0)$ , 求 $\frac{\partial z}{\partial x}, \frac{\partial^{2} z}{\partial x \partial y}$ . ( $6$ 分)
	
	\item 计算 $\iint_{D} x^{2} y^{2} \dd x \dd y$ , 其中 $D=\{(x, y) | 0 \leq x \leq 1,0 \leq y \leq 1\}$ . ( $6$ 分)
	
	\item 计算 $I=\iiint_{\Omega}\left(x^{2}+y^{2}\right) \dd v$ , 其中 $\Omega$ 为旋转抛物面 $z=x^{2}+y^{2}$ 与平面 $z=4$ 所围成的区域. ( $6$ 分)
	
	\item $L$ 是圆环区域 $D : 1 \leq x^{2}+y^{2} \leq 4$ 的正向边界曲线, 计算曲线积分 $\oint_{L} \sqrt{x^{2}+y^{2}} \dd x+\left[x y^{2}+y \ln \left(x+\sqrt{x^{2}+y^{2}}\right)\right] \dd y$ . ( $8$ 分)
	
	\item 计算 $\iint_{\Sigma} \frac{2}{z} \dd S$ , 其中 $\Sigma$ 是球面 $x^{2}+y^{2}+z^{2}=1$ 在平面 $z=\frac{1}{2}$ 上方的部分. ( $8$ 分)
	
	\item 判断级数 $\sum_{n=1}^{\infty} \frac{3^{n}}{n \cdot 2^{n}}$ 的敛散性. ( $6$ 分)
	
	\item 求幂级数 $\sum_{n=0}^{\infty}(n+1) x^{n}$ 在收敛域 $(-1,1)$ 的和函数 $s(x)$ . ( $6$ 分)
	
	\item 求过点 $(3,-2,1)$ ,  且与直线 $\frac{x-1}{1}=\frac{y+1}{1}=\frac{z-2}{3}$ 平行的直线方程. ( $6$ 分)
	
	\item 设 $ z = e^{xy} + \cos(x + y)$ , 求 $\dd z$ . ( $6$ 分)
	
	\item 计算 $\iint_{D}\frac{y}{x}\dd x\dd y$ , $D$ 是由直线 $ y = 2x,y = x, x = 2, x = 4$ 围成的闭区域. ( $6$ 分)
	
	\item 计算 $\iiint_{\Omega}z\dd x\dd y\dd z$ , 其中 $\Omega$ 由平面 $z = 3$ 与旋转抛物面 $x^2 + y^2 = 3z$ 围成的区域. ( $6$ 分)
	
	\item 计算 $\int_{L} 2 x y \dd x+x^{2} \dd y$ , $L$  为抛物线 $y=x^{2}$ 上从 $O(0,0)$ 到 $B(1,1)$ 的一段弧. ( $6$ 分)
	
	\item 利用高斯公式计算 $\oiint_{\Sigma} 2 x z \dd y \dd z+y z \dd z \dd x-z^{2} \dd x \dd y$ , 其中 $\Sigma$ 为由上半圆锥面 $z=\sqrt{x^{2}+y^{2}}$ 与上半球面 $z=\sqrt{2-x^{2}-y^{2}}$ 所围立体 $\Omega$ 的表面, 取外侧. ( $8$ 分)
	
	\item  判断级数 $\sum_{n=1}^{\infty} n 2^{n}$ 的敛散性. ( $6$ 分)
	
	\item  求幂级数 $\sum_{n=0}^{\infty}(n+1) x^{n}$ 在收敛域 $(-1,1)$ 的和函数 $s(x)$ . ( $6$ 分)
	
	\item $z=f\left(y^{2}-x^{2}\right)$ , 其中 $f(u)$ 有连续的二阶偏导数, 求 $\frac{\partial^{2} z}{\partial x^{2}}$ . ( $8$ 分)
	
	\item 计算 $\int_{L}\left(\ee^{x} \sin y-2 y\right) \dd x+\left(\ee^{x} \cos y-2\right) \dd y$ , $L$ 为由点 $A(1,0)$ 到 $B(0,1)$ , 再到 $C(-1,0)$ 的有向折线. ( $8$ 分)
	
	\item 计算 $\oiint_{\Sigma} x y^{2} \dd y \dd z+y z^{2} \dd z \dd x+z x^{2} \dd x \dd y$ , 其中 $\Sigma$ 为球体 $x^{2}+y^{2}+z^{2} \leq 4$ 及锥体 $z=\sqrt{x^{2}+y^{2}}$ 的公共部分的外表面. ( $8$ 分)
	
	\item 求级数 $\sum_{n=2}^{\infty} 2 n x^{n}$ 的收敛域及和函数. ( $8$ 分)
	
	\item 计算曲面积分 $\iint_{\Sigma}\left(x^{2}+y^{2}\right) \dd S$ , 其中 $\Sigma$ 为锥面 $z=\sqrt{3\left(x^2+y^2\right)}$ 被平面 $z=3$ 截下的带锥顶的部分. ( $8$ 分)
	
	\item 求函数 $z=x^2+y^2$ 在适合条件 $\frac{x}{2}+\frac{y}{3}=1$ 下的极小值. ( $7$ 分)
	
	\item 求方程 $y^{\prime \prime}-3 y^{\prime}+2 y=3 \ee^{x}$ 的通解. ( $8$ 分)
	
	\item 把 $f(x)=x,(0<x<\uppi)$ 展开为余弦级数. ( $7$ 分)
	
	\item 已知曲线积分 $\int_{(0,0)}^{(x, y)}\left[\ee^{x}(x+1)^{n}+\frac{n}{x+1} f(x)\right] y \dd x+f(x) \dd y$ 与路径无关, 其中 $f(x)$ 可微, $f(0)=0$ , 试确定 $f(x)$ , 并计算曲线积分的值. ( $8$ 分)
	
	\item 求过点 $(2,5,-3)$ 且与直线 $\begin{cases}
	x=5-2t\\
	y=1+t\\
	z=7
	\end{cases}$ 垂直的平面方程. ( $5$ 分)
	
	\item 由 $\ee^{x}-x y z=0$ 确定了函数 $z=z(x,y)$ , 求 $\frac{\partial z}{\partial x}$ . ( $5$ 分)
	
	\item 计算 $I=\iint_{D}\left(x^{2}+y^{2}\right) \dd x \dd y$ , 其中 $D=\left\{(x, y) | 1 \leq x^{2}+y^{2} \leq 4\right\}$ . ( $5$ 分)
	
	\item 利用格林公式, 计算 $\oint_{L}\left(2 x^{2} y-2 y\right) \dd x+\left(\frac{1}{3} x^{3}-2 x\right) \dd y$ , 其中 $L$ 为以 $y=x,y=x^{2}$ , 围成区域的正向边界. ( $8$ 分)
	
	\item 设 $\Sigma$ 是由旋转抛物面 $z=x^{2}+y^{2}$ 与平面 $z=2$ 所围成的封闭曲面, 取外侧. 用高斯公式计算 $\iint_{\Sigma} 4\left(1-y^{2}\right) \dd z \dd x+z(8 y+1) \dd x \dd y$ . ( $8$ 分)
	
	\item 求幂级数 $\sum_{n=0}^{\infty} \frac{x^{2 n+1}}{2 n+1}$ 在收敛域 $(-1,1)$ 内的和函数. ( $8$ 分)
	
	\item 求微分方程 $y''-2 y'+y=\ee^{x}$ 的通解. ( $8$ 分)
	
	\item 设函数 $f(x)$ 在 $[a,b]$ 上连续且 $f(x)>0$ , 证明 $\int_{a}^{b} f(x) \dd x \int_{a}^{b} \frac{1}{f(x)} \dd x \geq(b-a)^{2}$ . ( $5$ 分)

	\item 设 $u=f(x,xy)$ , $f$ 具有二阶连续偏导数, 求 $\frac{\partial u}{\partial x},\frac{\partial^2u}{\partial x\partial y}$ . ( $7$ 分)
	
	\item 求曲面 $\ee^{z}-z+x y=3$ 在点 $(2,1,0)$ 处的切平面及法线方程. ( $7$ 分)
	
	\item 设 $\Omega$ 是曲面 $\Sigma_1:z=\sqrt{x^2+y^2}$ 与 $\Sigma_2:2-x^2-y^2$ 所围成的立体, 求 $\Omega$ 的体积 $V$ 与表面积 $S$ . ( $10$ 分)
	
	\item 计算 $\iint_{\Sigma}\left(z+x y^{2}\right) \dd y \dd z+\left(y z^{2}-x z\right) \dd z \dd x+\left(x^{2} z+x^{3}\right) \dd x \dd y$ , 其中 $\Sigma$ 为 $x^{2}+y^{2}+z^{2}=4(z \leq 0)$ , 取下侧. ( $10$ 分)
	
	\item 计算 $\int_{L}\left(2 x y^{3}-y^{2} \cos x\right) \dd x+\left(1-2 y \sin x+3 x^{2} y^{2}\right) \dd y$ , 其中 $L$ 为抛物线 $2x=\uppi y^2$ 从点 $O(0,0)$ 到点 $A\left( \frac{\uppi}{2},1 \right)$ 的一段弧. ( $10$ 分)
	
	\item 求幂级数 $\sum_{n=1}^{\infty}nx^{n-1}$ 的收敛域与和函数. ( $8$ 分)
\end{enumerate}