\documentclass[fontset=sikou,punct=kaiming]{ctexbook}
\usepackage{xeCJKfntef}
\setcounter{secnumdepth}{4}
\ctexset{
	subsubsection=
	{
		numbering = true,
		name = {,、},
		number = \chinese{subsubsection},
		aftername = {\hspace{0pt}}
	},
}
\usepackage[a4paper,inner=1.5cm,textwidth=46\ccwd,top=2cm,bottom=2cm]{geometry}
\setlength{\headheight}{13pt}
\usepackage{fancyhdr}
\usepackage{lastpage}
\usepackage{calc,xcolor}
\newcommand{\progressbar}[2]{%
  \textcolor{black}{\rule{0.6\textwidth * \real{#1} / \real{#2}}{1.5ex}}%
	\textcolor{black!30}{\rule{0.6\textwidth - 0.6\textwidth * \real{#1} / \real{#2}}{1.5ex}}%
}

\pagestyle{fancy}
\fancyhf{}
\fancyhead[C]{\sffamily \booktitle}
\fancyhead[ER]{\sffamily \leftmark}
\fancyhead[OL]{\sffamily \rightmark}
\fancyhead[EL,OR]{\sffamily\thepage}

\fancypagestyle{plain}{%
	\fancyhf{}
	\cfoot{\sffamily\thepage}
	\renewcommand\headrulewidth{0pt}
	\renewcommand\footrulewidth{0pt}
}
\makeatletter
\fancypagestyle{myfancy}{
	\pagestyle{fancy}
	\fancyhead[C]{\sffamily \booktitle}
	\fancyhead[ER]{\sffamily \leftmark}
	\fancyhead[OL]{\sffamily \rightmark}
	\fancyhead[EL,OR]{\sffamily\thepage}
	\fancyfoot[EC]{%
		\@ifundefined{lastpage@lastpage}{%
		\progressbar{\thepage}{100}%
		}{%
		\progressbar{\thepage}{\lastpage@lastpage}%
		}%
	}
	\fancyfoot[OC]{\sffamily\fbox{仅供学习使用,严禁商业使用。此版本为第~\version{} 版,若想查看是否为最新版,去看 QQ: 2098881095 的空间}}
}
\makeatother
\usepackage[toc]{multitoc}
\usepackage{amsmath,amsthm}
\usepackage[partial=upright]{unicode-math}
\usepackage{physics,siunitx}
\setmainfont{XITS}
\setmathfont[StylisticSet = 8]{XITS Math}
\usepackage{sourcesanspro,sourcecodepro}
\newtheoremstyle{titi}
	{5pt}%
	{5pt}%
	{}%
	{0pt}%
	{\bfseries}%
	{}%
	{0pt}%
	{\thmname{#1}\thmnumber{#2}\thmnote{(#3)}}%
\theoremstyle{titi}
\newtheorem{ti}{}[subsubsection]
\renewcommand{\theti}{\arabic{ti}.}
\newtheorem*{solution}{解}
% \usepackage{ntheorem}
% {
% 	\theoremstyle{nonumberplain}
% 	\theoremheaderfont{\bfseries}
% 	\theorembodyfont{\normalfont}
% 	\newtheorem{solution}{解}
% }
% {
% 	\theoremstyle{change}
% 	\theoremheaderfont{\bfseries}
% 	\theorembodyfont{\normalfont}
% 	\newtheorem{ti}{}[subsubsection]
% }
% \renewcommand{\theti}{\arabic{ti}.}
\usepackage{tabularx,multirow}
\newcolumntype{Z}{>{\centering\arraybackslash}X}
\usepackage[hidelinks,bookmarksopen=true,bookmarksnumbered=true]{hyperref}
\usepackage{bookmark}
\newcommand{\ee}{\symup e}
\newcommand{\ii}{\symup i}
\DeclareMathOperator{\Ln}{Ln}
\newcommand{\LL}{\symscr{L}}
\newcommand{\EE}{\symbb{E}\,}
\newcommand{\bA}{\symbfit A}
\newcommand{\bB}{\symbfit B}
\newcommand{\bC}{\symbfit C}
\newcommand{\bD}{\symbfit D}
\newcommand{\bE}{\symbfit E}
\newcommand{\bX}{\symbfit X}
\newcommand{\astt}{*}
\newcommand{\balpha}{\symbfit \alpha}
\newcommand{\bbeta}{\symbfit \beta}
\newcommand{\TT}{\mathrm{T}}
\renewcommand{\leq}{\leqslant}
\renewcommand{\geq}{\geqslant}
\DeclareMathOperator{\Cov}{Cov}
\DeclareMathOperator{\Prj}{Prj}
% \makeatletter
% \let\mycases={\cases}
% \newenvironment{mycases}{%
%   \matrix@check\mycases\let\@ifnextchar\new@ifnextchar%
%   \left\lbrace%
%   \def\arraystretch{1.2}%
%   \array{@{}l}%
% }{%
%   \endarray\right.%
% }
% \makeatother
\def\theenumi{\arabic{enumi}}
\def\labelenumi{(\theenumi)}
\def\kuo{\mbox{(\hspace{\ccwd})}}
\def\kuoA{(\nobreak\makebox[\ccwd]{A}\nobreak)}
\def\kuoB{(\nobreak\makebox[\ccwd]{B}\nobreak)}
\def\kuoC{(\nobreak\makebox[\ccwd]{C}\nobreak)}
\def\kuoD{(\nobreak\makebox[\ccwd]{D}\nobreak)}
\def\hua{\CJKunderline*[hidden = true]{瞻彼阕者虚室生白}}
\def\huaa#1{\underline{ #1 }}
\long\def\guanggao{
	\vspace*{\fill}
	\begin{center}\thispagestyle{empty}
		广告位招租
	\end{center}
	\vspace*{\fill}
}
\setlength{\lineskip}{2.5pt}
\setlength{\lineskiplimit}{2.5pt}
\renewcommand\arraystretch{1.2}

\usepackage{calc,ifthen}

\newlength{\choicelengtha}
\newlength{\choicelengthb}
\newlength{\choicelengthc}
\newlength{\choicelengthd}
\newlength{\choicelengthe}
\newlength{\maxlength}

\makeatletter
\newcommand{\fourch}[4]{
  \par
  \settowidth{\choicelengtha}{A.#1}
  \settowidth{\choicelengthb}{B.#2}
  \settowidth{\choicelengthc}{C.#3}
  \settowidth{\choicelengthd}{D.#4}
  \ifthenelse{\lengthtest{\choicelengtha>\choicelengthb}}{\setlength{\maxlength}{\choicelengtha}}{\setlength{\maxlength}{\choicelengthb}}
  \ifthenelse{\lengthtest{\choicelengthc>\maxlength}}{\setlength{\maxlength}{\choicelengthc}}{}
  \ifthenelse{\lengthtest{\choicelengthd>\maxlength}}{\setlength{\maxlength}{\choicelengthd}}{}
  \ifthenelse{\lengthtest{\maxlength>0.8\linewidth}}
  {%
    \noindent%
    \begin{tabular}{@{}p{\linewidth}@{}}
      \setlength\tabcolsep{0pt}
      \@hangfrom{\textsf A.}#1 \\
      \@hangfrom{\textsf B.}#2 \\
      \@hangfrom{\textsf C.}#3 \\
      \@hangfrom{\textsf D.}#4 \\
    \end{tabular}
  }%
  {%
    \ifthenelse{\lengthtest{\maxlength>0.22\linewidth}}
    {%
      \noindent%
      \begin{tabular}{@{}p{0.48\linewidth}@{\hspace*{0.04\linewidth}}p{0.48\linewidth}@{}}
        \setlength\tabcolsep{0pt}
        \@hangfrom{\textsf A.}#1 & \@hangfrom{\textsf B.}#2 \\
        \@hangfrom{\textsf C.}#3 & \@hangfrom{\textsf D.}#4 \\
      \end{tabular}
    }%
    {%
      \noindent%
      \begin{tabular}{@{}*{3}{p{0.22\linewidth}@{\hspace*{0.04\linewidth}}}p{0.22\linewidth}@{}}
        \setlength\tabcolsep{0pt}
        \@hangfrom{\textsf A.}#1  & \@hangfrom{\textsf B.}#2 & \@hangfrom{\textsf C.}#3 & \@hangfrom{\textsf D.}#4 \\
      \end{tabular}
    }%
  }%
  \unskip\unskip
}
\makeatother
\def\onech{\fourch}
\def\twoch{\fourch}
\def\booktitle{JXUST 工科数学复习宝典}
\def\version{1.04}

\title{\booktitle(第~\version{} 版)}
\author{电气 173 何骏炜}
\begin{document}
	\frontmatter
	\pdfbookmark{标题页}{title}
	\maketitle
	\chapter{声明}
本汇总不得用于商业用途, 最新版下载地址: \href{https://github.com/sikouhjw/Mathematical-rearrangement}{Github} (用电脑点击链接即可), 不保证题目、答案的正确性, 如有错误可通过QQ群(见图~\ref{fig:1}~)\footnote{991832226}或者邮箱\footnote{489765924@qq.com}联系我们
\begin{figure}[htbp]
	\centering
	\includegraphics[width=0.3\textwidth]{2weima.png}
	\caption{二维码}\label{fig:1}
\end{figure}

点击 \href{https://github.com/sikouhjw/Engineering-mathematics-sunflower-treasure-book}{Github} 后,找到 $\mathrm{Emstbook.pdf}$ 后点击,点击 $\mathrm{Download}$ 即可

「复习题」是历年真题, 「难度与考试近似的题」是如果你太无聊给你写的。(为什么这都还有人问?)

	\tableofcontents
	\bookmark[dest=\HyperLocalCurrentHref, level=chapter]{目录}
	\newpage
	\guanggao
	\mainmatter
	\pagestyle{myfancy}
	\chapter{《高等数学》试卷汇总}
\section{《高等数学(一)》期中}
\subsection{2018-2019 A7}
\subsubsection{选择题}
\begin{ti}
	$\lim_{x \to 0} x \cos \frac{1}{x} = $\kuo
	\fourch{$-1$}{$1$}{$0$}{不存在}
\end{ti}

\begin{ti}
	$\lim_{x \to x_{0}} f(x_{0}) = 0$ 及\kuo, 则 $\lim_{x \to x_{0}} f(x) g(x) = 0$
	\twoch{$g(x)$ 为任意函数时}{当 $g(x)$ 为有界函数时}{仅当 $\lim_{x \to x_{0}} g(x) = 0$ 时}{仅当 $\lim_{x \to x_{0}} g(x)$ 存在时}
\end{ti}

\begin{ti}
	函数 $f(x)$ 在点 $x_{0}$ 处有定义, 是 $f(x)$ 在该点处连续的\kuo
	\fourch{必要条件}{充分条件}{充要条件}{无关条件}
\end{ti}

\begin{ti}
	已知 $y = \sin x$, 则 $y^{(10)} = $\kuo
	\fourch{$- \sin x$}{$\sin x$}{$- \cos x$}{$\cos x$}
\end{ti}

\begin{ti}
	函数 $y = x + \arctan x$ 在 $(-\infty,+\infty)$ 上\kuo
	\fourch{不连续}{连续不可导}{单调递减}{单调递增}
\end{ti}

\begin{ti}
	设 $f(x) = \begin{cases}
		\frac{2}{3}x^{3}, & x \leqslant 1 \\
		x^{2}, & x > 1
	\end{cases}$, 则 $f(x)$ 在点 $x = 1$ 处的\kuo
	\twoch{左、右导数都存在}{左导数存在, 右导数不存在}{左导数不存在, 右导数存在}{左、右导数均不存在}
\end{ti}

\begin{ti}
	$f'(x_{0}) = 0, f''(x_{0}) > 0$ 是函数 $f(x)$ 在点 $x = x_{0}$ 处取得极小值的一个\kuo
	\twoch{充要条件}{充分不必要条件}{必要不充分条件}{既不充分也不必要条件}
\end{ti}

\begin{ti}
	$f(x)$ 在 $(-\infty,+\infty)$ 内可导, 且 $\forall x_{1},x_{2}$, 当 $x_{1} > x_{2}$ 时 $f(x_{1}) > f(x_{2})$, 则\kuo
	\fourch{任意 $x$, $f'(x) > 0$}{任意 $x$, $f'(-x) \leqslant 0$}{$f(-x)$ 单调递增}{$-f(-x)$ 单调递增}
\end{ti}

\begin{ti}
	指出曲线 $y = \frac{x}{3 - x}$ 的渐近线\kuo
	\twoch{没有水平渐近线, 也没有垂直渐近线}{有垂直渐近线, 没有水平渐近线}{既有垂直渐近线, 又有水平渐近线}{没有垂直渐近线, 只有水平渐近线}
\end{ti}

\subsubsection{填空题}
\begin{ti}
	$\lim_{x \to 0} \frac{2x - \sin x}{x^{4} - x} = $ \hua
\end{ti}

\begin{ti}
	当 $x \to 0$, $\sin 3x$ 是 $\ee^{x} - 1$ 的 \hua 阶无穷小
\end{ti}

\begin{ti}
	函数 $f(x) = \begin{cases}
		\sqrt{1 + x} - 1, & x \ne 0 (x \geqslant -1) \\
		k, & x = 0
	\end{cases}$ 在 $x = 0$ 处连续, 则 $k = $ \hua 
\end{ti}

\begin{ti}
	设 $f(x) = \frac{3}{5 - x} + \frac{x^{2}}{5}$, 则 $f'(0) = $ \hua
\end{ti}

\begin{ti}
	曲线 $\begin{cases}
		x = 2 \ee^{t} \\
		y = \ee^{-t}
	\end{cases}$ 在 $t = 0$ 处切线方程为\hua
\end{ti}

\begin{ti}
	设 $f(x)$ 一阶可导, $y = f(1 + \sin x)$, 则 $y' = $ \hua
\end{ti}

\begin{ti}
	若 $f(x)$ 在 $[a,b]$ 上连续, 在 $(a,b)$ 内可导, 则至少存在一点 $\xi \in (a,b)$ 使得 $f(b) - f(a) = $ \hua
\end{ti}

\begin{ti}
	函数 $f(x) = x + 2 \cos x$ 在 $\bigl[ 0, \frac{\uppi}{2} \bigr]$ 上的最大值为 \hua
\end{ti}

\begin{ti}
	曲线 $y = x^{3} - 3 x^{2} + 5$ 的拐点为 \hua
\end{ti}

\subsubsection{大题}
\begin{ti}
	求 $\lim_{x \to \infty} \frac{4 x^{4} - 3 x^{2} + 1}{2 x^{4} + 5 x^{2} - 6}$
\end{ti}

\begin{ti}
	求 $\lim_{x \to 0} \bigl( 1 + 2x^{2} \bigr)^{x - 2}$
\end{ti}

\begin{ti}
	设 $y = \ee^{\arctan \sqrt{x}}$, 求 $y'$ 及 $\dd{y}$
\end{ti}

\begin{ti}
	求由方程 $x + y - \ee^{2x} + \ee^{y} = 0$ 确定的隐函数导数 $\frac{\dd{y}}{\dd{x}}$
\end{ti}

\begin{ti}
	设 $f(x) = \begin{cases}
		2x + 1, & x < 0, \\
		x^{2} + 1, & x \geqslant 0,
	\end{cases}$ 讨论 $f(x)$ 在 $x = 0$ 处的连续性与可导性
\end{ti}

\begin{ti}
	求 $\lim_{x \to 1} \frac{x^{3} - 1 + \ln x}{\ee^{x} - \ee}$
\end{ti}

\begin{ti}
	讨论函数 $y = x \ee^{x}$ 的图形性态并绘图
\end{ti}

\section{《高等数学(一)》期末}
\subsection{2018-2019 A15}
\subsubsection{选择题}
\begin{ti}
	设 $f(x)=
	\begin{cases}
	\frac{x^2-3x+2}{x-2}, & x\ne2\\
	a, & x=2
	\end{cases}
	$ 为连续函数, 则 $a=$ \kuo
	\fourch{任意}{0}{2}{1}
\end{ti}

\begin{ti}
	设 $f(x)=
	\begin{cases}
	x^2\sin\frac{1}{x}, & x\ne0\\
	0, & x=0
	\end{cases}
	$, 则 $f'(0)=$ \kuo
	\fourch{1}{$-1$}{0}{不存在}
\end{ti}

\begin{ti}
	函数 $f(x)$ 有连续二阶导数, 且 $f(0)=0$, $f'(0)=1$, $f''(0)=-2$, 则 $\lim_{x\to0}\frac{f(x)-x}{x^2}=$ \kuo
	\fourch{不存在}{0}{$-1$}{$-2$}
\end{ti}

\begin{ti}
	$\int \sec x\tan x\dd{x}=$ \kuo
	\fourch{$\tan x+C$}{$\sec x+C$}{$\arctan x+C$}{$\arccot x+C$}
\end{ti}

\begin{ti}
	设 $f(x)$ 的一个原函数是 $\ee^{-2x}$, 则 $f(x)$ 等于\kuo
	\fourch{$-2 \ee^{-2 x}$}{$-4 \ee^{-2x}$}{$\ee^{-2x}$}{$4 \ee^{-2x}$}
\end{ti}

\begin{ti}
	$\int\cos\ee^x\dd{\ee^x}=$ \kuo
	\fourch{$\sin\ee^x+C$}{$-\sin\ee^x+C$}{$\arcsin\ee^x+C$}{$-\arcsin\ee^x+C$}
\end{ti}

\begin{ti}
	若 $\int_{0}^{1}(x+k)\dd{x}=2$, 则 $k=$ \kuo
	\fourch{0}{1}{$-1$}{$\frac{3}{2}$}
\end{ti}

\begin{ti}
	下列反常积分中收敛的是\kuo
	\fourch{$\int_{1}^{+\infty}\frac{1}{\sqrt{x}}\dd{x}$}{$\int_{1}^{+\infty}\frac{1}{1+x^2}\dd{x}$}{$\int_{1}^{+\infty}\frac{1}{1+x}\dd{x}$}{$\int_{2}^{+\infty}\frac{1}{x\ln x}\dd{x}$}
\end{ti}

\subsubsection{填空题}
\begin{ti}
	$\lim_{x\to0}\frac{\sin^2x}{x}=$ \hua
\end{ti}

\begin{ti}
	$y=x^n$ ($n$ 是正整数)的 $n$ 阶导数是 \hua
\end{ti}

\begin{ti}
	当 $a = $ \hua{} 时函数 $f(x)=a\sin x+\frac13\sin3x$ 在 $x=\frac{\uppi}{3}$ 处取极大值
\end{ti}

\begin{ti}
	$\int\sec x(\sec x-\tan x)\dd{x}=$ \hua
\end{ti}

\begin{ti}
	经过点 $(1,2)$ 且其切线的斜率为 $2x$ 的曲线方程是 \hua
\end{ti}

\begin{ti}
	通过定积分的几何意义知 $\int_{-1}^{1}(1+x)\sqrt{1-x^2}\dd{x}=$ \hua
\end{ti}

\begin{ti}
	比较大小 $\int_{1}^{3}x^2\dd{x}$ \hua{} $\int_{1}^{3}x^3\dd{x}$
\end{ti}

\begin{ti}
	由 $y=x^2+1,y=0,x=1,x=0$ 所围平面图形绕 $x$ 轴旋转一周所得立体的体积用定积分表示为 \hua
\end{ti}

\subsubsection{大题}
\begin{ti}
	求 $\lim_{x\to0}\left(1+\frac{1}{2x}\right)^{x+3}$
\end{ti}

\begin{ti}
	计算 $\lim_{x\to0}\frac{ x-\int_0^x\ee^{-t^2}\dd{t}}{x^3}$
\end{ti}

\begin{ti}
	设 $y=\sin\left(x^2\right)$, 求 $\frac{\dd{y}}{\dd{x}},\frac{\dd^2{y}}{\dd{x^2}}$
\end{ti}

\begin{ti}
	计算不定积分 $\int\sin^3x\dd{x}$
\end{ti}

\begin{ti}
	计算不定积分 $\int x\ln x\dd{x}$
\end{ti}

\begin{ti}
	计算广义积分 $\int_{2}^{+\infty}\frac{1}{x(\ln x)^2}\dd{x}$
\end{ti}

\begin{ti}
	计算 $\int_{1}^{5}\frac{\sqrt{u-1}}{u}\dd{u}$
\end{ti}

\begin{ti}
	已知
	\[
	I_n=\int_{0}^{\frac{\uppi}{2}}\sin^nt\dd{t}=
	\begin{cases}
	\frac{n-1}{n}\cdot\frac{n-3}{n-2}\cdot\frac{n-5}{n-4}\cdots\frac{2}{3}, & n\text{为大于}1\text{的奇数}\\
	\frac{n-1}{n}\cdot\frac{n-3}{n-2}\cdots\frac{1}{2}\cdot\frac{\uppi}{2}, & n\text{为正偶数}
	\end{cases},
	\]
	求星形线 $
	\begin{cases}
	x=a\cos^3t\\
	y=a\sin^3t
	\end{cases}
	$ 围成的平面图形的面积, 其中 $a>0$
\end{ti}

\begin{ti}
	设 $f(x),g(x)$ 均在 $[a,b]$ 上连续, 证明在 $(a,b)$ 内至少存在一个 $\xi$, 使得
	\[
		f(\xi)\int_{\xi}^{b}g(x)\dd{x}=g(\xi)\int_{a}^{\xi}f(x)\dd{x}
	\]
	(提示:辅助函数 $F(x) = \int_{a}^{x}f(t)\dd{t}\int_{x}^{b}g(t)\dd{t}$)
\end{ti}

\subsection{2019-2020 B12}
\subsubsection{选择题}
缺

\subsubsection{填空题}
\begin{ti}
	设 $f(x) = \ln(1 + x)$, 则 $\dd{f(x)} \bigr|_{\vartriangle x = 0.01}^{x = 2} = $ \hua
\end{ti}

\begin{ti}
	$\lim_{x \to 0} \frac{1 - \cos 2x}{x \sin x} = $ \hua
\end{ti}

\begin{ti}
	$\int_{0}^{1} x^{3} \dd{x} = $ \hua
\end{ti}

\begin{ti}
	函数 $f(x) = 4 + 8x^{3} - 3x^{4}$ 在 $x = $ \hua{} 取最大值
\end{ti}

\begin{ti}
	$\int_{-\frac{\uppi}{2}}^{\frac{\uppi}{2}} (x|\cos x| + \cos x) \dd{x} = $ \hua
\end{ti}

\begin{ti}
	用参数方程表示的曲线 $x = x(t), y = y(t)$ 在 $t \in [a,b]$ 对应的一段长度用定积分表示为 \hua
\end{ti}

\begin{ti}
	如果 $\ee^{-x}$ 是函数 $f(x)$ 的一个原函数, 则 $\int f(x) \dd{x} = $ \hua
\end{ti}

\begin{ti}
	$\int \frac{\cos 2x}{\cos x - \sin x} \dd{x} = $ \hua
\end{ti}

\subsubsection{综合题}
\begin{ti}
	计算 $\lim_{x \to \infty} \frac{\int_{0}^{x} \ee^{t^{2}} \dd{t}}{\ee^{x^{2}}}$
\end{ti}

\begin{ti}
	设 $\begin{cases}
		x = \ee^{t},\\
		y = t \ee^{t},
	\end{cases}$ 求 $\frac{\dd^{2}y}{\dd{x^{2}}}$
\end{ti}

\begin{ti}
	求 $\lim_{x \to 0} \frac{\tan x - \sin x}{\sin^{3}x}$
\end{ti}

\begin{ti}
	计算 $\int_{0}^{1} x^{2} \sqrt{1 - x^{2}} \dd{x}$
\end{ti}

\begin{ti}
	计算 $\int_{\ee}^{\ee^{2}} \frac{\dd{x}}{x \ln x}$
\end{ti}

\begin{ti}
	计算不定积分 $\int x \cos x \dd{x}$
\end{ti}

\begin{ti}
	计算 $\int \frac{\ee^{x}}{1 + \ee^{x}} \dd{x}$
\end{ti}

\begin{ti}
	求证:在 $[-1,1]$ 上 $\arcsin x + \arccos x = \frac{\uppi}{2}$
\end{ti}

\begin{ti}
	求由曲线 $y = x^{2}$ 及直线 $y = x$ 围成的平面图形的面积
\end{ti}

\subsection{2019-2020 B20}
\subsubsection{选择题}
\begin{ti}[3 分]
	函数 $f(x)$ 在点 $x_{0}$ 处有定义, 是 $f(x)$ 在该点处连续的 \kuo
	\fourch{充分条件}{必要条件}{充要条件}{无关的条件}
\end{ti}

\begin{ti}[3 分]
	若 $f(x) = \begin{cases}
		\ee^{ax}, & x < 0 \\
		b + \sin 2x, & x \geqslant 0
	\end{cases}$ 在 $x = 0$ 处可导, 则 $a,b$ 的值应为 \kuo
	\fourch{$a = -2, b = 1$}{$a = 1, b = 2$}{$a = 2, b = 1$}{$a = 2, b = -1$}
\end{ti}

\begin{ti}[3 分]
	函数 $y = \frac{1}{x}$ 在 $(0,1)$ 内的最小值是 \kuo
	\fourch{不存在}{0}{1}{任何小于 $1$ 的数}
\end{ti}

\begin{ti}[3 分]
	$\int a^{x} \dd{x} = $ \kuo
	\fourch{$\frac{1}{a} a^{x + 1} + C$}{$a^{x} + C$}{$a^{x} \ln a + C$}{$\frac{a^{x}}{\ln a} + C$}
\end{ti}

\begin{ti}[3 分]
	设 $f(x)$ 的一个原函数为 $F(x)$, 则 $\int f(2x) \dd{x} = $ \kuo
	\fourch{$2 F \bigl( \frac{x}{2} \bigr) + C$}{$F \bigl( \frac{x}{2} \bigr) + C$}{$F(2x) + C$}{$\frac{1}{2} F(2x) + C$}
\end{ti}

\begin{ti}[3 分]
	一物体由静止开始运动, 位移函数为 $s(t)$, $s(0) = 0$, 经 $t$ 秒后的速度是 $3t^{2} (\si{m/s})$, 问:在 \SI{3}{s} 后物体离开出发点的距离是多少?\kuo
	\fourch{$81$ 米}{$27$ 米}{$9$ 米}{$3$ 米}
\end{ti}

\begin{ti}[3 分]
	以下等式成立的是 \kuo
	\twoch{$\int_{-2}^{-1} \frac{1}{x} \dd{x} = \ln |-1| - \ln|-2|$}{$\int_{-1}^{1} \frac{1}{x} \dd{x} = \ln |1| - \ln|-1|$}{$\int_{-\frac{\uppi}{2}}^{\frac{\uppi}{2}} \cot x \dd{x} = \ln \bigl|\sin \frac{\uppi}{2}\bigr| - \ln \bigl|\sin\bigl( -\frac{\uppi}{2} \bigr)\bigr|$}{$\int_{-1}^{1} \bigl( x^{2} + \frac{1}{x} \bigr) \dd{x} = \int_{-1}^{1} x^{2} \dd{x}$}
\end{ti}

\begin{ti}
	设 $f(x) = \int_{0}^{x} \sin t^{2} \dd{t}, g(x) = x^{2}$, 则当 $x \to 0$ 时, $f(x)$ 是 $g(x)$ 的 \kuo
	\twoch{同阶但不等价的无穷小}{等价无穷小}{高阶无穷小}{低阶无穷小}
\end{ti}

\subsubsection{填空题}
\begin{ti}[3 分]
	设 $y = f(\sin x)$, $f$ 可微, 则 $\dd{y} = $ \hua
\end{ti}

\begin{ti}[3 分]
	$\lim_{n \to \infty} \frac{(n + 1)(3n + 2)(n + 3)}{4 n^{3}} = $ \hua
\end{ti}

\begin{ti}[3 分]
	$\int \csc x \cot x \dd{x} = $ \hua
\end{ti}

\begin{ti}[3 分]
	曲线 $y = x^{2} + \frac{1}{x - 1}$ 的铅直渐近线是 \hua
\end{ti}

\begin{ti}[3 分]
	$\int \bigl( 3 \ee^{x} + \frac{2}{x} \bigr) \dd{x} = $ \hua
\end{ti}

\begin{ti}[3 分]
	用参数方程表示的曲线 $x = x(t), y = y(t)$ 在 $t \in [a,b]$ 对应的一段长度用定积分表示为 \hua
\end{ti}

\begin{ti}[3 分]
	设函数 $f(x)$ 连续, 且 $f(x) = 2x - \int_{0}^{2} f(t) \dd{t}$, 则 $f(x) = $ \hua
\end{ti}

\begin{ti}[3 分]
	通过定积分的几何意义知 $\int_{-1}^{1} (1 + x) \sqrt{1 - x^{2}} \dd{x} = $ \hua
\end{ti}

\subsubsection{综合题}
\begin{ti}[6 分]
	计算 $\lim_{x \to 0} \frac{\int_{0}^{x} t(t + \sin t) \dd{t}}{x^{3}}$
\end{ti}

\begin{ti}[5 分]
	计算不定积分 $\int \frac{x^{2}}{x^{2} + 1} \dd{x}$
\end{ti}

\begin{ti}[6 分]
	求 $\lim_{x \to 0} \bigl( 1 + \frac{x}{2} \bigr)^{\frac{3}{x}}$
\end{ti}

\begin{ti}[6 分]
	计算 $\int_{\frac{1}{\uppi}}^{\frac{2}{\uppi}} \frac{1}{x^{2}} \cos \frac{1}{x} \dd{x}$
\end{ti}

\begin{ti}[6 分]
	$y = x^{3} + \sin x + \ln x + \ee^{x}$, 求 $y'$
\end{ti}

\begin{ti}[6 分]
	计算 $\int_{0}^{1} x \ee^{3x} \dd{x}$
\end{ti}

\begin{ti}[6 分]
	求曲线 $y = x^{2}$ 与 $x = y^{2}$ 所围成图形的面积
\end{ti}

\begin{ti}[5 分]
	计算 $\int \frac{\dd{x}}{x^{2} \sqrt{1 - x^{2}}}$
\end{ti}

\begin{ti}[6 分]
	求 $y = x^{3} + 3x^{2} - 9x + 14$ 在 $[-4,4]$ 上的最大值与最小值
\end{ti}

\subsection{2020-2021 A13}
\subsubsection{选择题(每小题3分,共24分)}
\begin{ti}
	下列等式中成立的是 \kuo
	\fourch{$\lim_{n\to\infty} \bigl( 1 + \frac{2}{n} \bigr)^n = \ee$}
	{$\lim_{n\to\infty} \bigl( 1 + \frac{1}{2n} \bigr)^n = \ee$}
	{$\lim_{n\to\infty} \bigl( 1 + \frac{1}{n} \bigr)^{2n} = \ee$}
	{$\lim_{n\to\infty} \bigl( 1 + \frac{1}{n} \bigr)^{n+2} = \ee$}
\end{ti}

\begin{ti}
	设 $y = f(\ln x)$, 则 $\dd{y} = $ \kuo
	\fourch{$f'(\ln x) \dd{x}$}{$f'(\ln x) \dd{\ln x}$}{$f'\bigl( \frac{1}{x} \bigr) \dd{x}$}{$f'(\ln x) \dd{\frac{1}{x}}$}
\end{ti}

\begin{ti}
	方程 $x^5 + x - 1 = 0$ 是 \kuo
	\fourch{有三个不同的实根}{有且仅有两个不同的实根}{没有实根}{有且仅有一个实根}
\end{ti}

\begin{ti}
	若 $\int f(x) \dd{x} = x^2 + C$, 则 $\int x f\bigl(x^2\bigr) \dd{x} = $ \kuo
	\fourch{$2x^4 + C$}{$2x^2 + C$}{$\frac{1}{2} x^4 + C$}{$\frac{1}{2} x^2 + C$}
\end{ti}

\begin{ti}
	设 $I = \int \sin x \cos x \dd{x}$, 则 $I = $ \kuo
	\fourch{$\frac{1}{2} \sin^2 x + C$}{$\frac{1}{2} \cos^2 x + C$}{$\frac{1}{4} \cos 2x + C$}{$\frac{1}{4} \sin 2x + C$}
\end{ti}

\begin{ti}
	若当 $a \leqslant x \leqslant b$, 有 $f(x) \leqslant g(x)$, 则 \kuo
	\fourch{$\int f(x) \dd{x} \leqslant \int g(x) \dd{x}$}
	{$\int_a^b f(x) \dd{x} < \int_a^b g(x) \dd{x}$}
	{$\int_a^b f(x) \dd{x} \leqslant \int_a^b g(x) \dd{x}$}
	{$\int f(x) \dd{x} < \int g(x) \dd{x}$}
\end{ti}

\begin{ti}
	以下等式成立的是 \kuo
	\fourch{$\int_{-1}^1 \cot x \dd{x} = \ln  \lvert\sin 1\rvert - \ln \lvert\sin(-1)\rvert$}
	{$\int_{-3}^{-2} \frac{1}{x} \dd{x} = \ln\lvert-2\rvert - \ln\lvert-3\rvert$}
	{$\int_{-1}^1 \bigl( x^2 + \frac{1}{x^3} \bigr) \dd{x} = \int_{-1}^1 x^2 \dd{x}$}
	{$\int_{-1}^1 \frac{1}{x^2} \dd{x} = 2 \int_0^1 \frac{1}{x^2} \dd{x}$}
\end{ti}

\begin{ti}
	星形线 $\left\lbrace\begin{array}{@{}l}
		x = a \cos^3 t \\
		y = a \sin^3 t
	\end{array}\right.$ (其中 $a>0$) 的周长为 \kuo
	\fourch
	{$4 \int_0^{\frac{\uppi}{2}} \sqrt{ \bigl[ \dd{\bigl( a \cos^3 t \bigr)} \bigr]^2 + \bigl[ \dd{\bigl( a \sin^3 t \bigr)} \bigr]^2 }$}
	{$4\int_0^{\frac{\uppi}{2}} \uppi a^2 \sin^6 t \dd{\bigl( a \cos^3 t \bigr)}$}
	{$4 \int_0^{a} \sqrt{ \bigl[ \dd{\bigl( a \cos^3 t \bigr)} \bigr]^2 + \bigl[ \dd{\bigl( a \sin^3 t \bigr)} \bigr]^2 }$}
	{$4 \int_0^{\frac{\uppi}{2}} a \sin^3t \dd{\bigl( a \cos^3 t \bigr)}$}
\end{ti}

\subsubsection{填空题(每空3分,共24分)}
\begin{ti}
	$\lim_{x \to 0} \frac{\ln(1+2x)}{\sin 5x} = $ \hua
\end{ti}

\begin{ti}
	设 $y=3^x$, 则 $y' = $ \hua
\end{ti}

\begin{ti}
	函数 $y = 2x^3 - 3x^2$ 在 $x = $ \hua{} 处有极大值
\end{ti}

\begin{ti}
	$\int \frac{\cos 2x}{\cos^2 x \sin^2 x} \dd{x} = $ \hua
\end{ti}

\begin{ti}
	函数 $f(x)$ 在 $(-\infty,+\infty)$ 上连续, 则 $\dd\bigl[ \int f(x) \dd{x} \bigr]$ 等于 \hua
\end{ti}

\begin{ti}
	由几何意义知定积分 $\int_{-1}^1 \bigl( x^4 \sin^3 x + 3 \sqrt{1-x^2} \bigr) \dd{x} = $ \hua
\end{ti}

\begin{ti}
	设函数 $f(x)$ 连续, 且 $f(x) = 3x^2 - \int_0^1 f(x) \dd{x}$, 则 $f(x) = $ \hua
\end{ti}

\begin{ti}
	由曲线 $y=x^2$ 及直线 $y=x$ 围成的平面图形绕 $x$ 轴旋转一周所得的旋转体的体积用定积分表示为 \hua
\end{ti}

\subsubsection{综合题(请写出求解或证明过程,9 小题,共 52 分)}
\begin{ti}[6分]
	求 $\lim_{x \to 0} \frac{4x^3 - 2x^2 + x}{3x^2 + 2x}$
\end{ti}

\begin{ti}[6分]
	计算 $\lim_{x \to a} \frac{1}{x-a} \int_a^x f(t) \dd{t}$, 其中 $f(x)$ 连续
\end{ti}

\begin{ti}[6分]
	设 $\left\lbrace\begin{array}{@{}l}
		x = 1+t^2 \\
		y = 3t^3+4
	\end{array}\right.$, 求 $\frac{\dd[2]{y}}{\dd{x^2}}$
\end{ti}

\begin{ti}[5分]
	求不定积分 $\int \frac{1}{x} \ln x \dd{x}$
\end{ti}

\begin{ti}[5分]
	计算不定积分 $\int x \arctan x \dd{x}$
\end{ti}

\begin{ti}[6分]
	计算 $\int_0^{+\infty} \frac{\dd{x}}{\sqrt{x} + x\sqrt{x}}$
\end{ti}

\begin{ti}[6分]
	$\int_{-\frac{\uppi}{4}}^{\frac{\uppi}{2}} \sqrt{\cos x - \cos^3 x} \dd{x}$
\end{ti}

\begin{ti}[6分]
	求由曲线 $y=\ee^x$ 与直线 $y=\ee x$、$y$ 轴围成的图形的面积
\end{ti}

\begin{ti}[6分]
	证明恒等式:$\arctan x + \arccot x = \frac{\uppi}{2}$ $(-\infty<x<\infty)$
\end{ti}

\section{《高等数学(二)》期中}
\subsection{2017-2018}
\subsubsection{选择题}
\begin{ti}
	下列方程中, 设 $y_{1},y_{2}$ 是它的解, 可以推知 $y_{1} + y_{2}$ 也是它的解的方程是 \kuo
	\twoch{$y' + p(x)y + q(x) = 0$}{$y'' + p(x)y' + q(x)y = 0$}{$y'' + p(x)y' + q(x)y = f(x)$}{$y'' + p(x)y' + q(x) = 0$}
\end{ti}

\begin{ti}
	微分方程 $\frac{\dd{y}}{\dd{x}} = \frac{y}{x} \ln \frac{y}{x}$ 是 \kuo
	\twoch{可分离变量的微分方程}{齐次微分方程}{一阶线性非齐次微分方程}{二阶微分方程}
\end{ti}

\begin{ti}
	二阶齐次线性微分方程 $y'' + y = 0$ 的通解为 $y = $ \kuo
	\twoch{$C_{1} \cos x$}{$C_{2} \sin x$}{$(C_{1} + C_{2})(\cos x + \sin x)$}{$C_{1} \cos x + C_{2} \sin x$}
\end{ti}

\begin{ti}
	两向量 $\vec{a} \cdot \vec{b}$ 平行的充要条件是 \kuo
	\fourch{$\vec{a} \cdot \vec{b} = \vec{0}$}{$\vec{a} \cdot \vec{b} = 0$}{$\vec{a} \times \vec{b} = \vec{0}$}{$\vec{a} \times \vec{b} = 0$}
\end{ti}

\begin{ti}
	设 $\vec{a} = (2,4,-1), \vec{b} = (0,-2,2)$, 则同时与 $\vec{a},\vec{b}$ 垂直的单位向量 $\vec{n} = $ \kuo
	\twoch{$6 \vec{i} - 4 \vec{j} - 4 \vec{k}$}{$-6 \vec{i} + 4 \vec{j} + 4 \vec{k}$}{$\frac{1}{| \vec{a} + \vec{b} |} \bigl( 6 \vec{i} - 4 \vec{j} - 4 \vec{k} \bigr)$}{$\frac{\pm 1}{| \vec{a} \times \vec{b} |} \bigl( 6 \vec{i} - 4 \vec{j} - 4 \vec{k} \bigr)$}
\end{ti}

\begin{ti}
	若平面 $Ax + By + Cz + D = 0$ 过 $x$ 轴, 则 \kuo
	\fourch{$A = D = 0$}{$B = 0, C \ne 0$}{$B \ne 0, C = 0$}{$B = C = 0$}
\end{ti}

\begin{ti}
	设 $f(x,y) = x^{2} + (y - 3) \arctan \frac{x}{y}$, 则 $f_{x}(2,3) = $ \kuo
	\fourch{1}{2}{3}{4}
\end{ti}

\begin{ti}
	设 $z = x^{2}y + 3xy^{2} + x$, 则 $\frac{\partial^{2}z}{\partial x \partial y} = $ \kuo
	\fourch{$x + 3y$}{$2x + 3y$}{$2x + 6y$}{$2x + 6y + 1$}
\end{ti}

\begin{ti}
	若 $z = \ee^{-x + y}$, 则 $\dd{z}\bigr|_{(1,1)} = $ \kuo
	\fourch{$- \dd{x} + \dd{y}$}{$\dd{x} - \dd{y}$}{$\dd{x} + \dd{y}$}{$- \dd{x} - \dd{y}$}
\end{ti}

\subsubsection{填空题}
\begin{ti}
	微分方程 $(y'')^{3} + y' \sin x = 0$ 的阶数为 \hua
\end{ti}

\begin{ti}
	方程 $y \dd{x} = x \dd{y}$ 的通解为 $y = $ \hua
\end{ti}

\begin{ti}
	微分方程 $y^{(4)} \sin x - y' = \ee^{x}$ 的通解中所含相互独立的任意常数的个数为 \hua
\end{ti}

\begin{ti}
	设向量 $\vec{a}$ 的方向角为 $\alpha,\beta,\gamma$, 满足 $\cos \alpha = 1$ 时, 向量 $\vec{\alpha}$ 垂直于 \hua{} 坐标面
\end{ti}

\begin{ti}
	过点 $(1,0,1)$ 及以 $(2,2,4)$ 为方向向量的直线的参数方程是 \hua
\end{ti}

\begin{ti}
	圆锥面 $z = \sqrt{x^{2} + y^{2}}$ 与平面 $z = 2$ 所围立体在 $xoy$ 平面上投影曲线方程是 \hua
\end{ti}

\begin{ti}
	函数 $z = \ln(x^{2} + y^{2} - 4)$ 的定义域 $D$ 是 \hua
\end{ti}

\begin{ti}
	设 $z = 3 x^{2}y^{2} + \ee^{x^{2}y}$, 则 $\frac{\partial z}{\partial x} = $ \hua
\end{ti}

\begin{ti}
	设 $z = \ln(2x + y)$, 则 $\frac{\partial^{2}z}{\partial x \partial y} = $ \hua
\end{ti}

\subsubsection{综合题}
\begin{ti}
	求方程 $\frac{\dd{y}}{\dd{x}} + \frac{1 - 2x}{x^{2}} y - 1 = 0$ 的通解
\end{ti}

\begin{ti}
	求方程 $y'' - 2y' - 3y = 3x + 1$ 的通解
\end{ti}

\begin{ti}
	设 $z = \ee^{u + v}$, 而 $u = xy,v = x - y$, 求 $\frac{\partial z}{\partial x},\frac{\partial z}{\partial y}$
\end{ti}

\begin{ti}
	求曲面 $2x^{2} + 3y^{2} + z^{2} = 6$ 在点 $(1,1,1)$ 处的切平面和法线方程
\end{ti}

\begin{ti}
	求函数 $f(x,y) = x + y - x^{2} - y^{2}$ 在曲线 $x^{2} + y^{2} = 1$ 上的最大值和最小值
\end{ti}

\subsection{2018-2019 B10}
\subsubsection{选择题}
\begin{ti}
	微分方程 $(y')^3+3\sqrt{y''}+x^4y'''=\sin x$ 的阶数是\kuo
	\fourch{1}{4}{2}{3}
\end{ti}

\begin{ti}
	设 $f(x,y)=x-y-\sqrt{x^2+y^2}$ , 则 $f_{x}(3,4)=$\kuo
	\fourch{$\frac{3}{5}$}{$\frac{2}{5}$}{$-\frac{2}{5}$}{$\frac{1}{5}$}
\end{ti}

\begin{ti}
	微分方程 $y'=\frac{y}{x}$ 的一个特解是\kuo
	\fourch{$y=2x$}{$\ee^y=x$}{$y=x^2$}{$y=\ln x$}
\end{ti}

\begin{ti}
	若 $z=\ln\sqrt{1+x^2+y^2}$ , 则 $\left.\dd{z}\right|_{(1,1)}=$\kuo
	\fourch{$\frac{\dd{x}+\dd{y}}{3}$}{$\frac{\dd{x}+\dd{y}}{2}$}{$\frac{\dd{x}+\dd{y}}{1}$}{$3(\dd{x}+\dd{y})$}
\end{ti}

\begin{ti}
	设直线 $L:\begin{cases}
	x+3y+2z+1=0\\
	2x-y-10z+3=0
	\end{cases}$ , 平面 $\eta:4x-2y+z-2=0$ , 则\kuo
	\fourch{$L$ 在 $\eta$ 上}{$L$ 平行于 $\eta$}{$L$ 垂直于 $\eta$}{$L$ 与 $\eta$ 斜交}
\end{ti}

\begin{ti}
	方程 $y'+3xy=6x^2y$ 是\kuo
	\twoch{二阶微分方程}{非线性微分方程}{一阶线性非齐次微分方程}{可分离变量的微分方程}
\end{ti}

\begin{ti}
	曲面 $\frac{x^2}{9}-\frac{y^2}{4}+\frac{z^2}{4}=1$ 与平面 $x=y$ 的交线是\kuo
	\fourch{两条直线}{双曲线}{椭圆}{抛物线}
\end{ti}

\begin{ti}
	设 $z=\ee^{x^2y}$ , 则 $\frac{\partial^2z}{\partial x\partial y}=$\kuo
	\twoch{$2y\left(1+x^3\right)\ee^{x^2y}$}{$\ee^{x^2y}$}{$2x\left(1+x^2y\right)\ee^{x^2y}$}{$2x\ee^{x^2y}$}
\end{ti}

\begin{ti}
	下列结论正确的是\kuo
	\twoch{$\vec{a}\times\left(\vec{b}-\vec{c}\right)=\vec{a}\times\vec{b}-\vec{a}\times\vec{c}$}{若 $\vec{a}\times\vec{b}=\vec{a}\times\vec{c}$ 且 $\vec{a}\ne\vec{0}$ , 则 $\vec{b}=\vec{c}$}{$\vec{a}\times\vec{b}=\vec{b}\times\vec{a}$}{若 $\left|\vec{a}\right|=1,\left|\vec{b}\right|=1$ , 则 $\left|\vec{a}\times\vec{b}\right|=1$}
\end{ti}


\subsubsection{填空题}
\begin{ti}
	平面过点 $(2,0,0),(0,1,0),(0,0,0.5)$ , 则该平面的方程是 \hua{}
\end{ti}

\begin{ti}
	设 $y_1$ 是 $y''+p(x)y'+q(x)y=f(x)$ 的解, $y_2$ 是 $y''+p(x)y'+q(x)y=f(x)$ 的解, 则 $y_1+y_2$ 是 \hua{}方程的解
\end{ti}

\begin{ti}
	设 $z=y\arctan x$ , 则 $\left.\mathrm{grad}\,z\right|_{(1,2)}=$ \hua{}
\end{ti}

\begin{ti}
	过点 $P(0,2,4)$ 且与两平面 $x+2z=1$ 和 $y-2z=2$ 平行的直线方程是 \hua{}
\end{ti}

\begin{ti}
	设 $f(x,y)=\arcsin\frac{y}{x}$ , 则 $f_y(1,0)=$ \hua{}
\end{ti}

\begin{ti}
	$y=\ee^x$ 是微分方程 $y''+py'+6y=0$ 的一个特解, 则 $p=$ \hua{}
\end{ti}

\begin{ti}
	已知平面 $\eta_1:A_1x+B_1y+C_1z+D_1=0$ 与平面 $\eta_2:A_2x+B_2y+C_2z+D_2=0$, 则 $\eta_1\perp\eta_2$ 的充要条件是 \hua{}
\end{ti}

\begin{ti}
	微分方程 $y''+2y'+5y=0$ 的通解为 $y=$ \hua{}
\end{ti}

\begin{ti}
	设 $z=\ee^{xy}+\cos\left(x^2+y\right)$, 则 $\frac{\partial z}{\partial y}=$ \hua{}
\end{ti}

\subsubsection{大题}
\begin{ti}
	求方程 $\frac{\dd{z}}{\dd{x}}=-z+4x$ 的通解
\end{ti}

\begin{ti}
	求曲线 $2z+1=\ln(xy)+\ee^z$ 在点 $M_{0}(1,1,0)$ 处的切平面和法线方程
\end{ti}

\begin{ti}
	设由方程组 $\begin{cases}
	x+y+z=0\\
	x^2+y^2+z^2=1
	\end{cases}$
	确定了隐函数 $x=x(z),y=y(z)$ , 求 $\frac{\dd{x}}{\dd{z}},\frac{\dd{y}}{\dd{z}}$
\end{ti}

\begin{ti}
	求方程 $y''+6y'+13y=\ee^t$ 的通解
\end{ti}

\begin{ti}
	设 $z=x^2y+\sin x+\varphi(xy+1)$ , 且 $\varphi(u)$ 具有一阶连续导数, 求 $\frac{\partial z}{\partial x},\frac{\partial z}{\partial y}$
\end{ti}


\subsection{2018-2019 B10 答案}
\subsubsection{选择题}
\begin{ti}
	微分方程 $(y')^3+3\sqrt{y''}+x^4y'''=\sin x$ 的阶数是\kuoD
	\fourch{1}{4}{2}{3}
\end{ti}

\begin{ti}
	设 $f(x,y)=x-y-\sqrt{x^2+y^2}$ , 则 $f_{x}(3,4)=$\kuoB
	\fourch{$\frac{3}{5}$}{$\frac{2}{5}$}{$-\frac{2}{5}$}{$\frac{1}{5}$}
\end{ti}

\begin{ti}
	微分方程 $y'=\frac{y}{x}$ 的一个特解是\kuoA
	\fourch{$y=2x$}{$\ee^y=x$}{$y=x^2$}{$y=\ln x$}
\end{ti}

\begin{ti}
	若 $z=\ln\sqrt{1+x^2+y^2}$ , 则 $\left.\dd{z}\right|_{(1,1)}=$\kuoA
	\fourch{$\frac{\dd{x}+\dd{y}}{3}$}{$\frac{\dd{x}+\dd{y}}{2}$}{$\frac{\dd{x}+\dd{y}}{1}$}{$3(\dd{x}+\dd{y})$}
\end{ti}

\begin{ti}
	设直线 $L:\begin{cases}
	x+3y+2z+1=0\\
	2x-y-10z+3=0
	\end{cases}$ , 平面 $\eta:4x-2y+z-2=0$ , 则\kuoC
	\fourch{$L$ 在 $\eta$ 上}{$L$ 平行于 $\eta$}{$L$ 垂直于 $\eta$}{$L$ 与 $\eta$ 斜交}
\end{ti}

\begin{ti}
	方程 $y'+3xy=6x^2y$ 是\kuoD
	\twoch{二阶微分方程}{非线性微分方程}{一阶线性非齐次微分方程}{可分离变量的微分方程}
\end{ti}

\begin{ti}
	曲面 $\frac{x^2}{9}-\frac{y^2}{4}+\frac{z^2}{4}=1$ 与平面 $x=y$ 的交线是\kuoB
	\fourch{两条直线}{双曲线}{椭圆}{抛物线}
\end{ti}

\begin{ti}
	设 $z=\ee^{x^2y}$ , 则 $\frac{\partial^2z}{\partial x\partial y}=$\kuoC
	\twoch{$2y\left(1+x^3\right)\ee^{x^2y}$}{$\ee^{x^2y}$}{$2x\left(1+x^2y\right)\ee^{x^2y}$}{$2x\ee^{x^2y}$}
\end{ti}

\begin{ti}
	下列结论正确的是\kuoA
	\twoch{$\vec{a}\times\left(\vec{b}-\vec{c}\right)=\vec{a}\times\vec{b}-\vec{a}\times\vec{c}$}{若 $\vec{a}\times\vec{b}=\vec{a}\times\vec{c}$ 且 $\vec{a}\ne\vec{0}$ , 则 $\vec{b}=\vec{c}$}{$\vec{a}\times\vec{b}=\vec{b}\times\vec{a}$}{若 $\left|\vec{a}\right|=1,\left|\vec{b}\right|=1$ , 则 $\left|\vec{a}\times\vec{b}\right|=1$}
\end{ti}


\subsubsection{填空题}
\begin{ti}
	平面过点 $(2,0,0),(0,1,0),(0,0,0.5)$ , 则该平面的方程是 \huaa{$\frac{x}{2}+y+2z=1$}
\end{ti}

\begin{ti}
	设 $y_1$ 是 $y''+p(x)y'+q(x)y=f(x)$ 的解, $y_2$ 是 $y''+p(x)y'+q(x)y=f(x)$ 的解, 则 $y_1+y_2$ 是 \huaa{$y''+p(x)y'+$} \huaa{$q(x)y=2f(x)$} 方程的解
\end{ti}

\begin{ti}
	设 $z=y\arctan x$ , 则 $\left.\mathrm{grad}\,z\right|_{(1,2)}=$ \huaa{$\dd{x}+\frac{\uppi}{4}\dd{y}$}
\end{ti}

\begin{ti}
	过点 $P(0,2,4)$ 且与两平面 $x+2z=1$ 和 $y-2z=2$ 平行的直线方程是 \huaa{$\frac{x}{-2}=\frac{y-2}{2}=\frac{z-4}{1}$}
\end{ti}

\begin{ti}
	设 $f(x,y)=\arcsin\frac{y}{x}$ , 则 $f_y(1,0)=$ \huaa{$1$}
\end{ti}

\begin{ti}
	$y=\ee^x$ 是微分方程 $y''+py'+6y=0$ 的一个特解, 则 $p=$ \huaa{$-7$}
\end{ti}

\begin{ti}
	已知平面 $\eta_1:A_1x+B_1y+C_1z+D_1=0$ 与平面 $\eta_2:A_2x+B_2y+C_2z+D_2=0$, 则 $\eta_1\perp\eta_2$ 的充要条件是 \huaa{$A_1A_2+B_1B_2+C_1C_2=0$}
\end{ti}

\begin{ti}
	微分方程 $y''+2y'+5y=0$ 的通解为 $y=$ \huaa{$C_1\ee^{-x}\sin(2x)+C_2\ee^{-x}\cos(2x)$}
\end{ti}

\begin{ti}
	设 $z=\ee^{xy}+\cos\left(x^2+y\right)$, 则 $\frac{\partial z}{\partial y}=$ \huaa{$x\ee^{xy}-\sin\left(x^2+y\right)$}
\end{ti}

\subsubsection{大题}
\begin{ti}
	求方程 $\frac{\dd{z}}{\dd{x}}=-z+4x$ 的通解
	\begin{solution}
		运用一阶线性非齐次微分方程公式, 得
		\begin{align*}
			z&=\ee^{-\int\dd{x}}\left( \int 4x\ee^{\int \dd{x}}\dd{x}+C\right) =\ee^{-x}\left( \int 4x\ee^{x}\dd{x}+C\right) \\
			&=\ee^{-x}\left( 4(x-1)\ee^{x}+C\right) =4(x-1)+C\ee^{-x}
		\end{align*}
	\end{solution}
\end{ti}

\begin{ti}
	求曲线 $2z+1=\ln(xy)+\ee^z$ 在点 $M_{0}(1,1,0)$ 处的切平面和法线方程
\end{ti}

\begin{ti}
	设由方程组 $\begin{cases}
		x+y+z=0\\
		x^2+y^2+z^2=1
		\end{cases}$
	确定了隐函数 $x=x(z),y=y(z)$ , 求 $\frac{\dd{x}}{\dd{z}},\frac{\dd{y}}{\dd{z}}$
	\begin{solution}
		对方程组 $\begin{cases}
		x+y+z=0\\
		x^2+y^2+z^2=1
		\end{cases}$ 两式求微分, 得
		\begin{equation*}
			\begin{cases}
			\dd{x}+\dd{y}+\dd{z}=0\\
			2x\dd{x}+2y\dd{y}+2z\dd{z}=0
			\end{cases}
		\end{equation*}
		解得
		\begin{equation*}
			\begin{cases}
			\frac{\dd{x}}{\dd{z}}=-\frac{x+2z}{2x+z}\\
			\frac{\dd{y}}{\dd{z}}=-\frac{y+2x}{2y+z}
			\end{cases}
		\end{equation*}
	\end{solution}
\end{ti}

\begin{ti}
	求方程 $y''+6y'+13y=\ee^t$ 的通解
	\begin{solution}
		方程 $y''+6y'+13y=\ee^t$ 对应的齐次方程 $y''+6y'+13y=0$ 的特征方程为 $r^2+6r+13=0$ , 解得 $r=-3\pm2\ii$, 那么齐次方程的通解为 $C_1\ee^{-3t}\sin(2t)+C_2\ee^{-3t}\cos(2t)$
		
		设特解为 $a\ee^{t}$, 代入方程 $y''+6y'+13y=\ee^t$ 后解得 $a=\frac{1}{20}$
		
		综上, 方程 $y''+6y'+13y=\ee^t$ 的通解为 $C_1\ee^{-3t}\sin(2t)+C_2\ee^{-3t}\cos(2t)+\frac{\ee^x}{20}$
	\end{solution}
\end{ti}

\begin{ti}
	设 $z=x^2y+\sin x+\varphi(xy+1)$ , 且 $\varphi(u)$ 具有一阶连续导数, 求 $\frac{\partial z}{\partial x},\frac{\partial z}{\partial y}$
	\begin{solution}
		$\frac{\partial z}{\partial x}=2xy+\cos x+y\varphi'(xy+1)$ , $\frac{\partial z}{\partial y}=x^2+x\varphi'(xy+1)$
	\end{solution}
\end{ti}

\subsection{2019-2020}
线上考试
\subsubsection{选择题}
\begin{ti}
	$y = \ee^{2x}$ 是微分方程 $y'' + py' + 6y = 0$ 的一个特解, 则此方程的通解是 \kuo
	\twoch{$y=C_1 \ee^{2x} + C_2 \ee^{-3x}$}{$y = (C_1 + xC_2) \ee^{2x}$}{$y = C_1 \ee^{2x} + C_2 \ee^{3x}$}{$y = \ee^{2x} (C_1 \sin 3x + C_2 \cos 3x)$}
\end{ti}

\begin{ti}
	非齐次线性微分方程 $x'' - 2x' + x = (t+2) \ee^{2t}$ 的一个待定的解形式 $x^\astt = $ \kuo
	\fourch{$t(At+B)\ee^{2t}$}{$(At+B)\ee^{2t}$}{$At^2 \ee^{2t}$}{$t^2 (At+B)\ee^{2t}$}
\end{ti}

\begin{ti}
	方程 $y' + 3xy = 6x^2$ 是 \kuo
	\twoch{可分离变量的微分方程}{齐次微分方程}{一阶线性非齐次微分方程}{二阶微分方程}
\end{ti}

\begin{ti}
	微分方程 $\bigl(y'\bigr)^4 + 3 \sqrt{y''} + y^2 = \sin x$ 的阶数是 \kuo
	\fourch{1}{2}{3}{4}
\end{ti}

\begin{ti}
	微分方程 $y' = \frac{y}{x}$ 的通解是 \kuo
	\fourch{$y = C\ee^{x}$}{$\ee^{y} = x$}{$\ee^{y} = Cx$}{$y = Cx$}
\end{ti}

\begin{ti}
	方程 $y'' - 2y' - 3y = 0$ 的通解是 \kuo
	\twoch{$y = C_1 \ee^{-x} + C_2 \ee^{3x}$}{$y = \frac{C_1}{x} + C_2 x^2$}{$y = C_1 \ee^{x} + C_2 \ee^{-x}$}{$y = C_1 x + \frac{C_2}{x^3}$}
\end{ti}

\begin{ti}
	已知向量 $\vec a = (0,3,4), \vec b = (2,1,-2)$, 则 $\Prj_{\vec b} \vec a = $ \kuo
	\fourch{5}{$-\frac{1}{3}$}{$-\frac{5}{3}$}{$\frac{1}{3}$}
\end{ti}

\begin{ti}
	平面 $y + z = 1$ 的位置是 \kuo
	\fourch{过 $x$ 轴}{垂直于 $x$ 轴}{平行于 $x$ 轴}{平行于 $yOz$ 面}
\end{ti}

\begin{ti}
	设 $\vec a = (a_x,a_y,a_z), \vec b = (b_x,b_y,b_z)$ 均为非零向量, 且 $a_x b_x + a_y b_y + a_z b_z = 0$, 则 \kuo
	\fourch{$\vec a \parallel \vec b$}{$\vec a + \vec b = \vec 0$}{$\vec a = \lambda \vec b (\lambda \ne 0)$}{$\vec a \perp \vec b$}
\end{ti}

\begin{ti}
	两向量 $\vec a, \vec b$ 平行的充要条件是 \kuo
	\fourch{$\vec a \cdot \vec b = \vec 0$}{$\vec a \cdot \vec b = 0$}{$\vec a \times \vec b = \vec 0$}{$\vec a \times \vec b = 0$}
\end{ti}

\begin{ti}
	设 $\vec a = (2,4,-1), \vec b = (0,-2,2)$, 则同时与 $\vec a$、$\vec b$ 垂直的单位向量 $\vec n = $ \kuo
	\twoch{$6 \vec i - 4 \vec j - 4 \vec k$}
	{$- 6 \vec i + 4 \vec j + 4 \vec k$}
	{$\frac{1}{\bigl| \vec a + \vec b \bigr|} \bigl( 6 \vec i - 4 \vec j - 4 \vec k \bigr)$}
	{$\frac{\pm 1}{\bigl| \vec a \times \vec b \bigr|} \bigl( 6 \vec i - 4 \vec j - 4 \vec k \bigr)$}
\end{ti}

\begin{ti}
	下列平面中通过坐标原点的平面是 \kuo
	\twoch{$x=1$}{$x + 2z + 3y + 4 = 0$}{$3(x-1) - y + (z+3) = 0$}{$x + y + z = 1$}
\end{ti}

\begin{ti}
	两向量 $\vec a, \vec b$ 平行的充要条件是 \kuo
	\fourch{$\vec a \cdot \vec b = \vec 0$}{$\vec a \cdot \vec b = 0$}{$\vec a \times \vec b = \vec 0$}{$\vec a \times \vec b = 0$}
\end{ti}

\begin{ti}
	曲线 $\begin{cases}
		x = t, \\
		y = 2t^2, \\
		z = 3t^3,
	\end{cases}$ 在点 $(1,2,3)$ 处的一个切向量为 \kuo
	\fourch{$(1,2,3)$}{$(2,4,6)$}{$(1,4,9)$}{$(1,4,8)$}
\end{ti}

\begin{ti}
	设 $f(x,y) = x^2 + (y-3) \arctan \frac{x}{y}$, 则 $f_x(2,3) = $ \kuo
	\fourch{1}{2}{3}{4}
\end{ti}

\begin{ti}
	函数 $z = x^2 + y^2$ 在点 $(0,0)$ 处 \kuo
	\twoch{连续, 偏导数存在}{连续, 偏导数不存在}{不连续, 但偏导数存在}{不连续, 且偏导数不存在}
\end{ti}

\begin{ti}
	若 $z = \arctan \frac{x}{y}$, 则 $\dd{z} = $ \kuo
	\fourch{$\frac{x \dd{y} - y \dd{x}}{x^2+y^2}$}{$\frac{x \dd{x} - y \dd{y}}{x^2+y^2}$}{$\frac{y \dd{x} + x \dd{y}}{x^2+y^2}$}{$\frac{y \dd{x} - x \dd{y}}{x^2+y^2}$}
\end{ti}

\begin{ti}
	设 $z = x^2 y + 3x y^2 + x$, 则 $\frac{\partial^2 z}{\partial x \partial y} = $ \kuo
	\fourch{$x+3y$}{$2x+3y$}{$2x+6y$}{$2x+6y+1$}
\end{ti}

\begin{ti}
	设 $z = x^y$, 则 $\frac{\partial^2 z}{\partial y^2} = $ \kuo
	\fourch{$x^y$}{$\frac{x^y}{\ln^2x}$}{$y (y-1) x^{y-2}$}{$x^y \ln^2x$}
\end{ti}

\begin{ti}
	$z = 3x^2y - xy^3 + 8$ 在 $A(1,2)$ 处沿 $A$ 到 $B(3,0)$ 方向的方向导数 $\frac{\partial z}{\partial l} = $ \kuo
	\fourch{$\frac{5\sqrt{2}}{2}$}{$-\frac{5\sqrt{2}}{2}$}{$-\frac{13\sqrt{2}}{2}$}{$\frac{13\sqrt{2}}{2}$}
\end{ti}

\subsubsection{计算题}
\begin{ti}
	求方程 $y'' - 4y' + 4y = 2xe^{2x}$ 的通解
\end{ti}

\begin{ti}
	求圆锥面 $z = \smqty({x^2+y^2})$ 与平面 $z = 2$ 所围立体在 $xOy$ 平面上的投影区域
\end{ti}

\begin{ti}
	求函数 $f(x,y) = x^3 - y^3 - 3xy + 1$ 的极值
\end{ti}

\begin{ti}
	求曲面 $z = y + \ln x - \ln z$ 在点 $M_0(1,1,1)$ 处的切平面和法线方程
\end{ti}

\begin{ti}
	求方程 $\frac{\dd{y}}{\dd{x}} = \frac{y - 3x}{x}$ 的通解
\end{ti}

\begin{ti}
	设 $z = \ee^{u+v}$, 而 $u = xy$, $v = x - y$, 求 $\frac{\partial z}{\partial x}, \frac{\partial z}{\partial y}$
\end{ti}

\subsection{难度与考试近似的题}
\subsubsection{选择题}
\begin{ti}
	下列平面中通过坐标原点的平面是 \kuo
	\twoch{$3(x - 1) - y + (z + 3) = 0$}{$x + y + z = 1$}{$x = 1$}{$x + 2z + 3y + 4 = 0$}
\end{ti}

\begin{ti}
	直线 $L: \frac{x + 3}{-2} = \frac{y + 4}{-7} = \frac{z}{3}$ 与平面 $\pi: 4x - 2y - 2z = 3$ 的关系是 \kuo
	\fourch{$L$ 在 $\pi$ 上}{相交但不垂直}{平行}{垂直相交}
\end{ti}

\begin{ti}
	设 $f(x,y) = xy + (2y - 1) \arccos \frac{x}{y}$, 则 $f_{x}(1,2) = $ \kuo
	\fourch{$2 + 2\sqrt{3}$}{$2 + \sqrt{3}$}{$2 - 2\sqrt{3}$}{$2 - \sqrt{3}$}
\end{ti}

\begin{ti}
	设 $z = y^{x}$, 则 $\frac{\partial^{2}z}{\partial x \partial y} = $ \kuo
	\fourch{$y^{x-1} (1 + x \ln y)$}{$y^{x} \ln^{2}x$}{$x y^{x - 1} \ln x$}{$y^{x - 1} (x + \ln y)$}
\end{ti}

\begin{ti}
	$z = 3x^{2}y - xy^{3} + 8$ 在 $A(1,2)$ 处沿 $A$ 到 $B(3,0)$ 方向的方向导数 $\frac{\partial z}{\partial l} = $ \kuo
	\fourch{$-\frac{13 \sqrt{2}}{2}$}{$\frac{13 \sqrt{2}}{2}$}{$\frac{5 \sqrt{2}}{2}$}{$- \frac{5 \sqrt{2}}{2}$}
\end{ti}

\begin{ti}
	微分方程 $(y'')^{3} + 3y''' + y^{4} = x$ 的阶数是 \kuo
	\fourch{1}{2}{3}{4}
\end{ti}

\begin{ti}
	微分方程 $y' = y$ 的通解是 \kuo
	\fourch{$y = c \ee^{-\frac{x}{2}}$}{$y = c \ee^{-x}$}{$cy = \ee^{x^{2}}$}{$y = c\ee^{x}$}
\end{ti}

\begin{ti}
	方程 $y' + 3xy = 6x^{2}$ 是 \kuo
	\twoch{一阶线性非齐次微分方程}{二阶微分方程}{可分离变量的微分方程}{齐次微分方程}
\end{ti}

\begin{ti}
	设 $\vec{a} = (a_{x},a_{y},a_{z}), \vec{b} = (b_{x},b_{y},b_{z})$ 均为非零向量, 且 $a_{x}b_{x} + a_{y}b_{y} + a_{z}b_{z} = 0$, 则 \kuo
	\fourch{$\vec{a} = \lambda \vec{b} (\lambda \ne 0)$}{$\vec{a} \perp \vec{b}$}{$\vec{a} \parallel \vec{b}$}{$\vec{a} + \vec{b} = \vec{0}$}
\end{ti}

\subsubsection{填空题}
\begin{ti}
	设 $z = \cos(x - y)$, 则 $\frac{\partial z}{\partial y} = $ \hua
\end{ti}

\begin{ti}
	设 $f(x,y) = \arctan \frac{y}{x}$, 则 $f_{x}(1,1) = $ \hua
\end{ti}

\begin{ti}
	设 $z = 2x^{2}y + \sin(xy)$, 则 $\frac{\partial^{2}z}{\partial x^{2}} = $ \hua
\end{ti}

\begin{ti}
	微分方程 $y'' \sin x - y' = \ln x$ 的通解中所含相互独立的任意常数的个数为 \hua
\end{ti}

\begin{ti}
	方程 $\frac{\dd{y}}{\dd{x}} = \frac{\bigl( \frac{y}{x} \bigr)^{2}}{\frac{y}{x} - 1}$ 的通解是 $y = $ \hua
\end{ti}

\begin{ti}
	微分方程 $y'' - 5y' + 4y = 0$ 的通解为 $y = $ \hua
\end{ti}

\begin{ti}
	设 $|\vec{a}| = 3, \bigl| \vec{b} \bigr| = 4$, 且 $\vec{a} \perp \vec{b}$, 则 $\bigl| \bigl( \vec{a} + \vec{b} \bigr) \times \bigl( \vec{a} - \vec{b} \bigr) \bigr| = $ \hua
\end{ti}

\begin{ti}
	直线 $\begin{cases}
		3x - 2y + z + 1 = 0 \\
		2x + y - z - 2 = 0
	\end{cases}$ 的方向向量是 \hua
\end{ti}

\begin{ti}
	曲线 $\begin{cases}
		y^{2} + z^{2} - 2x = 0 \\
		z = 2
	\end{cases}$ 在 $xoy$ 面上的投影曲线方程是 \hua
\end{ti}

\subsubsection{综合题}
\begin{ti}[6 分]
	求方程 $\frac{\dd{y}}{\dd{x}} = y + \cos x$ 的通解
\end{ti}

\begin{ti}[10 分]
	求方程 $s'' - 4s = t + 1$ 的通解
\end{ti}

\begin{ti}[10 分]
	设函数 $z = z(x,y)$ 由方程 $\cos x + 3y - z = \ee^{z}$ 所确定, 求 $\dd{z}$
\end{ti}

\begin{ti}[10 分]
	求曲面 $\ee^{z} - z + x^{2}y = 3$ 在点 $(1,2,0)$ 处的切平面及法线方程
\end{ti}

\begin{ti}[10 分]
	某工厂生产两种产品甲和乙, 出售单价分别为 $10$ 元与 $9$ 元, 生产 $x$ 单位的产品甲与生产 $y$ 单位的产品乙的总费用是 $300 + 3x + 2y + 0.01 (3x^{2} + xy + 3y^{2})$ 元, 求取得最大利润时, 两种产品的产量各为多少单位?
\end{ti}


\section{《高等数学(二)》期末}

\subsection{2017-2018 A}
\subsubsection{选择题(每小题 $3$ 分,共 $24$ 分)}
\begin{ti}
	微分方程 $y''-6 y'+9 y=\left(6 x^{2}+2\right) \ee^{x}$ 的待定特解的一个形式可为 \kuo
	\twoch{$y^{*}=\left(a x^{2}+b x+c\right) \ee^{x}$}{$y^{*}=x\left(a x^{2}+b x+c\right) \ee^{x}$}{$y^{*}=x^{2}\left(a x^{2}+b x+c\right) \ee^{x}$}{$y^{*}=x^{2}\left(x^{2}+1\right) \ee^{x}$}
\end{ti}

\begin{ti}
	设向量 $\vec{a}$ 的三个方向角为 $\alpha$、$\beta$、$\gamma$ , 且已知 $\alpha=60^{\circ}$、$\beta=120^{\circ}$ , 则 $\gamma=$ \kuo
	\fourch{$120^{\circ}$}{$60^{\circ}$}{$45^{\circ}$}{$30^{\circ}$}
\end{ti}

\begin{ti}
	设 $z=\arctan \ee^{x y}$ , 则 $\frac{\partial z}{\partial y}=$ \kuo
	\fourch{$-\frac{x \ee^{x y}}{\sqrt{1-\ee^{2 x y}}}$}{$\frac{x \ee^{x y}}{\sqrt{1-\ee^{2 x y}}}$}{$-\frac{x \ee^{x y}}{1+\ee^{2 x y}}$}{$\frac{x \ee^{x y}}{1+\ee^{2 x y}}$}
\end{ti}

\begin{ti}
	$D$ 为平面区域 $x^{2}+y^{2} \leqslant 4$ , 利用二重积分的性质, $\iint_{D}\left(x^{2}+4 y^{2}+9\right) \dd{x} \dd{y}$ 的最佳估值区间为 \kuo
	\fourch{$[36 \uppi, 52 \uppi]$}{$[36 \uppi, 100 \uppi]$}{$[52 \uppi, 100 \uppi]$}{$[9 \uppi, 25 \uppi]$}
\end{ti}

\begin{ti}
	设 $\Omega=\left\{(x, y, z) | x^{2}+y^{2}+z^{2} \leqslant 2, x \geqslant 0\right\}$ , 则以下等式错误的是 \kuo
	\fourch{$\iiint_{\Omega} x^{2} y \dd{V}=0$}{$\iiint_{\Omega}(x+y) \dd{V}=0$}{$\iiint_{\Omega} z \dd{V}=0$}{$\iiint_{\Omega} x y \dd{V}=0$}
\end{ti}

\begin{ti}
	设 $L$ 为直线 $y=y_0$ 上从点 $A(0,y_0)$ 到点 $B(3,y_0)$ 的有向直线段, 则 $\int_{L} 2 \dd{y}=$ \kuo
	\fourch{$6$}{$6y_0$}{$3y_0$}{$0$}
\end{ti}

\begin{ti}
	$\Sigma$ 为平面 $x+y+z=1$ 与三坐标面所围区域表面的外侧, 则 $\iint_{\Sigma}(2 y+3 z) \dd{y} \dd{z}+(x+2 z) \dd{z} \dd{x}+(y+1) \dd{x} \dd{y}=$ \kuo
	\fourch{$0$}{$\frac{1}{6}$}{$\frac{2}{3}$}{$\frac{5}{3}$}
\end{ti}

\begin{ti}
	交错级数 $\sum_{n=1}^{\infty}(-1)^{n-1} \frac{1}{3^{n-1}}$ \kuo
	\fourch{发散}{条件收敛}{绝对收敛}{无法确定}
\end{ti}

\subsubsection{填空题(每空 $3$ 分,共 $24$ 分)}
\begin{ti}
	以 $y_{1}=\ee^{x}, y_{2}=x \ee^{x}$ 为特解的阶数最低的常系数齐次线性微分方程是 \hua{}
\end{ti}

\begin{ti}
	直线 $L:\begin{cases}
		x=3t-2\\
		y=t+2\\
		z=2t-1
	\end{cases}$ 和平面 $\pi:2 x+3 y+3 z-8=0$ 的交点是 \hua{}
\end{ti}

\begin{ti}
	设 $z=xy^3$ , 则 $\dd{z}=$ \hua{}
\end{ti}

\begin{ti}
	交换二次积分的积分次序后, $\int_{0}^{2} \dd{y} \int_{y^{2}}^{2 y} f(x, y) \dd{x}=$ \hua{}
\end{ti}

\begin{ti}
	设 $\Omega=\{-1 \leqslant x \leqslant 1,-1 \leqslant y \leqslant 3,0 \leqslant z \leqslant 2\}$ , 则 $\iiint_{\Omega} \dd{x} \dd{y} \dd{z}=$ \hua{}
\end{ti}

\begin{ti}
	设 $L$ 为由三点 $(0,0),(3,0),(3,2)$ 围成的平面区域 $D$ 的正向边界曲线, 由格林公式知 $\int_{L}(3 x-y+4) \dd{x}+(5 y+3 x-6) \dd{y}=$ \hua{}
\end{ti}

\begin{ti}
	设 $\Sigma$ 是上半圆锥面 $z=\sqrt{x^{2}+y^{2}}(0 \leqslant z \leqslant 1)$ , 则曲面积分 $\iint_{\Sigma}\left(x^2+y^2\right)\dd{s}=$ \hua{}
\end{ti}

\begin{ti}
	级数 $\sum_{n=1}^{\infty}\left(\frac{1}{n(n+1)}-\frac{1}{2^{n}}\right)$ 的和为 \hua{}
\end{ti}

\subsubsection{综合题($8$ 小题,共 $52$ 分)}
\begin{ti}[$6$ 分]
	求方程 $\frac{\dd{y}}{\dd{x}}=\frac{x y}{1+x^{2}}$ 的通解
\end{ti}

\begin{ti}[$6$ 分]
	设 $z=\ln \left(x^{2}-y\right)$ , 而 $y=\tan x$ , 求 $\frac{\dd{z}}{\dd{x}}$
\end{ti}

\begin{ti}[$6$ 分]
	计算 $\iint_{D}\left(x^{2}+y^{2}\right) \dd{x} \dd{y}$ , $D$ 为曲线 $x^{2}-2 x+y^{2}=0, y=0$ 围成的在第一象限的闭区域
\end{ti}

\begin{ti}[$6$ 分]
	计算三重积分 $\iiint_{\Omega} z \dd{x} \dd{y} \dd{z}$ , 其中 $\Omega$ 是由圆锥面 $z=\sqrt{x^{2}+y^{2}}$ 与球面 $z=\sqrt{2-x^{2}-y^{2}}$ 围成的区域
\end{ti}

\begin{ti}[$8$ 分]
	用高斯公式计算 $\oiint_{\Sigma}\left(a^{2} x+x^{3}\right) \dd{y} \dd{z}+y^{3} \dd{z} \dd{x}+z^{3} \dd{x} \dd{y}$ , 其中 $\Sigma$ 为球面 $x^{2}+y^{2}+z^{2}=a^{2}$ , 取外侧
\end{ti}

\begin{ti}[$8$ 分]
	用格林公式计算 $\oint_{C} x^{2} y \dd{x}-x y^{2} \dd{y}$ , 其中 $C$ 为圆周 $x^2+y^2=4$ , 取正向
\end{ti}

\begin{ti}[$6$ 分]
	判断级数 $\sum_{n=1}^{\infty} \frac{1}{2^{n-1}(2 n-1)}$ 的敛散性
\end{ti}

\begin{ti}[$6$ 分]
	在区间 $(-1,1)$ 内求幂级数 $\sum_{n=1}^{\infty} \frac{x^{n}}{n}$ 的和函数 $s(x)$
\end{ti}

\subsection{2018-2019}
\subsubsection{选择题}
\begin{ti}
	设 $z = x^y$, 则 $\frac{\partial z}{\partial y} = $ \kuo
	\fourch{$x^y$}{$yx^{y-1}$}{$x^y \ln x$}{$\frac{x^y}{\ln x}$}
\end{ti}

\begin{ti}
	下列级数中收敛的是 \kuo
	\fourch{$\sum_{n=1}^{+\infty} \frac{1}{n}$}{$\sum_{n=1}^{+\infty}(-1)^n$}{$\sum_{n=1}^{+\infty}\frac{n}{2n+1}$}{$\sum_{n=1}^{+\infty} (-1)^n \frac{1}{n}$}
\end{ti}

\begin{ti}
	设有界闭区域 $D$ 由分段光滑曲线 $L$ 所围成, $L$ 取正向, 函数 $P(x,y)$, $Q(x,y)$ 在 $D$ 上具有一阶连续偏导数, 则 $\oint_L P \dd{x} + Q \dd{y} = $ \kuo
	\fourch{
		$\iint_D \Bigl( \frac{\partial Q}{\partial x} - \frac{\partial P}{\partial y} \Bigr) \dd{x} \dd{y}$
	}{
		$\iint_D \Bigl( \frac{\partial P}{\partial x} - \frac{\partial Q}{\partial y} \Bigr) \dd{x} \dd{y}$
	}{
		$\iint_D \Bigl( \frac{\partial P}{\partial y} - \frac{\partial Q}{\partial x} \Bigr) \dd{x} \dd{y}$
	}{
		$\iint_D \Bigl( \frac{\partial Q}{\partial y} - \frac{\partial P}{\partial x} \Bigr) \dd{x} \dd{y}$
	}
\end{ti}

\begin{ti}
	非齐次线性微分方程 $x'' - 2x' + x = (t + 2) \ee^{2t}$ 的一个待定特解形式 $x^* = $ \kuo
	\fourch{$t( At+B ) \ee{2t}$}{$(At + B) \ee{2t}$}{$A t^2 \ee^{2t}$}{$t^2 (At + B) \ee{2t}$}
\end{ti}

\begin{ti}
	设 $D$ 是由 $y=x$, $y=2x$ 及 $y=2$ 围成的, 那么 $\iint_D f(x,y) \dd{x} \dd{y} = $ \kuo
	\fourch{
		$\int_0^2 \dd{y} \int_{\frac{y}{2}}^y f(x,y) \dd{x}$
	}{
		$\int_0^2 \dd{y} \int_y^{\frac{y}{2}} f(x,y) \dd{x}$
	}{
		$\int_0^2 \dd{x} \int_{2x}^2 f(x,y) \dd{y}$
	}{
		$\int_0^2 \dd{x} \int_x^{2x} f(x,y) \dd{y}$
	}
\end{ti}

\begin{ti}
	设 $\Sigma$ 是圆柱面 $x^2 + y^2 = a^2$ 与平面 $z=1$、$z=3$ 围成区域的表面, 取外侧, 则曲面积分 $\oiint_\Sigma \bigl( xy^2 + z \bigr) \dd{x} \dd{y} + \bigl( x^2 z + y \bigr) \dd{x} \dd{z} + \bigl( x + yz^2 \bigr) \dd{y} \dd{z} = $ \kuo
	\fourch{
		$2 \uppi a^2$
	}{
		$6 \uppi a^2$
	}{
		$12 \uppi a^2$
	}{$0$}
\end{ti}

\begin{ti}
	过 $(0,2,4)$ 且与两平面 $x+2z=1$ 和 $y-3z=2$ 平行的直线方程是 \kuo
	\fourch{
		$\frac{x}{1} = \frac{y-2}{0} = \frac{z-4}{2}$
	}{
		$\frac{x}{0} = \frac{y-2}{1} = \frac{z-4}{-3}$
	}{
		$\frac{x}{-2} = \frac{y-2}{-3} = \frac{z-4}{1}$
	}{
		$\frac{x}{-2} = \frac{y-2}{3} = \frac{z-4}{1}$
	}
\end{ti}

\begin{ti}
	设积分区域 $D: 1 \leqslant x^2 + y^2 \leqslant 4$, 则二重积分 $\iint_D \sqrt{x^2 + y^2} \dd{x} \dd{y} = $ \kuo
	\fourch{
		$\int_0^{2\uppi} \dd{\theta} \int_{\rho}^4 \dd{\rho}$
	}{
		$\int_0^{2\uppi} \dd{\theta} \int_1^2 \rho \dd{\rho}$
	}{
		$\int_0^{2\uppi} \dd{\theta} \int_1^2 \rho^2 \dd{\rho}$
	}{
		$\int_0^{2\uppi} \dd{\theta} \int_0^1 \rho^2 \dd{\rho}$
	}
\end{ti}

\subsubsection{填空题}
\begin{ti}
	微分方程 $y'=3y$ 的通解是 $y=$ \hua{}
\end{ti}

\begin{ti}
	已知向量 $\vec{a}$ 与 $\vec{b}$ 方向相反, 且 $\left|\vec{b}\right|=3|\vec{a}|$ , 则 $\vec{b}$ 由 $\vec{a}$ 表示为 $\vec{b}=$ \hua{}
\end{ti}

\begin{ti}
	设 $z=x^3 y+\sin(x+y)$ , 则 $\frac{\partial z}{\partial x}=$ \hua{}
\end{ti}

\begin{ti}
	交换二次积分的积分次序后, $\int_0^1 \dd{x}\int_{x^3}^{x} f(x,y)\dd{y}=$ \hua{}
\end{ti}

\begin{ti}
	设 $\Omega=\left\{ 0\leqslant x\leqslant 3, 0\leqslant y\leqslant \frac{\uppi}{2}, 0\leqslant z\leqslant 2 \right\}$ , 则 $\iiint_{\Omega}xz\sin y\dd{x}\dd{y}\dd{z}=$ \hua{}
\end{ti}

\begin{ti}
	$L$ 为曲线 $y=x^2$ 上从点 $(0,0)$ 到点 $(1,1)$ 的一段弧, 则 $\int_{L}\sqrt{y}\dd{s}=$ \hua{}
\end{ti}

\begin{ti}
	$L$ 为圆周 $x^2+y^2=a^2$ 的正向边界曲线, 由格林公式知 $\oint_{L} \bigl( x^2y\cos x+2xy\sin x-y^2 \ee^x \bigr)\dd{x} + \bigl( x^2\sin x-2y\ee^x + x \bigr)\dd{y}=$ \hua{}
\end{ti}

\begin{ti}
	若 $p$ 满足条件 \hua{}, 则级数 $\sum_{n=1}^{+\infty}\frac{1}{n^{p-3}}$ 一定发散
\end{ti}

\subsubsection{综合题}
\begin{ti}
	求过点 $(3,1,3)$ , 且与直线 $
	\begin{cases}
		x-2y+2z-7=0\\
		x+5y-z+1=0
	\end{cases}
	$ 垂直的平面方程
\end{ti}

\begin{ti}
	已知 $z=\ee^{2x-y}$ , 而 $x=\sin t,y=t^2$, 求 $\frac{\dd{z}}{\dd{t}}$
\end{ti}

\begin{ti}
	计算 $\iint_{D} x\ee^{-y^2}\dd{x}\dd{y}$, $D$ 是曲线 $x=0,y=x^2,y=\sqrt{2}$ 围成的在第一象限的闭区域
\end{ti}

\begin{ti}
	计算以 $xoy$ 面上的圆周 $x^2+y^2=4$ 围成的闭区域为底, 以旋转抛物面 $z=x^2+y^2$ 为顶的曲顶柱体的体积
\end{ti}

\begin{ti}
	计算曲面积分 $\iint_{\Sigma}\left( x^2+y^2 \right)\dd{s}$ , 其中 $\Sigma$ 为锥面 $z=\sqrt{x^2+y^2}$ 被平面 $z=2$ 截下的带锥顶部分
\end{ti}

\begin{ti}
	利用高斯公式计算 $I=\iint_{\Sigma}\left( x^3 z+2x \right)\dd{y}\dd{z}-x^2 yz\dd{z}\dd{x}-x^2 z^2\dd{x}\dd{y}$ , 其中 $\Sigma$ 是由开口向下的旋转抛物面 $z=2-x^2-y^2$ 与平面 $z=1$ 围成立体 $\Omega$ 的表面, 取外侧
\end{ti}

\begin{ti}
	判断级数 $\sum_{n=1}^{+\infty}\frac{2\cdot 4\cdot 6\cdot\cdots\cdot (2n)}{1\cdot 4\cdot 7\cdot\cdots\cdot (3n-2)}$ 的敛散性
\end{ti}

\begin{ti}
	求级数 $\sum_{n=1}^{+\infty}nx^{n+1}(x\in(-1,1))$ 的和函数
\end{ti}

\subsection{2019-2020 11A}
\subsubsection{选择题(每小题 $3$ 分,共 $24$ 分)}
\begin{ti}
	方程 $y''-3 y'+2 y=\ee^{x}$ 的待定特解 $y^*$ 的一个形式是 $y^*=$ \kuo
	\fourch{$\ee^x$}{$ax^2\ee^x$}{$a\ee^x$}{$ax\ee^x$}
\end{ti}

\begin{ti}
	过点 $(3,1,-2)$ 且通过直线 $\frac{x-4}{5}=\frac{y+3}{2}=\frac{z}{1}$ 的平面方程 \kuo
	\twoch{$5x+2y+z-15=0$}{$\frac{x-3}{8}=\frac{y-1}{-9}=\frac{z+2}{-22}$}{$8x-9y-22z-59=0$}{$\frac{x-3}{5}=\frac{y-1}{2}=\frac{z+2}{1}$}
\end{ti}

\begin{ti}
	设 $f(x,y)=\ln\left(x+\frac{y}{2x} \right)$ , 则 $f_{y}(1,0)=$ \kuo
	\fourch{$1$}{$\frac{1}{2}$}{$\frac{1}{3}$}{$0$}
\end{ti}

\begin{ti}
	$D=\{ (x,y)|0\leqslant x\leqslant 1,0\leqslant y\leqslant 2 \}$ , 利用二重积分的性质, $\iint_{D}\frac{1}{\sqrt{x^2+y^2+2xy+16}}\dd{x}\dd{y}$ 的最佳估值区间为 \kuo
	\fourch{$\left[ \frac{2}{5},\frac{1}{2} \right]$}{$\left[ \frac{1}{5},\frac{1}{2} \right]$}{$\left[ \frac{2}{5},1 \right]$}{$\left[ \frac{1}{2},1 \right]$}
\end{ti}

\begin{ti}
	$\Omega$ 由柱面 $x^2+y^2=1$ 、平面 $z=1$ 及三个坐标面围成的在第一卦限内的闭区域, 则 $\iiint_{\Omega}xy\dd{V}=$ \kuo
	\twoch{$\int_{0}^{\uppi}\dd{\theta}\int_{0}^{1}\dd{\rho}\int_{0}^{1}\rho^3\sin\theta\cos\theta\dd{z}$}{$\int_{0}^{2\uppi}\int_{0}^{1}\dd\rho\int_{0}^{1}\rho^2\sin\theta\cos\theta\dd{z}$}{$\int_{0}^{\frac{\uppi}{2}}\dd{\theta}\int_{0}^{1}\dd\rho\int_{0}^{1}\rho^2\sin\theta\cos\theta\dd{z}$}{$\int_{0}^{\frac{\uppi}{2}} \dd{\theta} \int_{0}^{1} \dd{\rho} \int_{0}^{1} \rho^{3} \sin \theta \cos \theta \dd{z}$}
\end{ti}

\begin{ti}
	设 $L$ 是 $xoy$ 平面上的有向曲线, 下列曲线积分中, \kuo 是与路径无关的
	\twoch{$\int_{L} 3 y x^{2} \dd{x}+x^{3} \dd{y}$}{$\int_{L} y \dd{x}-x \dd{y}$}{$\int_{L} 2 x y \dd{x}-x^{2} \dd{y}$}{$\int_{L} 3 y x^{2} \dd{x}+y^{3} \dd{y}$}
\end{ti}

\begin{ti}
	设 $L$ 为圆周 $\begin{cases}
		x=a\cos t\\
		y=a\sin t
	\end{cases}(0\leqslant t\leqslant 2\uppi)$ , 则 $\oint_{L}\left(x^{2}+y^{2}\right) \dd{s}=$ \kuo
	\fourch{$a^3$}{$\uppi a^3$}{$2\uppi a^3$}{$3\uppi a^3$}
\end{ti}

\begin{ti}
	下列级数中收敛的是 \kuo
	\fourch{$\sum_{n=1}^{\infty} \frac{n}{n+1}$}{$\sum_{n=1}^{\infty} \frac{1}{n \sqrt{n+1}}$}{$\sum_{n=1}^{\infty} \frac{1}{2(n+1)}$}{$\sum_{n=1}^{\infty} \frac{1}{\sqrt{n+1}}$}
\end{ti}

\subsubsection{填空题(每空 $3$ 分, 共 $24$ 分)}
\begin{ti}
	微分方程 $\frac{\dd{y}}{\dd{x}}=-3 y+\ee^{2 x}$ 的通解是 $y=$ \hua{}
\end{ti}

\begin{ti}
	平行于 $y$ 轴且通过曲线 $\begin{cases}
		x^{2}+y^{2}+4 z^{2}=1\\
		x^{2}=y^{2}+z^{2}
	\end{cases}$ 的柱面方程是 \hua{}
\end{ti}

\begin{ti}
	设 $z=x^{2} y+x y^{2}$ , 则 $\dd{z}=$ \hua{}
\end{ti}

\begin{ti}
	$\iint_{D} y^{2} \sin ^{3} x \dd{x} \dd{y}=$ \hua{}(区域 $D$ 为: $-4 \leqslant x \leqslant 4,-1 \leqslant y \leqslant 1$)
\end{ti}

\begin{ti}
	设 $D$ 为平面闭区域: $x^{2}+y^{2} \leqslant 1$ , 则 $\iint_{D} \sqrt{x^{2}+y^{2}} \dd{x} \dd{y}$ 化为极坐标系下二次积分的表达式为 \hua{}
\end{ti}

\begin{ti}
	设 $L$ 是任意一条分段光滑的有向闭曲线, 则 $\oint_{L} 2 x y \dd{x}+x^{2} \dd{y}=$ \hua{}
\end{ti}

\begin{ti}
	$I=\iint_{\Sigma}(x+z \sin y) \dd{y} \dd{z}+(y+x \sin z) \dd{z} \dd{x}+z \dd{x} \dd{y}=$ \hua{}, 其中 $\Sigma$ 为球面 $x^{2}+y^{2}+z^{2}=4(z \geqslant 0)$ 与平面 $z=0$ 围成区域的表面, 取外侧
\end{ti}

\begin{ti}
	级数 $\sum_{n=1}^{\infty}(-1)^{n} \frac{1}{n} x^{n}$ 的收敛半径为 \hua{}
\end{ti}

\subsubsection{综合题(请写出求解过程,$8$ 小题,共 $52$ 分)}
\begin{ti}[$6$ 分]
	求过点 $(2,1,1)$ , 且与直线 $\begin{cases}
		x-y+3 z-7=0\\
		3 x+5 y-2 z+1=0
	\end{cases}$ 垂直的平面方程
\end{ti}

\begin{ti}[$6$ 分]
	设 $z=f\left(\ee^{x+y}, \sin (x y)\right)$ , 且 $f$ 具有一阶连续偏导数, 求 $\frac{\partial z}{\partial x}, \frac{\partial z}{\partial y}$
\end{ti}

\begin{ti}[$8$ 分]
	计算 $\iint_{D}\left(x^{2}+y\right) \dd{x} \dd{y}$ , $D$ 是曲线 $y=x^{2}, x=y^{2}$ 围成的闭区域
\end{ti}

\begin{ti}[$6$ 分]
	计算 $\iiint_{\Omega}\left(x^{2}+y^{2}\right) \dd{x} \dd{y} \dd{z}$ , 其中 $\Omega$ 是由圆锥面 $z^{2}=x^{2}+y^{2}$ 及平面 $z=2$ 围成的闭区域
\end{ti}

\begin{ti}[$6$ 分]
	计算 $\int_{\Gamma} x^{3} \dd{x}+3 z y^{2} \dd{y}-x^{2} y \dd{z}$ , 其中 $\Gamma$ 是从点 $A(2,2,1)$ 到原点 $O$ 的直线段 $AO$
\end{ti}

\begin{ti}[$8$ 分]
	空间区域 $\Omega$ 由开口向下的旋转抛物面 $z=1-x^{2}-y^{2}$ 与平面 $z=0$ 所围, $\Omega$ 的表面取外侧为 $\Sigma$ , 利用高斯公式计算 $\oiint_{\Sigma} x^{2} y z^{2} \dd{y} \dd{z}-x y^{2} z^{2} \dd{z} \dd{x}+z(1+x y z) \dd{x} \dd{y}$
\end{ti}

\begin{ti}[$6$ 分]
	判断级数 $\sum_{n=1}^{\infty} \frac{n^{\ee}}{\ee^{n}}$ 的敛散性
\end{ti}

\begin{ti}[$6$ 分]
	求幂级数 $\sum_{n=0}^{\infty}(2 n+1) x^{2 n}(x \in(-1,1))$ 的和函数
\end{ti}

\subsection{2019-2020 5A}
\subsubsection{选择题}
\begin{ti}
	微分方程 $y' = 2y$ 的一个特解是 \kuo
	\fourch{$y = \ee^{x^2}$}{$y = \ee^{2x}$}{$y = \ee^{-\frac{x}{2}}$}{$y = \ee^{-x}$}
\end{ti}

\begin{ti}
	方程 $\frac{x^2}{4} + \frac{y^2}{4} + \frac{z^2}{9} = 1$ 在三维空间中表示 \kuo
	\fourch{椭球面}{柱面}{抛物面}{平面}
\end{ti}

\begin{ti}
	若 $z = x^4 y^3 + 2x$, 则 $\dd{z}|_{(1,2)} =$ \kuo
	\fourch{$\bigl( 4x^3 y^3 + 2 \bigr)\dd{x} + 3x^4 y^2 \dd{y}$}{$12 \dd{x} + 34 \dd{y}$}{$34 \dd{x} + 12 \dd{y}$}{$34 \dd{x} - 12 \dd{y}$}
\end{ti}

\begin{ti}
	设积分区域 $D:$ $x^2+y^2 \leqslant 1$, $y \geqslant 0$, 则二重积分 $\iint_D \ee^{\sqrt{x^2 + y^2}} \dd{x} \dd{y} =$ \kuo
	\fourch{$\int_0^{2\uppi} \dd{\theta} \int_0^1 \rho \ee^{\rho} \dd{\rho}$}{$\int_0^{2\uppi} \dd{\theta} \int_0^1 \ee^{\rho} \dd{\rho}$}{$\int_0^{\uppi} \dd{\theta} \int_0^1 \ee^{\rho} \dd{\rho}$}{$\int_0^{\uppi} \dd{\theta} \int_0^1 \rho \ee^{\rho} \dd{\rho}$}
\end{ti}

\begin{ti}
	设 $\Omega = \bigl\{ (x,y,z) | x^2 + y^2 + z^2 \leqslant 2, x \geqslant 0 \bigr\}$, 则以下等式错误的是 \kuo
	\fourch{$\iiint_{\Omega} x^2 y \dd{v} = 0$}{$\iiint_{\Omega} (x+y) \dd{v} = 0$}{$\iiint_{\Omega} z \dd{v} = 0$}{$\iiint_{\Omega} xy \dd{v} = 0$}
\end{ti}

\begin{ti}
	设 $L$ 是椭圆 $\frac{x^2}{4} + \frac{y^2}{3} = 1$, 其周长记为 $a$, 则 $\oint_L \bigl( 3x^2 + 4y^2 \bigr) \dd{S} =$ \kuo
	\fourch{$6a$}{$3a$}{$12a$}{$0$}
\end{ti}

\begin{ti}
	$L$ 为平面曲线 $x = a \cos t$, $y = b \sin t$ ($0 \leqslant t \leqslant 2\uppi$) 围成区域的边界, 取正向, 则 $\int_L y \dd{x} - x \dd{y} =$ \kuo
	\fourch{$\uppi ab$}{$-\uppi ab$}{$2\uppi ab$}{$-2\uppi ab$}
\end{ti}

\begin{ti}
	$\sum_{n=0}^{\infty} (-1)^n \frac{x^{2n}}{n!}$ 的和函数 $S(x) =$ \kuo
	\fourch{$\ee^{-x^2}$}{$\ee^{x^2}$}{$-\ee^{x^2}$}{$\cos x$}
\end{ti}

\subsubsection{填空题}
\begin{ti}
	以 $\ee^{2x}$, $\ee^{3x}$ 为解的阶数最低的常系数线性齐次微分方程是 \hua{}
\end{ti}

\begin{ti}
	过点 $(1,1,1)$ 与直线 $\begin{cases}
		3x - y + 2z + 2 = 0 \\
		x - 2y + 3z - 5 = 0
	\end{cases}$ 的平面方程 \hua{}
\end{ti}

\begin{ti}
	设 $z = 2\ee^{xy}$, 则 $\frac{\partial^2 z}{\partial x \partial y} =$ \hua{}
\end{ti}

\begin{ti}
	若 $D = \{ 0 \leqslant x \leqslant 2, 0 \leqslant y \leqslant 4 \}$, 则积分 $\iint_D xy \dd{x} \dd{y} =$ \hua{}
\end{ti}

\begin{ti}
	设 $\Omega$ 是圆锥面 $z = \sqrt{x^2+y^2}$ 和球面 $z = \sqrt{4-x^2-y^2}$ 围成, 则 $\iiint_{\Omega} y \sqrt{x^2+y^2} \dd{x} \dd{y} \dd{z} =$ \hua{}
\end{ti}

\begin{ti}
	设 $L$ 为 $x$ 轴上横坐标从 $1$ 到 $2$ 的一段, 则 $\int_L \bigl( 2y - x^2 \bigr) \dd{x} + \bigl( -2x + y^2 \bigr) \dd{y} =$ \hua{}
\end{ti}

\begin{ti}
	$\Sigma$ 为圆锥面 $x^2 + y^2 = z^2$ 与平面 $z = h$ ($h > 0$) 围成区域的表面, 取外侧. 则 $\iint_{\Sigma} (x + y - z) \dd{y} \dd{z} + (z - x) \dd{z} \dd{x} + (x - y) \dd{x} \dd{y} =$ \hua{}
\end{ti}

\begin{ti}
	级数 $\sum_{n=1}^{\infty} \frac{\sin n \alpha}{(n+1)^2}$ 是 \hua{} 收敛(填绝对或条件收敛)
\end{ti}

\subsubsection{综合题}
\begin{ti}
	求过点 $(3,-2,0)$ 且与直线 $\frac{x-1}{1} = \frac{y+1}{-1} = \frac{z-2}{1}$ 平行的直线方程
\end{ti}

\begin{ti}
	设 $z = \ee^{x+y} + \sin xy$, 求 $\frac{\partial z}{\partial x}$, $\frac{\partial^2z}{\partial x \partial y}$
\end{ti}

\begin{ti}
	计算 $\iint_D y \dd{x} \dd{y}$, $D$ 是曲线 $y = x^2$, $y = 1$ 围成的区域
\end{ti}

\begin{ti}
	求平面 $z = 2$ 与球面 $x^2 + y^2 + z^2 = 8$ 所围成在平面 $z = 2$ 之上的部分的体积 $V$
\end{ti}

\begin{ti}
	计算 $\iint_{\varSigma} \Bigl( z + 2x + \frac{4}{3}y \Bigr) \dd{S}$, 其中 $\varSigma$ 为平面 $\frac{x}{2} + \frac{y}{3} + \frac{z}{4} = 1$ 在第一卦限的部分
\end{ti}

\begin{ti}
	利用高斯公式计算 $\iint_{\varSigma} yz \dd{z} \dd{x} + 2 \dd{x} \dd{y}$, $\varSigma$ 是半球球面 $x^2 + y^2 + z^2 = 4$ ($z \geqslant 0$) 与 $z = 0$ 围成区域 $\Omega$ 的表面, 取外侧
\end{ti}

\begin{ti}
	判断级数 $\sum_{n=1}^{\infty} n \Bigl( \frac{3}{5} \Bigr)^n$ 的敛散性
\end{ti}

\begin{ti}
	求级数 $\sum_{n=1}^{\infty} n x^n$ ($x \in (-1,1)$) 的和函数
\end{ti}

\subsection{难度与考试近似的题}
\subsubsection{选择题}
\begin{ti}
	微分方程 $y'=p(x) y$ 的通解是 \kuo
	\fourch{$y=\ee^{\int p(x) \dd{x}}$}{$y=C \ee^{\int-p(x) \dd{x}}$}{$y=C \ee^{\int p(x) \dd{x}}$}{$y=C p(x)$}
\end{ti}

\begin{ti}
	已知曲线 $\begin{cases}
		x^{2}+y^{2}+z^{2}=2\\
		x+y+z=a
	\end{cases}$ 在 $yoz$ 坐标面上的投影曲线为 $\begin{cases}
		y^{2}+y z+z^{2}=1\\
		x=0
	\end{cases}$, 则 $a=$ \kuo
	\fourch{$-1$}{$0$}{$1$}{$2$}
\end{ti}

\begin{ti}
	设 $z=\ee^{y} \tan x$, 则 $\dd{z}=$ \kuo
	\twoch{$\ee^{y} \tan x \dd{x}+\ee^{y} \sec ^{2} x \dd{y}$}{$\frac{\ee^{y}}{1+x^{2}} \dd{x}+\ee^{y} \tan x \dd{y}$}{$\ee^{x} \tan y \dd{x}+\ee^{x} \sec ^{2} y \dd{y}$}{$\ee^{y} \sec ^{2} x \dd{x}+\ee^{y} \tan x \dd{y}$}
\end{ti}

\begin{ti}
	设积分区域 $D : x^{2}+y^{2} \leqslant 4$ , 则二重积分 $\iint_{D} \sqrt{x^{2}+y^{2}} \dd{x} \dd{y}=$ \kuo
	\fourch{$\int_{0}^{2 \uppi} \dd{\theta} \int_{0}^{2} \rho^{2} \dd{\rho}$}{$\int_{0}^{2 \uppi} \dd{\theta} \int_{\rho}^{4} \dd{\rho}$}{$\int_{0}^{2 \uppi} \dd{\theta} \int_{0}^{1} \rho^{2} \dd{\rho}$}{$\int_{0}^{2 \uppi} \dd{\theta} \int_{1}^{2} \rho \dd{\rho}$}
\end{ti}

\begin{ti}
	设 $\Omega$ 由圆锥面 $z=1-\sqrt{x^{2}+y^{2}}$ 与平面 $z=0$ 围成的闭区域, 则 $\iiint_{\Omega} z \dd{V}=$ \kuo
	\twoch{$\int_{0}^{\uppi} \dd{\theta} \int_{0}^{1} \rho \dd{\rho} \int_{0}^{1-\rho} z \dd{z}$}{$\int_{0}^{2 \uppi} \dd{\theta} \int_{0}^{1} \dd{\rho} \int_{0}^{1-\rho} z \dd{z}$}{$\int_{0}^{\uppi} \dd{\theta} \int_{0}^{1} \dd{\rho} \int_{0}^{1-\rho} z \dd{z}$}{$\int_{0}^{2 \uppi} \dd{\theta} \int_{0}^{1} \rho \dd{\rho} \int_{0}^{1-\rho} z \dd{z}$}
\end{ti}

\begin{ti}
	设 $L$ 为圆周 $\begin{cases}
		x=a \cos t\\
		y=a \sin t
	\end{cases}(0\leqslant t\leqslant 2\uppi)$ , 则 $\oint_{L}\left(x^{2}+y^{2}\right) \dd{s}=$ \kuo
	\fourch{$a^3$}{$\uppi a^3$}{$2\uppi a^3$}{$3\uppi a^3$}
\end{ti}

\begin{ti}
	$L$ 为平面闭区域: $-1 \leqslant x \leqslant 1,0 \leqslant y \leqslant 1$ 的正向边界, 则 $\int_{L}\left(\frac{1}{2} y+3 x \ee^{x}\right) \dd{x}-\left(\frac{1}{2} x-y \sin y\right) \dd{y}=$ \kuo
	\fourch{$-2$}{$2$}{$-1$}{$1$}
\end{ti}

\begin{ti}
	设幂级数 $\sum_{n=1}^{\infty} a_{n} x^{n}$ 的收敛半径为 $R(0<R<+\infty)$ , 则幂级数 $\sum_{n=1}^{\infty} a_{n}\left(\frac{x}{2}\right)^{n}$ 的收敛半径为 \kuo
	\fourch{$\frac{R}{2}$}{$2R$}{$R$}{$\frac{2}{R}$}
\end{ti}

\begin{ti}
	微分方程 $\frac{\dd^{2} y}{\dd{x}^{2}}-3 \frac{\dd{y}}{\dd{x}}+2 y=x \ee^{3 x}$ 的待定特解 $y^{*}$ 的一个形式是 \kuo
	\twoch{$y^{*}=(a x+b)+c \ee^{3 x}$}{$y^{*}=(a x+b)+c x \ee^{3 x}$}{$y^{*}=(a x+b) \ee^{3 x}$}{$y^{*}=(a x+b) x \ee^{3 x}$}
\end{ti}

\begin{ti}
	过点 $(3,2,-7)$  且在三坐标轴上的截距相等, 则此平面方程是 \kuo
	\fourch{$x+y+z+2=0$}{$z+y+z-2=0$}{$x-y+z-2=0$}{$x-y-z-2=0$}
\end{ti}

\begin{ti}
	设 $L$ 是平面有向曲线, 下列曲线积分中, \kuo 是与路径无关的
	\twoch{$\int_{L}\left(y \ee^{x}+x^{2}-y\right) \dd{x}+\left(x+\ee^{x}-2 y^{2}\right) \dd{y}$}{$\int_{L}(\cos x+y) \dd{x}+(x+\cos y) \dd{y}$}{$\int_{L}(\cos x-y) \dd{x}+(x+\cos y) \dd{y}$}{$\int_{L}\left(\frac{1}{2} y+3 x \ee^{x}\right) \dd{x}-\left(\frac{1}{2} x-y \sin y\right) \dd{y}$}
\end{ti}

\begin{ti}
	设 $\Sigma$ 是平面 $x=1, y=1, z=1$ 与三个坐标面围成区域的表面, 取外侧, 则曲面积分 $\iint_{\Sigma} 2 x \dd{y} \dd{z}+2 z \dd{z} \dd{x}+3 y \dd{x} \dd{y}=$ \kuo
	\fourch{$0$}{$2$}{$4$}{$7$}
\end{ti}

\begin{ti}
	级数 $1+\left(\frac{1}{2}\right)^{2}+\left(\frac{1}{3}\right)^{2}+\cdots+\left(\frac{1}{n}\right)^{2}+\cdots$ 是 \kuo
	\fourch{幂级数}{调和级数}{$p$ 级数}{等比级数}
\end{ti}

\begin{ti}
	方程 $\left(3 x^{2}+y \cos x\right) \dd{x}+\left(\sin x-4 y^{3}\right) \dd{y}=0$ 是 \kuo
	\twoch{可分离变量微分方程}{一阶线性方程}{全微分方程}{\textsf{A}、\textsf{B}、\textsf{C} 均不对}
\end{ti}

\begin{ti}
	$z=f(x, y)$ 在 $\left(x_{0}, y_{0}\right)$ 可微, 则 $\frac{\partial z}{\partial x}, \frac{\partial z}{\partial y}$ 在 $\left(x_{0}, y_{0}\right)$ \kuo
	\fourch{连续}{不连续}{不一定存在}{一定存在}
\end{ti}

\begin{ti}
	级数 $\sum_{n=2}^{\infty}\left(\frac{1}{\sqrt{n}-1}-\frac{1}{\sqrt{n}+1}\right)$ 是 \kuo
	\fourch{发散}{收敛}{条件收敛}{绝对收敛}
\end{ti}

\begin{ti}
	曲面 $z=\sqrt{x^{2}+y^{2}}$ 与平面 $z=1$ 所围立体的体积为 \kuo
	\twoch{$\iiint_{\Omega}\left(x^{2}+y^{2}\right) \dd{V}$}{$\int_{0}^{2 \pi} \dd{\theta} \int_{0}^{1} r \dd{r} \int_{r}^{1} \dd{z}$}{$\int_{-1}^{1} \dd{x} \int_{-\sqrt{1-x^{2}}}^{\sqrt{1-x^{2}}} \dd{y} \int_{0}^{x^{2}+y^{2}} \dd{z}$}{$\int_{0}^{2 \pi} \dd{\theta} \int_{0}^{1} r \dd{r} \int_{0}^{1} \dd{z}$}
\end{ti}

\begin{ti}
	方程 $y''-3 y'+2 y=3 x-\ee^{x}$ 的特解形式为 \kuo
	\fourch{$(a x+b) \ee^{x}$}{$a x+b+c x \ee^{x}$}{$a x+b+c \ee^{x}$}{$(a x+b) x \ee^{x}$}
\end{ti}

\begin{ti}
	设 $\overrightarrow{AB}$ 与 $u$ 轴的夹角为 $\frac{\uppi}{3}$ , 则 $\overrightarrow{AB}$ 在 $u$ 轴上的投影是 \kuo
	\fourch{$\overrightarrow{AB}\cos\frac{\uppi}{3}$}{$\overrightarrow{AB}\sin\frac{\uppi}{3}$}{$\left|\overrightarrow{AB}\right|\cos\frac{\uppi}{3}$}{$\left|\overrightarrow{AB}\right|\sin\frac{\uppi}{3}$}
\end{ti}

\begin{ti}
	过点 $M_{1}(3,-2,1), M_{2}(-1,0,2)$ 的直线方程是 \kuo
	\twoch{$-4(x-3)+2(y+2)+(z-1)=0$}{$\frac{x-3}{4}=\frac{y+2}{2}=\frac{z-1}{1}$}{$\frac{x+1}{4}=\frac{y}{2}=\frac{z-2}{1}$}{$\frac{x-3}{4}=\frac{y+2}{-2}=\frac{z-1}{-1}$}
\end{ti}

\begin{ti}
	直线 $\begin{cases}
		x+y+3 z=0\\
		x-y-z=0
	\end{cases}$ 与平面 $x-y-z+1=0$ 的夹角是 \kuo
	\fourch{$60^\circ$}{$0^\circ$}{$30^\circ$}{$90^\circ$}
\end{ti}

\begin{ti}
	设 $f(x,y)=\begin{cases}
		\frac{1}{xy}\sin\left(x^2y\right), & \text{当}xy\ne0,\\
		0, & \text{当}xy=0,
	\end{cases}$ 则当 $y\ne0$ 时, $f_{x}(0,y)=$ \kuo
	\fourch{$0$}{$1$}{$2$}{不存在}
\end{ti}

\begin{ti}
	曲线 $\begin{cases}
		z=\frac{x^2}{2}+\frac{y^2}{4},\\
		y=2.
	\end{cases}$ 在点 $\left( 1,2,\frac{3}{2} \right)$ 处的切线与 $x$ 轴的正向所成的倾角是 \kuo
	\fourch{$\arctan 1$}{$30^\circ$}{$60^\circ$}{$90^\circ$}
\end{ti}

\begin{ti}
	$D$ 是矩形闭区域 $0 \leqslant x \leqslant 1, \quad 0 \leqslant y \leqslant 2 \quad, I=\iint_{D}(x+y+1) \dd{x} \dd{y}$ , 利用二重积分的性质, $I$ 的最佳估计区间为 \kuo
	\fourch{$[0,1]$}{$[0,2]$}{$[1,3]$}{$2,8$}
\end{ti}

\begin{ti}
	$L$ 为 $y=x^2$ 上从 $A(1,1)$ 到 $B(0,0)$ 的一段弧, 则 $\int_{L}x\dd{y}=$ \kuo
	\fourch{$\int_{0}^{1} 2 x^{2} \dd{x}$}{$\int_{1}^{0} x \dd{y}$}{$\int_{1}^{0} 2 x^{2} \dd{x}$}{$\int_{0}^{1} \sqrt{y} \dd{y}$}
\end{ti}

\begin{ti}
	当 $\sum_{n=1}^{\infty}\left(a_{n}+b_{n}\right)$ 收敛时, $\sum_{n=1}^{\infty} a_{n}$ 与 $\sum_{n=1}^{\infty} b_{n}$ \kuo
	\fourch{可能不同时收敛}{不可能同时收敛}{必同时收敛}{必同时发散}
\end{ti}

\begin{ti}
	非齐次线性微分方程 $x''-2 x'+5 x=t \ee^{t} \sin 2 t$ 的特解形式 $x^*=$ \kuo
	\twoch{$(A t+B) \ee^{t} \sin 2 t$}{$\ee^{t}[(A t+B) \cos 2 t+(C t+D) \sin 2 t]$}{$t(A t+B) \ee^{t} \sin 2 t$}{$t \ee^{t}[(A t+B) \cos 2 t+(C t+D) \sin 2 t]$}
\end{ti}

\begin{ti}
	设向量 $\vec{a}=(1,2,3)$ 、 $\vec{b}=(2,0,1)$ , 则向量 $\vec{a}\times\vec{b}$ 在 $y$ 轴上的分向量为 \kuo
	\fourch{$5$}{$5\vec{j}$}{$-5$}{$-5\vec{j}$}
\end{ti}

\begin{ti}
	两向量 $\vec{a}$、$\vec{b}$ 平行的充要条件是 \kuo
	\fourch{$\vec{a}\cdot\vec{b}=0$}{$\vec{a}\times\vec{b}=0$}{$\vec{a}\cdot\vec{b}=\vec{0}$}{$\vec{a}\times\vec{b}=\vec{0}$}
\end{ti}

\begin{ti}
	$f(x,y)$ 在点 $(x_0,y_0)$ 处两个偏导数存在是 $f(x,y)$ 在 $(x_0,y_0)$ 处可微的 \kuo
	\fourch{必要条件}{充分条件}{充分必要条件}{以上都不是}
\end{ti}

\begin{ti}
	设上半球 $V=\left\{(x, y, z) | x^{2}+y^{2}+z^{2} \leqslant 1, z \geqslant 0\right\}$, 则以下等式错误的是 \kuo
	\fourch{$\iiint_{V} x \dd{V}=0$}{$\iiint_{V} y \dd{V}=0$}{$\iiint_{V} z \dd{V}=0$}{$\iiint_{V} xy \dd{V}=0$}
\end{ti}

\begin{ti}
	设 $f(x)=\begin{cases}
		x, & x\in[-\uppi,0)\\
		1, & x\in[0,\uppi)
	\end{cases}$ 的傅里叶级数的和函数为 $S(x)$, 则 $S(0)=$ \kuo
	\fourch{$0$}{$1$}{$\frac{1}{2}$}{$-\frac{1}{2}$}
\end{ti}

\begin{ti}
	级数 $\sum_{n=1}^{\infty}(-1)^{n} \frac{1}{n}$ \kuo
	\fourch{发散}{条件收敛}{绝对收敛}{以上都不对}
\end{ti}

\subsubsection{填空题}
\begin{ti}
	以 $\ee^x,x\ee^x$ 为解的阶数最低的常系数线性齐次微分方程是 \hua{}
\end{ti}

\begin{ti}
	过点 $A(1,-2,1)$ 且以 $\vec{n}=(1,2,3)$ 为法向量的平面方程是 \hua{}
\end{ti}

\begin{ti}
	设 $z=\sin \left(x^{2}+y\right)$ , 则 $\frac{\partial^{2} z}{\partial x \partial y}=$ \hua{}
\end{ti}

\begin{ti}
	设 $D$ 是圆环形闭区域 $1 \leqslant x^{2}+y^{2} \leqslant 4$ , 那么 $\iint_{D} \sqrt{x^{2}+y^{2}} \dd{\sigma}=$ \hua{}
\end{ti}

\begin{ti}
	设 $\Omega$ 为球体: $x^{2}+y^{2}+z^{2} \leqslant 4$ , 则 $\iiint_{\Omega} x^{2} \sin (y z) \dd{x} \dd{y} \dd{z}=$ \hua{}
\end{ti}

\begin{ti}
	$L$ 为抛物线 $x=y^2$ 上从点 $(1,-1)$ 到 $(1,1)$ 的一段弧, 则 $\int_{L} x y \dd{y}=$ \hua{}
\end{ti}

\begin{ti}
	$\oiint_{\Sigma}(x y+z) \dd{x} \dd{y}+(x z+y) \dd{x} \dd{z}+(x+y z) \dd{y} \dd{z}=$ \hua{}, 其中 $\Sigma$ 是由六张平面 $x=1,x=2,y=1,y=2,z=1,z=3$ 围成的六面体的表面, 取内侧
\end{ti}

\begin{ti}
	级数 $\frac{1}{3}+\frac{1}{\sqrt{3}}+\frac{1}{\sqrt[3]{3}}+\cdots+\frac{1}{\sqrt[n]{3}}+\cdots$ 是 \hua{}(填收敛或发散)
\end{ti}

\begin{ti}
	微分方程 $y'=p(x) y$ 的通解是 $y=$ \hua{}
\end{ti}

\begin{ti}
	设 $\vec{a}$ 与轴 $\vec{l}$ 的夹角为 $\frac{\uppi}{6}$, 且 $|\vec{a}|=4$, 则 $\Prj_{\vec{l}} \vec{a}=$ \hua{}
\end{ti}

\begin{ti}
	设 $f(x, y)=\tan \left(x y^{2}\right)$ , 则 $f_{x}(0,2)=$ \hua{}
\end{ti}

\begin{ti}
	交换二次积分次序的积分次序后, $\int_{1}^{2} \dd{x} \int_{2-x}^{\sqrt{2 x-x^{2}}} f(x, y) \dd{y}=$ \hua{}
\end{ti}

\begin{ti}
	已知 $\Omega$ 是由旋转抛物面 $z=x^{2}+y^{2}$ 与上半球面 $z=\sqrt{2-x^{2}-y^{2}}$ 围成的区域, 则 $\iiint_{\Omega} x y z \dd{x} \dd{y} \dd{z}=$ \hua{}
\end{ti}

\begin{ti}
	设 $\Sigma$ 是球面 $x^{2}+y^{2}+z^{2}=1$, 则 $\iint_{\Sigma}\left(x^{2}+y^{2}+z^{2}\right) \dd{s}=$ \hua{}
\end{ti}

\begin{ti}
	积分 $\oint_{L}\left(x^{2}-y\right) \dd{x}+\left(y^{2}+x\right) \dd{y}=$ \hua{}, 其中 $L$ 为圆周 $(x-1)^{2}+y^{2}=a^{2}$ 的正向
\end{ti}

\begin{ti}
	级数 $\sum_{n=1}^{\infty}(-1)^{n-1} \frac{1}{\sqrt{n}}$ 是 \hua{}收敛(填条件收敛或绝对收敛)
\end{ti}

\begin{ti}
	设 $z=x^{y}$ , 则 $\frac{\partial z}{\partial y}=$ \hua{}
\end{ti}

\begin{ti}
	积分 $\iint_{D} x y \dd{x} \dd{y}=$ \hua{}, 其中 $D$ 为 $0 \leqslant x \leqslant 2, 0 \leqslant y \leqslant 4$
\end{ti}

\begin{ti}
	$L$ 为 $y=x^2$ 点 $(0,0)$ 到 $(1,1)$ 的一段弧, 则 $\int_{L} \sqrt{y} \dd{s}=$ \hua{}
\end{ti}

\begin{ti}
	级数 $\sum_{n=1}^{\infty} \frac{(-1)^{n}}{n^{p}}$ 当 $p$ 满足 \hua{}时条件收敛
\end{ti}

\begin{ti}
	方程 $y \ee^{x} \dd{x}-\left(1+\ee^{x}\right) \dd{y}=0$ 的通解为 \hua{}
\end{ti}

\begin{ti}
	设 $z=\ln \sqrt{1+x^{2}+y^{2}}$, 则 $\dd\left.z\right|_{(1,1)}=$ \hua{}
\end{ti}

\begin{ti}
	函数 $z=x^{2}+y^{2}$ 在点 $P(1,2)$ 沿从点 $(1,2)$ 到点 $(2,2+\sqrt{3})$ 的方向上的方向导数为 \hua{}
\end{ti}

\begin{ti}
	改换二次积分的积分次序: $\int_{0}^{1} \dd{y} \int_{0}^{y} f(x, y) \dd{x}=$ \hua{}
\end{ti}

\begin{ti}
	平面 $x+y+z=1$ 含在圆柱面 $x^{2}+y^{2}=2 x$ 内部的那部分平面面积为 \hua{}
\end{ti}

\begin{ti}
	$L$ 为圆周 $x^{2}+y^{2}=1$ , 则 $\int_{L}\left(x^{2}+y^{2}\right) \dd{s}=$ \hua{}
\end{ti}

\begin{ti}
	$\Sigma$ 是 $xoy$ 平面上的圆域: $x^{2}+y^{2} \leqslant 1$, 取下侧, 则 $\iint_{\Sigma} \dd{x} \dd{y}=$ \hua{}
\end{ti}

\begin{ti}
	级数 $\sum_{n=1}^{\infty} \frac{3^{n}+4^{n}}{7^{n}}$ 的和为 \hua{}
\end{ti}

\begin{ti}
	$\ee^{x^{2}}$ 的 $x$ 的幂级数展开式为 \hua{}
\end{ti}

\begin{ti}
	微分方程 $x'''-2x''-x'+2x=0$ 的通解是 \hua{}
\end{ti}

\begin{ti}
	过点 $(1,0,1)$ 及以 $(1,2,3)$ 为方向向量的直线的对称式方程为 \hua{}
\end{ti}

\begin{ti}
	函数 $z=x^y$ 的全微分 $\dd{z}=$ \hua{}
\end{ti}

\begin{ti}
	二元函数 $u=x^{2}-x y+y^{2}$ 在点 $(-1,1)$ 处沿方向 \hua{}的方向导数最大
\end{ti}

\begin{ti}
	交换二次积分的次序 $\int_0^1 \dd{y}\int_{-1}^{-y} f(x,y)\dd{x}=$ \hua{}
\end{ti}

\begin{ti}
	若 $L$ 为抛物线 $y^2=2x$ 上介于 $(2,-2)$ 与 $(2,2)$ 两点间的曲线段, 则 $\int_{L} y\dd{s}=$ \hua{}
\end{ti}

\begin{ti}
	若 $\Sigma$ 是曲面 $x^2+y^2+z^2=1$, 则 $\iint_{\Sigma} \dd{s}=$ \hua{}
\end{ti}

\begin{ti}
	函数 $f(x)=3^x$ 的幂级数展开式为 \hua{}
\end{ti}

\subsubsection{综合题}
\begin{ti}[$6$ 分]
	求过点 $(2,0,-3)$ , 且过直线 $\begin{cases}
		x-2 y+4 z-7=0\\
		3 x+5 y-2 z+1=0
		\end{cases}$ 垂直的平面方程
\end{ti}

\begin{ti}[$6$ 分]
	设 $z=x^{y}(x>0)$ , 求 $\frac{\partial z}{\partial x}, \frac{\partial^{2} z}{\partial x \partial y}$
\end{ti}

\begin{ti}[$6$ 分]
	计算 $\iint_{D} x^{2} y^{2} \dd{x} \dd{y}$ , 其中 $D=\{(x, y) | 0 \leqslant x \leqslant 1,0 \leqslant y \leqslant 1\}$
\end{ti}

\begin{ti}[$6$ 分]
	计算 $I=\iiint_{\Omega}\left(x^{2}+y^{2}\right) \dd{V}$, 其中 $\Omega$ 为旋转抛物面 $z=x^{2}+y^{2}$ 与平面 $z=4$ 所围成的区域
\end{ti}

\begin{ti}[$8$ 分]
	$L$ 是圆环区域 $D : 1 \leqslant x^{2}+y^{2} \leqslant 4$ 的正向边界曲线, 计算曲线积分 $\oint_{L} \sqrt{x^{2}+y^{2}} \dd{x}+\Bigl[x y^{2}+y \ln \Bigl(x+\sqrt{x^{2}+y^{2}}\Bigr)\Bigr] \dd{y}$
\end{ti}

\begin{ti}[$8$ 分]
	计算 $\iint_{\Sigma} \frac{2}{z} \dd{s}$, 其中 $\Sigma$ 是球面 $x^{2}+y^{2}+z^{2}=1$ 在平面 $z=\frac{1}{2}$ 上方的部分
\end{ti}

\begin{ti}[$6$ 分]
	判断级数 $\sum_{n=1}^{\infty} \frac{3^{n}}{n \cdot 2^{n}}$ 的敛散性
\end{ti}

\begin{ti}[$6$ 分]
	求幂级数 $\sum_{n=0}^{\infty}(n+1) x^{n}$ 在收敛域 $(-1,1)$ 的和函数 $s(x)$
\end{ti}

\begin{ti}[$6$ 分]
	求过点 $(3,-2,1)$, 且与直线 $\frac{x-1}{1}=\frac{y+1}{1}=\frac{z-2}{3}$ 平行的直线方程
\end{ti}

\begin{ti}[$6$ 分]
	设 $ z = e^{xy} + \cos(x + y)$, 求 $\dd{z}$
\end{ti}

\begin{ti}[$6$ 分]
	计算 $\iint_{D}\frac{y}{x}\dd{x}\dd{y}$, $D$ 是由直线 $ y = 2x,y = x, x = 2, x = 4$ 围成的闭区域
\end{ti}

\begin{ti}[$6$ 分]
	计算 $\iiint_{\Omega}z\dd{x}\dd{y}\dd{z}$, 其中 $\Omega$ 由平面 $z = 3$ 与旋转抛物面 $x^2 + y^2 = 3z$ 围成的区域
\end{ti}

\begin{ti}[$6$ 分]
	计算 $\int_{L} 2 x y \dd{x}+x^{2} \dd{y}$, $L$ 为抛物线 $y=x^{2}$ 上从 $O(0,0)$ 到 $B(1,1)$ 的一段弧
\end{ti}

\begin{ti}[$8$ 分]
	利用高斯公式计算 $\oiint_{\Sigma} 2 x z \dd{y} \dd{z}+y z \dd{z} \dd{x}-z^{2} \dd{x} \dd{y}$, 其中 $\Sigma$ 为由上半圆锥面 $z=\sqrt{x^{2}+y^{2}}$ 与上半球面 $z=\sqrt{2-x^{2}-y^{2}}$ 所围立体 $\Omega$ 的表面, 取外侧
\end{ti}

\begin{ti}[$6$ 分]
	判断级数 $\sum_{n=1}^{\infty} n 2^{n}$ 的敛散性
\end{ti}

\begin{ti}[$6$ 分]
	求幂级数 $\sum_{n=0}^{\infty}(n+1) x^{n}$ 在收敛域 $(-1,1)$ 的和函数 $s(x)$
\end{ti}

\begin{ti}[$8$ 分]
	$z=f\left(y^{2}-x^{2}\right)$, 其中 $f(u)$ 有连续的二阶偏导数, 求 $\frac{\partial^{2} z}{\partial x^{2}}$
\end{ti}

\begin{ti}[$8$ 分]
	计算 $\int_{L}\left(\ee^{x} \sin y-2 y\right) \dd{x}+\left(\ee^{x} \cos y-2\right) \dd{y}$, $L$ 为由点 $A(1,0)$ 到 $B(0,1)$, 再到 $C(-1,0)$ 的有向折线
\end{ti}

\begin{ti}[$8$ 分]
	计算 $\oiint_{\Sigma} x y^{2} \dd{y} \dd{z}+y z^{2} \dd{z} \dd{x}+z x^{2} \dd{x} \dd{y}$, 其中 $\Sigma$ 为球体 $x^{2}+y^{2}+z^{2} \leqslant 4$ 及锥体 $z=\sqrt{x^{2}+y^{2}}$ 的公共部分的外表面
\end{ti}

\begin{ti}[$8$ 分]
	求级数 $\sum_{n=2}^{\infty} 2 n x^{n}$ 的收敛域及和函数
\end{ti}

\begin{ti}[$8$ 分]
	计算曲面积分 $\iint_{\Sigma}\left(x^{2}+y^{2}\right) \dd{s}$, 其中 $\Sigma$ 为锥面 $z=\sqrt{3\left(x^2+y^2\right)}$ 被平面 $z=3$ 截下的带锥顶的部分
\end{ti}

\begin{ti}[$7$ 分]
	求函数 $z=x^2+y^2$ 在适合条件 $\frac{x}{2}+\frac{y}{3}=1$ 下的极小值
\end{ti}

\begin{ti}[$8$ 分]
	求方程 $y^{\prime \prime}-3 y^{\prime}+2 y=3 \ee^{x}$ 的通解
\end{ti}

\begin{ti}[$7$ 分]
	把 $f(x)=x,(0<x<\uppi)$ 展开为余弦级数
\end{ti}

\begin{ti}[$8$ 分]
	已知曲线积分 $\int_{(0,0)}^{(x, y)}\left[\ee^{x}(x+1)^{n}+\frac{n}{x+1} f(x)\right] y \dd{x}+f(x) \dd{y}$ 与路径无关, 其中 $f(x)$ 可微, $f(0)=0$, 试确定 $f(x)$ , 并计算曲线积分的值
\end{ti}

\begin{ti}[$5$ 分]
	求过点 $(2,5,-3)$ 且与直线 $\begin{cases}
		x=5-2t\\
		y=1+t\\
		z=7
	\end{cases}$ 垂直的平面方程
\end{ti}

\begin{ti}[$5$ 分]
	由 $\ee^{x}-x y z=0$ 确定了函数 $z=z(x,y)$, 求 $\frac{\partial z}{\partial x}$
\end{ti}

\begin{ti}[$5$ 分]
	计算 $I=\iint_{D}\left(x^{2}+y^{2}\right) \dd{x} \dd{y}$, 其中 $D=\left\{(x, y) | 1 \leqslant x^{2}+y^{2} \leqslant 4\right\}$
\end{ti}

\begin{ti}[$8$ 分]
	利用格林公式, 计算 $\oint_{L}\left(2 x^{2} y-2 y\right) \dd{x}+\left(\frac{1}{3} x^{3}-2 x\right) \dd{y}$, 其中 $L$ 为以 $y=x,y=x^{2}$, 围成区域的正向边界
\end{ti}

\begin{ti}[$8$ 分]
	设 $\Sigma$ 是由旋转抛物面 $z=x^{2}+y^{2}$ 与平面 $z=2$ 所围成的封闭曲面, 取外侧. 用高斯公式计算 $\iint_{\Sigma} 4\left(1-y^{2}\right) \dd{z} \dd{x}+z(8 y+1) \dd{x} \dd{y}$
\end{ti}

\begin{ti}[$8$ 分]
	求幂级数 $\sum_{n=0}^{\infty} \frac{x^{2 n+1}}{2 n+1}$ 在收敛域 $(-1,1)$ 内的和函数
\end{ti}

\begin{ti}[$8$ 分]
	求微分方程 $y''-2 y'+y=\ee^{x}$ 的通解
\end{ti}

\begin{ti}[$5$ 分]
	设函数 $f(x)$ 在 $[a,b]$ 上连续且 $f(x)>0$, 证明 $\int_{a}^{b} f(x) \dd{x} \int_{a}^{b} \frac{1}{f(x)} \dd{x} \geq(b-a)^{2}$
\end{ti}

\begin{ti}[$7$ 分]
	设 $u=f(x,xy)$, $f$ 具有二阶连续偏导数, 求 $\frac{\partial u}{\partial x},\frac{\partial^2u}{\partial x\partial y}$
\end{ti}

\begin{ti}[$7$ 分]
	求曲面 $\ee^{z}-z+x y=3$ 在点 $(2,1,0)$ 处的切平面及法线方程
\end{ti}

\begin{ti}[$10$ 分]
	设 $\Omega$ 是曲面 $\Sigma_1:z=\sqrt{x^2+y^2}$ 与 $\Sigma_2:2-x^2-y^2$ 所围成的立体, 求 $\Omega$ 的体积 $V$ 与表面积 $S$
\end{ti}

\begin{ti}[$10$ 分]
	计算 $\iint_{\Sigma}\left(z+x y^{2}\right) \dd{y} \dd{z}+\left(y z^{2}-x z\right) \dd{z} \dd{x}+\left(x^{2} z+x^{3}\right) \dd{x} \dd{y}$, 其中 $\Sigma$ 为 $x^{2}+y^{2}+z^{2}=4(z \leqslant 0)$, 取下侧
\end{ti}

\begin{ti}[$10$ 分]
	计算 $\int_{L}\left(2 x y^{3}-y^{2} \cos x\right) \dd{x}+\left(1-2 y \sin x+3 x^{2} y^{2}\right) \dd{y}$, 其中 $L$ 为抛物线 $2x=\uppi y^2$ 从点 $O(0,0)$ 到点 $A\left( \frac{\uppi}{2},1 \right)$ 的一段弧
\end{ti}

\begin{ti}[$8$ 分]
	求幂级数 $\sum_{n=1}^{\infty}nx^{n-1}$ 的收敛域与和函数
\end{ti}
	\chapter{线性代数试卷汇总}
	\chapter{概率统计试卷汇总}

\section{复习题 1}
\subsubsection{选择题(每题 $3$ 分, 共 $21$ 分)}
\begin{enumerate}
	\item 从 $0,1,2,\ldots,9$ 中任意选出 $3$ 个不同的数字, 三个数字中不含 $0$ 与 $5$ 的概率是 (\hspace{1pc})
	\fourch{$\frac{1}{15}$}
	{$\frac{2}{15}$}
	{$\frac{14}{15}$}
	{$\frac{7}{15}$}
	
	\item 某人射击中靶的概率为 $\frac{3}{4}$ . 若射击直到中靶为止, 则射击次数为 $3$ 的概率为 (\hspace{1pc})
	\fourch{$\left(\frac{3}{4}\right)^3$}
	{$\left(\frac{1}{4}\right)^2\times\frac{3}{4}$}
	{$\left(\frac{1}{4}\right)^3$}
	{$\left(\frac{3}{4}\right)^2\times\frac{1}{4}$}
	
	\item 设随机变量 $X$ 的概率密度 $f(x)$ 满足 $f(-x)=f(x)$ , $F(x)$ 是分布函数, 则 (\hspace{1pc})
	\twoch{$F(-a)=1-F(a)$}
	{$F(-a)=\frac{1}{2}F(a)$}
	{$F(-a)=F(a)$}
	{$F(-a)=\frac{1}{2}-F(a)$}
	
	\item 设二维随机变量 $(X,Y)$ 的分布律为 $P\left\{X=i,Y=j\right\}=c\cdot i\cdot j,i=1,2,3,j=1,2,3$ , 则 $c=$ (\hspace{1pc})
	\fourch{$\frac{1}{12}$}
	{$\frac{1}{3}$}
	{$\frac{1}{36}$}
	{$\frac{1}{2}$}
	
	\item 设随机变量 $X$ 服从均匀分布, 其概率密度为 $f(x)=
	\begin{cases}
	\frac{1}{2}, & 1<x<3\\
	0, & \text{其他}
	\end{cases}
	$ , 则 $D(X)=$ (\hspace{1pc})
	\fourch{$3$}
	{$\frac{1}{3}$}
	{$\frac{1}{2}$}
	{$2$}
	
	\item 设总体 $X\sim N\left(0,\sigma^2\right)$ , $X_1,X_2,\ldots,X_n$ 是总体 $X$ 的一个样本, $\overline{X},S^2$ 分别为样本均值和样本方差, 则下列样本函数中, 服从 $\chi^2(n)$ 分布的是 (\hspace{1pc})
	\fourch{$\sum_{i=1}^{n}X_i^2$}
	{$\frac{\overline{X}}{S/\sqrt{n-1}}$}
	{$\frac{(n-1)S^2}{\sigma^2}$}
	{$\frac{1}{\sigma^2}\sum_{i=1}^{n}X_i^2$}
	
	\item 设 $X_1,X_2,\ldots,X_n$ 是来自正态总体 $N\left(\mu,\sigma^2\right)$ 的一个样本, $\sigma^2$ 未知, $\overline{X}$是样本均值, $S^2=\frac{1}{n-1}\sum_{i=1}^{n}\left(X_i-\overline{X}\right)^2$ , 如果 $\overline{X}-k\frac{S}{\sqrt{n}}$ 是 $\mu$ 的置信度为 $1-\alpha$ 的单侧置信下限, 则 $k$ 应取 (\hspace{1pc})
	\fourch{$t_{1-\alpha}(n)$}
	{$t_{\alpha}(n)$}
	{$t_{\alpha}(n-1)$}
	{$t_{\alpha/2}(n-1)$}	
\end{enumerate}

\subsubsection{填空题(每题 $3$ 分, 共 $21$ 分)}
\begin{enumerate}
	\item 设 $A,B$ 为随机事件, $P(A)=0.8$ , $P(A-B)=0.3$ , 则 $P\left(\overline{AB}\right)=$\underline{\hspace{8pc}}
	
	\item 设随机变量 $X$ 的分布律为 $P\left\{x=k\right\}=c(0.5)^k,k=1,2,3,\ldots$ , 则常数 $c=$\underline{\hspace{8pc}}
	
	\item 设随机变量 $X$ 的概率密度为 $f(x)=
	\begin{cases}
	3x^2, & 0<x<1\\
	0, & \text{其他}
	\end{cases}
	$ , 则 $P\left\{\left|X\right|<0.2\right\}=$\underline{\hspace{8pc}}
	
	\item 设随机变量 $X$ 的概率密度为 $f(x)=
	\begin{cases}
	\frac{1}{c}, & 0<x<c\\
	0, & \text{其他}
	\end{cases}
	$ , 则 $\EE(X)=$\underline{\hspace{8pc}}
	
	\item 设二维随机变量 $(X,Y)$ 的概率密度为
	\begin{equation*}
		f(x,y)=
		\begin{cases}
		\sin x\cdot\cos y, & 0<x<\uppi/2,\ 0<y<\uppi/2\\
		0, & \text{其他}
		\end{cases}
		,
	\end{equation*}
	
	则 $P\left\{0<X<\uppi/4,\uppi/4<Y<\uppi/2\right\}=$\underline{\hspace{8pc}}
	
	\item 设随机变量 $X$ 的数学期望 $\EE(X)=\mu$ , 方差 $D(X)=\sigma^2$ , 则由切比雪夫不等式有 $P\{|X-\mu|\geq3\sigma\}\leq$\underline{\hspace{8pc}}
	
	\item 设 $X_1,X_2$ 是取自正态总体 $X\sim N\left(\mu,\sigma^2\right)$ 的一个容量为 $2$ 的样本, 则 $\mu$ 的无偏估计量 $\hat\mu_1=\frac{1}{2}X_1+\frac{1}{2}X_2$ , $\hat\mu_2=\frac{2}{3}X_1+\frac{1}{3}X_2$ , $\hat\mu_3=\frac{1}{4}X_1+\frac{3}{4}X_2$ 中最有效的是\underline{\hspace{8pc}}
\end{enumerate}

\subsubsection{解答题(共 $58$ 分)}
\begin{enumerate}
	\item ( $10$ 分)车间里有甲、乙、丙 $3$ 台机床生产同一种产品, 已知它们的次品率依次是 $0.05$ 、 $0.1$ 、 $0.2$ , 产品所占份额依次是 $20\%$ 、 $30\%$ 、 $50\%$ . 现从产品中任取 $1$ 件, 发现它是次品, 求次品来自机床乙的概率.
	
	\item ( $10$ 分)设随机变量 $X$ 的分布函数为 $F(x)=
	\begin{cases}
	k-k\ee^{-x^3}, & x>0\\
	0, & x\leq0
	\end{cases}
	$ , 试求:
	\begin{enumerate}
		\item 常数 $k$ ;
		\item $X$ 的概率密度 $f(x)$ .
	\end{enumerate}

	\item ( $10$ 分)设二维随机变量 $(X,Y)$ 的概率密度为:
	\begin{equation*}
		f(x,y)=
		\begin{cases}
		\frac{1}{4}, & 2\leq x\leq4,1\leq y\leq3\\
		0, & \text{其他}
		\end{cases},
	\end{equation*}
	试求 $(X,Y)$ 关于 $X$ 与 $Y$ 的边缘概率密度 $f_X(x)$ 与 $f_Y(y)$ , 并判断 $X$ 与 $Y$ 是否相互独立.
	
	\item ( $10$ 分)已知红黄两种番茄杂交的第二代结红果的植株与结黄果的植株的比率为 $3:1$ , 现种植杂交种 $400$ 株, 试用中心极限定理近似计算, 结红果的植株介于 $285$ 与 $315$ 之间的概率. $\left(\varPhi\left(\sqrt{3}\right)=0.9582,\varPhi\left(\sqrt{2}\right)=0.9207\right)$
	
	\item ( $8$ 分)设二维随机变量 $(X,Y)$ 的分布律为
	\begin{center}
		\begin{tabularx}{0.8\textwidth}{ZZZZ}
			\hline
			 & \multicolumn{3}{c}{$Y$}\\
			\cline{2-4}
			$X$ & $-1$ & $0$ & $1$\\
			\hline
			$-1$ & $\frac{1}{8}$ & $\frac{1}{8}$ & $\frac{1}{8}$\\
			$0$ & $\frac{1}{8}$ & $0$ & $\frac{1}{8}$\\
			$1$ & $\frac{1}{8}$ & $\frac{1}{8}$ & $\frac{1}{8}$\\
			\hline
		\end{tabularx}
	\end{center}
	求 $\mathrm{Cov}(X,Y)$ .
	
	\item ( $10$ 分)设 $X_1,X_2,\ldots,X_n$ 为总体 $X$ 的一个样本, 总体 $X$ 的概率密度为:
	\begin{equation*}
		f(x)=
		\begin{cases}
		(\alpha+1)x^\alpha, & 0<x<1\\
		0, & \text{其他}
		\end{cases},
	\end{equation*}
	求未知参数 $\alpha$ 的矩估计.
\end{enumerate}

\section{复习题 1 答案}
\subsubsection{选择题(每题 $3$ 分, 共 $21$ 分)}
\begin{enumerate}
	\item 从 $0,1,2,\ldots,9$ 中任意选出 $3$ 个不同的数字, 三个数字中不含 $0$ 与 $5$ 的概率是 (\hspace{0.25pc}D\hspace{0.25pc})
	\fourch{$\frac{1}{15}$}{$\frac{2}{15}$}{$\frac{14}{15}$}{$\frac{7}{15}$}
	
	\item 某人射击中靶的概率为 $\frac{3}{4}$ . 若射击直到中靶为止, 则射击次数为 $3$ 的概率为 (\hspace{0.25pc}B\hspace{0.25pc})
	\fourch{$\left(\frac{3}{4}\right)^3$}{$\left(\frac{1}{4}\right)^2\times\frac{3}{4}$}{$\left(\frac{1}{4}\right)^3$}{$\left(\frac{3}{4}\right)^2\times\frac{1}{4}$}
	
	\item 设随机变量 $X$ 的概率密度 $f(x)$ 满足 $f(-x)=f(x)$ , $F(x)$ 是分布函数, 则 (\hspace{0.25pc}A\hspace{0.25pc})
	\twoch{$F(-a)=1-F(a)$}{$F(-a)=\frac{1}{2}F(a)$}{$F(-a)=F(a)$}{$F(-a)=\frac{1}{2}-F(a)$}
	
	\item 设二维随机变量 $(X,Y)$ 的分布律为 $P\left\{X=i,Y=j\right\}=c\cdot i\cdot j,i=1,2,3,j=1,2,3$ , 则 $c=$ (\hspace{0.25pc}C\hspace{0.25pc})
	\fourch{$\frac{1}{12}$}{$\frac{1}{3}$}{$\frac{1}{36}$}{$\frac{1}{2}$}
	
	\item 设随机变量 $X$ 服从均匀分布, 其概率密度为 $f(x)=
	\begin{cases}
	\frac{1}{2}, & 1<x<3\\
	0, & \text{其他}
	\end{cases}
	$ , 则 $D(X)=$ (\hspace{0.25pc}B\hspace{0.25pc})
	\fourch{$3$}{$\frac{1}{3}$}{$\frac{1}{2}$}{$2$}
	
	\item 设总体 $X\sim N\left(0,\sigma^2\right)$ , $X_1,X_2,\ldots,X_n$ 是总体 $X$ 的一个样本, $\overline{X},S^2$ 分别为样本均值和样本方差, 则下列样本函数中, 服从 $\chi^2(n)$ 分布的是 (\hspace{0.25pc}D\hspace{0.25pc})
	\fourch{$\sum_{i=1}^{n}X_i^2$}{$\frac{\overline{X}}{S/\sqrt{n-1}}$}{$\frac{(n-1)S^2}{\sigma^2}$}{$\frac{1}{\sigma^2}\sum_{i=1}^{n}X_i^2$}
	
	\item 设 $X_1,X_2,\ldots,X_n$ 是来自正态总体 $N\left(\mu,\sigma^2\right)$ 的一个样本, $\sigma^2$ 未知, $\overline{X}$是样本均值, $S^2=\frac{1}{n-1}\sum_{i=1}^{n}\left(X_i-\overline{X}\right)^2$ , 如果 $\overline{X}-k\frac{S}{\sqrt{n}}$ 是 $\mu$ 的置信度为 $1-\alpha$ 的单侧置信下限, 则 $k$ 应取 (\hspace{0.25pc}C\hspace{0.25pc})
	\fourch{$t_{1-\alpha}(n)$}{$t_{\alpha}(n)$}{$t_{\alpha}(n-1)$}{$t_{\alpha/2}(n-1)$}	
\end{enumerate}

\subsubsection{填空题(每题 $3$ 分, 共 $21$ 分)}
\begin{enumerate}
	\item 设 $A,B$ 为随机事件, $P(A)=0.8$ , $P(A-B)=0.3$ , 则 $P\left(\overline{AB}\right)=$\underline{\hspace{1pc}$0.5$\hspace{1pc}}
	
	\item 设随机变量 $X$ 的分布律为 $P\left\{x=k\right\}=c(0.5)^k,k=1,2,3,\ldots$ , 则常数 $c=$\underline{\hspace{1pc}$1$\hspace{1pc}}
	
	\item 设随机变量 $X$ 的概率密度为 $f(x)=
	\begin{cases}
	3x^2, & 0<x<1\\
	0, & \text{其他}
	\end{cases}
	$ , 则 $P\left\{\left|X\right|<0.2\right\}=$\underline{\hspace{1pc}$\frac{1}{125}$\hspace{1pc}}
	
	\item 设随机变量 $X$ 的概率密度为 $f(x)=
	\begin{cases}
	\frac{1}{c}, & 0<x<c\\
	0, & \text{其他}
	\end{cases}
	$ , 则 $\EE(X)=$\underline{\hspace{1pc}$\frac{c}{2}$\hspace{1pc}}
	
	\item 设二维随机变量 $(X,Y)$ 的概率密度为
	\begin{equation*}
	f(x,y)=
	\begin{cases}
	\sin x\cdot\cos y, & 0<x<\uppi/2,\ 0<y<\uppi/2\\
	0, & \text{其他}
	\end{cases}
	,
	\end{equation*}
	
	则 $P\left\{0<X<\uppi/4,\uppi/4<Y<\uppi/2\right\}=$\underline{\hspace{1pc}$\left( \frac{2-\sqrt{2}}{2} \right)^2$\hspace{1pc}}
	
	\item 设随机变量 $X$ 的数学期望 $\EE(X)=\mu$ , 方差 $D(X)=\sigma^2$ , 则由切比雪夫不等式有 $P\{|X-\mu|\geq3\sigma\}\leq$\underline{\hspace{1pc}$\frac{1}{9}$\hspace{1pc}}
	
	\item 设 $X_1,X_2$ 是取自正态总体 $X\sim N\left(\mu,\sigma^2\right)$ 的一个容量为 $2$ 的样本, 则 $\mu$ 的无偏估计量 $\hat\mu_1=\frac{1}{2}X_1+\frac{1}{2}X_2$ , $\hat\mu_2=\frac{2}{3}X_1+\frac{1}{3}X_2$ , $\hat\mu_3=\frac{1}{4}X_1+\frac{3}{4}X_2$ 中最有效的是\underline{\hspace{1pc}$\hat\mu_1$\hspace{1pc}}
\end{enumerate}

\subsubsection{解答题(共 $58$ 分)}
\begin{enumerate}
	\item ( $10$ 分)车间里有甲、乙、丙 $3$ 台机床生产同一种产品, 已知它们的次品率依次是 $0.05$ 、 $0.1$ 、 $0.2$ , 产品所占份额依次是 $20\%$ 、 $30\%$ 、 $50\%$ . 现从产品中任取 $1$ 件, 发现它是次品, 求次品来自机床乙的概率.
	\begin{solution}
		设抽取的产品为次品的事件为 $A$ , 抽取的次品来自机床甲的事件为 $B_1$ , 抽取的次品来自机床乙的事件为 $B_2$ , 抽取的次品来自机床丙的事件为 $B_3$ .
		
		根据全概率公式
		\begin{equation*}
			\begin{aligned}
			P(A)&=P(A|B_1)P(B_1)+P(A|B_2)P(B_2)+P(A|B_3)P(B_3)\\
			&=0.05\times0.2+0.1\times0.3+0.2\times0.5=0.14
			\end{aligned}
		\end{equation*}
		根据贝叶斯公式
		\begin{equation*}
			P(B_2|A)=\frac{P(AB_2)}{P(A)}=\frac{P(A|B_2)P(B_2)}{P(A)}=\frac{0.1\times0.3}{0.14}=\frac{3}{14}
		\end{equation*}
	\end{solution}
	
	\item ( $10$ 分)设随机变量 $X$ 的分布函数为 $F(x)=
	\begin{cases}
	k-k\ee^{-x^3}, & x>0\\
	0, & x\leq0
	\end{cases}
	$ , 试求:
	\begin{enumerate}
		\item 常数 $k$ ;
		\item $X$ 的概率密度 $f(x)$ .
	\end{enumerate}
	\begin{solution}
	  \begin{enumerate}
		\item 根据分布函数的性质 $\lim_{x\to+\infty}F(x)=k=1$
		\item $F(x)=
		\begin{cases}
		1-\ee^{-x^3}, & x>0\\
		0, & x\leq0 
		\end{cases}
		$ , 则 $f(x)=F'(x)=
		\begin{cases}
		3x^2\ee^{-x^3}, & x>0\\
		0, & x\leq0
		\end{cases}
		$
	  \end{enumerate}
	\end{solution}

	\item ( $10$ 分)设二维随机变量 $(X,Y)$ 的概率密度为:
	\begin{equation*}
	f(x,y)=
	\begin{cases}
	\frac{1}{4}, & 2\leq x\leq4,1\leq y\leq3\\
	0, & \text{其他}
	\end{cases},
	\end{equation*}
	试求 $(X,Y)$ 关于 $X$ 与 $Y$ 的边缘概率密度 $f_X(x)$ 与 $f_Y(y)$ , 并判断 $X$ 与 $Y$ 是否相互独立.
	\begin{solution}
		$f_X(x)=\int_{-\infty}^{+\infty}f(x,y)\dd y=
		\begin{cases}
		\int_{1}^{3}\frac{1}{4}\dd y, & 2\leq x\leq4\\
		0, & \text{其它}
		\end{cases}=\begin{cases}
		\frac{1}{2}, & 2\leq x\leq4\\
		0, & \text{其它}
		\end{cases}
		$
		
		同理 $f_Y(y)=
		\begin{cases}
		\frac{1}{2}, & 1\leq y\leq3\\
		0, & \text{其它}
		\end{cases}
		$ , $f_X(x)f_Y(y)=
		\begin{cases}
		\frac{1}{4}, & 2\leq x\leq4,1\leq y\leq 3\\
		0, & \text{其它}
		\end{cases}=f(x,y)
		$
		
		因此 $X$ 与 $Y$ 相互独立
	\end{solution}
	
	\item ( $10$ 分)已知红黄两种番茄杂交的第二代结红果的植株与结黄果的植株的比率为 $3:1$ , 现种植杂交种 $400$ 株, 试用中心极限定理近似计算, 结红果的植株介于 $285$ 与 $315$ 之间的概率. $\left(\varPhi\left(\sqrt{3}\right)=0.9582,\varPhi\left(\sqrt{2}\right)=0.9207\right)$
	\begin{solution}
		设结红果的植株的株数为 $X$ , $X\sim B(400,3/4)$ , 则 $\EE(X)=300$ , $D(X)=75$
		
		根据中心极限定理
			\begin{align*}
			 P(285\leq X\leq 315)&=P\left(\frac{-15}{\sqrt{75}}\leq\frac{X-300}{\sqrt{75}}\leq\frac{15}{\sqrt{75}}\right)=\varPhi\left(\sqrt{3}\right)-\varPhi\left(-\sqrt{3}\right)\\
			&=2\varPhi\left(\sqrt{3}\right)-1=0.9164
			\end{align*}
	\end{solution}
	
	\item ( $8$ 分)设二维随机变量 $(X,Y)$ 的分布律为
	\begin{center}
		\begin{tabularx}{0.8\textwidth}{ZZZZ}
			\hline
			\multirow{2}*{$X$} & \multicolumn{3}{c}{$Y$}\\
			\cline{2-4}
			 & $-1$ & $0$ & $1$\\
			\hline
			$-1$ & $\frac{1}{8}$ & $\frac{1}{8}$ & $\frac{1}{8}$\\
			$0$ & $\frac{1}{8}$ & $0$ & $\frac{1}{8}$\\
			$1$ & $\frac{1}{8}$ & $\frac{1}{8}$ & $\frac{1}{8}$\\
			\hline
		\end{tabularx}
	\end{center}
	求 $\mathrm{Cov}(X,Y)$ .
	\begin{solution}
		$\EE(X)=-1\times\frac{3}{8}+1\times\frac{3}{8}=0$ , 同理通过计算得 $\EE(Y)=0$ , $\EE(XY)=0$
		
		因此 $\text{Cov}(X,Y)=\EE(XY)-\EE(X)\EE(Y)=0$
	\end{solution}
	
	\item ( $10$ 分)设 $X_1,X_2,\ldots,X_n$ 为总体 $X$ 的一个样本, 总体 $X$ 的概率密度为:
	\begin{equation*}
	f(x)=
	\begin{cases}
	(\alpha+1)x^\alpha, & 0<x<1\\
	0, & \text{其他}
	\end{cases},
	\end{equation*}
	求未知参数 $\alpha$ 的矩估计.
	\begin{solution}
		$\EE(X)=\int_{0}^{1}(\alpha+1)x^{\alpha+1}\dd x=\frac{\alpha+1}{\alpha+2}$ , $\mu_1=\overline{X}=\sum_{i=1}^{n}\frac{X_i}{n}$ , 因此 $\alpha=\frac{2\overline{X}-1}{1-\overline{X}}$
	\end{solution}
\end{enumerate}

\section{复习题2}
此复习题非一份完整的考试卷, 而是多个不同的试卷拼凑.

\subsubsection{选择题(每题 $3$ 分)}
\begin{enumerate}
	\item 已知事件 $A$ , $B$ 满足 $P(AB)=P\left( \overline{A}\bigcap\overline{B} \right)$ , 且 $P(A)=0.4$ , 则 $P(B)=$ (\hspace{1pc})
	\fourch{0.4}
	{0.5}
	{0.6}
	{0.7}

	\item 有 $\gamma$ 个球, 随机地放在 $n$ 个盒子中 $(\gamma \leq n)$ , 则某指定的 $\gamma$ 个盒子中各有一球的概率为 (\hspace{1pc})
	\fourch{$\frac{\gamma !}{n^{\gamma}}$}
	{$\mathrm{C}_n^r\frac{\gamma !}{n^\gamma}$}
	{$\frac{n!}{\gamma^n}$}
	{$\mathrm{C}_\gamma^n\frac{n!}{\gamma^n}$}

	\item 设随机变量 $X$ 的概率密度为 $f(x)=c\ee^{-\left|x\right|}$ , 则 $c=$ (\hspace{1pc})
	\fourch{$-\frac{1}{2}$}{$0$}{$\frac{1}{2}$}{$1$}

	\item 掷一颗骰子$600$次, 求“一点”出现次数的均值为 (\hspace{1pc})
	\fourch{$50$}
	{$100$}
	{$120$}
	{$150$}

	\item 设每次试验成功的概率为 $p (0<p<1)$ , 重复进行试验直到第 $n$ 次才取得 $r (1\leq r\leq n)$ 次成功的概率为 (\hspace{1pc})
	\twoch{$\mathrm{C}_{n-1}^{r-1}p^r(1-p)^{n-r}$}
	{$\mathrm{C}_{n}^{r}p^r(1-p)^{n-r}$}
	{$\mathrm{C}_{n-1}^{r-1}p^{r-1}(1-p)^{n-r+1}$}
	{$p^r(1-p)^{n-r}$}

	\item 离散型随机变量 $X$ 的分布函数为 $F(x)$ , 则 $P(X=x_k)=$ (\hspace{1pc})
	\twoch{$P(x_{k-1}\leq X\leq x_k)$}
	{$F(x_{k+1})-F(x_{k-1})$}
	{$P(x_{k-1}<X<x_{k+1})$}
	{$F(x_{k})-F(x_{k-1})$}

	\item 设随机变量 $X, Y$ 是相互独立的两个随机变量, 其分布函数分别为 $F_X(x), F_Y(y)$ , 则随机变量 $Z=\max (X,Y)$ 的分布函数为 (\hspace{1pc})
	\twoch{$F_Z(z)=\max\left\{F_X(x),F_Y(y)\right\}$}
	{$F_Z(z)=\max\left\{\left|F_X(x)\right|,\left|F_Y(y)\right|\right\}$}
	{$F_Z(z)=F_X(x)F_Y(y)$}
	{$F_Z(z)=F_X(z)F_Y(z)$}

	\item 设随机变量 $(X, Y)$ 的方差 $D(X)=4, D(Y)=1$ , 相关系数 $\rho_{XY}=0.6$ , 则方差 $D(3X-2Y)=$ (\hspace{1pc})
	\fourch{$40$}
	{$34$}
	{$25.6$}
	{$17.6$}

	\item 设 $(X_1,X_2,\ldots,X_n)$ 为总体 $N\left(1,2^2\right)$ 的一个样本, $\overline X$ 为样本均值, 则下列结论中正确的是 (\hspace{1pc})
	\twoch{$\frac{\overline{X}-1}{2/\sqrt{n}}\sim t(n)$}
	{$\frac{1}{4}\sum_{i=1}^{n}(X_i-1)^2\sim F(n,1)$}
	{$\frac{\overline{X}-1}{\sqrt{2}/\sqrt{n}}\sim N(0,1)$}
	{$\frac{1}{4}\sum_{i=1}^{n}(X_i-1)^2\sim\chi^2(n)$}

	\item 设总体 $X$ 在 $(\mu-\rho,\mu+\rho)$ 上服从均匀分布, 则参数 $\mu$ 的矩估计量为 (\hspace{1pc})
	\fourch{$\frac{1}{\overline{X}}$}
	{$\frac{1}{n-1}\sum_{i=1}^{n}X_i$}
	{$\frac{1}{n-1}\sum_{i=1}^{n}X_i^2$}
	{$\overline{X}$}

	\item 设二维随机变量 $(X,Y)$ 的分布律为:
	\begin{center}
		\begin{tabularx}{0.8\textwidth}{ZZZ}
			\hline
			 & \multicolumn{2}{c}{$Y$}\\
			\cline{2-3}
			$X$ & 1 & 2\\
			\hline
			1 & $a$ & $\frac{2}{9}$\\
			\hline
			2 & $b$ & $\frac{4}{9}$\\
			\hline
		\end{tabularx}
	\end{center}
	若 $X$ 与 $Y$ 相互独立, 则 (\hspace{1pc})
	\fourch{$a=\frac{4}{9}, b=\frac{1}{9}$}
	{$a=\frac{1}{9}, b=\frac{4}{9}$}
	{$a=\frac{2}{9}, b=\frac{1}{9}$}
	{$a=\frac{1}{9}, b=\frac{2}{9}$}

	\item 设随机变量 $X$ 的概率密度为 $f(x)=
	\begin{cases}
	ax, & 0\leq x\leq 2\\
	0, & \text{其他}
	\end{cases}
	$, 则 $a=$ (\hspace{1pc})
	\fourch{$1$}
	{$\frac{1}{4}$}
	{$\frac{1}{2}$}
	{$\frac{1}{3}$}

	\item 设 $E(X)=2$ , $E(Y)=3$ , 则 $E(3X-4Y+5)=$ (\hspace{1pc})
	\fourch{1}
	{6}
	{$-6$}
	{$-1$}

	\item 设随机变量 $X\sim N(1,4)$ , 则下列随机变量中服从标准正态分布的是 (\hspace{1pc})
	\fourch{$Y=\frac{X-1}{4}$}
	{$Y=\frac{X-1}{2}$}
	{$Y=\frac{X+1}{4}$}
	{$Y=\frac{X+1}{2}$}

	\item 设盒中有 $4$ 支铅笔, $2$ 支钢笔, 从盒中任取 $2$ 支笔(不放回抽样), 则取得 $1$ 支铅笔和 $1$ 支钢笔的概率是 (\hspace{1pc})
	\fourch{$\frac{8}{15}$}
	{$\frac{4}{5}$}
	{$\frac{3}{5}$}
	{$\frac{7}{15}$}

	\item 设总体 $X$ 服从正态分布 $N(\mu,1)$ , $X_1$ , $X_2$ 是来自总体 $X$ 的一个样本, $\hat\mu_1=\frac{2}{3}X_1+\frac{1}{3}X_2$ , 
	$\hat\mu_2=\frac{1}{4}X_1+\frac{3}{4}X_2$ , $\hat\mu_3=\frac{1}{2}X_1+\frac{1}{2}X_2$ 都是 $\mu$ 的无偏估计量, 则其中方差最小的是 (\hspace{1pc})
	\fourch{$\hat\mu_3$}
	{$\hat\mu_2$}
	{$\hat\mu_1$}
	{一样大}

	\item 设随机变量 $X\sim\chi^2(m_1)$ , $Y\sim\chi^2(m_2)$ , 且 $X$ 与 $Y$ 相互独立, 则下列结果中不正确的是 (\hspace{1pc})
	\twoch{$X+Y\sim\chi^2(m_1+m_2)$}
	{$D(Y)=m_2$}
	{$D(X)=2m_1$}
	{$E(X)=m_1$}
 \end{enumerate}

 \subsubsection{填空题(每题 3 分)}
 \begin{enumerate}
	\item 已知 $P(B)=0.3$ , $P\left(\overline{A}\bigcup B\right)=0.7$ , 且 $A$ 与 $B$ 相互独立, 则 $P(A)=$\underline{\hspace{8pc}}
	
	\item 设随机变量 $X$ 服从参数为 $\lambda$ 的泊松分布, 且 $P\left\{X=0\right\}=\frac{1}{3}$ , 则 $\lambda=$\underline{\hspace{8pc}}
	
	\item 设 $X\sim N\left(2,\sigma^2\right)$ , 且 $P\left\{2<X<4\right\}=0.2$ , 则 $P\left\{X<0\right\}=$\underline{\hspace{8pc}}
	
	\item 已知 $D(X)=2$ , $D(Y)=1$ , 且 $X$ 和 $Y$ 相互独立, 则 $D(X-2Y)=$\underline{\hspace{8pc}}
	
	\item 设 $S^2$ 是从 $N(0,1)$ 中抽取容量为 $16$ 的样本方差, 则 $D\left(S^2\right)=$\underline{\hspace{8pc}}

	\item 一批电子元件共有 $100$ 个, 次品数为 $5$ . 连续两次不放回地从中任取一个, 则第二次才取得正品的概率为\underline{\hspace{8pc}}
	
	\item 设事件 $A$ 与 $B$ 相互独立, $P(A)=0.2$ , $P(B)=0.3$ , 则 $P\left(A\bigcup B\right)=$\underline{\hspace{8pc}}
	
	\item 设 $\overline{X}$ 为总体 $X\sim N(3,4)$ 中抽取的样本 $(X_1,X_2,X_3,X_4)$ 的均值, 则 $P\left(-1<\overline{X}<5\right)=$
	
	\underline{\hspace{8pc}}
	
	\item 甲、乙两人独立地进行射击, 甲击中的概率为 $0.9$ , 乙击中的概率为 $0.8$ , 则甲中乙不中的概率等于\underline{\hspace{8pc}}
	
	\item 设 $P(A)=P(B)=P(C)=\frac{1}{3}$ , $ P(AB)=P(AC)=0$ , $ P(BC)=\frac{1}{5}$. 则 $A, B, C$ 中至少有一个发生的概率为\underline{\hspace{8pc}}
	
	\item 设二维随机变量 $(X,Y)$ 的概率密度为 $f(x,y)=
	\begin{cases}
	 C\ee^{-3x}\sin y, & x>0, 0<y<\frac{\uppi}{2}\\
	0, & \text{其他}
	\end{cases}
	$ , 则 $C=$
	
	\underline{\hspace{8pc}}

	\item 设随机变量 $X$ 服从参数为 $3$ 的指数分布, 则 $P\left\{X\geqslant  1\right\}=$\underline{\hspace{8pc}}

	\item 总体 $Y\sim N\left(\mu,\sigma^2\right)$ , $\overline{Y}=\frac{1}{n}\sum_{i=1}^{n}Y_i$ 为样本均值, $S$ 为样本标准差, 当 $\sigma$ 为未知时, $\mu$ 的置信度为 $1-\alpha (0<\alpha<1)$ 的双侧置信区间为\underline{\hspace{8pc}}

	\item 设 $X$ , $Y$ 为两个随机变量, 已知 $E(X)=1$ , $E(Y)=2$ , $E(XY)=5$ , 则 $\Cov(X,Y-4)=$
	
	\underline{\hspace{8pc}}
	
	\item 若 $P(A)=0.5$ , $P\left(B\overline{A}\right)=0.2$ , 则 $P(A+B)=$\underline{\hspace{8pc}}
	
	\item 已知随机变量 $X\sim
	\begin{bmatrix}
	-1 & 0 & 2 & 5\\
	0.25 & 0.25 & 0.25 & 0.25
	\end{bmatrix}
	$ , 那么 $E(X)=$\underline{\hspace{8pc}}
	
	\item 设 $\hat\theta$ 是未知参数 $\theta$ 的一个无偏估计量, 则 $E(\hat\theta)=$\underline{\hspace{8pc}}

	\item 随机变量 $X$ 服从均匀分布 $U(1,3)$ , 则 $P(X>2)=$\underline{\hspace{8pc}}
	
	\item 设随机变量 $X\sim B(100,0.15)$ , 则 $E(X)=$\underline{\hspace{8pc}}
	
	\item 设随机变量 $X\sim N(3,4)$ , 已知 $\varPhi(1)=0.8413$ , 则 $P(X<1)=$\underline{\hspace{8pc}}
	
	\item 设随机变量 $X$ 的概率密度函数为 $f(x)=
	\begin{cases}
	3x^2, & 0\leq x\leq 1\\
	0, & \text{其它}
	\end{cases}
	$, 则$ P\left(X<\frac{1}{2}\right)=$
	
	\underline{\hspace{8pc}}

	\item 设随机变量 $X\sim
	\begin{bmatrix}
	0 & 1 & 3\\
	0.5 & 0.35 & 0.15
	\end{bmatrix}
	$ , 则 $P(X<2)=$\underline{\hspace{8pc}}
	
	\item 设随机变量 $X$ 的期望存在, 则 $E(X-E(X))=$\underline{\hspace{8pc}}
	
	\item 设 $X$ 为随机变量, 已知 $D(X)=2$ , 那么 $D(3X-5)=$\underline{\hspace{8pc}}
 \end{enumerate}

 \subsubsection{计算题}
 \begin{enumerate}
	\item (10分)设随机变量 $X$ 与 $Y$ 具有概率密度: $f(x,y)=
	\begin{cases}
	\frac{1}{8}(x+y) & 0\leq x\leq 2, 0\leq y\leq 2\\
	0 & \text{其它}
	\end{cases}
	$ . 试求: $D(X)$ , $D(Y)$ , 与 $D(2X-3Y)$ .

	\item (10分)某电子计算机主机有 $100$ 个终端, 每个终端有 $80\%$ 的时间被使用. 
	若各个终端是否被使用是相互独立的, 试求至少有 $15$ 个终端空闲的概率. $(\varPhi(1.25)=0.8944, \varPhi(0.31)=0.6217$
	
	\item (10分)试求正态总体 $N\left(\mu,0.5^2\right)$ 的容量分别为 $10$ , $15$ 的两独立样本均值差的绝对值大于 $0.4$ 的概率. $(\varPhi(1.96)=0.975)$
	
	\item (10分)设总体 $X$ 的密度函数为 $f(x)=
	\begin{cases}
	\frac{2}{\theta^2}(\theta-x), & 0<x<\theta\\
	0, & \text{其它}
	\end{cases}
	$ , $\theta>0$ , $X_1,X_2,\ldots,X_{10}$ 为来自总体 $X$ 的样本, 试求当样本观察值分别为 $0.5$ , $1.3$ , $0.6$ , $1.7$ , $2.2$ , $1.2$ , $0.8$ , $1.5$ , $2.0$ , $1.6$ 时未知参数 $\theta$ 的矩估计值.

	\item (10分)某商店拥有某产品共计 $12$ 件, 其中 $4$ 件次品, 已经售出 $2$ 件, 现从剩下的 $10$ 件产品中任取一件, 求这件是正品的概率.
	
	\item (10分)设某种电子元件的寿命服从正态分布 $N(40,100)$ , 随机地取 $5$ 个元件, 求恰有两个元件寿命小于 $50$ 的概率. $(\varPhi(1)=0.8413, \varPhi(2)=0.9772)$

	\item (12分)设总体 $X$ 的分布律为 $P\left\{X=k\right\}=(1-p)^{k-1}p, k=1,2,\ldots.$ ( $p$ 为未知参数), $X_1,X_2,\ldots,X_n$ 是总体 $X$ 的一个样本, 求 $p$ 的极大似然估计量.

	\item (10分)两台车床加工同样的零件, 第一台出现不合格品的概率是 $0.03$ , 第二台出现不合格品的概率是 $0.06$ , 加工出来的零件放在一起, 并且已知第一台加工的零件数比第二台加工的零件数多一倍.
	\begin{enumerate}
		\item 求任取一个零件是合格品的概率.
		\item 如果取出的零件是不合格品, 求它是由第二台车床加工的概率.
	\end{enumerate}

	\item (10分)某仪器装了 $3$ 个独立工作的同型号电子元件, 其寿命(单位: 小时)都服从同一指数分布, 密度函数为 $f(x)=
	\begin{cases}
	\frac{1}{600}\ee^{-\frac{x}{600}}, & x>0\\
	0, & \text{其它}
	\end{cases}
	$ , 试求此仪器在最初使用的 $200$ 小时内, 至少有一个此种电子元件损坏的概率.
\end{enumerate}
	\chapter{复变函数试卷汇总}

\section{复习题 1}
\subsubsection{选择题(每小题 $3$ 分, 共 $15$ 分)}
\begin{enumerate}
	\item $\frac{(\sqrt{3}-\ii)^{4}}{(1-\ii)^{8}}=$ (\hspace{1pc})
	\twoch{$-\frac{1}{2}+\frac{\sqrt{3}}{2}\ii$}{$-\frac{1}{8}\left(1+\sqrt{3}\ii\right)$}{$\frac{1}{8}\left(-1+\sqrt{3} \ii\right)$}{$-\frac{1}{2}-\frac{\sqrt{3}}{2} \ii$}
	
	\item 设 $f(z)=2 x^{3}+3 y^{3} \ii$ , 则 $f(z)$ (\hspace{1pc})
	\twoch{处处不可导}{仅在 $6x^2=9y^2$ 上可导, 处处不解析}{处处解析}{仅在 $(0,0)$ 点可导}
	
	\item 下列等式正确的是 (\hspace{1pc})
	\twoch{$\Ln \mathrm{i}=\left(2 k \uppi-\frac{\uppi}{2}\right) \ii, \ln \ii=\frac{\uppi}{2} \ii$}{$\Ln \ii=\left( 2k\uppi+\frac{\uppi}{2}\right)\ii,\ln\ii=-\frac{\uppi}{2}\ii $}{$\Ln \ii=\left(2 k \uppi+\frac{\uppi}{2}\right) \ii, \ln \ii=\frac{\uppi}{2} \ii$}{$\Ln \ii=\left(2 k \uppi-\frac{\uppi}{2}\right) \ii, \ln \ii=-\frac{\uppi}{2} \ii$}
	
	\item $z=0$ 是函数 $\frac{1-\cos z}{z-\sin z}$ 的 (\hspace{1pc})
	\fourch{本性奇点}{可去奇点}{二级极点}{一级极点}
	
	\item 设 $\mathrm{C}$ 为 $z=(1-\ii)t$ , $t$ 从 $1$ 到 $0$ 的一段, 则 $\int_{\mathrm{C}} \overline{z} \dd z=$ (\hspace{1pc})
	\fourch{$-1$}{$1$}{$-\ii$}{$\ii$}
\end{enumerate}

\subsubsection{填空题(每小题 $3$ 分, 共 $15$ 分)}
\begin{enumerate}
	\item 若 $z+|z|=2+\ii$ , 则 $z=$\underline{\hspace{8pc}}
	
	\item 若 $\mathrm{C}$ 为正向圆周 $|z|=\frac{1}{2}$ , 则 $\oint_{\mathrm{C}} \frac{1}{z-2} \dd z=$\underline{\hspace{8pc}}
	
	\item 若 $z=2-\uppi\ii$ , 则 $\ee^{z}=$\underline{\hspace{8pc}}
	
	\item 若 $f(z)=\cos z^2$ , 则 $f(z)$ 在 $z=0$ 处泰勒展开式中 $z^4$ 项的系数 $a_4=$\underline{\hspace{8pc}}
	
	\item 函数 $f(t)=\sin t$ 的拉普拉斯变换 $F(s)=$\underline{\hspace{8pc}}
\end{enumerate}

\subsubsection{计算题(70分)}
\begin{enumerate}
	\item 设 $u(x,y)=x-2xy$ 且 $f(0)=0$ , 求解析函数 $f(z)=u+\ii v$ . ( $10$ 分)
	
	\item 计算积分 $\oint_{\mathrm{C}}\frac{2\ee^x}{z^5}\dd z$ 的值, 其中 $\mathrm{C}$ 为正向圆周 $|z|=1$ . ( $7$ 分)
	
	\item 计算积分 $\oint_{\mathrm{C}}\frac{3z+5}{z^2-z}\dd z$ 的值, 其中 $\mathrm{C}$ 为正向圆周 $|z|=\frac{1}{2}$ . ( $7$ 分)
	
	\item 求函数 $\frac{1-\cos z}{z^3}$ 在有限奇点处的留数. ( $7$ 分)
	
	\item 求函数 $\frac{2z^2+1}{z^2+2z}$ 在有限奇点处的留数. ( $7$ 分)
	
	\item 将 $f(z)=\frac{z}{(z-2)(z-6)}$ 在 $2<|z|<6$ 内展开为洛朗级数. ( $10$ 分)
	
	\item 若函数 $f(z)=a y^{3}+b x^{2} y+\ii\left(x^{3}+c x y^{2}\right)$ 是复平面上的解析函数, 求 $a,b,c$ 的值. ( $12$ 分)
	
	\item 利用拉普拉斯变换解常微分方程初值问题: $\begin{cases}
	x''(t)+6x'(t)+9x(t)=\ee^{-3t}\\
	x(0)=0, x'(0)=0
	\end{cases}$ . ( $10$ 分)
\end{enumerate}



\section{复习题 1 答案}
\subsubsection{选择题(每小题 $3$ 分, 共 $15$ 分)}
\begin{enumerate}
	\item $\frac{(\sqrt{3}-\ii)^{4}}{(1-\ii)^{8}}=$ (\hspace{0.25pc}D\hspace{0.25pc})
	\twoch{$-\frac{1}{2}+\frac{\sqrt{3}}{2}\ii$}{$-\frac{1}{8}\left(1+\sqrt{3}\ii\right)$}{$\frac{1}{8}\left(-1+\sqrt{3} \ii\right)$}{$-\frac{1}{2}-\frac{\sqrt{3}}{2} \ii$}
	
	\item 设 $f(z)=2 x^{3}+3 y^{3} \ii$ , 则 $f(z)$ (\hspace{0.25pc}B\hspace{0.25pc})
	\twoch{处处不可导}{仅在 $6x^2=9y^2$ 上可导, 处处不解析}{处处解析}{仅在 $(0,0)$ 点可导}
	
	\item 下列等式正确的是 (\hspace{0.25pc}C\hspace{0.25pc})
	\twoch{$\Ln \mathrm{i}=\left(2 k \uppi-\frac{\uppi}{2}\right) \ii, \ln \ii=\frac{\uppi}{2} \ii$}{$\Ln \ii=\left( 2k\uppi+\frac{\uppi}{2}\right)\ii,\ln\ii=-\frac{\uppi}{2}\ii $}{$\Ln \ii=\left(2 k \uppi+\frac{\uppi}{2}\right) \ii, \ln \ii=\frac{\uppi}{2} \ii$}{$\Ln \ii=\left(2 k \uppi-\frac{\uppi}{2}\right) \ii, \ln \ii=-\frac{\uppi}{2} \ii$}
	
	\item $z=0$ 是函数 $\frac{1-\cos z}{z-\sin z}$ 的 (\hspace{0.25pc}D\hspace{0.25pc})
	\fourch{本性奇点}{可去奇点}{二级极点}{一级极点}
	
	\item 设 $\mathrm{C}$ 为 $z=(1-\ii)t$ , $t$ 从 $1$ 到 $0$ 的一段, 则 $\int_{\mathrm{C}} \overline{z} \dd z=$ (\hspace{0.25pc}A\hspace{0.25pc})
	\fourch{$-1$}{$1$}{$-\ii$}{$\ii$}
\end{enumerate}

\subsubsection{填空题(每小题 $3$ 分, 共 $15$ 分)}
\begin{enumerate}
	\item 若 $z+|z|=2+\ii$ , 则 $z=$\underline{\hspace{1pc}$\frac{3}{4}+\ii$\hspace{1pc}}
	
	\item 若 $\mathrm{C}$ 为正向圆周 $|z|=\frac{1}{2}$ , 则 $\oint_{\mathrm{C}} \frac{1}{z-2} \dd z=$\underline{\hspace{1pc}$0$\hspace{1pc}}
	
	\item 若 $z=2-\uppi\ii$ , 则 $\ee^{z}=$\underline{\hspace{1pc}$-\ee^2$\hspace{1pc}}
	
	\item 若 $f(z)=\cos z^2$ , 则 $f(z)$ 在 $z=0$ 处泰勒展开式中 $z^4$ 项的系数 $a_4=$\underline{\hspace{1pc}$-\frac{1}{2}$\hspace{1pc}}
	
	\item 函数 $f(t)=\sin t$ 的拉普拉斯变换 $F(s)=$\underline{\hspace{1pc}$\frac{1}{s^2+1}$\hspace{1pc}}
\end{enumerate}

\subsubsection{计算题(70分)}
\begin{enumerate}
	\item 设 $u(x,y)=x-2xy$ 且 $f(0)=0$ , 求解析函数 $f(z)=u+\ii v$ . ( $10$ 分)
	\begin{solution}
		解析函数的 $u,v$ 必定满足 $\mathrm{C}.-\mathrm{R}.$ 方程, 即
		\begin{equation*}
			\begin{cases}
			\frac{\partial u}{\partial x}=\frac{\partial v}{\partial y}\\
			\frac{\partial u}{\partial y}=-\frac{\partial v}{\partial x}
			\end{cases}
		\end{equation*}
		$\frac{\partial v}{\partial y}=\frac{\partial u}{\partial x}=1-2 y$ , $\frac{\partial v}{\partial y}$ 对 $y$ 积分得 $v=y-y^{2}+\varphi(x)$
		
		$\frac{\partial u}{\partial y}=-2 x=-\frac{\partial v}{\partial x}=-\varphi^{\prime}(x)$ , 可以得出 $\varphi(x)=x^{2}+C$
		
		由于 $f(0)=0$ , 因此 $C=0$ ,即 $f(z)=x-2 x y+\ii\left(y-y^{2}+x^{2}\right)$
	\end{solution}
	
	\item 计算积分 $\oint_{\mathrm{C}}\frac{2\ee^x}{z^5}\dd z$ 的值, 其中 $\mathrm{C}$ 为正向圆周 $|z|=1$ . ( $7$ 分)
	\begin{solution}
		根据高阶导数公式 $f^{(n)}(z_0)=\frac{n!}{2\uppi\ii}\oint_{\mathrm{C}}\frac{f(z)}{(z-z_0)^{n+1}}\dd z$ , 那么
		\begin{equation*}
			\oint_{\mathrm{C}} \frac{2 \ee^{z}}{(z-0)^{5}} \dd z=\frac{2 \uppi \ii}{4 !}\left.\left(2 \ee^{z}\right)^{(4)}\right|_{z=0}=\frac{\uppi \ii}{6}
		\end{equation*}
	\end{solution}
	
	\item 计算积分 $\oint_{\mathrm{C}}\frac{3z+5}{z^2-z}\dd z$ 的值, 其中 $\mathrm{C}$ 为正向圆周 $|z|=\frac{1}{2}$ . ( $7$ 分)
	\begin{solution}
		\begin{equation*}
			\oint_{\mathrm{C}}\frac{3z+5}{z^2-z}\dd z=2\uppi\ii\underset{z=0}{\Res}\frac{3z+5}{z(z-1)}=2\uppi\ii\left.\frac{3z+5}{z-1}\right|_{z=0}=-10\uppi\ii
		\end{equation*}
	\end{solution}
	
	\item 求函数 $\frac{1-\cos z}{z^3}$ 在有限奇点处的留数. ( $7$ 分)
	\begin{solution}
		对 $\cos z$ 进行洛朗展开, $\cos z=1+\sum_{n=1}^{\infty}(-1)^n\frac{z^{2n}}{(2n)!}$ , 那么 $1-\cos z=\sum_{n=1}^{\infty}(-1)^{n+1}\frac{z^{2n}}{(2n)!}$
		
		那么 $\frac{1-\cos z}{z^3}=\sum_{n=1}^{\infty}(-1)^{n+1}\frac{z^{2n-3}}{(2n)!}$ , 根据洛朗系数公式, $\underset{z=0}{\Res}\frac{1-\cos z}{z^3}=c_{-1}=\frac{1}{2}$
	\end{solution}
	
	\item 求函数 $\frac{2z^2+1}{z^2+2z}$ 在有限奇点处的留数. ( $7$ 分)
	\begin{solution}
		\begin{equation*}
			\underset{z=0}{\Res}\frac{2z^2+1}{z^2+2z}=\left.\frac{2z^2+1}{z+2} \right|_{z=0}=\frac{1}{2} , 
			\underset{z=-2}{\Res}\frac{2z^2+1}{z^2+2z}=\left.\frac{2z^2+1}{z}\right|_{z=-2}=-\frac{9}{2}
		\end{equation*}
		
	\end{solution}
	
	\item 将 $f(z)=\frac{z}{(z-2)(z-6)}$ 在 $2<|z|<6$ 内展开为洛朗级数. ( $10$ 分)
	\begin{solution}
		\begin{align*}
			f(z)&=\frac{z}{4}\left( \frac{1}{z-6}-\frac{1}{z-2}\right) =\frac{z}{4}\left( -\frac{1}{6}\frac{1}{1-z/6}-\frac{1}{z}\frac{1}{1-2/z}\right) \\
			&=\frac{z}{4}\left( -\frac{1}{6}\sum_{n=0}^{\infty}(z/6)^n-\frac{1}{z}\sum_{n=0}^{\infty}(2/z)^n\right)\\
			&=-\frac{1}{4}\left( \sum_{n=0}^{\infty}(z/6)^{n+1}+\sum_{n=0}^{\infty}(2/z)^n\right)  
		\end{align*}
	\end{solution}
	
	\item 若函数 $f(z)=a y^{3}+b x^{2} y+\ii\left(x^{3}+c x y^{2}\right)$ 是复平面上的解析函数, 求 $a,b,c$ 的值. ( $12$ 分)
	\begin{solution}
		若 $f(z)$ 为解析函数, 则其实部、虚部满足 $\mathrm{C}.-\mathrm{R}.$ 方程, 设 $u=ay^3+bx^2y$ , $v=x^3+cxy^2$ , 则有
		\begin{equation*}
			\begin{cases}
			\frac{\partial u}{\partial x}=2 b x y=2 c x y=\frac{\partial v}{\partial y}\\
			\frac{\partial u}{\partial y}=3 a y^{2}+b x^{2}=-3 x^{2}-c y^{2}=-\frac{\partial v}{\partial x}
			\end{cases}
		\end{equation*}
		解得\begin{equation*}
			\begin{cases}
			a=1\\
			b=c=-3
			\end{cases}
		\end{equation*}
		
	\end{solution}
	
	\item 利用拉普拉斯变换解常微分方程初值问题: $\begin{cases}
	x''(t)+6x'(t)+9x(t)=\ee^{-3t}\\
	x(0)=0, x'(0)=0
	\end{cases}$ . ( $10$ 分)
	\begin{solution}
		设 $\LL[x]=X(s)$ , 对等式两边作拉普拉斯变换
		\begin{align*}
			\LL[x''+6x'+9x]&=s^2X(s)-sx(0)-x'(0)+6sX(s)-6x(0)+9X(s)\\
			&=s^2X(s)+6sX(s)+9X(s)=\frac{1}{s+3}
		\end{align*}
		那么有 $X(s)=\frac{1}{(s+3)^3}$ , 根据拉普拉斯变换的微分性质 $F''(s)=\LL[t^2f(t)]$
		\begin{equation*}
			\frac{1}{(s+3)^3}=\frac{1}{2}\left(\frac{1}{s+3} \right)''=\frac{\LL[t^2\ee^{-3t}]}{2}
		\end{equation*}
		那么 $x(t)=\frac{t^2\ee^{-3t}}{2}$
	\end{solution}
\end{enumerate}
	\backmatter
	\chapter{后记}
首先向一路披荆斩棘做到这里的读者表示祝贺,至少在精神上你已经成
为一名合格的江理学子。从此你生是江理的人,死是江理的鬼。

面对后面纷繁复杂的课程,守望“三实”的初心。同学们要带着“为人诚实、基础扎实、工作踏实”的精神气质,真诚对待每一门课,仔细倾听老师的讲课,尊重课程就是尊重自己,爱学习是最大的爱自己。愿“三实”品质伴随同学们漫漫课程路,创造理想的成绩。
\begin{flushright}
	二〇二〇年六月于赣州
\end{flushright}
\end{document}